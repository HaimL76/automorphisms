\documentclass[12pt]{article}
\makeatletter
\newcommand*{\rom}[1]{\expandafter\@slowromancap\romannumeral #1@}
\makeatother
\usepackage{amsfonts, amssymb}
\usepackage{mathrsfs, mathdots} 
\usepackage{amsmath}
\usepackage{float}
\usepackage{amsthm}
\usepackage{tikz-cd}
\usepackage{xcolor}
\usepackage{xparse}
\usepackage{setspace}
\usepackage{xfrac}
\usepackage{yfonts}

\newtheorem{theorem}{Theorem}[subsection]
\newtheorem{proposition}[theorem]{Proposition}
\newtheorem{corollary}[theorem]{Corollary}
\newtheorem{lemma}[theorem]{Lemma}
\newtheorem{notations}[theorem]{Notations}
\newtheorem{definition}[theorem]{Definition}
\newtheorem{example}[theorem]{Example}
\newtheorem{remark}[theorem]{Remark}

\ExplSyntaxOn
\NewDocumentCommand{\cycle}{ O{\;} m }
 {
  (
  \alec_cycle:nn { #1 } { #2 }
  )
 }

\seq_new:N \l_alec_cycle_seq
\cs_new_protected:Npn \alec_cycle:nn #1 #2
 {
  \seq_set_split:Nnn \l_alec_cycle_seq { , } { #2 }
  \seq_use:Nn \l_alec_cycle_seq { #1 }
 }
\ExplSyntaxOff
\usepackage{titling}
\newcommand{\subtitle}[1]{%
  \posttitle{%
    \par\end{center}
    \begin{center}\large#1\end{center}
    \vskip0.5em}%
}
\usepackage{arydshln}
\setcounter{tocdepth}{2}
\setlength{\dashlinedash}{1.2pt}
\setlength{\dashlinegap}{1.5pt}
\setlength{\arrayrulewidth}{0.2pt}
\title{The pro-isomorphic zeta-functions of some nilpotent Lie algebras over integer rings}
\subtitle{Research proposal for a Ph.D. Thesis\\
Under the supervision of Dr. Michael M. Schein\\
Department of Mathematics, Bar-Ilan University}
\author{Haim Lavi}
\date{\today}
\begin{document}
\maketitle
\newpage
\tableofcontents
\newpage
\begin{abstract}
Let $G$ be any group. For any natural number $n\in\mathbb{N}$, let $a_{n}^{\leq}(G)$ be the number of subgroups $H\leq G$ such that $[G:H]=n$. If $G$ is finitely-generated, then $a_{n}^{\leq}(G)<\infty$, and we can define a Dirichlet series of the form $\zeta_G(s):=\sum_{n=1}^\infty a_{n}^{\leq}(G){n}^{-s}$, where $s\in\mathbb{C}$. Assume, in addition, that $G$ is also nilpotent and torsion-free, then this function has some properties of the Riemann zeta-function $\zeta$, such as the Euler decomposition of $\zeta$ into a product of local factors indexed by primes. A version of this zeta-function counts pro-isomorphic subgroups, and an analogous function may be defined for appropriate Lie rings. We study here the pro-isomorphic zeta-functions for a family of nilpotent Lie rings of unbounded nilpotency class. We shall compute the automorphism groups of these Lie rings explicitly and use them to prove uniformity of the local factors of the pro-isomorphic zeta-functions, and aim to determine them explicitly.
\end{abstract}
\section{Scientific Background}
\subsection{Introduction}
Although we will work with Lie algebras, for motivation we first present analogous and more natural questions in the context of groups.
\begin{proposition} \label{prop:finite.number.subgroups}
Let $G$ be any finitely generated group, and let $n\in\mathbb{N}$ be any natural number. Then there is a finite number of subgroups $H\leq G$ such that $[G:H]=n$.
\end{proposition}
This proposition gives rise to an entire branch of group theory, called \textbf{subgroup growth}. We denote by \[a_{n}^{\leq}(G):=|\{H\leq{G} : [G:H]=n\}|\]
the number of subgroups of $G$ of index $n$ and consider the sequence $\{a_{n}^{\leq}(G)\}_{n=1}^{\infty}$. The subject of subgroup growth aims to relate the properties of this sequence to the algebraic structure of $G$. We denote by \[a_{n}^{\trianglelefteq}(G):=|\{H\trianglelefteq{G} : [G:H]=n\}|\]
the number of \textbf{normal} subgroups of $G$. We now define another type of subgroups of $G$.
\begin{definition}
\label{def:pro.isomorphic}
Let $G$ be a group. A subgroup $H\leq G$ is called \textbf{pro-isomorphic} if $\widehat{H}\cong\widehat{G}$, where $\widehat{H}$ and $\widehat{G}$ are the profinite closures of $H$ and $G$, respectively.
\end{definition}
We denote by \[a_{n}^{\wedge}(G):=|\{H\leq G : \widehat{H}\cong\widehat{G}, [G:H]=n\}|\]
the number of \textbf{pro-isomorphic} subgroups of $G$.

Now we define generating functions determined by these sequences. 
\begin{definition}
\label{def:zeta.function}
Let $G$ be a finitely-generated group, and let $\ast\in\{\leq,\trianglelefteq,\wedge\}$, then we define the zeta-function $\zeta_{G}^{\ast}(s)$ to be the Dirichlet series $\sum_{n=1}^{\infty}a_{n}^{\ast}(G){n}^{-s}$. 
\end{definition}
\begin{proposition}
Let $G$ be a $\mathcal{T}$-group, i.e. finitely-generated, nilpotent and torsion-free group, and let $\ast\in\{\leq,\trianglelefteq,\wedge\}$. Then the zeta-functions for $G$ have the following properties:\par
\textbf{Polynomial growth and convergence}. There exist constants $b$ and $C$ such that $a_{n}^{\leq}(G)\leq{C}{n}^{b}$ for all $n\in\mathbb{N}$. Hence $\zeta_{G}^{\ast}(s)$ converges on some right half plane $\mathrm{Re}\,s>\alpha$. The abscissa of convergence \[\alpha^{\ast}:=\inf\{\alpha : \zeta_{G}^{\ast}(s)\,\mathrm{converges}\,\mathrm{on}\,\mathrm{Re}\,s>\alpha\}\] are known to be rational numbers.

\textbf{Euler decomposition}.
We have that $\zeta_{G}^{\ast}(s)=\prod_{p}\zeta_{G,p}^{\ast}(s)$, where $\zeta_{G,p}^{\ast}(s)=\sum_{k=0}^{\infty}a_{p^k}^{\ast}(G)p^{-ks}$.

\textbf{Rationality}. For all primes $p$, there is a rational function $W_{p}^{\ast}\in\mathbb{Q}(X)$ such that $\zeta_{G,p}^{\ast}(s)=W_{p}^{\ast}(p^{-s})$.
\end{proposition}
\begin{definition}
Let $\zeta_{L,p}^{\ast}$ be a \textbf{finitely-uniform} zeta-function, i.e. there are rational functions $W_{1}^{\ast}(X,Y),\dots,W_{r}^{\ast}(X,Y)\in\mathbb{Q}(X,Y)$, such that for all $p$, there is some $1\leq{i}\leq{r}$ such that $\zeta_{G,p}^{\ast}(s)=W_{i}^{\ast}(p,p^{-s})$. We say $W_{i}^{\ast}$ satisfies a \textbf{functional equation} if $W_{i}^{\ast}(X^{-1},Y^{-1})=\pm{X}^{a}{Y}^{b}{W_{i}^{\ast}(X,Y)}$, where $a,b\in\mathbb{N}\cup\{0\}$, $X^{a}Y^{b}$ being called the \textbf{symmetry factor}.    
\end{definition}
Let $G$ be a $\mathcal{T}$-group, then $\zeta_{G,p}^{\leq}(s)$ satisfies a functional equation, for all but finitely many $p$, with the same symmetry factor. If $G$ is a $\mathcal{T}$-group of nilpotency class $2$, same is true for $\zeta_{G,p}^{\trianglelefteq}(s)$.

There are specific examples of $\mathcal{T}$-groups of nilpotency class $3$, such that $\zeta_{G,p}^{\trianglelefteq}(s)$ is \textbf{uniform}, i.e. $\zeta_{G,p}^{\trianglelefteq}(s)=W(p,p^{-s})$, but $W(p,p^{-s})$ does not satisfy a functional equation.

In the case of pro-isomorphic zeta-functions, there are results showing that under restrictive conditions, $\zeta_{G,p}^{\wedge}(s)$ is uniform and satisfies a functional equation, see, for instance, \cite{BermanKlopschOnn}. However, there are examples of $G$ of nilpotency class $4$ such that $\zeta_{G,p}^{\wedge}(s)$ is uniform but does not have a functional equation.

If any Dirichlet series, in particular $\zeta_{G}^{\ast}(s)$, convergences on some subset of $\mathbb{C}$, one may reconstruct its coefficients $a_{n}^{\ast}(G)$, the sequence of our interest, using Perron's formula, which is an implementation of a reverse Mellin transform; see, for instance, \cite{MontgomeryVaughan}.\par
As an example, consider the group $G=\mathbb{Z}$, for which every finite-index subgroup $H\leq{\mathbb{Z}}$ is of the form $H=n\mathbb{Z}=\langle n\rangle$ for some $n\in\mathbb{N}$. Thus $H\cong \mathbb{Z}$, as both are infinite cyclic groups, and so $\widehat{H}\cong\widehat{\mathbb{Z}}$. Since we have only one subgroup of index $n$ for every $n\in\mathbb{N}$, then $a_{n}^{\ast}(\mathbb{Z})=1$, for $\ast\in\{\leq,\trianglelefteq,\wedge\}$. Thus, its pro-isomorphic zeta-function is $\zeta_{\mathbb{Z}}^{\wedge}(s)=\sum_{i=1}^{\infty}n^{-s}=\zeta(s)$, the Riemann zeta-function, which is known to converge for $\mathrm{Re}\,s>1$. It is known that the Riemann zeta-function decomposes into an infinite product of local zeta-functions, that is, $\zeta(s)=\prod_p\zeta_p(s)=\prod_p\sum_{k=0}^\infty p^{-ks}=\prod_p\frac{1}{1-p^{-s}}$, where the product runs over all the prime numbers.\par
This research concentrates on the growth of \textbf{pro-isomorphic} subgroups defined above, hence we shall restrict our further discussion to the pro-isomorphic case only.
\subsection{Linearization}
We may transfer the ideas from the above discussion about groups to a linear context, where we can use tools from linear algebra.
If $L$ is a $\mathbb{Z}$-Lie algebra, namely a free $\mathbb{Z}$-module of finite rank with a Lie bracket operation, then consider the number $a_{n}^{\wedge}(L)$ of subalgebras $M\leq L$ of index $n$ such that $M\otimes_{\mathbb{Z}}\mathbb{Z}_p\cong L\otimes_{\mathbb{Z}}\mathbb{Z}_p$ for all primes $p$. It is known that $a_{n}^{\wedge}(L)<\infty$ for all $n\in\mathbb{N}$. The Dirichlet series $\zeta_{L}^{\wedge}(s):=\sum_{n=1}^{\infty}{a_{n}^{\wedge}}(L)n^{-s}$, is called the \textbf{pro-isomorphic zeta-function} of $L$. By the Mal'cev correspondence, to every $\mathcal{T}$-group $G$ one may associate a Lie algebra $L=L(G)$ such that $\zeta_{G,p}^{\wedge}(s)=\zeta_{L,p}^{\wedge}(s)$ for all but finitely many primes $p$. If $G$ has nilpotency class $2$, one may obtain the equality for all primes.\par
Let $L$ be a $\mathbb{Z}$-Lie algebra. Choose a basis $(b_1,\dots,b_r)$, where $r=\mathrm{rank}\,L$. Let $\mathcal{L}_{p}=L\otimes_{\mathbb{Z}}\mathbb{Q}_p$, for any $p$. This is a $\mathbb{Q}_p$-Lie algebra, and our choice of basis allows us to identify the automorphism group $G(\mathbb{Q}_p)=Aut_{\mathbb{Q}_p}(\mathcal{L}_{p})$ with a subgroup of $GL_r(\mathbb{Q}_p)$. Note that $\mathcal{L}_{p}$ contains a $\mathbb{Z}_p$-lattice $L_{p}=L\otimes_{\mathbb{Z}}\mathbb{Z}_p$. If $\varphi\in G(\mathbb{Q}_p)$, then $\varphi(L_{p})=L_{p}$ if and only if $\varphi\in G(\mathbb{Z}_p)=G(\mathbb{Q}_p)\cap GL_r(\mathbb{Z}_p)$. Similarly, $\varphi(L_{p})\subseteq L_{p}$ if and only if $\varphi\in G^{+}(\mathbb{Q}_p):=G(\mathbb{Q}_p)\cap {M}_r(\mathbb{Z}_p)$, where ${M}_r(\mathbb{Z}_p)$ is the collection of $r\times r$ matrices with entries in $\mathbb{Z}_p$. Note that $G^{+}(\mathbb{Q}_p)$ is a monoid, not a group.\par
There is a bijection between the set $G(\mathbb{Z}_p)\backslash G^{+}(\mathbb{Q}_p)$ of right cosets of $G(\mathbb{Z}_p)$ and the set $\{M\leq L_{p} : M\cong L_{p}\}$ of $L_{p}$-subalgebras which are isomorphic to $L_{p}$ itself. This bijection takes $G(\mathbb{Z}_p)g$ to $M=\varphi(L_{p})$, for $\varphi\in G(\mathbb{Z}_p)g$.
Observe that $[L_{p}:M]=|\det\varphi|_p^{-1}$, where $|\cdot|_p$ is the $p$-adic norm on $\mathbb{Q}_{p}$, and therefore,
\begin{equation}
\label{equation.proisomorphic.zeta}
\zeta_{L,p}^{\wedge}(s)=\underset{\overset{\scriptscriptstyle M\leq L_{p}}{\scriptscriptstyle M\cong L_{p}}}{\sum}[L_{p}:M]^{-s}=\underset{\scriptscriptstyle G(\mathbb{Z}_p)\varphi\in G(\mathbb{Z}_p)\backslash G^{+}(\mathbb{Q}_p)}{\sum}|\det\varphi|_p^s
\end{equation}.
\subsection{$p$-adic Integration}
We fix a specific right Haar measure $\mu$ on the group $G(\mathbb{Q}_{p})$ by normalizing $\mu(G(\mathbb{Z}_p))=1$. Now we are able to express $\zeta_{L,p}^{\wedge}(s)$ as a $p$-adic integral, as first done by du Sautoy and Lubotzky \cite{DuSautoyLubotzky}. Since $|\det{g}|_{p}^{s}$ is constant on right cosets of $G(\mathbb{Z}_{p})$, we have $|\det\varphi|_{p}^{s}=\displaystyle\int_{G(\mathbb{Z}_p)\varphi}{|\det{g}|_{p}^{s}}d\mu(g)$. By \eqref{equation.proisomorphic.zeta}, we have \[\zeta_{L,p}^{\wedge}(s)=\underset{\scriptscriptstyle G(\mathbb{Z}_p)\varphi\in G(\mathbb{Z}_p)\backslash{G^{+}(\mathbb{Q}_p)}}{\sum}|\det\varphi|_{p}^{s}=\underset{\scriptscriptstyle{G(\mathbb{Z}_p)}\varphi\in G(\mathbb{Z}_p)\backslash{G^{+}(\mathbb{Q}_p)}}{\sum}\displaystyle{\int_{G(\mathbb{Z}_p)\varphi}}{|\det{g}|_{p}^{s}}d\mu(g)=\]\[=\displaystyle\int_{G^{+}(\mathbb{Z}_p)\varphi}{|\det{g}|_{p}^{s}}d\mu(g)\].
\section{Research Goals and Methodology}
\subsection{The Lie algebras $L_{n}$}
Let $R$ be a commutative ring, and let $L_{n}(R)$ be the set of strictly upper triangular $n\times{n}$ matrices with elements in $R$. The Lie bracket $[A,B]=AB-BA$ gives $L_{n}(R)$ the structure of a nilpotent $R$-Lie algebra. Set $L_{n}=L_{n}(\mathbb{Z})$ and $\mathcal{L}_{n}=L_{n}(\mathbb{Q})$, while $L_{n,p}=L_{n}(\mathbb{Z}_{p})$ and $\mathcal{L}_{n,p}=L_{n}(\mathbb{Q}_{p})$.

Let $e_{ij}\in{L_{n}}$ be the matrix with $1$ in the $(i,j)$ entry and $0$ elsewhere. Then $B:=\{e_{ij} : 1\leq{i}<{j}\leq{n}\}$ is a basis of $L_{n}$, which we order as follows: $(e_{12},\dots,e_{n-1,n},e_{13},\dots,e_{n-2,n},e_{14},\dots,e_{n-3,n},\dots,e_{1n})$. In other words we order the elements of $B$ so that $e_{ij}<e_{kl}$ if either $j-i<l-k$ or if $j-i=l-k$ and $i<k$. Note that $\gamma_{1}L_{n},\dots,\gamma_{n-1}L_{n}$ are spanned by final segments of $B$. The aim of our research is to study the pro-isomorphic zeta function of $L_{n}$. Berman has computed $\zeta_{L_{4,p}}^{\wedge}(s)$ explicitly for all $p$ and shown that $\zeta_{L_{5,p}}^{\wedge}(s)$ are uniform.
\begin{remark}
Let $U_n(R)$ be the group of $n\times{n}$ upper unitriangular matrices over $R$. By the Mal'cev correspondence, ${\zeta^\wedge_{U_{n}(\mathbb{Z}_{p})}}(s)={\zeta^\wedge_{L_{n,p}}}(s)$, for all but finitely many primes $p$. This relates the subject of our research to the group-theoretic motivation presented at the beginning of this paper.
\end{remark}
\subsection{Research goals}
The project consists of three major steps:
\begin{enumerate}
\item
Computing the automorphism group $G_{n}(\mathbb{Q}_{p})$ of the $\mathbb{Q}_p$-Lie algebras $\mathcal{L}_{n,p}$, for all $n\in\mathbb{N}$ and all primes $p$, and understanding the submonoid $G_{n}^{+}(\mathbb{Q}_{p})$.
\item
Showing that the pro-isomorphic zeta-functions $\zeta_{L_{n,p}}^{\wedge}(s)$ are uniform, or finitely-uniform, for all $n\in\mathbb{N}$. We expect the method used by Berman in the case $n=5$ to generalize.
\item
Giving an explicit uniform formula for the zeta-functions $\zeta_{L_{n,p}}^{\wedge}(s)$ for specific values of $n$, if not for all $n\in\mathbb{N}$.
We aim to compute an explicit formula for $\zeta_{L_{n,p}}^{\wedge}(s)$ for all $n \geq 4$ and all primes $p$ in terms of a suitable combinatorical framework that is compatible with the complicated structure of $G_{n}^{+}(\mathbb{Q}_p)$. We will study whether $\zeta_{L_{n,p}}(s)$ have functional equations and in particular whether they satisfy the conjecture of \cite{BermanKlopschOnn}, which says that if $\zeta_{L,p}^{\wedge}(s)$ has functional equations and $L$ is graded, then the exponent $b$ in the symmetry factor must be the weight of a minimal grading of $L$. Note that $L_{n}$ is naturally graded by its lower central series, with minimal grading of weight $\sum_{i=1}^{n-1}{i(n-i)}$. In the event that we do not succeed in this task in general, we still expect to complete Berman's unfinished calculation in the case $n=5$. In this case, the combinatorics is just simple enough to handle directly, possibly with some assistance from a computer.
\end{enumerate}

Automorphism groups of Lie algebras, and in particular $G_{n}(\mathbb{Q}_{p})$, have been studied before from a different point of view, and there are classical results showing that any automorphism may be expressed as a product of automorphisms of a specific type; see, for instance, the main result of Gibbs \cite{Gibbs}.  These statements are not explicit enough for our purposes; indeed, the submonoid $G_{n}^{+}(\mathbb{Q}_p)$ arises for us as the domain of integration of a $p$-adic integral.  In order to calculate this integral, it will be useful to decompose the automorphism group $G_{n}(\mathbb{Q}_p)$ into a repeated semi-direct product of groups with a simple structure.

We will need to analyze the monoid $G_{n}^{+}(\mathbb{Q}_p)$ and its $G_{n}(\mathbb{Z}_p)$ right-cosets, as we have seen above. This will give us both the integrand, which is $\det\varphi$ for every $G_{n}(\mathbb{Z}_p)$ right-coset $G_{n}(\mathbb{Z}_p)\varphi$, and the domain of integration, which is the monoid $G_{n}^{+}(\mathbb{Q}_p)$. It is expected that $G_{n}^{+}(\mathbb{Q}_{p})$ will have a complicated structure, and expressing $\zeta_{L_{n,p}}^{\wedge}(s)$ explicitely in a way that allows us to verify functional equations will depend on finding a suitable combinatorical structure. However, we expect that the method of Berman's proof of uniformity in the case $n=5$, which relies on generalizations of results by Stanley on generating functions of polyhedral cell complexes, will generalize to all $n\geq{4}$.

Let $K$ be a number field of degree $d=[K:\mathbb{Q}]$, and let $\mathcal{O}_K$ be its ring of integers. Let $L$ be a $\mathbb{Z}$-Lie algebra of rank $r$. We can naturally consider the base extension $L_{K}=L\otimes_{\mathbb{Z}}\mathcal{O}_{K}$ as a $\mathbb{Z}$-Lie algebra of rank $rd$, and by extension of scalars we can consider also $L_{K,p}=(L\otimes_{\mathbb{Z}}\mathcal{O}_K)\otimes_{\mathbb{Z}}\mathbb{Q}_p$ as a $\mathbb{Q}_p$-Lie algebra of the same rank. If $\mathcal{L}_{p}$ has a certain property originally introduced by Segal, which we call \textbf{rigidity}, then a simple modification of the calculation of $\zeta_{L,p}^{\wedge}(s)$ gives $\zeta_{L_{K},p}^{\wedge}(s)$ for all $K$ without any substantial extra effort. Berman, Glazer and Schein \cite{BermanGlazerSchein} give a criteria for rigidity in terms of the centralizers of elements of $L_{p}$. Note that the criterion does not necessarily apply for all $p$. We shall study whether the criterion applies to the $\mathbb{Q}_p$-Lie algebras $\mathcal{L}_{n,p}$ of our work. If so, then uniformity of $\zeta_{L,p}^{\wedge}(s)$ will imply finite uniformity of $\zeta_{L_{K},p}(s)$. Indeed, for each of the finitely many decomposition types of a prime in $\mathcal{O}_K$, there is a rational function in two variables $W\in\mathbb{Q}(X,Y)$ such that $\zeta_{\mathcal{L}\otimes\mathcal{O}_K,p}^{\wedge}(s)=W(p,p^{-s})$ for all $p$ of that decomposition type.
\subsection{Representing matrices of $L_{n,p}$-automorphisms}
To sum up what is already known about $L_{n,p}$-automorphisms, let $\varphi\in{G_{n}}(\mathbb{Q}_p)$. Let $M$ be the matrix of $\varphi$ with respect to the ordered basis $B$, where $M$ acts from the right on row vectors.  Thus:
\[
M=\begin{pmatrix}
\varphi(e_{12})\\
\varphi(e_{23})\\
\varphi(e_{n-1,n})\\
\hdashline
\varphi(e_{13})\\
\vdots\\
\varphi(e_{n-2,n})\\
\hdashline
\vdots\\
\hdashline
\varphi(e_{1n})\\
\end{pmatrix}\\
\]
Since the elements $\{\gamma_{k}\mathcal{L}_{n,p}\}_{k=1}^{n-1}$ of the lower central series are characteristic subalgebras, $M$ has the block-diagonal form
\[M=\begin{pmatrix}
M_{11}&\vline&M_{12}&\vline& M_{13}
&\dots&\vline&M_{1,n-2}&\vline&M_{1,n-1}\\
\hline
0&\vline&M_{22}&\vline&M_{23}&\dots&\vline 
&M_{2,n-2} &\vline& M_{2,n-1}\\
\hline
\vdots & \vline & \vdots&\vline & \vdots & \ddots &\vline & \vdots &\vline& \vdots\\
\hline
0 & \vline & 0&\vline & 0 & \dots &\vline & M_{2,n-2} &\vline& M_{2,n-1}\\
\hline
0 & \vline & 0 &\vline & 0 & \dots &\vline & 0 &\vline& M_{n-1,n-1}\\
\end{pmatrix}
\]
where each block $M_{ij}$ is an $(n-i)\times{(n-j)}$ matrix.\par
Note that $\begin{pmatrix}
M_{11} & \vline & M_{12}\\
\hline
0 & \vline & M_{22}\\
\end{pmatrix}$ gives an element of $Aut_{\mathbb{Q}_{p}}(\sfrac{\mathcal{L}_{n,p}}{\gamma_{3}\mathcal{L}_{n,p}})$. Berman determined this automorphism group for all $n$ and $p$, see \cite[Prop. 3.6]{Berman}. It follows that the block $M_{11}$ must be either diagonal or anti-diagonal.
\subsection{Preliminary results}
\label{preliminary.results}
We have made progress towards step 1 of our research program, namely determining the automorphism groups $G_{n}(\mathbb{Q}_p)$, based on Berman's results stated above.
Define $\eta\in{G_{n}(\mathbb{Q}_p)}$ by $\eta(e_{ij}):=(-1)^{j-i-1}e_{n+1-j,n+1-i}$, for $1\leq{i}\leq{n-1}$ and ${i+1}\leq{j}\leq{n}$. We check that $\eta\in{G_{n}(\mathbb{Q}_{p})}$. Replacing $\varphi$ by $\varphi\circ\eta$, we assume without loss of generality that $M_{11}$ is diagonal. Indeed, let $G_{n}^{0}(\mathbb{Q}_{p})\leq{G_{n}(\mathbb{Q}_{p})}$ be the subgroup of automporhisms with diagonal block $M_{11}$, then $G_{n}(\mathbb{Q}_{p})=G_{n}^{0}(\mathbb{Q}_{p})\coprod{G_{n}(\mathbb{Q}_{p})}\eta$, and we proceed to determine $G_{n}^{0}(\mathbb{Q}_{p})$.
We check, for any $\lambda_{1},\lambda_{2},\dots,\lambda_{n-1}\in\mathbb{Q}_{p}^{\ast}$, that the diagonal matrix \[h=\mathrm{diag}(\lambda_{1},\lambda_{2},\dots,\lambda_{n-1},\lambda_{1}\lambda_{2},\lambda_{2}\lambda_{3}\dots,\lambda_{n-2}\lambda_{n-1},\dots,\lambda_{1}\lambda_{2}\cdots\lambda_{n-2}\lambda_{n-1})\]
represents an automorphism of $\mathcal{L}$. Thus, multiplying $\varphi$ from the right by a unique such automorphism, we may now assume that $M$ has $1$'s on the diagonal. By this print we know
$H(\mathbb{Q}_{p}):=\{\mathrm{diag}(\lambda_{1},\lambda_{2},\dots) : \lambda_{1},\lambda_{2},\dots\in\mathbb{Q}_{p}^{\ast}\}$ is the \textbf{reductive part} of $\mathbb{Q}_{p}$, while $N(\mathbb{Q}_{p}):=\{\varphi\in{G_{n}(\mathbb{Q}_{p})}$ with $1$'s
 in diagonal $\}$ is the \textbf{unipotent radical} of $G_{n}(\mathbb{Q}_{p})$.
Every $g\in{G_{n}(\mathbb{Q}_{p})}$ has a unique decomposition $g=uh$, with $u\in{N}(\mathbb{Q}_{p})$, $h\in{H(\mathbb{Q}_{p})}$. We aim to determine the structure of the unipotent radical $N(\mathbb{Q}_p)$ by decomposing it into a semidirect product of abelian subgroups. 
We can simplify the domain of integration, for the $p$-adic integral that we aim to calculate, at the price of replacing a single integral by multiple integrals. As we saw earlier, the calculation of $\zeta_{L}^{\wedge}(s)$ requires computing $G_{n}(\mathbb{Z}_p)$ and $G_{n}^{+}(\mathbb{Q}_p)$ first. Assuming we have already computed $G_{n}(\mathbb{Q}_p)$, based on the strategy that we have presented above, we need to identify $G_{n}(\mathbb{Z}_p)$ as a subgroup of $G_{n}(\mathbb{Q}_p)$, which is expected not to be difficult, and continue from there to identify the monoid $G_{n}^{+}(\mathbb{Q}_p)$, which is expected to be a substantial challenge. By applying \textbf{Fubini's theorem} for semidirect products of topological groups, we have that \[\zeta_{\mathcal{L}}^{\wedge}(s)=\displaystyle\int_{G_{n}^{+}(\mathbb{Q}_p)}|\det\varphi|_p^sd\mu_{G_{n}(\mathbb{Z}_p)\varphi}=\displaystyle\int_{H^+(\mathbb{Q}_p)}\left(\displaystyle\int_{N_{h}^+}|\det{uh}|_p^sd\mu_{N(\mathbb{Q}_p)}\right)d\mu_{H(\mathbb{Q}_p)}\]
where $H^+(\mathbb{Q}_p)$ consists of all $h\in H(\mathbb{Q}_p)$ that appear in the decomposition $\varphi=uh$ for some $\varphi\in G_{n}^{+}(\mathbb{Q}_p)$, and, for a given $h\in H^+(\mathbb{Q}_p)$, we set $N_h^+(\mathbb{Q}_p):=\{u\in N(\mathbb{Q}_p) : uh\in{G}^+(\mathbb{Q}_p)\}$. The integrand of the inner integral is constant, so the integral amounts to computing the measure of the set $N_{h}^+(\mathbb{Q}_p)$. The advantage that we gain by this decomposition is that it simplifies the calculation of the integral. The integral function $|\det{uh}|_p^s=|\det{h}|_p^s$ depends only on $h$, so computing the inner integral amounts to finding the measure of $N_{h}^+(\mathbb{Q}_p)$. For the unipotent matrix $u$, the determinant is $1$, for the digonal matrix $h$, we have the following proposition,
\begin{proposition}
\label{prop.h.matrix.determinant}
Let $n\geq{2}$, and let \[h=\mathrm{diag}(\lambda_{1},\lambda_{2},\dots,\lambda_{n-1},\lambda_{1}\lambda_{2},\lambda_{2}\lambda_{3}\dots,\lambda_{n-2}\lambda_{n-1},\dots,\lambda_{1}\lambda_{2}\cdots\lambda_{n-2}\lambda_{n-1})\]
as defined above, then $\det{h}=\prod_{i=1}^{n-1}\lambda_i^{i(n-i)}$.
\end{proposition}
We show this by induction on $n$.
We conclude this section with the observation that the sets $N_{h}^+(\mathbf{{Q}_p)}$ arising in our computation are quite complicated. We will decompose $N(\mathbb{Q}_p)$ into an iterated semidirect product of a large number of subgroups, each abelian with a simple structure. This will decompose the integral over $N_{h}^+(\mathbb{Q}_p)$ into a multiple integral that can be computed explicitly, given a suitable combinatorical framework. Hence, we strive to decompose $N(\mathbb{Q}_p)$ itself into a product of finitely many simpler subgroups, $N(\mathbb{Q}_p)= N(\mathbb{Q}_p)_1\rtimes_{\phi} N(\mathbb{Q}_p)_2\rtimes_{\phi}\cdots\rtimes_{\phi} N(\mathbb{Q}_p)_{m}$, where $m$ is the number of subgroups in the decomposition of $N(\mathbb{Q}_p)$, which means that $\displaystyle\int_{G_{n}^{+}(\mathbb{Q}_p)}|\det\varphi|_p^sd\mu_{G_{n}(\mathbb{Q}_p)}=$\[=\displaystyle\int_{H^+(\mathbb{Q}_p)}\left(\displaystyle\int_{\mathcal{N}^+_{m}}\cdots\left(\displaystyle\int_{\mathcal{N}^+_3}\left(\displaystyle\int_{\mathcal{N}^+_2}\left(\displaystyle\int_{\mathcal{N}^+_1}|\det\varphi|_p^sd\mu_{\mathcal{N}_1}\right)d\mu_{\mathcal{N}_2}\right)d\mu_{\mathcal{N}_3}\right)\cdots d\mu_{\mathcal{N}_{m}}\right)d\mu_{H(\mathbb{Q}_p)}\]
where we denote $\mathcal{N}_i:=N(\mathbb{Q}_p)_i$ and $\mathcal{N}^+_i:=N^+(\mathbb{Q}_p)_i$, for every $1\leq i\leq m$. One checks that every $\mathcal{N}^+_i$ depends on $h,u_1,u_2,\dots,u_{i-1}$, if $\varphi=u_{m}\cdots{u_2}{u_1}h$, where $h\in H^+(\mathbb{Q}_p)$ and $u_k\in\mathcal{N}_{k}$, for every $1\leq k\leq i-1$. 
All the subgroups in the decomposition of $N(\mathbb{Q}_p)$ are obviously unipotent as well, which means that their determinants are also $1$. This means that computing the inner integrals amounts to determining the measure of the sets $\mathcal{N}^+_i$ in terms of $h,u_1,u_2,\dots,u_{n-1}$.
\begin{thebibliography}{2}
\bibitem{Berman} Mark. N. Berman,
Proisomorphic zeta functions of groups
, Ph.D. thesis, University of Oxford,
2005.
\bibitem{BermanGlazerSchein} Mark N. Berman, Itay Glazer, and Michael M. Schein, Pro-isomorphic zeta functions of nilpotent groups and lie rings under base extension, Trans. Amer. Math. Soc. 375 (2022), 1051–1100.
\bibitem{BermanKlopschOnn} M. N. Berman, B. Klopsch, and U. Onn,
On pro-isomorphic zeta functions of $D^{\ast}$-groups of even Hirsch length, arXiv/1511.06360v3.
\bibitem{DuSautoyLubotzky} M.P.F. du Sautoy and A. Lubotzky, Functional equations and uniformity for
local zeta functions of nilpotent groups, Amer. J. Math. 118 (1996), no. 1, 39–
90.
\bibitem{Gibbs} John A. Gibbs, Automorphisms of certain unipotent groups, J. Algebra 14 (1970), 203-228.
\bibitem{GrunewaldSegalSmith} F. J. Grunewald, D. Segal, and G. C. Smith, Subgroups of finite index in nilpotent groups,
Invent. Math. 93 (1988), no. 1, 185–223.
\bibitem{LubotzkyMannSegal} Alexander Lubotzky, Avinoam Mann, and Dan Segal,
Finitely generated groups of polynomial
subgroup growth
, Israel J. Math.
82
(1993), no. 1-3, 363–371.
\bibitem{MontgomeryVaughan} Hugh L. Montgomery and Robert C. Vaughan, Multiplicative Number Theory I. Classical Theory, Cambridge Studies in Advanced Mathematics 97.
\end{thebibliography}
\end{document}