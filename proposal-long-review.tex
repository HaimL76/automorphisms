\documentclass[12pt]{article}
\makeatletter
\newcommand*{\rom}[1]{\expandafter\@slowromancap\romannumeral #1@}
\makeatother
\usepackage{amsfonts, amssymb}
\usepackage{mathrsfs, mathdots} 
\usepackage{amsmath}
\usepackage{float}
\usepackage{amsthm}
\usepackage{tikz-cd}
\usepackage{xcolor}
\usepackage{xparse}
\usepackage{setspace}
\usepackage{xfrac}
\usepackage{yfonts}
\setcounter{MaxMatrixCols}{11}

\newtheorem{theorem}{Theorem}[subsection]
\newtheorem{proposition}[theorem]{Proposition}
\newtheorem{corollary}[theorem]{Corollary}
\newtheorem{lemma}[theorem]{Lemma}
\newtheorem{notations}[theorem]{Notations}
\newtheorem{definition}[theorem]{Definition}
\newtheorem{example}[theorem]{Example}
\newtheorem{remark}[theorem]{Remark}

\newtheorem{theorem2}{Theorem}[section]
\newtheorem{proposition2}[theorem2]{Proposition}
\newtheorem{corollary2}[theorem2]{Corollary}
\newtheorem{lemma2}[theorem2]{Lemma}
\newtheorem{notations2}[theorem2]{Notations}
\newtheorem{definition2}[theorem2]{Definition}
\newtheorem{example2}[theorem2]{Example}
\newtheorem{remark2}[theorem2]{Remark}

\ExplSyntaxOn
\NewDocumentCommand{\cycle}{ O{\;} m }
 {
  (
  \alec_cycle:nn { #1 } { #2 }
  )
 }

\seq_new:N \l_alec_cycle_seq
\cs_new_protected:Npn \alec_cycle:nn #1 #2
 {
  \seq_set_split:Nnn \l_alec_cycle_seq { , } { #2 }
  \seq_use:Nn \l_alec_cycle_seq { #1 }
 }
\ExplSyntaxOff
\usepackage{titling}
\newcommand{\subtitle}[1]{%
  \posttitle{%
    \par\end{center}
    \begin{center}\large#1\end{center}
    \vskip0.5em}%
}
\usepackage{arydshln}
\setcounter{tocdepth}{2}
\setlength{\dashlinedash}{1.2pt}
\setlength{\dashlinegap}{1.5pt}
\setlength{\arrayrulewidth}{0.2pt}
\title{The pro-isomorphic zeta-functions of some nilpotent Lie algebras over integer rings}
\subtitle{Research proposal for a Ph.D. Thesis\\
under the supervision of Prof. Michael M. Schein\\
Department of Mathematics, Bar-Ilan University}
\author{Haim Lavi}
\date{\today}
\begin{document}
\maketitle
\newpage
\tableofcontents
\newpage
\begin{abstract}
Let $G$ be any group. For any natural number $n\in\mathbb{N}$, let $a_{n}^{\leq}(G)$ be the number of subgroups $H\leq G$ such that $[G:H]=n$. If $G$ is finitely-generated, then $a_{n}^{\leq}(G)<\infty$, and we can define a Dirichlet series of the form $\zeta_G(s):=\sum_{n=1}^\infty a_{n}^{\leq}(G){n}^{-s}$, where $s\in\mathbb{C}$. Assume, in addition, that $G$ is also nilpotent and torsion-free, then this function has some properties of the Riemann zeta-function $\zeta$, such as the Euler decomposition of $\zeta$ into a product of local factors indexed by primes. A version of this zeta-function counts pro-isomorphic subgroups, and an analogous function may be defined for appropriate Lie rings. We study here the pro-isomorphic zeta-functions for a family of nilpotent Lie rings of unbounded nilpotency class. We shall compute the automorphism groups of these Lie rings explicitly and use them to prove uniformity of the local factors of the pro-isomorphic zeta-functions, and aim to determine them explicitly.
\end{abstract}
\section{Scientific Background}
\subsection{Introduction}
Although we will work with Lie algebras, for motivation we first present analogous and more natural questions in the context of groups.
\begin{proposition} \label{prop:finite.number.subgroups}
Let $G$ be any finitely generated group, and let $n\in\mathbb{N}$ be any natural number. Then there is a finite number of subgroups $H\leq G$ such that $[G:H]=n$.
\end{proposition}
For a proof, see, for instance \cite[Corollary 1.1.2]{LubotzkySegal}.
This proposition gives rise to an entire branch of group theory, called \textbf{subgroup growth}. We denote by \[a_{n}^{\leq}(G):=|\{H\leq{G} : [G:H]=n\}|\]
the number of subgroups of $G$ of index $n$ and consider the sequence $\{a_{n}^{\leq}(G)\}_{n=1}^{\infty}$. The subject of subgroup growth aims to relate the properties of this sequence to the algebraic structure of $G$. We denote by \[a_{n}^{\trianglelefteq}(G):=|\{H\trianglelefteq{G} : [G:H]=n\}|\]
the number of \textbf{normal} subgroups of $G$ of index $n$. We now define another type of subgroups of $G$.
\begin{definition}
\label{def:pro.isomorphic}
Let $G$ be a group. A subgroup $H\leq G$ is called \textbf{pro-isomorphic} if $\widehat{H}\cong\widehat{G}$, where $\widehat{H}$ and $\widehat{G}$ are the profinite closures of $H$ and $G$, respectively.
\end{definition}
We denote by \[a_{n}^{\wedge}(G):=|\{H\leq G : \widehat{H}\cong\widehat{G}, [G:H]=n\}|\]
the number of \textbf{pro-isomorphic} subgroups of $G$ of index $n$.

Now we define generating functions determined by these sequences. 
\begin{definition}
\label{def:zeta.function}
Let $G$ be a finitely-generated group, and let\\ $\ast\in\{\leq,\trianglelefteq,\wedge\}$, then we define the zeta-function $\zeta_{G}^{\ast}(s)$ to be the Dirichlet series $\sum_{n=1}^{\infty}a_{n}^{\ast}(G){n}^{-s}$. 
\end{definition}
\begin{proposition}
Let $G$ be a $\mathcal{T}$-group, i.e. finitely-generated, nilpotent and torsion-free group, and let $\ast\in\{\leq,\trianglelefteq,\wedge\}$. Then the zeta-functions for $G$ have the following properties:\par
\textbf{Polynomial growth and convergence}. There exist constants $b$ and $C$ such that $a_{n}^{\leq}(G)\leq{C}{n}^{b}$ for all $n\in\mathbb{N}$. Hence $\zeta_{G}^{\ast}(s)$ converges on some right half plane $\mathrm{Re}\,s>\alpha$. The abscissae of convergence \[\alpha^{\ast}:=\inf\{\alpha : \zeta_{G}^{\ast}(s)\,\mathrm{converges}\,\mathrm{on}\,\mathrm{Re}\,s>\alpha\}\] are known to be rational numbers. See \cite[Theorem 1.1]{DuSautoyGrunewald} for $\{\leq,\trianglelefteq\}$ and see \cite[Theorem 7.2]{HrushovskiMartinRideau} for all three types.

\textbf{Euler decomposition}.
We have that $\zeta_{G}^{\ast}(s)=\prod_{p}\zeta_{G,p}^{\ast}(s)$, where the product runs over all primes $p$ and where $\zeta_{G,p}^{\ast}(s)=\sum_{k=0}^{\infty}a_{p^k}^{\ast}(G)p^{-ks}$\\ \cite[Proposition 4]{GrunewaldSegalSmith}.

\textbf{Rationality}. For all primes $p$, there is a rational function $W_{p}^{\ast}\in\mathbb{Q}(X)$ such that $\zeta_{G,p}^{\ast}(s)=W_{p}^{\ast}(p^{-s})$ \cite[Theorem 1]{GrunewaldSegalSmith}.
\end{proposition}
\begin{definition}
Let $\zeta_{L,p}^{\ast}$ be a \textbf{finitely uniform} zeta-function, i.e. there are rational functions $W_{1}^{\ast}(X,Y),\dots,W_{r}^{\ast}(X,Y)\in\mathbb{Q}(X,Y)$, such that for all $p$, there is some $1\leq{i}\leq{r}$ such that $\zeta_{G,p}^{\ast}(s)=W_{i}^{\ast}(p,p^{-s})$. We say $W_{i}^{\ast}$ satisfies a \textbf{functional equation} if $W_{i}^{\ast}(X^{-1},Y^{-1})=\pm{X}^{a}{Y}^{b}{W_{i}^{\ast}(X,Y)}$, where $a,b\in\mathbb{N}\cup\{0\}$. The factor $X^{a}Y^{b}$ is called the \textbf{symmetry factor}. See a comprehensive discussion on functional equations for zeta-functions in \cite{Voll}.
\end{definition}
Let $G$ be a $\mathcal{T}$-group, then $\zeta_{G,p}^{\leq}(s)$ satisfies a functional equation, for all but finitely many $p$, with the same symmetry factor. If $G$ is a $\mathcal{T}$-group of nilpotency class $2$, the same is true for $\zeta_{G,p}^{\trianglelefteq}(s)$ \cite[Theorem C]{Voll}.

A more general result for zeta-functions counting normal subgroups of groups of arbitrary class can be found in \cite[Theorem 1.2]{Voll2} where, under a specific condition, it is shown that the zeta-functions satisfy a functional equation.

There are specific examples of $\mathcal{T}$-groups of nilpotency class $3$, such that $\zeta_{G,p}^{\trianglelefteq}(s)$ is \textbf{uniform}, i.e. $\zeta_{G,p}^{\trianglelefteq}(s)=W(p,p^{-s})$, but $W(p,p^{-s})$ does not satisfy a functional equation, see \cite[Theorem 2.40]{DuSautoyWoodward} and \cite[Theorem 2.44]{DuSautoyWoodward}.

In the case of pro-isomorphic zeta-functions, some condition is necessary, since there are results showing that under restrictive conditions, $\zeta_{G,p}^{\wedge}(s)$ is uniform and satisfies a functional equation, see, for instance, \cite[Theorem 1.1]{Berman} and \cite[Theorem B]{DuSautoyLubotzky}. However, there are examples of $G$ of nilpotency class $4$ such that $\zeta_{G,p}^{\wedge}(s)$ is uniform but does not have a functional equation, see \cite[Theorem 1.1]{BermanKlopsch} for the establishment of the uniformity of such an example, and further on in that paper for the absence of a functional equation. 

If any Dirichlet series, in particular $\zeta_{G}^{\ast}(s)$, convergences on some subset of $\mathbb{C}$, one may reconstruct its coefficients $a_{n}^{\ast}(G)$, the sequence of our interest, using Perron's formula, which is an implementation of a reverse Mellin transform; see, for instance, \cite[Theorem 5.1]{MontgomeryVaughan}.\par
As an example, consider the group $G=\mathbb{Z}$, for which every finite-index subgroup $H\leq{\mathbb{Z}}$ is of the form $H=n\mathbb{Z}=\langle n\rangle$ for some $n\in\mathbb{N}$. Thus $H\cong \mathbb{Z}$, as both are infinite cyclic groups, and so $\widehat{H}\cong\widehat{\mathbb{Z}}$. Since we have only one subgroup of index $n$ for every $n\in\mathbb{N}$, then $a_{n}^{\ast}(\mathbb{Z})=1$. Thus, its pro-isomorphic zeta-function is $\zeta_{\mathbb{Z}}^{\wedge}(s)=\sum_{i=1}^{\infty}n^{-s}=\zeta(s)$, the Riemann zeta-function, which is known to converge for $\mathrm{Re}\,s>1$. It is known that the Riemann zeta-function decomposes into an infinite product of local zeta-functions, that is, $\zeta(s)=\prod_p\zeta_p(s)=\prod_p\sum_{k=0}^\infty p^{-ks}=\prod_p\frac{1}{1-p^{-s}}$, where the product runs over all the prime numbers.\par
This research concentrates on the growth of \textbf{pro-isomorphic} subgroups defined above, hence we shall restrict our further discussion to the pro-isomorphic case only.
\subsection{Linearization}
We may transfer the ideas from the above discussion about groups to a linear context, where we can use tools from linear algebra.
If $L$ is a $\mathbb{Z}$-Lie algebra, namely a free $\mathbb{Z}$-module of finite rank with a Lie bracket operation, then consider the number $a_{n}^{\wedge}(L)$ of subalgebras $M\leq L$ of index $n$ such that $M\otimes_{\mathbb{Z}}\mathbb{Z}_p\cong L\otimes_{\mathbb{Z}}\mathbb{Z}_p$ for all primes $p$. It is known that $a_{n}^{\wedge}(L)<\infty$ for all $n\in\mathbb{N}$. The Dirichlet series $\zeta_{L}^{\wedge}(s):=\sum_{n=1}^{\infty}{a_{n}^{\wedge}}(L)n^{-s}$ is called the \textbf{pro-isomorphic zeta-function} of $L$. By the Mal'cev correspondence, to every $\mathcal{T}$-group $G$ one may associate a Lie algebra $L=L(G)$ such that $\zeta_{G,p}^{\wedge}(s)=\zeta_{L,p}^{\wedge}(s)$ for all but finitely many primes $p$. If $G$ has nilpotency class $2$, one may obtain the equality for all primes \cite[Theorem 4.1]{GrunewaldSegalSmith}.\par
Let $L$ be a $\mathbb{Z}$-Lie algebra. Choose a basis $(b_1,\dots,b_r)$, where $r=\mathrm{rank}\,L$. Let $\mathcal{L}_{p}=L\otimes_{\mathbb{Z}}\mathbb{Q}_p$, for any prime $p$. This is a $\mathbb{Q}_p$-Lie algebra, and our choice of basis allows us to identify the automorphism group $G(\mathbb{Q}_p)=Aut_{\mathbb{Q}_p}(\mathcal{L}_{p})$ with a subgroup of $GL_r(\mathbb{Q}_p)$. Note that $\mathcal{L}_{p}$ contains a $\mathbb{Z}_p$-lattice $L_{p}=L\otimes_{\mathbb{Z}}\mathbb{Z}_p$. If $\varphi\in G(\mathbb{Q}_p)$, then $\varphi(L_{p})=L_{p}$ if and only if $\varphi\in G(\mathbb{Z}_p)=G(\mathbb{Q}_p)\cap GL_r(\mathbb{Z}_p)$. Similarly, $\varphi(L_{p})\subseteq L_{p}$ if and only if $\varphi\in G^{+}(\mathbb{Q}_p):=G(\mathbb{Q}_p)\cap {M}_r(\mathbb{Z}_p)$, where ${M}_r(\mathbb{Z}_p)$ is the collection of $r\times r$ matrices with entries in $\mathbb{Z}_p$. Note that $G^{+}(\mathbb{Q}_p)$ is a monoid, not a group.\par
We consider automorphisms as acting from the right, hence, we observe that there is a bijection between the set $G(\mathbb{Z}_p)\backslash G^{+}(\mathbb{Q}_p)$ of right cosets of $G(\mathbb{Z}_p)$ and the set $\{M\leq L_{p} : M\cong L_{p}\}$ of $\mathbb{Z}_{p}$-subalgebras of $L_{p}$ which are isomorphic to $L_{p}$ itself. This bijection takes $G(\mathbb{Z}_p)g$ to $M=\varphi(L_{p})$, for $\varphi\in G(\mathbb{Z}_p)g$.
Observe that $[L_{p}:M]=|\det\varphi|_p^{-1}$, where $|\cdot|_p$ is the $p$-adic norm on $\mathbb{Q}_{p}$ (recall that $|p|_{p}=p^{-1}$), and therefore,
\begin{equation}
\label{eq:proisomorphic.zeta}
\zeta_{L,p}^{\wedge}(s)=\underset{\overset{\scriptscriptstyle M\leq L_{p}}{\scriptscriptstyle M\cong L_{p}}}{\sum}[L_{p}:M]^{-s}=\underset{\scriptscriptstyle G(\mathbb{Z}_p)\varphi\in G(\mathbb{Z}_p)\backslash G^{+}(\mathbb{Q}_p)}{\sum}|\det\varphi|_p^s
\end{equation}.
\subsection{$p$-adic Integration}
We fix a specific right Haar measure $\mu$ on the locally compact group $G(\mathbb{Q}_{p})$ by normalizing $\mu(G(\mathbb{Z}_p))=1$. Now we are able to express $\zeta_{L,p}^{\wedge}(s)$ as a $p$-adic integral, as first done by Grunewald, Segal and Smith \cite[Proposition 4.2]{GrunewaldSegalSmith}. Since $|\det{g}|_{p}^{s}$ is constant on right cosets of $G(\mathbb{Z}_{p})$, we have \[|\det\varphi|_{p}^{s}=\displaystyle\int_{G(\mathbb{Z}_p)\varphi}{|\det{g}|_{p}^{s}}d\mu(g).\]
By \eqref{eq:proisomorphic.zeta}, we have \[\zeta_{L,p}^{\wedge}(s)=\underset{\scriptscriptstyle G(\mathbb{Z}_p)\varphi\in G(\mathbb{Z}_p)\backslash{G^{+}(\mathbb{Q}_p)}}{\sum}|\det\varphi|_{p}^{s}=\underset{\scriptscriptstyle{G(\mathbb{Z}_p)}\varphi\in G(\mathbb{Z}_p)\backslash{G^{+}(\mathbb{Q}_p)}}{\sum}\displaystyle{\int_{G(\mathbb{Z}_p)\varphi}}{|\det{g}|_{p}^{s}}d\mu(g)\]
which results in \begin{equation}
\label{eq:p-adic.integral}
\zeta_{L,p}^{\wedge}(s)=\displaystyle\int_{G^{+}(\mathbb{Q}_p)}{|\det{g}|_{p}^{s}}d\mu(g)
\end{equation}
\section{Research Goals and Methodology}
\subsection{The Lie algebras $L_{n}$}
Let $R$ be a commutative ring, and let $L_{n}(R)$ be the set of strictly upper triangular $n\times{n}$ matrices with elements in $R$. The commutator Lie bracket $[A,B]=AB-BA$ gives $L_{n}(R)$ the structure of a nilpotent $R$-Lie algebra. Set $L_{n}=L_{n}(\mathbb{Z})$ and $\mathcal{L}_{n}=L_{n}(\mathbb{Q})$, while $L_{n,p}=L_{n}(\mathbb{Z}_{p})$ and $\mathcal{L}_{n,p}=L_{n}(\mathbb{Q}_{p})$.

Let $e_{ij}\in{L_{n}}$ be the matrix with $1$ in the $(i,j)$ entry and $0$ elsewhere. Then $B:=\{e_{ij} : 1\leq{i}<{j}\leq{n}\}$ is a basis of $L_{n}$, which we order as follows: $(e_{12},\dots,e_{n-1,n},e_{13},\dots,e_{n-2,n},e_{14},\dots,e_{n-3,n},\dots,e_{1n})$. In other words we order the elements of $B$ so that $e_{ij}<e_{kl}$ if either $j-i<l-k$ or if $j-i=l-k$ and $i<k$. Note that $\gamma_{1}L_{n},\dots,\gamma_{n-1}L_{n}$ are spanned by final segments of $B$. The aim of our research is to study the pro-isomorphic zeta function of $L_{n}$. Berman has computed $\zeta_{L_{4,p}}^{\wedge}(s)$ explicitly for all $p$ \cite[\S{3.3.7}]{BermanThesis} and has shown that $\zeta_{L_{5,p}}^{\wedge}(s)$ are uniform \cite[Proposition 3.12]{BermanThesis}.
\begin{remark}
Let $U_n(R)$ be the group of $n\times{n}$ upper unitriangular matrices over $R$. By the Mal'cev correspondence, ${\zeta^\wedge_{U_{n}(\mathbb{Z}_{p})}}(s)={\zeta^\wedge_{L_{n,p}}}(s)$, for all but finitely many primes $p$. This relates the subject of our research to the group-theoretic motivation presented at the beginning of this paper.
\end{remark}
\subsection{Research goals}
The project consists of three major steps:
\begin{enumerate}
\item
Computing the automorphism group $G_{n}(\mathbb{Q}_{p})$ of the $\mathbb{Q}_p$-Lie algebras $\mathcal{L}_{n,p}$, for all $n\in\mathbb{N}$ and all primes $p$, and understanding the submonoid $G_{n}^{+}(\mathbb{Q}_{p})$.
\item
Showing that the pro-isomorphic zeta-functions $\zeta_{L_{n,p}}^{\wedge}(s)$ are uniform for all $n\in\mathbb{N}$. We expect the method used by Berman in the case $n=5$ to generalize.
\item
Giving an explicit uniform formula for the zeta-functions $\zeta_{L_{n,p}}^{\wedge}(s)$.
We aim to compute an explicit formula for $\zeta_{L_{n,p}}^{\wedge}(s)$ for all $n \geq 4$ and all primes $p$ in terms of a suitable combinatorial framework that is compatible with the complicated structure of $G_{n}^{+}(\mathbb{Q}_p)$. We will study whether $\zeta_{L_{n,p}}^{\wedge}(s)$ have functional equations and in particular whether they satisfy the conjecture of \cite[Conjecture 1.8]{BermanKlopschOnn}, which says that if $\zeta_{L,p}^{\wedge}(s)$ has functional equations and $L$ is graded, then the exponent $b$ in the symmetry factor must be the weight of a minimal grading of $L$. Note that $L_{n}$ is naturally graded by its lower central series, with minimal grading of weight $\sum_{i=1}^{n-1}{i(n-i)}$. In the event that we do not succeed in this task in general, we still expect to complete Berman's unfinished calculation in the case $n=5$. In this case, the combinatorics is just simple enough to handle directly, possibly with some assistance from a computer.
\end{enumerate}
We make several remarks about these research goals.
\begin{enumerate}
    \item 
Automorphism groups of Lie algebras, and in particular $G_{n}(\mathbb{Q}_{p})$, have been studied before from a different point of view, and there are classical results showing that any automorphism may be expressed as a product of automorphisms of a specific type; see, for instance, the main result of Gibbs \cite{Gibbs}.  These statements are not explicit enough for our purposes; indeed, the submonoid $G_{n}^{+}(\mathbb{Q}_p)$ arises for us as the domain of integration of a $p$-adic integral. In order to calculate this integral, it will be useful to decompose the automorphism group $G_{n}(\mathbb{Q}_p)$ into a repeated semi-direct product of groups with a simple structure.
\item 
We will need to analyze the monoid $G_{n}^{+}(\mathbb{Q}_p)$ and its $G_{n}(\mathbb{Z}_p)$ right-cosets, as we have seen above. This will give us both the integrand, which is $|\det\varphi|_{p}^{s}$, and the domain of integration, which is the monoid $G_{n}^{+}(\mathbb{Q}_p)$. It is expected that $G_{n}^{+}(\mathbb{Q}_{p})$ will have a complicated structure, and expressing $\zeta_{L_{n,p}}^{\wedge}(s)$ explicitly in a way that allows us to verify functional equations will depend on finding a suitable class of combinatorially defined functions in terms of which $\zeta_{L_{n,p}}$ may be expressed;\\ cf. \cite[Theorem 4.21]{CarnevaleScheinVoll}. However, we expect that the method of Berman's proof of uniformity in the case $n=5$, which relies on generalizations of results by Stanley on generating functions of polyhedral cell complexes, will generalize to all $n\geq{4}$.
\item 
Let $K$ be a number field of degree $d=[K:\mathbb{Q}]$, and let $\mathcal{O}_K$ be its ring of integers. Let $L$ be a $\mathbb{Z}$-Lie algebra of rank $r$. We can naturally consider the base extension $L_{K}=L\otimes_{\mathbb{Z}}\mathcal{O}_{K}$ as a $\mathbb{Z}$-Lie algebra of rank $rd$, and by extension of scalars we can consider also $\mathcal{L}_{K,p}=(L\otimes_{\mathbb{Z}}\mathcal{O}_K)\otimes_{\mathbb{Z}}\mathbb{Q}_p$ as a $\mathbb{Q}_p$-Lie algebra of the same rank. If $\mathcal{L}_{p}$ has a certain property originally introduced by Segal \cite{Segal}, which we call rigidity, then Berman, Glazer and Schein have shown that a simple modification of the calculation of $\zeta_{L,p}^{\wedge}(s)$ gives $\zeta_{L_{K},p}^{\wedge}(s)$ for all $K$ without any substantial extra effort \cite[Proposition 3.14]{BermanGlazerSchein}. They give a criterion for rigidity in terms of the centralizers of elements of $L_{p}$ \cite[Theorem 3.9]{BermanGlazerSchein}. Note that the criterion does not necessarily apply for all $p$. We shall study whether the criterion applies to the $\mathbb{Q}_p$-Lie algebras $\mathcal{L}_{n,p}$ of our work. If so, then uniformity of $\zeta_{L,p}^{\wedge}(s)$ will imply finite uniformity of $\zeta_{L_{K},p}^{\wedge}(s)$. Indeed, for each of the finitely many decomposition types of a prime in $\mathcal{O}_K$, there would be a rational function in two variables $W\in\mathbb{Q}(X,Y)$ such that $\zeta_{\mathcal{L}\otimes\mathcal{O}_K,p}^{\wedge}(s)=W(p,p^{-s})$ for all $p$ of that decomposition type.
\end{enumerate}
\subsection{Representing matrices of automorphisms of $\mathcal{L}_{n,p}$}
To sum up what is already known about automorphisms of $\mathcal{L}_{n,p}$,\\ let $\varphi\in{G_{n}}(\mathbb{Q}_p)$ and let $M$ be the matrix of $\varphi$ with respect to the ordered basis $B$, where $M$ acts from the right on row vectors.  Thus:
\[
M=\begin{pmatrix}
\varphi(e_{12})\\
\varphi(e_{23})\\
\varphi(e_{n-1,n})\\
\hdashline
\varphi(e_{13})\\
\vdots\\
\varphi(e_{n-2,n})\\
\hdashline
\vdots\\
\hdashline
\varphi(e_{1n})\\
\end{pmatrix}\\
\]
Since the elements $\{\gamma_{k}\mathcal{L}_{n,p}\}_{k=1}^{n-1}$ of the lower central series are characteristic subalgebras, $M$ has the block-diagonal form
\[M=\begin{pmatrix}
M_{11} & \vline & M_{12} & \vline & M_{13} & \vline & \dots & \vline & M_{1,n-2} & \vline & M_{1,n-1}\\
\hline
0 & \vline & M_{22} & \vline & M_{23} & \vline & \dots & \vline & M_{2,n-2} & \vline & M_{2,n-1}\\
\hline
\vdots & \vline & \vdots & \vline & \vdots & \vline & \ddots & \vline & \vdots & \vline & \vdots\\
\hline
0 & \vline & 0 & \vline & 0 & \vline & \dots & \vline & M_{2,n-2} & \vline & M_{2,n-1}\\
\hline
0 & \vline & 0 &\vline & 0 & \vline & \dots & \vline & 0 & \vline & M_{n-1,n-1}\\
\end{pmatrix}
\]
where each block $M_{ij}$ is an $(n-i)\times{(n-j)}$ matrix.\par
Note that $\begin{pmatrix}
M_{11} & \vline & M_{12}\\
\hline
0 & \vline & M_{22}\\
\end{pmatrix}$ gives an element of $Aut_{\mathbb{Q}_{p}}(\sfrac{\mathcal{L}_{n,p}}{\gamma_{3}\mathcal{L}_{n,p}})$. Berman determined this automorphism group for all $n$ and $p$ \cite[Proposition 3.6]{BermanThesis}. It follows that the block $M_{11}$ must be either diagonal or anti-diagonal.
\section{Preliminary results}
\label{preliminary.results}
We have made progress towards the first step of our research program, namely determining the automorphism groups $G_{n}(\mathbb{Q}_p)$, based on Berman's results cited above.
Define $\eta\in{G_{n}(\mathbb{Q}_p)}$ by $\eta(e_{ij}):=(-1)^{j-i-1}e_{n+1-j,n+1-i}$,\\ for $1\leq{i}<{j}\leq{n}$. Replacing $\varphi$ by $\varphi\circ\eta$ if necessary, we may assume without loss of generality that $M_{11}$ is diagonal. Indeed, let $G_{n}^{0}(\mathbb{Q}_{p})\leq{G_{n}(\mathbb{Q}_{p})}$ be the subgroup of automorphisms with diagonal block $M_{11}$,\\ then $G_{n}(\mathbb{Q}_{p})=G_{n}^{0}(\mathbb{Q}_{p})\coprod{G_{n}(\mathbb{Q}_{p})}\eta$, and as in \cite[Proposition 2.1]{DuSautoyLubotzky} we may replace the domain of integration in \eqref{eq:p-adic.integral} by $G_{n}^{0}(\mathbb{Q}_{p})\cap{G_{n}^{+}(\mathbb{Q}_{p})}$ after a suitable renormalization of the Haar measure. We proceed to determine $G_{n}^{0}(\mathbb{Q}_{p})$.
We check, for any $\lambda_{1},\lambda_{2},\dots,\lambda_{n-1}\in\mathbb{Q}_{p}^{\ast}$, that the diagonal matrix \[h=\mathrm{diag}(\lambda_{1},\lambda_{2},\dots,\lambda_{n-1},\lambda_{1}\lambda_{2},\lambda_{2}\lambda_{3}\dots,\lambda_{n-2}\lambda_{n-1},\dots,\lambda_{1}\lambda_{2}\cdots\lambda_{n-2}\lambda_{n-1})\]
represents an automorphism of $\mathcal{L}_{n,p}$. Thus every $g\in{G_{n}^{0}(\mathbb{Q}_{p})}$ has a unique decomposition $g=uh$, where $h$ is of the above form and $u$ has $1$'s on the diagonal. It is easy to see that $\det{h}=\prod_{i=1}^{n-1}\lambda_i^{i(n-i)}$ by induction on $n$. Since $\det{u}=1$, it follows that $|\det{g}|_{p}^{s}=|\det{h}|_{p}^{s}$.

The collection $H(\mathbb{Q}_{p})$ of diagonal matrices $h$ as above is the reductive part of $G_{n}^{0}(\mathbb{Q}_{p})$. The collection $N(\mathbb{Q}_{p})$ of matrices $u$ as above is the unipotent radical of $G_{n}^{0}(\mathbb{Q}_{p})$. We aim to determine the structure of the unipotent radical $N(\mathbb{Q}_p)$ by decomposing it into an iterative semidirect product of abelian subgroups.
For all $2\leq{r}\leq{n-1}$ we denote by $N_{r}\leq{N(\mathbb{Q}_{p})}$ the subgroup of all automorphisms of $\mathcal{L}_{n,p}$, such that\\ $M_{11}=I_{n}$ and $M_{12}=M_{13}=\cdots=M_{1,r-1}=0$. In other words, $N_{r}$ is the kernel of the natural map $G_{n}^{0}(\mathbb{Q}_{p})\rightarrow{Aut_{\mathbb{Q}_{p}}(\sfrac{\mathcal{L}_{n,p}}{\gamma_{r}\mathcal{L}_{n,p}})}$. We can describe $N_{n-1}$ explicitly as the following set of block matrices \[
N_{n-1}=\left\{\begin{pmatrix}
I_{n-1} & 0 & \cdots & 0 & M_{1,n-1}\\
0 & I_{n-2} & \cdots & 0 & 0\\
\vdots & \vdots & \ddots & \vdots & \vdots\\
0 & 0 & \cdots & I_{2} & 0\\
0 & 0 & \cdots & 0 & 1\\
\end{pmatrix}\right\}
\]
where $M_{1,n-1}$ is an arbitrary $(n-1)\times{1}$ matrix with entries in $\mathbb{Q}_{p}$ and the $0$ blocks are zero matrices of suitable size. The map $\varphi\mapsto{M_{1,n-1}}$ for all $\varphi\in{N_{n-1}}$ gives an isomorphism $N_{n-1}\cong\mathbb{Q}_{p}^{n-1}$.
\begin{proposition2}
\label{block.1r.structure}
Consider the $(n-1)\times{(n-r)}$ matrix $M_{1r}=(a_{ij})$. Then,
\begin{enumerate}
    \item Let $2\leq{r}<{n-2}$. If $a_{ij}\neq{0}$, then either $i=j$ or $i=j+r-1$, and we have the relation $a_{i+r,i+1}=-a_{ii}$.
    \item Let $r=n-2$. If $a_{ij}\neq{0}$, then either $i=j$ or $i=j+r-1$ or $(i,j)\in\{(1,2),(n-1,1)\}$, with the same relation as above.
\end{enumerate}
\end{proposition2}
\begin{proof}
From the relation $[\varphi(e_{k,k+1}),\varphi(e_{l,l+1})]=0$ where $l>k+1$, we deduce that $a_{ij}\neq{0}$ only if either $i=j$ or $i=j+r-1$ or\\ $(i,j)\in\{(r+1,1),(r+2,1),(n-r-2,n-r),(n-r-1,n-r)\}$. If $r<n-2$ then it follows from the conditions
\begin{align*}
[\varphi(e_{n-r-2,n-r-1}),\varphi(e_{n-r-2,n-r})]=0\\
[\varphi(e_{n-r-1,n-r}),\varphi(e_{n-r-2,n-r})]=0\\
[\varphi(e_{r+1,r+2}),\varphi(e_{r+1,r+3})]=0\\
[\varphi(e_{r+2,r+3}),\varphi(e_{r+1,r+3})]=0\\
\end{align*}
that the four exceptional cases cannot occur. Similar considerations eliminate two of the exceptional cases when $r=n-2$.
\end{proof}
\begin{proposition2}
\label{phi.r.k}
Let $2\leq{r}\leq{n-2}$, and let $0\leq{k}\leq{n-r}$, and let $a\in\mathbb{Q}_{p}$. We extend our notation of basis elements to include $e_{01}=e_{n,n+1}=0$.
\begin{enumerate}
\item There is an automorphism $\varphi_{r,k}(a)\in{G_{n}^{0}(\mathbb{Q}_{p})}$ determined by \[
    \varphi_{r,k}(a)(e_{i,i+1})=\begin{cases}
        e_{k,k+1}+a{e_{k,k+r}} & : i=k\\
        e_{k+r,k+r+1}-a{e_{k+1,k+r+1}} & : i=k+r\\
        e_{i,i+1} & : i\notin\{k,k+r\}
    \end{cases}
\]
\item Suppose that $r=n-2$, let $(k,l)\in\{(1,2),(n-1,1)\}$, and let $a\in\mathbb{Q}_{p}$. There is an automorphism $\varphi_{n-2,k,l}(a)\in{G_{n}^{0}(\mathbb{Q}_{p})}$ determined by \[
    \varphi_{n-2,k,l}(a)(e_{i,i+1})=\begin{cases}
        e_{k,k+1}+a{e_{l,l+r}} & : i=k\\
        e_{i,i+1} & : i\neq{k}
    \end{cases}
\]
    We denote $\varphi'_{n-2}(a):=\varphi_{n-2,1,2}(a)$ and $\varphi''_{n-2}(a):=\varphi_{n-2,n-1,1}(a)$.
\end{enumerate}
\end{proposition2}
\begin{proof}
We need to verify that for all $1\leq{i<j}\leq{n}$ and $1\leq{l<m}\leq{n}$ we have the following relations \[
    [\varphi_{r,k}(a)(e_{ij}),\varphi_{r,k}(a)(e_{lm})]=\begin{cases}
    \varphi_{r,k}(a)(e_{im}) & : j=l\\
        -\varphi_{r,k}(a)(e_{lj}) & : i=m\\
        0 & : \mathrm{otherwise}
    \end{cases}
\]
We can verify explicitly for $n=4$ that these relations are true. Alternatively, Berman did this in \cite{BermanThesis}. For $n>4$, one considers the injection $\iota:\mathcal{L}_{n-1,p}\hookrightarrow\mathcal{L}_{n,p}$, mapping each $e_{i,i+1}\in\mathcal{L}_{n-1,p}$ to $e_{i,i+1}\in\mathcal{L}_{n,p}$ for all $1\leq{i}\leq{n-2}$. By induction it suffices to consider the cases $j=n$ and $m=n$. The exceptional cases $\varphi'_{n-2}(a)$ and $\varphi''_{n-2}(a)$ can easily be verified directly to be automorphisms.
\end{proof}
Fix some $r,k$ and denote by $N_{r,k}:=\{\varphi_{r,k}(a) : a\in\mathbb{Q}_{p}\}\subset{G_{n}(\mathbb{Q}_{p})}$ the set of all automorphisms of this form. It is easy to see that $N_{r,k}\leq{G_{n}^{0}(\mathbb{Q}_{p})}$ and $N_{r,k}\cong\mathbb{Q}_{p}$.
Also denote by $N'_{n-2}:=\{\varphi'_{n-2}(a) : a\in\mathbb{Q}_{p}\}$ and by\\ $N''_{n-2}:=\{\varphi''_{n-2}(a) : a\in\mathbb{Q}_{p}\}$, and it is easy to see that the same applies for $N'_{n-2}$ and $N''_{n-2}$.

The following proposition follows from a simple computation.
\begin{proposition2}
\label{prop:psi.automorphism}
Consider $\varphi\in{N_{r}}$.
\begin{enumerate}
    \item 
If $r<n-2$, denote by $\psi$ the automorphism \[\psi:=\varphi\circ\varphi_{r,n-r}(-a_{n-r,n-r})\circ\cdots\circ\varphi_{r,1}(-a_{11})\circ\varphi_{r,0}(-a_{r+1,1})\]
Then $\psi\in{N_{r+1}}$.
    \item 
If $r=n-2$, denote by $\psi_{n-2}$ the automorphism \[\psi_{n-2}:=\varphi\circ\varphi'_{n-2}(-a_{12})\circ\varphi''_{n-2}(-a_{n-1,1})\circ\]\[\circ\varphi_{n-2,2}(-a_{n-2,2})\circ\varphi_{n-2,1}(-a_{n-2,1})\circ\varphi_{n-2,0}(-a_{n-2,0})\]
Then $\psi_{n-2}\in{N_{n-1}}$.
\end{enumerate}
\end{proposition2}
\begin{corollary2}
\label{cor:Nr.decomposition}
We have the following decompositions:
\begin{enumerate}
    \item 
For all $2\leq{r}<{n-2}$, we have \[N_{r}=N_{r+1}\rtimes(N_{r,0}\rtimes(\cdots(N_{r,n-r-1}\rtimes{N_{r,n-r}})\cdots))\]
\item For $r=n-2$, we have 
\[N_{n-2}=N_{n-1}\rtimes({N_{n-2,0}}\rtimes(N_{n-2,1}\rtimes(N_{n-2,2}\rtimes(N''_{n-2}\rtimes{N'_{n-2}}))))\]
\end{enumerate}
\end{corollary2}
This is immediate from Proposition \ref{prop:psi.automorphism}.

Corollary \ref{cor:Nr.decomposition} provides a recursive decomposition of the unipotent radical $N(\mathbb{Q}_{p})$ as an iterated semidirect product of $N_{n-1}$ and subgroups isomorphic to $\mathbb{Q}_{p}$.

As we saw earlier, the calculation of $\zeta_{L_{n,p}}^{\wedge}(s)$ requires understanding $G_{n}(\mathbb{Z}_p)$ and $G_{n}^{+}(\mathbb{Q}_p)$ first. As $G_{n}(\mathbb{Z}_{p})$ is a group, its structure is easily deduced from the above.
\begin{proposition2}
\label{prop:G.n.Zp.decomposition}
For all $n\geq{4}$ the group $G_{n}^{0}(\mathbb{Z}_{p})$ has the decomposition $G_{n}^{0}(\mathbb{Z}_{p})=N(\mathbb{Z}_{p})\rtimes{H(\mathbb{Z}_{p})}$, where \[N(\mathbb{Z}_{p}):=M_{\binom{n}{2}}(\mathbb{Z}_{p})\cap{N(\mathbb{Q}_{p})}\]
\[H(\mathbb{Z}_{p}):=\{\mathrm{diag}(\lambda_{1},\lambda_{2},\dots,) : \lambda_1,\dots,\lambda_{n-1}\in\mathbb{Z}_{p}^{\ast}\}\]
Moreover, $N(\mathbb{Z}_{p})$ itself has the decomposition: \[N(\mathbb{Z}_{p})=\Tilde{N}_{2}(\mathbb{Z}_{p})=
\Tilde{N}_{n-1}\rtimes({\Tilde{N}_{n-2,0}}\rtimes(\Tilde{N}_{n-2,1}\rtimes(\Tilde{N}_{n-2,2}\rtimes(\Tilde{N}''_{n-2}\rtimes{\Tilde{N}'_{n-2}}))))\rtimes\cdots\]\[\cdots\rtimes{(\Tilde{N}_{2,0}\rtimes(\cdots(\Tilde{N}_{2,n-3}\rtimes{\Tilde{N}_{2,n-2}})\cdots))}
\]
where $\Tilde{N}_{r}=N_{r}\cap{N(\mathbb{Z}_{p})}$ and $\Tilde{N}_{r,k}=\{\varphi_{r,k}(a) : a\in\mathbb{Z}_{p}\}$
\end{proposition2}
By contrast, describing the structure of the monoid $G_{n}^{+}(\mathbb{Q}_p)$ is expected to be a substantial challenge.

By applying Fubini's theorem for semidirect products of topological groups \cite[Proposition 28]{Nachbin}, we have that \[\zeta_{L_{n,p}}^{\wedge}(s)=\displaystyle\int_{G_{n}^{+}(\mathbb{Q}_p)}|\det{\varphi}|_p^sd\mu_{G_{n}(\mathbb{Z}_p)\varphi}=\displaystyle\int_{H^+(\mathbb{Q}_p)}\left(\displaystyle\int_{N_{h}^+}|\det{uh}|_p^sd\mu_{N(\mathbb{Q}_p)}\right)d\mu_{H(\mathbb{Q}_p)}\]
where \[H^+(\mathbb{Q}_p):=\{\mathrm{diag}(\lambda_{1},\dots,\lambda_{n-1},\lambda_{1}\lambda_{2},\dots,\lambda_{1}\lambda_{2}\cdots\lambda_{n-2}\lambda_{n-1}) : \lambda_{i}\in\mathbb{Z}_{p}\setminus\{0\}\},\]
that is, $H^+(\mathbb{Q}_p)$ consists of all $h\in{H(\mathbb{Q}_{p})}$ that appear in the decomposition $\varphi=uh$ for some $\varphi\in{G_{n}^{+}(\mathbb{Q}_{p})}$, and, for a given $h\in{H^{+}(\mathbb{Q}_{p})}$, we set $N_{h}^{+}:=\{u\in{N(\mathbb{Q}_{p}) : uh\in{G_{n}^{+}(\mathbb{Q}_{p})}}\}$. The integrand is constant on $N_{h}^{+}$, so computing the inner integral amounts to finding the measure of $N_{h}^+$, which is complicated, but using the decomposition from Corollary \ref{cor:Nr.decomposition}, we can simplify $N_{h}^{+}$ at the price of replacing a single integral by multiple integrals.

Let $\mathcal{N}_{1},\mathcal{N}_{2},\dots,\mathcal{N}_{m}$, where $m=\binom{n}{2}$, be an enumeration of the subgroups
\[N_{2,n-2},N_{2,n-3},\dots,N_{2,0},N_{3,n-3},\dots,N_{3,0},\dots\]\[\dots,N_{n-2,2},N_{n-2,1},N_{n-2,0},N'_{n-2},N''_{n-2},N_{n-1}\]
Every $\varphi\in{G_{n}^{0}(\mathbb{Q}_{p})}$ can be written uniquely as $\varphi=u_{m}\cdots{u_{1}h}$, where $u_{i}\in{\mathcal{N}_{i}}$. Thus, by Fubini \[
\zeta_{L_{n,p}}^{\wedge}(s)=\displaystyle\int_{H^{+}}\displaystyle\int_{\mathcal{N}_{1}^{+}(h)}\displaystyle\int_{\mathcal{N}_{2}^{+}(h,u_{1})}\cdots\displaystyle\int_{\mathcal{N}_{m}^{+}(h,u_{1},\dots,u_{m-1})}|\det{h}|_{p}^{s}d\mu_{H}d\mu_{\mathcal{N}_{1}}\cdots{d\mu_{\mathcal{N}_{m}}}\]
where each $\mu_{\mathcal{N}_{i}}$ is Haar measure on $\mathcal{N}_{i}(\mathbb{Q}_{p})$ normalized so that\\ $\mu_{\mathcal{N}_{i}}(\mathcal{N}_{i}(\mathbb{Z}_{p}))=1$, and \[\mathcal{N}_{i}^{+}(h,u_{1},\dots,u_{i-1}):=\{u_{i}\in\mathcal{N}_{i} : \exists{u_{i+1},\dots,u_{m}}\,\mathrm{such}\,\mathrm{that}\,u_{m}\cdots{u_{1}}h\in{G_{n}^{+}(\mathbb{Q}_{p})}\}.\]
\begin{thebibliography}{2}
\bibitem{BermanThesis} M. N. Berman,
Proisomorphic zeta functions of groups, Ph.D. thesis, University of Oxford,
2005.
\bibitem{Berman}
M. N. Berman,
Uniformity and functional equations for local zeta functions of
K-split
algebraic groups, Amer. J. Math.
133
(2011), 1–27.
\bibitem{BermanGlazerSchein} M. N. Berman, I. Glazer and M. M. Schein, Pro-isomorphic zeta functions of nilpotent groups and Lie rings under base extension, Trans. Amer. Math. Soc. 375 (2022), 1051–1100.
\bibitem{BermanKlopsch}
M. N. Berman and B. Klopsch,
A nilpotent group without local functional equations
for pro-isomorphic subgroups, J. Group Theory
18
(2015), 489–510.
\bibitem{BermanKlopschOnn} M. N. Berman, B. Klopsch and U. Onn,
On pro-isomorphic zeta functions of $D^{\ast}$-groups of even Hirsch length, arXiv/1511.06360v5. To appear in Israel J. Math.
\bibitem{CarnevaleScheinVoll}
 A. Carnevale, M. M. Schein and C. Voll, Generalized Igusa functions and
 ideal growth in nilpotent Lie rings, Algebra Number Theory 18 (2024), 537-582.
\bibitem{DuSautoyGrunewald}
M. P. F. du Sautoy and F. J. Grunewald,
Analytic properties of zeta functions and subgroup
growth, Ann. of Math. (2)
152
(2000), 793–833.
\bibitem{DuSautoyLubotzky} M. P. F. du Sautoy and A. Lubotzky, Functional equations and uniformity for
local zeta functions of nilpotent groups, Amer. J. Math. 118 (1996), 39–90.
\bibitem{DuSautoyWoodward}
M. P. F. du Sautoy and L. Woodward,
Zeta functions of groups and rings, Lecture Notes
in Mathematics, vol. 1925, Springer-Verlag, Berlin, 2008.
\bibitem{Gibbs} J. A. Gibbs, Automorphisms of certain unipotent groups, J. Algebra 14 (1970), 203-228.
\bibitem{GrunewaldSegalSmith} F. J. Grunewald, D. Segal and G. C. Smith, Subgroups of finite index in nilpotent groups,
Invent. Math. 93 (1988), 185–223.
\bibitem{HrushovskiMartinRideau}
 E. Hrushovski, B. Martin and S. Rideau,
Definable equivalence relations and
zeta functions of groups, J. Eur. Math. Soc. (JEMS)
20
(2018), 2467–2537. 
\bibitem{LubotzkySegal}
A. Lubotzky and D. Segal, Subgroup growth, Progress in Mathematics, vol. 212,
 Birkhäuser Verlag, Basel, 2003.
\bibitem{LubotzkyMannSegal} A. Lubotzky, A. Mann and D. Segal,
Finitely generated groups of polynomial
subgroup growth, Israel J. Math.
82
(1993), 363–371.
\bibitem{MontgomeryVaughan} H. L. Montgomery and R. C. Vaughan, Multiplicative Number Theory I. Classical Theory, Cambridge Studies in Advanced Mathematics 97.
\bibitem{Nachbin}
L. Nachbin, The Haar Integral. D. Van Nostrand, 1965.
\bibitem{Segal}
 D. Segal, On the automorphism groups of certain Lie algebras, Math. Proc. Cambridge
 Philos. Soc. 106 (1989), 67–76.
\bibitem{Voll}
 C. Voll,
Functional equations for zeta
functions of groups and rings, Ann. of Math.
(2)
172
(2010), 1181–1218.
\bibitem{Voll2}
C. Voll, Local functional equations for submodule zeta functions associated to nilpotent
algebras of endomorphisms, Int. Math. Res. Not. IMRN (2019), 2137–2176.
\end{thebibliography}
\end{document}