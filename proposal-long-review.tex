\documentclass[12pt]{article}
\makeatletter
\newcommand*{\rom}[1]{\expandafter\@slowromancap\romannumeral #1@}
\makeatother
\usepackage{amsfonts, amssymb}
\usepackage{mathrsfs, mathdots} 
\usepackage{amsmath}
\usepackage{float}
\usepackage{amsthm}
\usepackage{tikz-cd}
\usepackage{xcolor}
\usepackage{xparse}
\usepackage{setspace}
\usepackage{xfrac}
\usepackage{yfonts}

\newtheorem{theorem}{Theorem}[subsection]
\newtheorem{proposition}[theorem]{Proposition}
\newtheorem{corollary}[theorem]{Corollary}
\newtheorem{lemma}[theorem]{Lemma}
\newtheorem{notations}[theorem]{Notations}
\newtheorem{definition}[theorem]{Definition}
\newtheorem{example}[theorem]{Example}
\newtheorem{remark}[theorem]{Remark}

\ExplSyntaxOn
\NewDocumentCommand{\cycle}{ O{\;} m }
 {
  (
  \alec_cycle:nn { #1 } { #2 }
  )
 }

\seq_new:N \l_alec_cycle_seq
\cs_new_protected:Npn \alec_cycle:nn #1 #2
 {
  \seq_set_split:Nnn \l_alec_cycle_seq { , } { #2 }
  \seq_use:Nn \l_alec_cycle_seq { #1 }
 }
\ExplSyntaxOff
\usepackage{titling}
\newcommand{\subtitle}[1]{%
  \posttitle{%
    \par\end{center}
    \begin{center}\large#1\end{center}
    \vskip0.5em}%
}
\usepackage{arydshln}
\setcounter{tocdepth}{2}
\setlength{\dashlinedash}{1.2pt}
\setlength{\dashlinegap}{1.5pt}
\setlength{\arrayrulewidth}{0.2pt}
\title{The pro-isomorphic zeta-functions of some nilpotent Lie algebras over integer rings}
\subtitle{Research proposal for a Ph.D. Thesis\\
Under the supervision of Dr. Michael M. Schein\\
Department of Mathematics, Bar-Ilan University}
\author{Haim Lavi}
\date{\today}
\begin{document}
\maketitle
\newpage
\tableofcontents
\newpage
\begin{abstract}
Let $G$ be any group. For any natural number $n\in\mathbb{N}$, let $a_n$ be the number of subgroups $H\leq G$, such that $[G:H]=n$. Assume $G$ is finitely-generated, then $a_n<\infty$, and we can define a Dirichlet series of the form $\zeta_G(s):=\sum_{n=1}^\infty a_n n^{-s}$, where $s\in\mathbb{C}$. Assume, in addition, that $G$ is also nilpotent and torsion-free, then this function has some properties of the Riemann $\zeta$-function, such as the Euler decomposition of $\zeta$ into a product of local factors indexed by primes. A version of this $\zeta$-function counts pro-isomorphic subgroups, and an analogous function may be defined for appropriate Lie rings. We study here the pro-isomorphic $\zeta$-functions for a family of nilpotent Lie rings of unbounded nilpotency class. We shall compute the automorphism groups of these Lie rings explicitly, prove uniformity of the local factors of the pro-isomorphic $\zeta$-functions, and aim to determine them explicitly.
\end{abstract}
\section{Scientific Background}
\subsection{Introduction}
Although we will work with Lie algebras, for motivation we first present analogous and more natural questions in the context of groups.\par
We start our discussion with the following proposition, which stands at the very foundation of our subject.
\begin{proposition} \label{prop:finite.number.subgroups}
Let $G$ be any finitely generated group, and let $n\in\mathbb{N}$ be any natural number. Then there is a finite number of subgroups $H\leq G$, such that $[G:H]=n$
\end{proposition}
This proposition gives rise to an entire subject in group theory, called \textbf{subgroup growth}. We denote by $a_n(G)$ the number of subgroups of $G$ of index $n$, and look at the sequence $\{a_n(G)\}_{n=1}^{\infty}$. The subject of subgroup growth aims to relate the properties of this sequence to the algebraic structure of $G$. For instance, Lubotzky, Mann and Segal showed in \cite{LubotzkyMannSegal} that $a_n(G)$ grows polynomially if and only if $G$ is virtually solvable of finite rank, that is, $G$ has a finite-index solvable subgroup, and all finitely-generated subgroups may be generated by a bounded number of generators. This research concentrates on the growth of \textbf{pro-isomorphic} subgroups, which we now define.
\begin{definition}
\label{def:profinite.closure}
Let $G$ be any group, and let $\mathcal{N}(G):=\{N_k\trianglelefteq G\}_{k\in I}$ be the set of all normal subgroups of $G$. We define a partial order on $\mathcal{N}(G)$ by reverse inclusion, that is for every two indices $i,j$ we say that $i\leq j$ if and only if $N_j\subseteq N_i$, hence for every $i\leq j$ there exists a natural projection map $\pi_{ji}:\sfrac{G}{N_
j}\rightarrow\sfrac{G}{N_i}$. The inverse limit \[\widehat{G}=\underset{\leftarrow}{lim}\{\sfrac{G}{N_k}\}_{k\in I}:=\{(h_k)_{k\in I}\in\prod_{k\in I}\sfrac{G}{N_k} : \pi_{ji}(h_j)=h_i,\forall i\leq j\}\] is called the \textbf{profinite closure} of $G$.
\end{definition}
\begin{definition}
\label{def:pro.isomorphic}
Let $G$ be any group. A subgroup $H\leq G$ is called \textbf{pro-isomorphic} if $\widehat{H}\cong\widehat{G}$.
\end{definition}
\begin{definition}
\label{def:zeta.pro.isomorphic}
Let $G$ be any group, and let \[\hat{a_n}(G):=\#\{H\leq G : \widehat{H}\cong\widehat{G}, [G:H]=n\}\] be the number of pro-isomorphic subgroups of $G$ of index $n$. Assume that $\hat{a_n}(G)<\infty$ for all $n$. The \textbf{pro-isomorphic $\zeta$-function} of $G$ is defined by $\hat{\zeta_G}(s):=\sum_{n=1}^{\infty}\hat{a_n}(G)n^{-s}$ for $s\in\mathbb{C}$.
\end{definition}
If our $\zeta$-function, which is a special case of the Dirichlet series, has some properties of convergence on some subset of $\mathbb{C}$, one may reconstruct its coefficients $\hat{a_n}(G)$, which, in our case, are the number of subgroups of our interest, using the \textbf{Perron's formula}, which is an implementation of a \textbf{reverse Mellin transform}, as discussed, for example, in \cite{MontgomeryVaughan}. This method, including the specific properties of convergence required for the reconstruction, is out of the scope of our research, and therefore will not be further discussed, at this stage.\par
It is known that if $\hat{a_n}(G)$ grows polynomially, then $\hat{\zeta_G}(s)$ converges on some right half-plane of $\mathbb{C}$. For instance, we take the group $G=(\mathbb{Z},+)$. The group $\mathbb{Z}$ is an abelian group, and every subgroup is of the form $H=n\mathbb{Z}=\langle n\rangle$, for some $n\in\mathbb{N}$, which means that $H\cong \mathbb{Z}$, as both are infinite cyclic groups, and so, $\widehat{H}\cong\widehat{\mathbb{Z}}$. Since we have only one subgroup of index $n$, for every $n\in\mathbb{N}$, then $a_n(\mathbb{Z})=\hat{a_n}(\mathbb{Z})=1$. Thus, its pro-isomorphic $\zeta$-function is $\hat{\zeta_{\mathbb{Z}}}=\sum_{i=1}^{\infty}n^{-s}=\zeta(s)$, the Riemann $\zeta$-function, which is known to converge for $Re(s)>1$.\par
After establishing the basic definitions, we recall a known fact that is a major motivation for this research, which says that the Riemann $\zeta$-function decomposes into an infinite product of local zeta-functions, that is, $\zeta(s)=\prod_p\zeta_p(s)=\prod_p\sum_{k=0}^\infty p^{-ks}=\prod_p\frac{1}{1-p^{-s}}$, where the product runs over all the prime numbers. Following this fact regarding the Riemann $\zeta$-function, we observe that for any finitely-generated, nilpotent and torsion-free group $G$, we have the same decomposition as above for the pro-isomorphic $\zeta$-function: $\hat{\zeta_G}(s)=\prod_p\hat{\zeta_{G,p}}(s)$, where $\hat{\zeta_{G,p}}(s):=\sum_{k=0}^\infty \hat{a_{p^k}}(G)p^{-ks}$. The general construction of $\zeta$-functions, as well as their Euler decomposition to local zeta-functions, for the study of subgroup growth, were well established by Grunewald, Segal and Smith, in \cite{GrunewaldSegalSmith}.
\subsection{Linearization}
We want to transfer the ideas from the above discussion about groups to a linear context, where we can use tools from linear algebra.
Hence, for finitely-generated torsion-free nilpotent groups $G$, we associate nilpotent Lie algebras over $\mathbb{Z}$. This, in general, is called the \textbf{Mal'cev correspondence}. 
If $L$ is a $\mathbb{Z}$-Lie algebra, namely a free $\mathbb{Z}$-module of finite rank with a Lie bracket operation, then consider the number $\hat{a_n}(L)$ of subalgebras $M\leq L$, where $n=[L:M]$, such that $M\otimes_{\mathbb{Z}}\mathbb{Z}_p\cong L\otimes_{\mathbb{Z}}\mathbb{Z}_p$, where $\mathbb{Z}_p$ is the ring of $p$-adic integers, for all primes $p$, and it is also known that $\hat{a_n}(L)<\infty$ for all $n\in\mathbb{N}$. The Dirichlet series $\hat{\zeta_L}(s):=\sum_{n=1}^{\infty}\hat{a_n}(L)n^{-s}$, is called the \textbf{pro-isomorphic zeta-function} of $L$. By the Mal'cev correspondence, to every finitely-generated, nilpotent, torsion-free group $G$, one may associate a Lie algebra $L=L(G)$, such that $\hat{\zeta_{G,p}}(s)=\hat{\zeta_{L,p}}(s)$, for all but finitely many primes $p$. If $G$ has nilpotency class $2$, one may obtain the equality for all primes. For this $L$, choose a basis $\mathcal{B}(\mathbb{Z})=\{b_1,\dots,b_r\}$, where $r=\mathrm{rank}L$. Let $\mathcal{L}_{p}=L\otimes_{\mathbb{Z}}\mathbb{Q}_p$, for any $p$. This is a $\mathbb{Q}_p$-Lie algebra, and our choice of basis allows us to identify the automorphism group $G(\mathbb{Q}_p)=Aut_{\mathbb{Q}_p}(\mathcal{L}_{p})$ with a subgroup of $GL_r(\mathbb{Q}_p)$. Note that $\mathcal{L}_{p}$ contains a $\mathbb{Z}_p$-lattice, $L_{p}=L\otimes_{\mathbb{Z}}\mathbb{Z}_p$. If $\varphi\in G(\mathbb{Q}_p)$, then $\varphi(L_{p})=L_{p}$ if and only if $\varphi\in G(\mathbb{Z}_p)=G(\mathbb{Q}_p)\cap GL_r(\mathbb{Z}_p)$. Here $GL_r(\mathbb{Z}_p)$ is the group of $r\times r$ matrices which are invertible over $\mathbb{Z}_p$. Similarly, $\varphi(L_{p})\subseteq L_{p}$ if and only if $\varphi\in G^+(\mathbb{Q}_p):=G(\mathbb{Q}_p)\cap \mathcal{M}_r(\mathbb{Z}_p)$, where $\mathcal{M}_r(\mathbb{Z}_p)$ is the collection of $r\times r$ matrices with entries in $\mathbb{Z}_p$. Note that $G^+(\mathbb{Q}_p)$ is a monoid, not a group.\par
Denote by $G(\mathbb{Z}_p)g$, where $g\in G^+(\mathbb{Q}_p)$, a right-coset of $G(\mathbb{Z}_p)$. One checks that the monoid $G^+(\mathbb{Q}_p)$ is a disjoint union of right-cosets of $G(\mathbb{Z}_p)$.\par
The discussion above reveals the construction we base our research upon.
We observe that there is a bijection between the set $G(\mathbb{Z}_p)\backslash G^+(\mathbb{Q}_p)$ of right-cosets of $G(\mathbb{Z}_p)$ and the set $\{M\leq L_{p} : M\cong L_{p}\}$ of $L_{p}$-subalgebras which are isomorphic to $L_{p}$ itself. This bijection takes $G(\mathbb{Z}_p)g$ to $M=\varphi(L_{p})$. For any $\varphi\in G(\mathbb{Z}_p)g$, this is well-defined. One checks that for every $\psi\in G(\mathbb{Z}_p)g$, we have that $\psi(L_{p})=\varphi(L_{p})=M$.
We end this part, as a preparation for the final part of this technical background review, with the following result, which states that for each right-coset $G(\mathbb{Z}_p)g$, if $M=\varphi(L_{p})$, where $\varphi\in G(\mathbb{Z}_p)g$, then $[L_{p}:M]=|\det\varphi|_p^{-1}$, where $|\det\varphi|_p$ is the $p$-adic norm of $\det\varphi$, and therefore,\par $\hat{\zeta_{L,p}}(s)=\underset{\overset{\scriptscriptstyle M\leq L_{p}}{\scriptscriptstyle M\cong L_{p}}}{\sum}[L_{p}:M]^{-s}=\underset{\scriptscriptstyle G(\mathbb{Z}_p)\varphi\in G(\mathbb{Z}_p)\backslash G^+(\mathbb{Q}_p)}{\sum}|\det\varphi|_p^s$.
Last for this section, we shall define the lower central series for Lie algebras, which will be our main working tool in the research of the automorphism groups.
\begin{definition}
Let $L$ be a Lie algebra, then the $\textbf{lower central series}$ of $L$ is defined by $\gamma_k L:=[L,\gamma_{k-1}L]$ where $k\in\mathbb{N}$, and $\gamma_1 L:=L$, same as defined for groups in \ref{def.lower.central.series}. A Lie algebra is said to be \textbf{nilpotent}, if $\gamma_k L=\{0\}$, for some finite $k$, also as defined above for groups in \ref{def.nilpotent.group}.
\end{definition}
When the algebra is clear from the context, we may omit the notation of the algebra in $\gamma_k L$, only to stay with the gamma notation $\gamma_k$. Sometimes, we will also use the gamma notation for nilpotnecy classes of groups.
\subsection{$p$-adic Integration}
In this final part of the technical background review, we finally get to the motivation for all the construction we have presented in the first parts. We now define a very central object for our research. We shall assume, without proof, the existence of such an object, under the prerequisites of the definition.
\begin{definition}
\label{def.right.haar.measure}
Let $\Gamma$ be a locally compact topological group, i.e. for all $\gamma\in\Gamma$, there is an open neighborhood of $\gamma\in U_{\gamma}$ and a compact subset $K_{\gamma}$, such that $U_{\gamma}\subset K_{\gamma}$. Then there is a measure $\mu$, with the following property: for any measurable subset $U\subseteq\Gamma$ and any $\gamma\in\Gamma$, $\mu(U\gamma)=\mu(U)$, where $U\gamma:=\{u\gamma : u\in U\}$. Such a measure $\mu$ is called a \textbf{right Haar measure}, and is unique up to multiplication by a non-zero constant.
\end{definition}
Equipped with this definition of a right Haar measure, we can finally make use of the construction from above. We start by claiming, without proof, that for every prime number $p$, the group $G(\mathbb{Q}_p)$ is a locally compact topological group. We also claim that the right Haar measure can be normalized such that $\mu(G(\mathbb{Z}_p))=1$. The measure of all the right-cosets of $G(\mathbb{Z}_p)$ equals to the measure of $G(\mathbb{Z}_p)$ itself, i.e. for every $g\in G^+(\mathbb{Q}_p)$, we have that $\mu(G(\mathbb{Z}_p)g)=\mu(G(\mathbb{Z}_p))=1$.
With this observation, we go directly to the calculation of the $p$-adic norm of the determinant of every $L_{p}$-automorphism, as a $p$-adic integral over our measure space.
First, we observe that given any $L_{p}$-automorphism in some right-coset $\varphi\in G(\mathbb{Z}_p)\varphi$, we have that $|\det\varphi|_p^s=\displaystyle\int_{G(\mathbb{Z}_p)\varphi}|\det\varphi|_p^sd\mu$, because $\mu(G(\mathbb{Z}_p)\varphi)=1$, and $|\det\varphi|_p^{-1}$ is fixed on $G(\mathbb{Z}_p)\varphi$.\par
Going back to our desired function, we observe that\par $\hat{\zeta_{L,p}}(s)=\underset{\scriptscriptstyle G(\mathbb{Z}_p)\varphi\in G(\mathbb{Z}_p)\backslash G^+(\mathbb{Q}_p)}{\sum}|\det\varphi|_p^s=\underset{\scriptscriptstyle G(\mathbb{Z}_p)\varphi\in G(\mathbb{Z}_p)\backslash G^+(\mathbb{Q}_p)}{\sum}\displaystyle\int_{G(\mathbb{Z}_p)\varphi}|\det\varphi|_p^sd\mu=\displaystyle\int_{G^+(\mathbb{Q}_p)}|\det\varphi|_p^sd\mu$.\par 
This calculation of the local $\zeta_p$-function as a $p$-adic integral was established by the work of du Sautoy and Lubotzky, in \cite{DuSautoyLubotzky}.
This integral is the main object we shall study in this research.
We end this part, of the technical background review, by a theorem and a couple of definitions, which stand in the center of our research goals.
\begin{theorem}
\label{thm.rational.function}
Let $p$ be a prime number, and let $s\in\mathbb{C}$, then $\hat{\zeta_{L,p}}(s)$ is rational, i.e. there is a rational function in one variable $W_p\in\mathbb{Q}(X)$ such that $\zeta_{L,p}(s)=W_p(p^{-s})$.
\end{theorem}
\begin{definition}
\label{def.uniform}
Let $L$ be a $\mathbb{Z}$-Lie algebra, and let $s\in\mathbb{C}$. Then $\hat{\zeta_L}(s)$ is called \textbf{uniform}, if there exists a rational function in two variables $W\in\mathbb{Q}(X,Y)$ such that for every prime number $p$, the local function $\zeta_{L,p}(s)=W(p,p^{-s})$, which means that $\zeta_{L,p}(s)$ has exactly the same form for all primes, since its value depends only on $p$ and $p^{-s}$.
\end{definition}
\begin{definition}
\label{def.finitely.uniform}
Let $L$ be a $\mathbb{Z}$-Lie algebra, and let $s\in\mathbb{C}$. Then $\hat{\zeta_L}(s)$ is called \textbf{ finitely uniform}, if there exists a finite set of rational function in two variables $W_1,W_2,\dots,W_m\in\mathbb{Q}(X,Y)$ such that for every prime number $p$, the local function $\zeta_{L,p}(s)=W_k(p,p^{-s})$, for some $1\leq k\leq m$. One observes immediately that this is a weaker condition on $L$, because a uniform zeta-function is simply a finitely uniform function where $m=1$.
\end{definition}
The theorem ensures that for any $\mathbb{Z}$-Lie algebra, all the local pro-isomorphic zeta-functions are rational functions in $p^{-s}$, but their form can vary between different primes. If we can prove uniformity, for some specific $\mathbb{Z}$-Lie algebra $L$, it means that all its local zeta-functions have exactly the same form as a rational function in $p$ and $p^{-s}$, and if we can prove finite uniformity, then we can at least show that all the local zeta-functions have only a finite number of different forms.
The uniformity is also established in the work of Grunewald, Segal and Smith, see \cite{GrunewaldSegalSmith}.
Proving the uniformity of the pro-isomorphic zeta-functions of the algebras we are studying is one of the main goals of our research.
\section{Research Goals and Methodology}
\subsection{The Lie algebras $L_{n}$}
In all the following, given some algebra $L$, we may denote the quotients $\sfrac{\gamma_{k}L}{\gamma_{l}L}$ by $\sfrac{\gamma_{k}}{\gamma_{l}}$, for some $1\leq{k}<{l}\leq{n-1}$, for abbreviation. We may also denote the quotient $\sfrac{L}{\gamma_{k}}$ by $\sfrac{\gamma_{1}}{\gamma_{k}}$.\par
Let $e_{ij}$ be an $n\times{n}$ matrix, in which all the elements are zero, except for the element in row $i$ and column $j$ which has $1$. On the set $\mathcal{E}_n=\{e_{ij} : 1\leq{i}\leq{n-1}\land{{i+1}\leq{j}\leq{n}}\}$ we define a bracket operation: for every $1\leq{k,l}\leq{n-1}$, define $[e_{k,k+1},e_{l,l+1}]:=e_{k,k+1}e_{l,l+1}-e_{l,l+1}e_{k,k+1}$. Let $\mathcal{R}$ be some commutative ring, then the standard operation of $\mathcal{R}$ on $\mathcal{E}_n$, along with the defined bracket operation, form a nilpotent $\mathcal{R}$-Lie algebra. Considering $\mathcal{R}=\mathbb{Z}$, we obtain a nilpotent $\mathbb{Z}$-Lie algebra of strictly upper triangular matrices over $\mathbb{Z}$, which we denote by $L_n$, with the standard bracket operation as its Lie brackets. As discussed above, this $\mathbb{Z}$-Lie algebra can be extended to a $\mathbb{Z}_p$-algebra, which we denote by $L_{n,p}$, and then to a $\mathbb{Q}_p$-algebra, which we denote by $\mathcal{L}_{n,p}$.
It is readily seen that the set of matrices of the form $e_{ij}$ where $i<j$, spans the whole $\mathbb{Z}$-Lie algebra $L_n$ and is $\mathbb{Z}$-linearly independent. Therefore, it forms a basis for $L_n$ as a free module over $\mathbb{Z}$, $\mathcal{B}_n(\mathbb{Z}):=\{e_{12},e_{13},\dots,e_{1n},e_{23},\dots,e_{2,n},\dots,e_{n-1,n}\}$, which we call \textbf{the standard basis of $L_n$}. One easily checks that $r=\mathrm{rank}L_n=|\mathcal{B}_n(\mathbb{Z})|=\binom{n}{2}$, which is the number of elements above the main diagonal for every $n\in\mathbb{N}$. To this standard basis we apply a linear order by defining $e_{ij}<e_{kl}$ if $j-i<l-k$ or if $j-i=l-k$ and $i<k$. In other words, we apply an order that divides $\mathcal{B}_n(\mathbb{Z})$ to basis elements of the quotients $\sfrac{L_n}{\gamma_2},\sfrac{\gamma_2}{\gamma_3},\dots,\sfrac{\gamma_{n-2}}{\gamma_{n-1}},\gamma_{n-1}$


Obviously, the same goes also for the extensions of $L_n$, namely $L_{n,p}$ and $\mathcal{L}_{n,p}$. The target of our research is studying the $\hat{\zeta_{L,p}}$-function on these $\mathbb{Z}_p$-Lie algebras, and the related constructions.
\begin{remark}
Let $\mathcal{R}$ be a commutative ring, and let $\mathcal{U}_n(\mathcal{R})$ be the group of $n\times{n}$ upper unitriangular matrices over $\mathcal{R}$, with the standard matrix multiplication as the group operation. Denote by $\mathcal{U}_{n,p}=\mathcal{U}_n(\mathbb{Z}_p)$ the unitriangular matrix group over $\mathbb{Z}_p$, then by Mal'cev correspondence, $\hat{\zeta_{\mathcal{U}_{n,p}}}(s)=\hat{\zeta_{L_{n,p}}}(s)$, for all but finitely many primes $p$. This relates the subject of research to the motivation presented at the beginning of this paper.
\end{remark}
\subsection{Research goals}
The project consists of three major steps:\par
1. \textbf{Computing the automorphism group of the $\mathbb{Q}_p$-Lie algebras $\mathcal{L}_{n,p}$, for all $n\in\mathbb{N}$ and all primes $p$.}\par
2. \textbf{Showing that the pro-isomorphic zeta-functions $\hat{\zeta_{L_{n,p}}}(s)$ are uniform for all $n\in\mathbb{N}$.}\par
3. \textbf{Giving an explicit uniform formula for the zeta-functions $\hat{\zeta_{L_{n,p}}}(s)$ for specific values of $n$, if not for all $n\in\mathbb{N}$.}
Specifically for $n=5$ we aim to continue the work of Mark N. Berman, who proved that $\hat{\zeta_{L_{5,p}}}(s)$ is uniform.\par
As we elaborate further, steps 1 and 2 are already known entirely for $n\leq 5$, and step 3 is known for $n\leq 4$.
We start with the first step of calculating $Aut_{\mathbb{Q}_p}(\mathcal{L}_{n,p})$. These automorphism groups have been studied for decades from a different point of view.  There are classical results showing that any automorphism may be expressed as a product of automorphisms of a specific type; see, for instance, the main result of Gibbs \cite{Gibbs}.  These results are not explicit enough for our purposes; indeed, the submonoid $G_n^+(\mathbb{Q}_p)$ arises for us as the domain of integration of a $p$-adic integral.  In order to calculate this integral, we need to decompose the automorphism group $G_n(\mathbb{Q}_p)$ into a repeated semi-direct product of groups with a simple structure.
\par
After we have analyzed the structure of $G_n(\mathbb{Z}_p)$, we will need to construct the monoid $G_n^+(\mathbb{Q}_p)$ and its $G_n(\mathbb{Z}_p)$ right-cosets, as we have seen above. This will give us both the function to integrate, which is $\det\varphi$ for every $G_n(\mathbb{Z}_p)$ right-coset $G_n(\mathbb{Z}_p)\varphi$, and the domain of integration, which is the monoid $G_n^+(\mathbb{Q}_p)$. We will use this information to analyze the behavior of the $p$-adic integral we have described above and prove that its calculation depends only on $p$, thus showing that the $\hat{\zeta_{L,p}}$-function is uniform.
\subsection{$L_{n,p}$-Lie algebras for $n>3$}
Mark N. Berman, in his doctoral thesis \cite{Berman}, has displayed an explicit formula for $\hat{\zeta_{L_{4,p}}}$, and proved that $\hat{\zeta_{L_{5,p}}}$ is indeed uniform. We aim to generalize his work to prove that $\hat{\zeta_{L_{n,p}}}$ is uniform for all $n$. We also aim to compute $\hat{\zeta_{\mathcal{L}_{n,p}}}(s)$ explicitly for all $n$, or at least to obtain explicit formulas for some $n\geq 5$, and specifically for $n=5$. By analyzing carefully Berman's work on $L_{4,p}$ and $L_{5,p}$, we gain the basic understanding of the expected structure of the local zeta-functions in the general case.
We begin our discussion of the first goal, which is computing $G_n(\mathbb{Z}_p)$, by first recalling that for every $v\in L_{n,p}$, where $n\geq3$, we present $\varphi(v)$ as the multiplication of $v$ by a matrix from the right $\varphi(v)=vM$. As stated earlier, $M$ is an $r\times r$ matrix, where $r=\mathrm{rank}L_{n,p}=\binom{n}{2}$, whose lines are set by the order we have defined above, i.e. considering the standard ordered basis \[\mathcal{B}_n=\{e_{
12},e_{23},\dots,e_{n-1,n},e_{13},\dots,e_{n-2,n},\dots,e_{1n}\}\] then $M$ is the following matrix,
$$
M=\begin{pmatrix}
\varphi(e_{12})\\
\varphi(e_{23})\\
\varphi(e_{n-1,n})\\
\hdashline
\varphi(e_{13})\\
\vdots\\
\varphi(e_{n-2,n})\\
\hdashline
\vdots\\
\hdashline
\varphi(e_{1n})\\
\end{pmatrix}\\
$$
Given an $\mathcal{L}_{n,p}$-automorphism $\varphi$, we denote by $\varphi_k:\gamma_k\mathcal{L}_{n,p}\rightarrow\gamma_k\mathcal{L}_{n,p}$ the operation of $\varphi$ on all the $n-k$ elements of the lower central series starting from $k$, that is, we consider only the images \[\varphi(e_{1,1+k}),\varphi(e_{2,2+k}),\dots,\varphi(e_{n-k,n}),\varphi(e_{1,2+k}),\dots,\varphi(e_{n-k-1,n}),\dots,\varphi(e_{1n})\]
For every $\varphi_k$, we have the induced map denoted by $\varphi_{kk}$, from the quotient algebra $\sfrac{\gamma_k}{\gamma_{k+1}}$ to itself, defined by $\varphi_{kk}(e_{l,l+k}+\gamma_{k+1}\mathcal{L}_{n,p}):=a_{1,1+k}e_{1,1+k}+a_{2,2+k}e_{2,2+k}+\cdots+a_{n-k,n}e_{n-k,n}+z_{k+1}$, where $z_{k+1}\in\gamma_{k+1}\mathcal{L}_{n,p}$, for every $1\leq{l}\leq{n-k}$. Clearly, $\varphi_{kk}$ is well-defined, since $\varphi_k(\gamma_k\mathcal{L}_{n,p})=\gamma_k\mathcal{L}_{n,p}$, for every $1\leq{k}\leq{n-1}$.
Following this division of $\mathcal{L}_{n,p}$ by the lower central series and its quotients, we view $M$ as a block matrix, \[M=\begin{pmatrix}
M_{11} & \vline & M_{12}&\vline & \dots& \vline & M_{1,n-2} & \vline&M_{1,n-1}\\
\hline
M_{21} & \vline & M_{22}&\vline & \dots &\vline & M_{2,n-2} &\vline& M_{2,n-1}\\
\hline
\vdots & \vline & \vdots&\vline & \ddots &\vline & \vdots &\vline& \vdots\\
\hline
M_{n-1,1} & \vline & M_{n-1,2}&\vline & \dots &\vline & M_{n-1,n-2} &\vline& M_{n-1,n-1}\\
\end{pmatrix}\]
each block is denoted by $M_{kl}\in\mathcal{M}_{m\times{r}}(\mathbb{Q}_p)$, where $m=\dim\sfrac{\gamma_k}{\gamma_{k+1}}$ and $r=\dim\sfrac{\gamma_l}{\gamma_{l+1}}$. From this, we can understand that the blocks on the main diagonal of $M$, which are the induced quotient maps defined above, are square matrices $\varphi_{kk}=M_{kk}\in\mathcal{M}_{n-k}(\mathbb{Q}_p)$. A trivial observation is that since all the elements of the lower central series of $\mathcal{L}_{n,p}$ are characteristic subalgebras, then $\sfrac{\varphi(\gamma_k\mathcal{L}_{n,p})}{\gamma_k\mathcal{L}_{n,p}}=0$, which means that all the matrix blocks $M_{kl}$, where $k<l$, must be zero, therefore $M$ has the form, \[M=\begin{pmatrix}
M_{11} & \vline & M_{12}&\vline & M_{13} & \dots& \vline & M_{1,n-2} & \vline&M_{1,n-1}\\
\hline
0 & \vline & M_{22}&\vline & M_{23} & \dots &\vline & M_{2,n-2} &\vline& M_{2,n-1}\\
\hline
\vdots & \vline & \vdots&\vline & \vdots & \ddots &\vline & \vdots &\vline& \vdots\\
\hline
0 & \vline & 0&\vline & 0 & \dots &\vline & M_{2,n-2} &\vline& M_{2,n-1}\\
\hline
0 & \vline & 0 &\vline & 0 & \dots &\vline & 0 &\vline& M_{n-1,n-1}\\
\end{pmatrix}
\]
\subsection{Preliminary results}
\label{preliminary.results}
\begin{proposition}
Let $\varphi\in{G_n(\mathbb{Q}_p)}$ be a $\mathcal{L}_{n,p}$-automorphism, and $M$ its representing matrix, divided into matrix blocks, as shown earlier. Then, $M_{11}\in\mathcal{M}_{n-1}(\mathbb{Q}_p)$ is either diagonal or anti-diagonal.
\end{proposition}
\begin{proposition}
Let $\mathcal{L}_{n,p}$ be the $\mathbb{Q}_p$-algebra of our research. Define the map $\eta_{n}(e_{ij}):=(-1)^{j-i-1}e_{n+1-j,n+1-i}$, for $1\leq{i}\leq{n-1}$ and ${i+1}\leq{j}\leq{n}$. Then $\eta$ is an involution, hence also an $\mathcal{L}_{n,p}$-automorphism.
\end{proposition}
One easily checks that the above is true.
\begin{proposition}
Let $\varphi\in{G_n(\mathbb{Q}_p)}$ be a $\mathcal{L}_{n,p}$-automorphism, and $M$ its representing matrix. Then if $M_{11}$ is anti-diagonal, replacing $\varphi$ with $\varphi\circ\eta_{n}$, we may assume, without loss of generality, that $M_{11}$ is diagonal.
\end{proposition}
We shall explain in details, in the research itself, that computing $\hat\zeta_{L_{n,p}}(s)$ is invariant to whether we assume $M_{11}$ is diagonal or we assume that $M_{11}$ is anti-diagonal. 
\begin{proposition}
Let $\lambda_{1},\lambda_{2},\dots,\lambda_{n-1}\in\mathbb{Q}_{p}^{\times}$ be the elements on the main diagonal of the matrix block $M_{11}$, where $M$ is the matrix representing $\varphi\in{G}_n(\mathbb{Q}_p)$. Then the main diagonal of $M$ is \[\lambda_{1},\lambda_{2},\dots,\lambda_{n-1},\lambda_{1}\lambda_{2},\lambda_{2}\lambda_{3},\dots,\lambda_{n-2}\lambda_{n-1},\dots,\lambda_{1}\lambda_{2}\cdots\lambda_{n-1}\].
\end{proposition}
This proposition is easily proved by recursively evaluating the relation $\varphi(e_{i,j})=[\varphi(e_{i,j-1}),\varphi(e_{j-1,j})]$, for all $1\leq{i}<{j}\leq{n-1}$.
\begin{proposition}
Let $\varphi\in{N_n(\mathbb{Q}_p)}$ be a $\mathcal{L}_{n,p}$-automorphism, with its representing matrix $M$. Then, for all $1\leq{r}\leq{n-1}$, the matrix block $M_{1r}=(m_{kl})\in\mathcal{M}_{n-1,n-r}(\mathbb{Q}_p)$ has the constraint that $m_{kl}=0$ for all ${k+1}\leq{l}\leq{n-r}$ and for all $1\leq{l}\leq{k-r}$. Specifically for $r=1$, $M_{11}=I_{n-1}$, which is an already known result.
\end{proposition}
We shall describe the strategy for proving this proposition, the full details shall be described in the research itself. For all $1\leq{i}\leq{n-1}$, $\varphi(e_{i,i+1})=a_{i,i+1}^{i}e_{i,i+1}+\sum_{r=2}^{n-1}\sum_{j=1}^{n-r}a_{j,j+r}e_{j,j+r}$. We aim to prove that for all $1\leq{r}\leq{n-1}$ and for all $1\leq{i}\leq{n-1}$, we have the constraints $a_{i+j,i+j+r}^{i}=0$ for all $1\leq{j}\leq{n-r-i}$, and $a_{i-r-(j-1),i-(j-1)}$ for $1\leq{j}\leq {i+1}$. We establish the assumption for $r=1$, because we already know that $M_{11}$ is a diagonal matrix. Moving to $r>1$, we show first that $a_{i+1,i+1+r}^{i}=0$ and that $a_{i-r,i}^{i}=0$, by utilizing the relation $[\varphi(e_{i,i+1}),\varphi(e_{i+1,i+3})]=0$ and the induction assumption.
In the same way, we prove that $a_{i-1-r,i-1}^{i}=0$, utilizing the relation $[\varphi(e_{i,i+1}),\varphi(e_{i-1,i+1})]=0$ and the assumption.
Then we prove that $a_{i+1,i+1+r}=0$ by showing, also considering the assumption, that unless we have this constraint, then if $\mathcal{C}_{\sfrac{\mathcal{L}_{n,p}}{\gamma_3}}(e_{i,i+1})$ is the centralizer of $e_{i,i+1}$ in the quotient $\sfrac{\mathcal{L}_{n,p}}{\gamma_3}$, we get that $\dim\mathcal{C}_{\sfrac{\mathcal{L}_{n,p}}{\gamma_3}}(\varphi(e_{i,i+1}))\neq\mathcal{C}_{\sfrac{\mathcal{L}_{n,p}}{\gamma_3}}(e_{i,i+1})$, which is a contradiction. Once establishing all these constraints for $M_{1,r+1}$, and for all $1\leq{i}\leq{n-1}$, then, fixing $r$, we prove that $a_{i+j,i+j+r}^{i}=0$, for all $1\leq{j}\leq{n-1}$, and that $a_{i-r-j}^{i}=0$, for all $0\leq{j}\leq{i-r-1}$, by induction on $j$, considering the assumption on $r$ itself.
\subsection{Base Extension}
Let $K$ be a number field of degree $d=[K:\mathbb{Q}]$, and let $\mathcal{O}_K$ be its ring of integers. Let $L$ be a $\mathbb{Z}$-Lie algebra of rank $r$. By base extension we can consider $L\otimes_{\mathbb{Z}}\mathcal{O}_K$ as a $\mathbb{Z}$-Lie algebra of rank $rd$, and by extension of scalars we can consider also $L_{K,p}=(L\otimes_{\mathbb{Z}}\mathcal{O}_K)\otimes_{\mathbb{Z}}\mathbb{Q}_p$ as a $\mathbb{Q}_p$-Lie algebra of the same rank. Berman-Glazer-Schein give a criterion in \cite{BermanGlazerSchein}, under which the pro-isomorphic zeta-function of $L_{K,p}$ can be calculated without a significant extra effort relative to that of $L_p$ itself. Note that the criterion does not necessarily apply for all $p$. We shall research whether the criterion applies to the $\mathbb{Q}_p$-Lie algebras $\mathcal{L}_{n,p}$ of our work. If so, then $\hat\zeta_{\mathcal{L}_n\otimes\mathcal{O}_K,p}$ will be finitely uniform (see \ref{def.finitely.uniform}). In other words, for each of the finitely many decomposition types of a prime in $\mathcal{O}_K$, there is a rational function in two variables $W\in\mathbb{Q}(X,Y)$ such that $\hat\zeta_{\mathcal{L}_n\otimes\mathcal{O}_K,p}(s)=W(p,p^{-s})$ for all $p$ of that decomposition type.
\begin{thebibliography}{2}
\bibitem{Berman} Mark. N. Berman,
Proisomorphic zeta functions of groups
, Ph.D. thesis, University of Oxford,
2005.
\bibitem{BermanGlazerSchein} Mark N. Berman, Itay Glazer, and Michael M. Schein, Pro-isomorphic zeta functions of nilpotent groups and lie rings under base extension, Trans. Amer. Math. Soc. 375 (2022), 1051–1100.
\bibitem{DuSautoyLubotzky} M.P.F. du Sautoy and A. Lubotzky, Functional equations and uniformity for
local zeta functions of nilpotent groups, Amer. J. Math. 118 (1996), no. 1, 39–
90.
\bibitem{Gibbs} John A. Gibbs, Automorphisms of certain unipotent groups, J. Algebra 14 (1970), 203-228.
\bibitem{GrunewaldSegalSmith} F. J. Grunewald, D. Segal, and G. C. Smith, Subgroups of finite index in nilpotent groups,
Invent. Math. 93 (1988), no. 1, 185–223.
\bibitem{LubotzkyMannSegal} Alexander Lubotzky, Avinoam Mann, and Dan Segal,
Finitely generated groups of polynomial
subgroup growth
, Israel J. Math.
82
(1993), no. 1-3, 363–371.
\bibitem{MontgomeryVaughan} Hugh L. Montgomery and Robert C. Vaughan, Multiplicative Number Theory I. Classical Theory, Cambridge Studies in Advanced Mathematics 97.
\end{thebibliography}
\end{document}