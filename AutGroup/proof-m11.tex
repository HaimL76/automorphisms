\documentclass[12pt]{article}
\usepackage{graphicx} % Required for inserting images
\makeatletter
\newcommand*{\rom}[1]{\expandafter\@slowromancap\romannumeral #1@}
\makeatother
\usepackage{amsfonts, amssymb}
\usepackage{mathrsfs, mathdots} 
\usepackage{amsmath}
\usepackage{float}
\usepackage{amsthm}
\usepackage{tikz-cd}
\usepackage{xcolor}
\usepackage{xparse}
\usepackage{setspace}
\usepackage{xfrac}
\usepackage{yfonts}

\newtheorem{theorem}{Theorem}[subsection]
\newtheorem{proposition}[theorem]{Proposition}
\newtheorem{corollary}[theorem]{Corollary}
\newtheorem{lemma}[theorem]{Lemma}
\newtheorem{notations}[theorem]{Notations}
\newtheorem{definition}[theorem]{Definition}
\newtheorem{example}[theorem]{Example}

\ExplSyntaxOn
\NewDocumentCommand{\cycle}{ O{\;} m }
 {
  (
  \alec_cycle:nn { #1 } { #2 }
  )
 }

\seq_new:N \l_alec_cycle_seq
\cs_new_protected:Npn \alec_cycle:nn #1 #2
 {
  \seq_set_split:Nnn \l_alec_cycle_seq { , } { #2 }
  \seq_use:Nn \l_alec_cycle_seq { #1 }
 }
\ExplSyntaxOff
\usepackage{titling}
\newcommand{\subtitle}[1]{%
  \posttitle{%
    \par\end{center}
    \begin{center}\large#1\end{center}
    \vskip0.5em}%
}
\usepackage{arydshln}
\setcounter{tocdepth}{2}
\setlength{\dashlinedash}{1.2pt}
\setlength{\dashlinegap}{1.5pt}
\setlength{\arrayrulewidth}{0.2pt}
\title{proof-m11}
\author{haiml76 }
\date{May 2024}

\begin{document}
\section{The computation of $G_n(\mathbb{Q}_p)$}
\subsection{The computation of the first block $M_{11}$}
\begin{proposition}
\label{prop.centralizer.dimension}
Let $x=\sum_{i=1}^{n-1}\lambda_i e_{i,i+1}$, where $\lambda_i\in\mathbb{Q}_p$ are not all zero. Then $\dim\sfrac{\mathcal{C}_{\sfrac{\gamma_1}{\gamma_3}}(x)}{\gamma_2}=\mathfrak{l}(x)+\mathfrak{m}(x)$, where $\mathfrak{l}(x)$ is the number of sequences of consecutive non-zero coefficients of the form $\lambda_j,\lambda_{j+1},\dots,\lambda_{j+k-1},\lambda_{j+k}$ and $\lambda_{j-1}=\lambda_{j+k+1}=0$ (that is, the sequences are separated by one of more zero coefficients)\footnote{We extend our notation of indices, to include also the case where $j=1$ or $j+k=n-1$, and define that $\lambda_{j-1}=\lambda_0=0$ or $\lambda_{j+k+1}=\lambda_n=0$, respectively}, and $\mathfrak{m}(x)$ is the number of zero coefficients $\lambda_j=0$, such that also $\lambda_{j-1}=\lambda_{j+1}=0$.
\end{proposition}
\begin{proof}
Let $y=\sum_{i=1}^{n-1}\mu_i e_{i,i+1}$, where $\lambda_i\in\mathbb{Q}_p$, be an element in the quotient $\sfrac{\gamma_1}{\gamma_2}$.
For every $1\leq i\leq n-1$, denote by $(\mathfrak{C}_i)$ the constraint equation $[\lambda_i e_{i,i+1},\mu_{i+1}e_{i+1,i+2}]-[\lambda_{i+1}e_{i+1,i+2},\mu_i e_{i,i+1}]=(\lambda_i\mu_{i+1}-\lambda_{i+1}\mu_i)e_{i,i+2}=0$, and it is clear that $y\in\sfrac{\mathcal{C}_{\sfrac{\gamma_1}{\gamma_3}}(x)}{\gamma_2}$ if and only if all the $(\mathfrak{C}_i)$ constraints are satisfied. We observe that each $\mu_i$ participates in two constraints, $(\mathfrak{C}_{i-1})$ and $(\mathfrak{C}_i)$, that is, $\lambda_{i-1}\mu_i-\lambda_i\mu_{i-1}=\lambda_i\mu_{i+1}-\lambda_{i+1}\mu_i=0$. If $\lambda_i=0$, then $\lambda_i\mu_{i-1}=\lambda_i\mu_{i+1}=0$, hence by constraint $(\mathfrak{C}_{i-1})$ we have that $\lambda_{i-1}\mu_i=0$, and by constraint $(\mathfrak{C}_i)$ we have that $\lambda_{i+1}\mu_i=0$. Hence, if either $\lambda_{i-1}$ or $\lambda_{i+1}$ are non-zero, then $\mu_i=0$. But if $\lambda_{i-1}=\lambda_{i+1}=0$, then both constraints are satisfied for any choice of $\mu_i$, which increases $\dim\sfrac{\mathcal{C}_{\sfrac{\gamma_1}{\gamma_3}}(x)}{\gamma_2}$ by $1$. We need to prove that for any sequence of $k$ consecutive zero coefficients of $x$, where $k\geq 3$, that is\footnote{Here again, we consider the non-existent $\lambda_0=\lambda_n=0$ as part of the sequence} $\lambda_{j+1}=\lambda_{j+2}=\cdots=\lambda_{j+k}=0$, for $1\leq j\leq n-2$, we have that the sequence $\mu_{j+2},\mu_{j+3},\dots,\mu_{j+k-1}$ of $k-2$ consecutive coefficients of $y$ is made of scalars of any choice, thus $\dim\sfrac{\mathcal{C}_{\sfrac{\gamma_1}{\gamma_3}}(x)}{\gamma_2}$ is increased by $k-2$. We prove that by simple induction on $k$. For $k=3$, we just proved that if $\lambda_{i-1}=\lambda_i=\lambda_{i+1}=0$, then $\mu_i$ can be any scalar. For $k+1$, we look at the sequence of $k+1$ zero coefficients, $\lambda_{j+1}=\lambda_{j+2}=\cdots=\lambda_{j+k}=\lambda_{j+k+1}=0$. By constraints $(\mathfrak{C}_{j+k-1})$ and $(\mathfrak{C}_{j+k})$, we have that $\lambda_{j+k-1}\mu_{j+k}-\lambda_{j+k}\mu_{j+k-1}=\lambda_{j+k}\mu_{j+k+1}-\lambda_{j+k+1}\mu_{j+k}=0$, and since $\lambda_{j+k-1}=\lambda_{j+k}=\lambda_{j+k+1}=0$, we have that $\mu_{j+k}$ can be any scalar, as we proved earlier. By the assumption, we have that all $k-2$ previous coefficients, that is $\mu_{j+2},\dots,\mu_{j+k-1}$ can also be any scalars, so in total the whole sequence of $k-1=(k+1)-2$ coefficients of $y$ can be any scalars, which proves the induction step. Suppose that we have $m$ sequences of three or more consecutive zero coefficients in $x$, whose lengths are $k_1,k_2,\dots,k_m$, then $\mathfrak{m}(x)=\sum_{l=1}^m k_l-2m$ is the total number of zero coefficients $\lambda_j=0$ such that also $\lambda_{j-1}=\lambda_{j+1}=0$, as proposed. Using again the two consecutive constraints, $(\mathfrak{C}_{i-1})$ and $(\mathfrak{C}_i)$,
suppose now that $\lambda_i\neq 0$. If $\lambda_{i-1}=0$, then by constraint $(\mathfrak{C}_{i-1})$ we must have that $\mu_{i-1}=0$, but if $\lambda_{i-1}\neq 0$, then by this constraint we have $\mu_i=\frac{\lambda_i\mu_{i-1}}{\lambda_{i-1}}$, which means that $\mu_i$ depends on $\mu_{i-1}$. Precisely the same way for constraint $(\mathfrak{C}_i)$, we have that if $\lambda_{i+1}=0$ then $\mu_{i+1}=0$, otherwise $\mu_{i+1}=\frac{\lambda_{i+1}\mu_i}{\lambda_i}$, which means that $\mu_{i+1}$ depends on $\mu_i$, and if also $\lambda_{i-1}\neq 0$, then $\mu_{i+1}=\frac{\lambda_{i+1}\frac{\lambda_i\mu_{i-1}}{\lambda_{i-1}}}{\lambda_i}=\frac{\lambda_{i+1}\mu_{i-1}}{\lambda_{i-1}}$, which means that both $\mu_{i+1}$ and $\mu_i$ depend on $\mu_{i-1}$. We need to prove this is true for any sequence of $k$ consecutive non-zero coefficients of $x$, that is, for a given sequence of coefficients, $\lambda_{j+1},\lambda_{j+2},\dots,\lambda_{j+k}$, where $1\leq j\leq n-1$, we need to prove that $\mu_{j+1}$ can be any scalar, while $\mu_{j+2},\mu_{j+3},\dots,\mu_{j+k}$ all depend on $\mu_{j+1}$. Here again, we use a simple induction on $k$. For $k=1,2,3$, we already proved this. For $k+1$, we look into a sequence of $k+1$ consecutive non-zero coefficients of $x$, that is $\lambda_{j+1},\dots,\lambda_{j+k},\lambda_{j+k+1}$. By constraint $(\mathfrak{C}_{j+k})$ we have $\lambda_{j+k}\mu_{j+k+1}-\lambda_{j+k+1}\mu_{j+k}=0$, and since $\lambda_{j+k}\neq 0$, we have $\mu_{j+k+1}=\frac{\lambda_{j+k+1}\mu_{j+k}}{\lambda_{j+k}}$, but by the assumption, $\mu_{j+k}=\frac{\lambda_{j+k}\mu_{j+k-1}}{\lambda_{j+k-1}}=\frac{\lambda_{j+k}}{\lambda_{j+k-1}}\frac{\lambda_{j+k-1}\mu_{j+k-2}}{\lambda_{j+k-2}}=\frac{\lambda_{j+k}}{\lambda_{j+k-1}}\frac{\lambda_{j+k-1}}{\lambda_{j+k-2}}\frac{\lambda_{j+k-2}\mu_{j+k-3}}{\lambda_{j+k-3}}=\cdots=\frac{\lambda_{j+k}\mu_{j+1}}{\lambda_{j+1}}$, hence $\mu_{j+k+1}=\frac{\lambda_{j+k+1}\mu_{j+k}}{\lambda_{j+k}}=\frac{\lambda_{j+k+1}}{\lambda_{j+k}}\frac{\lambda_{j+k}\mu_{j+1}}{\lambda_{j+1}}=\frac{\lambda_{j+k+1}\mu_{j+1}}{\lambda_{j+1}}$, which proves the induction step. Looking again at $(\mathfrak{C}_{i-1})$ and $(\mathfrak{C}_i)$, we observe that if $\lambda_{i-1}$ and $\lambda_{i+1}$ are non-zero and $\lambda_i=0$, then $\lambda_i\mu_{i-1}=\lambda_i\mu_{i+1}=0$, hence by constraint $(\mathfrak{C}_{i-1})$ we have that $\lambda_{i-1}\mu_i=0$, and by constraint $(\mathfrak{C}_i)$ we have that $\lambda_{i+1}\mu_i=0$, so $\mu_i=0$ by both constraints, which means that there is no dependency between $\mu_{i+1}$ and $\mu_{i-1}$. This shows that any zero coefficient between two non-zero coefficients of $x$ creates two separate sequences, each sequences increases $\dim\sfrac{\mathcal{C}_{\sfrac{\gamma_1}{\gamma_3}}(x)}{\gamma_2}$ by $1$, hence, if we have $l$ sequences of consecutive non-zero coefficients of $x$, they increase $\dim\sfrac{\mathcal{C}_{\sfrac{\gamma_1}{\gamma_3}}(x)}{\gamma_2}$ by $l$, and we denote $\mathfrak{l}(x)=l$. All this shows that $\dim\sfrac{\mathcal{C}_{\sfrac{\gamma_1}{\gamma_3}}(x)}{\gamma_2}=\mathfrak{l}(x)+\mathfrak{m}(x)$, as proposed.
\end{proof}
\begin{corollary}
\label{prop.n.geq.5.centralizer.codimension}
Let $\mathcal{L}_{n,p}$ be the $\mathbb{Q}_p$-Lie algebra associated with $\mathcal{U}_n(\mathbb{Z})$, where $n\geq 5$, then $\dim\mathcal{C}_{\sfrac{\gamma_1}{\gamma_3}}(x)=\dim\sfrac{\gamma_1}{\gamma_3}-1$ if and only if $x\in\{\lambda e_{12}+\gamma_2\mathcal{L}_{n,p}\}$ or $x\in\{\lambda e_{n-1,n}+\gamma_2\mathcal{L}_{n,p}\}$, for a non-zero scalar $\lambda\in\mathbb{Q}_p$.
\end{corollary}
\begin{proof}
We recall first that for any algebra $\mathcal{L}_{n,p}$, the first two elements of the lower central series, namely $\sfrac{\gamma_1}{\gamma_2}$ and $\sfrac{\gamma_2}{\gamma_3}$, are of sizes $n-1$ and $n-2$, respectively. Consider the Lie brackets operation $[x_1,x_2]$, where $x_1\in\sfrac{\gamma_1}{\gamma_2}$ and $x_2\in\sfrac{\gamma_2}{\gamma_3}$, or the opposite, then $[x_1,x_2]$ can be either zero, or $[x_1,x_2]\in\gamma_3$, which means that if we consider the centralizer of some $x\in\mathcal{L}_{n,p}$ in $\sfrac{\gamma_1}{\gamma_3}$, then any element $y\in\sfrac{\gamma_2}{\gamma_3}$ would either commute with $x$ or yield an element in $\gamma_3$, which also means that it commutes with $x$ in $\sfrac{\gamma_1}{\gamma_3}$. This shows that $\dim\mathcal{C}_{\sfrac{\gamma_1}{\gamma_3}}(x)\geq\dim\sfrac{\gamma_2}{\gamma_3}=n-2$. Suppose that $x=\lambda_1 e_{12}+z$ or $x=\lambda_{n-1}e_{n-1,n}+z$, where $z\in\gamma_2\mathcal{L}_{n,p}$, then looking only at the elements of $\sfrac{\gamma_1}{\gamma_2}$ in the linear combination that forms $x$, we have one sequence of one non-zero coefficient, hence $\mathfrak{l}(x)=1$, and one sequence of $n-1-1=n-2$ zero coefficients of $x$, hence we have\footnote{Again, we consider also $\lambda_0=0$ and $\lambda_n=0$, hence the sequence of zero coefficients is either $\lambda_2=\lambda_3=\cdots=\lambda_{n-2}=\lambda_{n-1}=\lambda_n=0$ if $\lambda_1\neq 0$, or $\lambda_0=\lambda_1=\lambda_2=\cdots=\lambda_{n-3}=\lambda_{n-2}=0$, if $\lambda_{n-1}\neq 0$, which means that the length of the entire sequence of zero coefficients is not $n-2$ but $n-1$, therefore by \ref{prop.centralizer.dimension} we have that $\mathfrak{m}(x)=n-1-2=n-3$} that $\mathfrak{m}(x)=n-2-1=n-3$. By \ref{prop.centralizer.dimension} we have that $\dim\sfrac{\mathcal{C}_{\sfrac{\gamma_1}{\gamma_3}}(x)}{\gamma_2}=\mathfrak{l}(x)+\mathfrak{m}(x)=1+(n-3)=n-2$. Adding all the elements of $\sfrac{\gamma_2}{\gamma_3}$, we have that $\dim\mathcal{C}_{\sfrac{\gamma_1}{\gamma_3}}(x)=(n-2)+(n-2)=2n-4=(n-1)+(n-2)-1=\dim\sfrac{\gamma_1}{\gamma_2}+\dim\sfrac{\gamma_2}{\gamma_3}-1=\dim\sfrac{\gamma_1}{\gamma_3}-1$, as proposed. To prove the opposite direction, we assume that $\dim\mathcal{C}_{\sfrac{\gamma_1}{\gamma_3}}(x)=\dim\sfrac{\gamma_1}{\gamma_3}-1=2n-4$, and since we already know that all the elements of $\dim\sfrac{\gamma_2}{\gamma_3}$ are in the centralizer of $x$, we have that $\dim\sfrac{\mathcal{C}_{\sfrac{\gamma_1}{\gamma_3}}(x)}{\gamma_2}=\dim\mathcal{C}_{\sfrac{\gamma_1}{\gamma_3}}(x)-\dim\sfrac{\gamma_2}{\gamma_3}=(2n-4)-(n-2)=n-2=(n-1)-1=\dim\sfrac{\gamma_1}{\gamma_2}-1$. Suppose that either $x=\lambda_i e_{i,i+1}+z$ where $1<i<n-1$, or $x=\lambda_{i_1}e_{i_1,i_1+1}+\cdots+\lambda_{i_m}e_{i_m,i_m+1}$ where we denote the number of $\sfrac{\gamma_1}{\gamma_2}$ coefficients\footnote{In words, $x$ has at least two non-zero coefficients in $\sfrac{\gamma_1}{\gamma_2}$} by $2\leq m\leq n-1$, in both cases $z\in\gamma_2\mathcal{L}_{n,p}$. In the first case, we have that $\mathfrak{l}(x)=1$, because there is only one non-zero coefficient of $x$ in $\sfrac{\gamma_1}{\gamma_2}$, and we have two sequences of zero coefficients of $x$, that is $\lambda_0=\lambda_1=\lambda_2=\cdots=\lambda_{i-2}=\lambda_{i-1}=0$ and $\lambda_{i+1}=\lambda_{i+2}=\lambda_{i+3}=\cdots=\lambda_{n-2}=\lambda_{n-1}=\lambda_n=0$, which means we have a sequence of zeros of length $(i-1)+1=i$ and a sequence of length $(n-1)-i+1=n-i$, hence by \ref{prop.centralizer.dimension} $\mathfrak{}$ we have\footnote{Here, and in all the following options, $\mathfrak{m}(x)$ has an upper bound which depend on the number of consecutive zero coefficients of $x$, because if any of these sequences is of length $k$, where $k<3$, then by \ref{prop.centralizer.dimension} we have that $\mathfrak{m}(x)=k-2\leq 0$, which means that this particular sequence does not increase $\dim\sfrac{\mathcal{C}_{\sfrac{\gamma_1}{\gamma_3}}(x)}{\gamma_2}$} that $\mathfrak{m}(x)\leq(i-2)+(n-i-2)=n-4$, which means that $\dim\sfrac{\mathcal{C}_{\sfrac{\gamma_1}{\gamma_3}}(x)}{\gamma_2}=\mathfrak{l}(x)+\mathfrak{m}(x)\leq 1+(n-4)=n-3=(n-1)-2=\dim\sfrac{\gamma_1}{\gamma_2}-2$. Adding all the elements of $\sfrac{\gamma_2}{\gamma_3}$, we have that $\dim\mathcal{C}_{\sfrac{\gamma_1}{\gamma_3}}(x)=\dim\sfrac{\mathcal{C}_{\sfrac{\gamma_1}{\gamma_3}}(x)}{\gamma_2}+\dim\sfrac{\gamma_2}{\gamma_3}\leq(n-3)+(n-2)=2n-5<2n-4$, which contradicts the assumption. Suppose that $x$ is of the second form, then $x$ may have one sequence of $k$ non-zero coefficients, where $k>1$, or $x$ may have two or more sequences of one or more non-zero coefficients. In the case that $x$ has one sequence of $k$ non-zero coefficients, $\mathfrak{l}(x)=1$, and the number of consecutive zero coefficients is either less or equal to $(n-1)-k+1=n-k$, which results in $\mathfrak{m}(x)\leq n-k-2$, if the sequence of non-zero coefficients contains\footnote{That is, $x=\lambda_1 e_{12}+\lambda_2 e_{23}+\cdots+\lambda_{k,k+1}e_{k,k+1}$ or $x=\lambda_{n-k}e_{n-k,n-k+1}+\lambda_{n-k+1}e_{n-k+1,n-k+2}+\cdots+\lambda_{n-1}e_{n-1,n}$} either $\lambda_1$ or $\lambda_{n-1}$, or the number of consecutive zero coefficients is divided into two separate sequences with a total number of consecutive zeros which is less or equal to $(n-1)-k+2=n-k+1$, which results in $\mathfrak{m}(x)\leq n-k+1-4=n-k-3$, in the case that the sequence of non-zero coefficients of $x$ does not contains either $\lambda_1$ or $\lambda_{n-1}$, which means that $\mathfrak{l}(x)+\mathfrak{m}(x)\leq 1+(n-k-2)=n-k-1$, and since $k>1$, we have that $\dim\sfrac{\mathcal{C}_{\sfrac{\gamma_1}{\gamma_3}}(x)}{\gamma_2}=\mathfrak{l}(x)+\mathfrak{m}(x)\leq n-2-1=n-3=(n-1)-2$, which again contradicts the assumption. Suppose that $x$ has $l$ sequences of non-zero coefficients in $\sfrac{\gamma_1}{\gamma_2}$, of lengths $k_1,k_2,\dots,k_l$, then $\mathfrak{l}(x)=l$, and the number of zeros is $n-1-\sum_{j=1}^l k_j+2=n+1-\sum_{j=1}^l k_j$ in $l+1$ sequences\footnote{Including $\lambda_0=0$ and $\lambda_n=0$}, considering the two options from earlier, whether any of the sequences of non-zero coefficients contains $\lambda_1$ or $\lambda_{n-1}$, or does not contain either of them. We observe that $n+1-\sum_{j=1}^l k_j$ is bounded by $n+1-l$, which is the case where $k_1=k_2=\cdots=k_l=1$. Suppose that $l=2$, and that each of the $3$ sequences of zero coefficients is of length greater or equal to $3$, then $\mathfrak{m}(x)\leq n+1-2-3\cdot 2=n-7$, hence $\dim\sfrac{\mathcal{C}_{\sfrac{\gamma_1}{\gamma_3}}(x)}{\gamma_2}=\mathfrak{l}(x)+\mathfrak{m}(x)\leq 2+n-7=n-5$, hence $\dim\mathcal{C}_{\sfrac{\gamma_1}{\gamma_3}}(x)=\dim\sfrac{\mathcal{C}_{\sfrac{\gamma_1}{\gamma_3}}(x)}{\gamma_2}+\dim\sfrac{\gamma_2}{\gamma_3}\leq (n-5)+(n-2)=2n-7<2n-2=\dim\sfrac{\gamma_1}{\gamma_3}-1$. We prove this is true for any number of sequences of non-zero coefficients of $x$ simple induction on $l$. For $l=2$ we already proved that, for $l+1$ we take the upper bound case, which means that we add a new non-zero coefficient which splits a sequence of zero coefficients such that each of the two parts of the sequence that we split has at least $3$ zero coefficients. Denote by $\mathfrak{l}_l(x)=l$ and $\mathfrak{m}_l(x)$ for $l$, and denote by $\mathfrak{l}_{l+1}(x)=l+1$ and $\mathfrak{m}_{l+1}(x)$ for $l+1$, then obviously $\mathfrak{l}_{l+1}+\mathfrak{m}_{l+1}\leq\mathfrak{l}_l+\mathfrak{m}_l$, because $\mathfrak{l}_{l+1}-\mathfrak{l}_l=1$ and $\mathfrak{m}_{l}-\mathfrak{m}_{l+1}\geq 1$, because the newly added non-zero coefficient does not only remove one zero coefficient from the total count, but may also remove one or two of its neighboring zero coefficients, since they are no longer surrounded by another zero coefficient, which proves the induction step. All this proves that if $\dim\mathcal{C}_{\sfrac{\gamma_1}{\gamma_3}}(x)=\dim\sfrac{\gamma_1}{\gamma_3}-1$, then the only option for $x$ is either $x=\lambda_1 e_{12}+z$ or $x=\lambda_{n-1}e_{n-1,n}+z$, where $z\in\gamma_2\mathcal{L}_{n,p}$, as proposed.
\end{proof}
\begin{corollary}
\label{prop.n.geq.5.centralizer.codimension}
Let $\mathcal{L}_{n,p}$ be the $\mathbb{Q}_p$-Lie algebra associated with $\mathcal{U}_n(\mathbb{Z})$, where $n=4$, then $\dim\mathcal{C}_{\sfrac{\gamma_1}{\gamma_3}}(x)=\dim\sfrac{\gamma_1}{\gamma_3}-1$ if and only if $x\in\{\lambda_1 e_{12}+\lambda_3 e_{34}+\gamma_2\mathcal{L}_{4,p}\}$, for $\lambda_1,\lambda_3\in\mathbb{Q}_p$ not both zero.
\end{corollary}
\begin{proof}
Suppose that $x=\lambda e_{12}+\mu e_{34}+z$, where $x\in\gamma_2\mathcal{L}_{4,p}$, then by $\ref{prop.centralizer.dimension}$ if $\lambda\neq 0$ and $\mu\neq 0$, we have that $\mathfrak{l}(x)=2$, and $\mathfrak{m}(x)=0$ because there are three isolated zero coefficients, $\lambda_0=\lambda_2=\lambda_4=0$, hence $\dim\sfrac{\mathcal{C}_{\sfrac{\gamma_1}{\gamma_3}}(x)}{\gamma_2}=\mathfrak{l}(x)+\mathfrak{m}(x)=2$, which means that $\dim\mathfrak{C}_{\sfrac{\gamma_1}{\gamma_3}}(x)=\dim\sfrac{\mathcal{C}_{\sfrac{\gamma_1}{\gamma_3}}(x)}{\gamma_2}+\dim\sfrac{\gamma_2}{\gamma_3}=2+(4-2)=4=(4-1)+(4-2)-1=\dim\sfrac{\gamma_1}{\gamma_2}+\dim\sfrac{\gamma_2}{\gamma_3}-1=\dim\sfrac{\gamma_1}{\gamma_3}-1$. If $\lambda_1=0$ or $\lambda_3=0$, that is $x=\lambda_1 e_{12}+z$ or $x=\lambda_{}$, where $z\in\gamma_2\mathcal{L}_{4,p}$, then $\mathfrak{l}(x)=1$ and we have a sequence of $3$ consecutive zero coefficients, either $\lambda_2=\lambda_3=\lambda_4=0$ or $\lambda_0=\lambda_1\lambda_2=0$, which means that $\mathfrak{l}(x)=3-2=1$, hence $\mathfrak{l}(x)+\mathfrak{m}(x)=1+1=2$, hence $\dim\mathcal{C}_{\sfrac{\gamma_1}{\gamma_3}}(x)=\dim\sfrac{\gamma_1}{\gamma_3}-1$, as proposed. To prove the opposite, we review the different options. Suppose that $x\in\gamma_2\mathcal{L}_{4,p}$, which means that $\lambda_0=\lambda_1=\lambda_2=\lambda_3=\lambda_4=0$, hence $\mathfrak{l}(x)=0$ and $\mathfrak{m}(x)=5-2=3$, so $\dim\sfrac{\mathcal{C}_{\sfrac{\gamma_1}{\gamma_3}}(x)}{\gamma_2}=\mathfrak{l}(x)+\mathfrak{m}(x)=0+3$, and therefore $\dim\mathcal{C}_{\sfrac{\gamma_1}{\gamma_3}}(x)=\dim\sfrac{\mathcal{C}_{\sfrac{\gamma_1}{\gamma_3}}(x)}{\gamma_2}+\dim\sfrac{\gamma_2}{\gamma_3}=3+(4-2)=5>\dim\sfrac{\gamma_1}{\gamma_3}-1$. The other option is that $x=\lambda_1 e_{12}+\lambda_2 e_{23}+\lambda_3 e_{34}+z$, where $\lambda_2\neq 0$, where $\lambda_1$ and $\lambda_3$ may be either zero or non-zero, and where $z\in\gamma_2\mathcal{L}_{4,p}$. Since $\lambda_2\neq 0$ there is no sequence of zero coefficients of $x$ which is greater than $2$, and also, we observe that whether $\lambda_1\neq 0$ or $\lambda_3\neq 0$ or both are non-zero, they are part of the sequence of non-zero coefficients which has $\lambda_2$, hence $\mathfrak{l}(x)=1$ and $\mathfrak{m}(x)=0$, so $\dim\mathcal{C}_{\sfrac{\gamma_1}{\gamma_3}}(x)=\dim\sfrac{\mathcal{C}_{\sfrac{\gamma_1}{\gamma_3}}(x)}{\gamma_2}+\dim\sfrac{\gamma_2}{\gamma_3}=\mathfrak{l}(x)+\mathfrak{m}+2=1+2=3<4=\dim\sfrac{\gamma_1}{\gamma_3}-1$, which means that if $x$ is not as proposed, then $\dim\mathcal{C}_{\sfrac{\gamma_1}{\gamma_3}}(x)\neq\dim\sfrac{\gamma_1}{\gamma_3}-1$.
\end{proof}
\begin{corollary}
Let $\varphi\in G_n(\mathbb{Q}_p)$, where $n\geq 5$, then the first block of $M$, the coefficient matrix of $\varphi$, is either diagonal or anti-diagonal.
\end{corollary}
\begin{proof}
We first look at $\varphi(e_{12})$. We observe\footnote{Since we review the elements in the centralizer of $e_{12}$ in the quotient $\sfrac{\gamma_1}{\gamma_3}$, we have that $[e_{12},e_{23}]=e_{13}\in\sfrac{\gamma_1}{\gamma_3}$ hence $e_{23}$ does not commute with $e_{12}$ in this quotient, but $[e_{12},e_{24}]=e_{14}\in\gamma_3\mathcal{L}_{n,p}$ which is zero in the quotient, so $e_{24}$ is commuting with $e_{12}$, and so it is an element of the centralizer} that $\mathcal{C}_{\sfrac{\gamma_1}{\gamma_3}}(e_{12})=\{e_{12},e_{34},e_{45},\dots,e_{n-1,n},e_{13},e_{23},e_{34},\dots,e_{n-2,n-1},e_{n-2,n}\}$, hence $\dim\mathcal{C}_{\sfrac{\gamma_1}{\gamma_3}}(e_12)=(\dim\sfrac{\gamma_1}{\gamma_2}-1)+\dim\sfrac{\gamma_2}{\gamma_3}=(n-1-1)+(n-2)=n-4=\dim\sfrac{\gamma_1}{\gamma_3}-1$, which means that $\dim\mathcal{C}_{\sfrac{\gamma_1}{\gamma_3}}(\varphi(e_{12}))=\dim\sfrac{\gamma_1}{\gamma_3}-1$ as well. Hence we can apply \ref{prop.centralizer.dimension}, and conclude that either  $\varphi(e_{12})=\lambda_1 e_{12}+z$ or $\varphi(e_12)=\lambda_{n-1}e_{n-1,n}$, where $z\in\gamma_2\mathcal{L}_{n,p}$.
\end{proof}
\end{document}