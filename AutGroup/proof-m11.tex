\documentclass[12pt]{article}
\usepackage{graphicx} % Required for inserting images
\makeatletter
\newcommand*{\rom}[1]{\expandafter\@slowromancap\romannumeral #1@}
\makeatother
\usepackage{amsfonts, amssymb}
\usepackage{mathrsfs, mathdots} 
\usepackage{amsmath}
\usepackage{float}
\usepackage{amsthm}
\usepackage{tikz-cd}
\usepackage{xcolor}
\usepackage{xparse}
\usepackage{setspace}
\usepackage{xfrac}
\usepackage{yfonts}

\newtheorem{theorem}{Theorem}[subsection]
\newtheorem{proposition}[theorem]{Proposition}
\newtheorem{corollary}[theorem]{Corollary}
\newtheorem{lemma}[theorem]{Lemma}
\newtheorem{notations}[theorem]{Notations}
\newtheorem{definition}[theorem]{Definition}
\newtheorem{example}[theorem]{Example}

\ExplSyntaxOn
\NewDocumentCommand{\cycle}{ O{\;} m }
 {
  (
  \alec_cycle:nn { #1 } { #2 }
  )
 }

\seq_new:N \l_alec_cycle_seq
\cs_new_protected:Npn \alec_cycle:nn #1 #2
 {
  \seq_set_split:Nnn \l_alec_cycle_seq { , } { #2 }
  \seq_use:Nn \l_alec_cycle_seq { #1 }
 }
\ExplSyntaxOff
\usepackage{titling}
\newcommand{\subtitle}[1]{%
  \posttitle{%
    \par\end{center}
    \begin{center}\large#1\end{center}
    \vskip0.5em}%
}
\usepackage{arydshln}
\setcounter{tocdepth}{2}
\setlength{\dashlinedash}{1.2pt}
\setlength{\dashlinegap}{1.5pt}
\setlength{\arrayrulewidth}{0.2pt}
\title{proof-m11}
\author{haiml76 }
\date{May 2024}

\begin{document}
\section{The computation of $G_n(\mathbb{Q}_p)$}
\subsection{The computation of the first block $M_{11}$}
\begin{proposition}
\label{prop.centralizer.dimension}
Let $x=\sum_{i=1}^{n-1}\lambda_i e_{i,i+1}$, where $\lambda_i\in\mathbb{Q}_p$ are not all zero. Then $\dim\mathcal{C}_{\sfrac{\gamma_1}{\gamma_3}}(x)=\mathfrak{l}(x)+\mathfrak{m}(x)$, where $\mathfrak{l}(x)$ is the number of sequences of consecutive non-zero coefficients of the form $\lambda_j,\lambda_{j+1},\dots,\lambda_{j+k-1},\lambda_{j+k}$ and $\lambda_{j-1}=\lambda_{j+k+1}=0$ (that is, the sequences are separated by one of more zero coefficients)\footnote{We extend our notation of indices, to include also the case where $j=1$ or $j+k=n-1$, and define that $\lambda_{j-1}=\lambda_0=0$ or $\lambda_{j+k+1}=\lambda_n=0$, respectively}, and $\mathfrak{m}(x)$ is the number of zero coefficients $\lambda_j=0$, such that also $\lambda_{j-1}=\lambda_{j+1}=0$.
\end{proposition}
\begin{proof}
Let $y=\sum_{i=1}^{n-1}\mu_i e_{i,i+1}$, where $\lambda_i\in\mathbb{Q}_p$.
For every $1\leq i\leq n-1$, denote by $(\mathfrak{C}_i)$ the constraint equation $[\lambda_i e_{i,i+1},\mu_{i+1}e_{i+1,i+2}]-[\lambda_{i+1}e_{i+1,i+2},\mu_i e_{i,i+1}]=(\lambda_i\mu_{i+1}-\lambda_{i+1}\mu_i)e_{i,i+2}=0$, and it is clear that $y\in\mathcal{C}_{\sfrac{\gamma_1}{\gamma_3}}(x)$ if and only if all the $(\mathfrak{C}_i)$ constraints are satisfied. We can see that each $\mu_i$ participates in two constraints, $(\mathfrak{C}_{i-1})$ and $(\mathfrak{C}_i)$, that is, $\lambda_{i-1}\mu_i-\lambda_i\mu_{i-1}=\lambda_i\mu_{i+1}-\lambda_{i+1}\mu_i=0$. We have several options. If $\lambda_i=0$, then $\lambda_i\mu_{i-1}=\lambda_i\mu_{i+1}=0$, hence by constraint $(\mathfrak{C}_{i-1})$ we have that $\lambda_{i-1}\mu_i=0$, and by constraint $(\mathfrak{C}_i)$ we have that $\lambda_{i+1}\mu_i=0$. Hence, if either $\lambda_{i-1}$ or $\lambda_{i+1}$ are non-zero, then $\mu_i=0$. But if $\lambda_{i-1}=\lambda_{i+1}=0$, then both constraints are satisfied for any choice of $\mu_i$, which increases $\dim\mathcal{C}_{\sfrac{\gamma_1}{\gamma_3}}(x)$ by $1$. By simple induction, we prove that for any sequence of $k$ consecutive zero $\lambda$ coefficients of $x$, namely $\lambda_{j+1}=\lambda_{j+2}=\cdots=\lambda_{j+k}=0$, for $1\leq j\leq n-1$, we have that the sequence $\mu_{j+2},\mu_{j+3},\dots,\mu_{j+k-1}$ of $\mu$ coefficients of $y$ is a sequence of any scalars, thus $\dim\mathcal{C}_{\sfrac{\gamma_1}{\gamma_3}}(x)$ is increased by $k-2$. For $k=3$, we just proved that if $\lambda_{i-1}=\lambda_i=\lambda_{i+1}=0$, then $\mu_i$ can be any scalar. For $k+1$, we look at the sequence of $k+1$ zero coefficients, $\lambda_{j+1}=\lambda_{j+2}=\cdots=\lambda_{j+k}=\lambda_{j+k+1}=0$. By constraints $(\mathfrak{C}_{j+k-1})$ and $(\mathfrak{C}_{j+k})$, we have that $\lambda_{j+k-1}\mu_{j+k}-\lambda_{j+k}\mu_{j+k-1}=\lambda_{j+k}\mu_{j+k+1}-\lambda_{j+k+1}\mu_{j+k}=0$, and since $\lambda_{j+k-1}=\lambda_{j+k}=\lambda_{j+k+1}=0$, we have that $\mu_{j+k}$ can be any scalar, as we proved earlier. By the assumption, we have that $\mu_{j+2},\dots,\mu_{j+k-1}$ can be any scalars, and by adding $\mu_{j+k}$ to this sequence, we prove the induction step. Let $1\leq j\leq n-1$ and $1\leq k\leq n-1-j$ be two indices, such that $\lambda_{j-1}=\lambda_{j+k+1}=0$, and $\lambda_j,\lambda_{j+1},\dots,\lambda_{j+k-1},\lambda_{j+k}$ are all non-zero, then by constraints $(\mathfrak{C}_j),(\mathfrak{C}_{j+1}),\dots,(\mathfrak{C}_{m-1})$, we have that $\mu_m=\frac{\lambda_m}{\lambda_{m-1}}\mu_{m-1}=\frac{\lambda_m}{\lambda_{m-1}}\frac{\lambda_{m-1}}{\lambda_{m-2}}\mu_{m-2}=\frac{\lambda_m}{\lambda_{m-2}}\mu_{m-2}=\cdots=\frac{\lambda_m}{\lambda_j}\mu_j$, for every $j+1\leq m\leq j+k-1$, which means that for any choice of the first coefficient in the sequence, namely $\mu_j$, all the next $\mu$ coefficients of the sequence, with indices from $j+1$ to $j+k$, depend on $\mu_j$.
\end{proof}
\begin{corollary}
\label{prop.n.geq.4.centralizer.codimension}
Let $\mathcal{L}_{n,p}$ be the $\mathbb{Q}_p$-Lie algebra associated with $\mathcal{U}_n(\mathbb{Z})$. If $n\geq 5$, then $\dim\mathcal{C}_{\sfrac{\gamma_1}{\gamma_3}}(x)=\dim\sfrac{\gamma_1}{\gamma_3}-1$ if and only if $x\in\{\lambda e_{12}+\gamma_2\mathcal{L}_{n,p}\}$ or $x\in\{\lambda e_{n-1,n}+\gamma_2\mathcal{L}_{n,p}\}$, for a non-zero scalar $\lambda\in\mathbb{Q}_p$. If $n=4$, then $\dim\mathcal{C}_{\sfrac{\gamma_1}{\gamma_3}}(x)=\dim\sfrac{\gamma_1}{\gamma_3}-1$ if and only if $x\in\{\lambda e_{12}+\mu e_{34}+\gamma_2\mathcal{L}_{n,p}\}$, for $\lambda,\mu\in\mathbb{Q}_p$ not both zero.
\end{corollary}
\begin{proof}
Let $z=\lambda_{j,j+2}e_{j,j+2}$, where $1\leq j\leq n-2$ and $\lambda_{j,j+2}\in\mathbb{Q}_p$, then for every $w\in\sfrac{\gamma_1}{\gamma_3}$, either $z$ commutes with $w$ or $[z,w]\in\gamma_3\mathcal{L}_{n,p}$, which means that $\lambda_{j,j+2}e_{j,j+2}\in\mathcal{C}_{\sfrac{\gamma_1}{\gamma_3}}$, for every $1\leq j\leq n-2$. Hence, $\sfrac{\gamma_2}{\gamma_3}=\langle e_{13},e_{24},\dots,e_{n-2,n}\rangle\subset\mathcal{C}_{\sfrac{\gamma_1}{\gamma_3}}(x)$. Therefore, we only need to discuss elements of the quotient $\sfrac{\gamma_1}{\gamma_2}$, for the purpose of this proof.
Suppose that $x=\lambda_1 e_{12}+z$, where $z\in\gamma_2 \mathcal{L}_{n,p}$, then we have one sequence of non-zero coefficients, namely $\lambda_1$, and we have $n-2$ zero coefficients $\lambda_2=\lambda_3=\cdots=\lambda_{n-1}=0$, from which $n-3$ are between two other zeros. Hence, by \ref{prop.centralizer.dimension}, we have that $\mathcal{C}_{\sfrac{\gamma_1}{\gamma_2}}(x)=1+(n-3)=n-2=(n-1)-1=\dim\sfrac{\gamma_1}{\gamma_2}-1$. Similarly, the same goes also for $x=\lambda_{n-1}e_{n-1,n}+z$.
Suppose that $\dim\mathcal{C}_{\sfrac{\gamma_1}{\gamma_2}}(x)=\dim\sfrac{\gamma_1}{\gamma_2}-1$, but $x=\sum_{i=1}^{n-1}\lambda_i e_{i,i+1}$, such that either of the following options is true:
\begin{enumerate}
  \item there is more than one sequence of consecutive non-zero coefficients in the linear combination that forms $x$.
  \item there is one sequence of consecutive non-zero coefficients, but at least one of those coefficients has index $2\leq j\leq n-2$, meaning it is not $\lambda_1$ nor $\lambda_{n-1}$.
\end{enumerate}
For the second option, we start by fixing one index $2\leq j\leq n-2$, and assume that $x=\lambda_j e_{j,j+1}$. The number of zero coefficients in $x$ is $n-1-1=n-2$, but $\lambda_j$ and the zeros in indices $j-1,j+1$ are neighboring, hence $m_1=n-2-2=n-4$, and then $\dim\mathcal{C}_{\sfrac{\gamma_1}{\gamma_2}}(x)=l_1+m_1=1+n-4=n-3<n-2=\dim\sfrac{\gamma_1}{\gamma_2}-1$.
We denote by $k$ the length of the sequence of consecutive non-zero parameters, and prove that for any $k>0$, where at least one non-zero coefficient $\lambda_j$ lies in $2\leq j\leq n-2$, $\dim\mathcal{C}_{\sfrac{\gamma_1}{\gamma_2}}(x)<n-2$, by simple induction on $k$. For $k=1$, we have just shown that. For $k>1$, there are $k-1$ additional zeros that are replaced by non-zero coefficients, where except for $\lambda_{j-1}$ and $\lambda_{j+1}$, all the other zeros were originally lying between two other zeros. If the original sequence was $\lambda_2 e_{23}$ or $\lambda_{n-2}e_{n-2,n-1}$, and the new sequence is $\lambda_1 e_{12},\lambda_2 e_{23}$ or $\lambda_{n-2}e_{n-2,n-1},\lambda_{n-1}e_{n-1,n}$, respectively, then $m_k=m_1$, but clearly, in any other case, $m_k<m_1$, while $l_k=l_1=1$ at any case. by the assumption, for the original sequence, $\dim\mathcal{C}_{\sfrac{\gamma_1}{\gamma_2}}(x)=l_1+m_1<n-2$, hence for the new sequence, $\dim\mathcal{C}_{\sfrac{\gamma_1}{\gamma_2}}(x)=l_k+m_k\leq l_1+m_1=n-3<n-2$. Now we check the first option, starting from the case where $x=\lambda_1 e_{12}+\lambda_{n-1}e_{n-1,n}$. In this case, $l_2=2$ and the number of zeros is $n-1-2=n-3$, but $\lambda_1$ and the zero in index $2$ are neighboring, and so are $\lambda_{n-1}$ and the zero in index $n-2$, hence $m_2=n-3-2=n-5$ zeros are lying between two other zeros, therefore $\dim\mathcal{C}_{\sfrac{\gamma_1}{\gamma_2}}(x)=l_2+m_2=n-5+2=n-3<n-2$. if we add another non-zero coefficient, then it must lie in some index $2\leq j\leq n-2$, for which we have already proved that $\dim\mathcal{C}_{\sfrac{\gamma_1}{\gamma_2}}(x)<n-2$, which completes the proof for $n\geq 5$. For $n=4$, we can check explicitly. Assume $x=\lambda e_{12}+\mu e_{34}$, denote an element in the centralizer of $x$ by $y=\rho e_{12}+\tau e_{23}+\nu e_{34}$, and we observe that $[x,y]=[\lambda e_{12},\tau e_{23}]+[\mu e_{34},\tau e_{23}]=\lambda\tau e_{13}-\tau\mu e_{24}=0$, hence $\tau=0$, while $\rho=*$ and $\nu=*$, so $\dim\mathcal{C}_{\sfrac{\gamma_1}{\gamma_2}}(x)=2=\dim\sfrac{\gamma_1}{\gamma_2}-1$, as requested, and it is readily seen that even if either $\lambda=0$ or $\mu=0$, but not both, then $\tau$ still has to be zero, in order to satisfy either $\tau\mu=0$ or $\lambda\tau=0$, respectively, and $\rho,\nu$ can still be anything, which means that in either case, where the coefficient of $e_{23}$ is zero but $x\neq 0$, we have $\dim\mathcal{C}_{\sfrac{\gamma_1}{\gamma_2}}(x)=2$. Assume $\dim\mathcal{C}_{\sfrac{\gamma_1}{\gamma_2}}(x)=\dim\sfrac{\gamma_1}{\gamma_2}-1=3-1=2$, then if $x$ is not of the suggested form, it means that $x=\lambda e_{12}+\sigma e_{23}+\mu e_{34}$, where $\sigma\neq 0$ and either $\lambda$ or $\mu$ or both can be zero.
If $x=\lambda e_{12}+\sigma e_{23}+\mu e_{34}$ and all coefficients are non-zero, then for every $y\in\mathcal{C}_{\sfrac{\gamma_1}{\gamma_2}}(x)$ denoted by $y=\rho e_{12}+\tau e_{23}+\nu e_{34}$, we have $[x,y]=[\lambda e_{12},\tau e_{23}]+[\sigma e_{23},\rho e_{12}]+[\sigma e_{23},\nu e_{34}]+[\mu e_{34},\tau e_{23}]=(\lambda\tau-\sigma\rho)e_{13}+(\sigma\nu-\mu\tau)e_{24}$, hence $\tau=\frac{\sigma}{\lambda}\rho$ and $\nu=\frac{\mu}{\sigma}\tau=\frac{\mu}{\sigma}\frac{\sigma}{\lambda}\rho=\frac{\mu}{\lambda}\rho$, but this means that $\dim\mathcal{C}_{\sfrac{\gamma_1}{\gamma_2}}(x)=1$, because both $\tau$ and $\nu$ depend on $\rho$. If either $\lambda$ or $\mu$ or both are zero, then either $\sigma\rho$ or $\sigma\mu$ or both are zero, which means that $\rho$ or $\nu$ or both are zero, since $\sigma\neq 0$, but this means that either $y=\tau e_{23}+\frac{\mu}{\sigma}\tau e_{34}$ or $y=\frac{\lambda}{\sigma}\tau e_{12}+\tau e_{23}$ or $y=\tau e_{23}$, respectively. Therefore, in either case, where $\sigma\neq 0$, we have $\dim\mathcal{C}_{\sfrac{\gamma_1}{\gamma_2}}(x)=1$, which completes the proof for $n=4$.
\end{proof}
\begin{corollary}
Let $\mathcal{L}_{n,p}$ be a $\mathbb{Q}_p$-Lie algebra, where $n\geq 4$, and let $\varphi\in G_n(\mathbb{Q}_p)$ be an $\mathcal{L}_{n,p}$-automorphism, then
$\varphi_{11}(e_{12})=\lambda_1 e_{12}$ and $\varphi_{11}(e_{n,n-1})=\lambda_{n-1}e_{n-1,n}$, or $\varphi_{11}(e_{12})=\lambda_{n-1} e_{n-1,n}$ and $\varphi_{11}(e_{n,n-1})=\lambda_1 e_{1,2}$.
\end{corollary}
\begin{proof}
We look at the centralizer of $e_{12}$ in the quotient $\sfrac{\gamma_1}{\gamma_3}$, namely $\mathcal{C}_{\sfrac{\gamma_1}{\gamma_3}}(e_{12})$. Clearly, for any $e_{i,i+2}\in\sfrac{\gamma_2}{\gamma_3}$, we have that $[e_{12},e_{i,i+2}]$ is either zero, or $i=2$ and then $[e_{12},e_{24}]=e_{14}\in\gamma_3\mathcal{L}_{n,p}$, which vanishes in the quotient $\sfrac{\gamma_1}{\gamma_3}$, which means that in either case it is zero in this quotient. Therefore, we look only at elements $e_{i,i+1}\in\sfrac{\gamma_1}{\gamma_2}$. It is readily seen that every element of the form $e_{i,i+1}$ where $i\neq 2$ commutes with $e_{12}$, hence $\mathcal{C}_{\sfrac{\gamma_1}{\gamma_2}}(e_{12})=\langle e_{12},e_{34},e_{45},\dots,e_{n-2,n-1},e_{n-1,n}\rangle$, so $\dim\mathcal{C}_{\sfrac{\gamma_1}{\gamma_2}}(e_{12})=\dim\sfrac{\gamma_1}{\gamma_2}-1$, but since $\varphi_{11}$ is an automorphism, it must preserve the dimension of the centralizer, meaning $\dim\mathcal{C}_{\sfrac{\gamma_1}{\gamma_2}}(\varphi_{11}(e_{12}))=\dim\mathcal{C}_{\sfrac{\gamma_1}{\gamma_2}}(e_{12})=\dim\sfrac{\gamma_1}{\gamma_2}-1$. But by corollary \ref{prop.n.geq.4.centralizer.codimension}, if $n\geq 5$, then $\varphi_{11}(e_{12})=\lambda e_{12}$ or $\varphi_{11}(e_{12})=\lambda e_{n-1,n}$, and it is readily seen that the same applies also for $\varphi_{11}(e_{n-1,n})$, and since $\varphi$ is injective, then clearly, if $\varphi_{11}(e_{12})=\lambda e_{12}$ then $\varphi_{11}(e_{n-1,n})=\lambda e_{n-1,n}$, and if $\varphi_{11}(e_{12})=\lambda e_{n-1,n}$ then $\varphi_{11}(e_{n-1,n})=\lambda e_{12}$. If $n=4$, then by the same corollary, $\varphi_{11}(e_{12})=\lambda e_{12}+\mu e_{34}$, where $\lambda$ and $\mu$ are not both zero, which means that the same proof does not hold. Therefore, we now look at the centralizer of $e_{12}$ in the algebra $\mathcal{L}_{4,p}$ itself. We denote by $\mathcal{C}_{\mathcal{L}_{4,p}}(e_{12})$ the centralizer of $e_{12}$ in the algebra, which is $\mathcal{C}_{\mathcal{L}_{4,p}}(e_{12})=\langle e_{12},e_{34},e_{13},e_{14}\rangle$, so $\dim\mathcal{C}_{\mathcal{L}_{4,p}}(e_{12})=4$. Denote by $x=\varphi(e_{12})=\lambda_{12}e_{12}+\lambda_{23}e_{23}+\lambda_{34}e_{34}+\lambda_{13}e_{13}+\lambda_{24}e_{24}+\lambda_{14}e_{14}\in\mathcal{L}_{4,p}$, and denote by $y=\mu_{12}e_{12}+\mu_{23}e_{23}+\mu_{34}e_{34}+\mu_{13}e_{13}+\mu_{24}e_{24}+\mu_{14}e_{14}\in\mathcal{C}_{\mathcal{L}_{4,p}}(e_{12})$, an element in the centralizer of $e_{12}$, hence $[x,y]=(\lambda_{12}\mu_{23}-\lambda_{23}\mu_{12})e_{13}+(\lambda_{23}\mu_{34}-\lambda_{34}\mu_{23})e_{24}+(\lambda_{12}\mu_{24}-\lambda_{24}\mu_{12}+\lambda_{13}\mu_{34}-\lambda_{34}\mu_{13})e_{14}=0$. Assume all the coefficients of the linear combination that forms $x$ are non-zero. Then, as seen earlier, we have that $\mu_{23}=\frac{\lambda_{23}}{\lambda_{12}}\mu_{12}$, and $\mu_{34}=\frac{\lambda_{34}}{\lambda_{23}}\mu_{23}=\frac{\lambda_{34}}{\lambda_{23}}\frac{\lambda_{23}}{\lambda_{12}}\mu_{12}=\frac{\lambda_{34}}{\lambda_{12}}\mu_{12}$, and also $\lambda_{12}\mu_{24}-\lambda_{24}\mu_{12}+\lambda_{13}\mu_{34}-\lambda_{34}\mu_{13}=0$, which means that $\mu_{24}=\frac{\lambda_{24}\mu_{12}+\lambda_{13}\mu_{34}-\lambda_{34}\mu_{13}}{\lambda_{12}}=\frac{\lambda_{24}\mu_{12}+\lambda_{13}\frac{\lambda_{34}}{\lambda_{12}}\mu_{12}-\lambda_{34}\mu_{13}}{\lambda_{12}}$, hence we can choose freely $\mu_{12}$, $\mu_{13}$ and $\mu_{14}$, while $\mu_{23}$ and $\mu_{34}$ depend on $\mu_{12}$, and $\mu_{24}$ depends on $\mu_{12}$ and $\mu_{13}$, which means that $\dim\mathcal{C}_{\mathcal{L}_{4,p}}(y)=3<4=\dim\mathcal{C}_{\mathcal{L}_{4,p}}(e_{12})$. Assume that all the coefficients of $x$ are non-zero, except for $\lambda_{23}=0$, then $\lambda_{12}\mu_{23}$ and $\lambda_{34}\mu_{23}$ must vanish, hence $\mu_{23}=0$, but then $\mu_{34}$ does not depend on $\mu_{23}$, which implies that it does not depend on $\mu_{12}$ either, and can be chosen freely, hence there is no change in the dimension of $\mathcal{C}_{\mathcal{L}_{4,p}}(e_{12})$ from the general case. Now we assume $x=\lambda_{12}e_{12}+z$, where $z\in\gamma_2\mathcal{L}_{4,p}$, and observe the three equations from above with the current assumption. The second equation $\lambda_{23}\mu_{34}-\lambda_{34}\mu_{23}=0$ completely falls, which from the other two we obtain that $\lambda_{12}\mu_{23}$ and $\lambda_{12}\mu_{24}$ must vanish, which means that $\mu_{23}=\mu_{24}=0$, while $\mu_{12}$, $\mu_{34}$, $\mu_{13}$ and $\mu_{14}$ can be chosen freely, which means that $\dim\mathcal{C}_{\mathcal{L}_{4,p}}(y)=4=\dim\mathcal{C}_{\mathcal{L}_{4,p}}(e_{12})$. One checks that the same applies also for $\varphi(e_{12})=\lambda_{34}e_{34}+z$, and that no other linear combination of $x$ satisfies that $\dim\mathcal{C}_{\mathcal{L}_{4,p}}(\varphi(e_{12}))=\dim\mathcal{C}_{\mathcal{L}_{4,p}}(e_{12})$.
\end{proof}
\end{document}
