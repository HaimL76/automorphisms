\documentclass[12pt]{article}
\usepackage{graphicx} % Required for inserting images
\makeatletter
\newcommand*{\rom}[1]{\expandafter\@slowromancap\romannumeral #1@}
\makeatother
\usepackage{amsfonts, amssymb}
\usepackage{mathrsfs, mathdots} 
\usepackage{amsmath}
\usepackage{float}
\usepackage{amsthm}
\usepackage{tikz-cd}
\usepackage{xcolor}
\usepackage{xparse}
\usepackage{setspace}
\usepackage{xfrac}
\usepackage{yfonts}

\newtheorem{theorem}{Theorem}[subsection]
\newtheorem{proposition}[theorem]{Proposition}
\newtheorem{corollary}[theorem]{Corollary}
\newtheorem{lemma}[theorem]{Lemma}
\newtheorem{notations}[theorem]{Notations}
\newtheorem{definition}[theorem]{Definition}
\newtheorem{example}[theorem]{Example}

\ExplSyntaxOn
\NewDocumentCommand{\cycle}{ O{\;} m }
 {
  (
  \alec_cycle:nn { #1 } { #2 }
  )
 }

\seq_new:N \l_alec_cycle_seq
\cs_new_protected:Npn \alec_cycle:nn #1 #2
 {
  \seq_set_split:Nnn \l_alec_cycle_seq { , } { #2 }
  \seq_use:Nn \l_alec_cycle_seq { #1 }
 }
\ExplSyntaxOff
\usepackage{titling}
\newcommand{\subtitle}[1]{%
  \posttitle{%
    \par\end{center}
    \begin{center}\large#1\end{center}
    \vskip0.5em}%
}
\usepackage{arydshln}
\setcounter{tocdepth}{2}
\setlength{\dashlinedash}{1.2pt}
\setlength{\dashlinegap}{1.5pt}
\setlength{\arrayrulewidth}{0.2pt}
\title{proof-m11}
\author{haiml76 }
\date{May 2024}

\begin{document}
\begin{proposition}
Let $x=\sum_{i=1}^{n-1}\lambda_i e_{i,i+1}$, where $\lambda_i\in\mathbb{Q}_p$ not all zero. Then $\dim\mathcal{C}_{\gamma_3}(x)=\#\{i : \lambda_i=0\}+1$
\end{proposition}
\begin{proof}
Let $y=\sum_{i=1}^{n-1}\mu_i e_{i,i+1}\in\mathcal{C}_{\gamma_3}(x)$, where $\lambda_i\in\mathbb{Q}_p$.
For every $1\leq i\leq n-1$, denote by $c_i$ the constraint equation $[\lambda_i e_{i,i+1},\mu_{i+1}e_{i+1,i+2}]-[\lambda_{i+1}e_{i+1,i+2},\mu_i e_{i,i+1}]=(\lambda_i\mu_{i+1}-\lambda_{i+1}\mu_i)e_{i,i+2}=0$. Let $1\leq j<k\leq n-1$ be two indices, such that $\lambda_{j-1}=0$, and $\lambda_{k+1}=0$, and $\lambda_j,\lambda_{j+1},\dots,\lambda_k$ are all non-zero\footnote{We extend our notation of indices, to include also the case where $j=1$ or $k=n-1$, and define that $\lambda_{j-1}=\lambda_0=0$ or $\lambda_{k+1}=\lambda_n=0$, respectively}, then by constraints $c_j,c_{j+1},\dots,c_{m-1}$, we have that $\mu_m=\frac{\lambda_m}{\lambda_{m-1}}\mu_{m-1}=\frac{\lambda_m}{\lambda_{m-1}}\frac{\lambda_{m-1}}{\lambda_{m-2}}\mu_{m-2}=\frac{\lambda_m}{\lambda_{m-2}}\mu_{m-2}=\cdots=\frac{\lambda_m}{\lambda_j}\mu_j$, for every $j+1\leq m\leq k-1$, which means that all $\mu$ coefficients of $y$, with indices from $j+1$ to $k$, depend on the first coefficient, namely $\mu_j$. We denote the free choice of $\mu_j$ by $\mu_j=\ast$. One easily checks that we can choose freely any coefficient $\mu_m$ from $j+1$ to $k$, instead of $\mu_j$, and all other coefficients in that range will depend on our choice of $\mu_m$. By constraint $c_{j-1}$, we have that $\lambda_{j-1}\mu_j-\lambda_j\mu_{j-1}=0$, but $\lambda_{j-1}=0$, hence $\lambda_j\mu_{j-1}$ must vanish, but $\lambda_j\neq 0$, which obviously means that $\mu_{j-1}=0$. Similarly, we have that $\mu_{k+1}=0$, due to constraint $c_k$. By constraint $c_{k+1}$, we have that $\lambda_{k+1}\mu_{k+2}-\lambda_{k+2}\mu_{k+1}=0$, but $\lambda_{k+1}=\mu_{k+1}=0$, hence, $\lambda_{k+1}\mu_{k+2}$ must vanish, but $\lambda_{k+1}=0$, which means that we need to look at constraint $c_{k+2}$, that is, $\lambda_{k+2}\mu_{k+3}-\lambda_{k+3}\mu_{k+2}$. We check the different options.
If $\lambda_{k+2}=0$, then $\lambda_{k+3}\mu_{k+2}$ must vanish. Therefore, if $\lambda_{k+3}\neq 0$, then $\mu_{k+2}=0$, but if $\lambda_{k+3}=0$, then $\mu_{k+2}=\ast$. If $\lambda_{k+2}\neq 0$, then again $\mu_{k+2}=\ast$.



Suppose that $j>1$, then by constraint $c_{j-1}$, we have that $\lambda_{j-1}\mu_j-\lambda_j\mu_{j-1}=0$, but $\lambda_{j-1}=0$, so $\lambda_j\mu_{j-1}$ must vanish, but $\lambda_j\neq 0$, so it must be that $\mu_{j-1}=0$. Same way, if $k<n-1$, then it must be that $\mu_{k+1}=0$. Let $1\leq j$
\end{proof}




We observe that for each $2\leq j\leq n-2$, $\mu_j$ is obviously determined by the two constraints $c_{j-1}$ and $c_j$, which means that we have several options for $\lambda_{j-1},\lambda_j,\lambda_{j+1}$. We look at the two equations: \[
c_{j-1}=(\lambda_{j-1}\mu_j-\lambda_j\mu_{j-1})e_{j-1,j+1}=0
\]\[
c_j=(\lambda_j\mu_{j+1}-\lambda_{j+1}\mu_j)e_{j,j+2}=0
\]
Obviously, if $\lambda_{j-1}=\lambda_j=\lambda_{j+1}=0$, then both $c_{j-1}$ and $c_j$ are invalid constraints, which means that $\mu_j$ can assume any value, we usually denote this by $\mu_j=\ast$.
Suppose that we have only two zeros, then if $\lambda_{j-1}=\lambda_j=0$ and $\lambda_{j+1}\neq 0$ or if $\lambda_j=\lambda_{j+1}=0$ and $\lambda_{j-1}\neq 0$, then we must also have that $\lambda_{j+1}\mu_j=0$ or $\lambda_{j-1}\mu_j=0$, respectively, which means that $\mu_j=0$. On the other hand, if $\lambda_{j-1}=\lambda_{j+1}=0$ and $\lambda_j\neq 0$, then we must have $
\lambda_j\mu_{j-1}=\lambda_j\mu_{j+1}=0$, which means that $\mu_{j-1}=\mu_{j+1}=0$, and since $\mu_j$ depends only on $c_{j-1}$ and $c_j$, we have that $\mu_j=\ast$.
Suppose that only one of the three $\lambda$ coefficients is zero, then if $\lambda_{j-1}=0$, we must have that $\mu_{j-1}=0$, and $\mu_{j+1}=\frac{\lambda_{j+1}}{\lambda_j}\mu_j$. If $lambda_j=0$, then we must have that $\lambda_{j-1}\mu_j=\lambda_{j+1}\mu_j=0$, which means that $\mu_j=0$. If $\lambda_{j+1}=0$, then we must have that $\mu_{j+1}=0$, and $\mu_j=\frac{\lambda_j}{\lambda_{j-1}}\mu_{j-1}$. If the three $\lambda$ coefficients are non-zero, then we must have that $\mu_j=\frac{\lambda_j}{\lambda_{j-1}}\mu_{j-1}$, and $\mu_{j+1}=\frac{\lambda_{j+1}}{\lambda_j}\mu_j=\frac{\lambda_{j+1}}{\lambda_j}\frac{\lambda_j}{\lambda_{j-1}}\mu_{j-1}=\frac{\lambda_{j+1}}{\lambda_{j-1}}\mu_{j-1}$. We conclude that every chain of non-zero consecutive coefficients $\lambda_j,\lambda_{j+1},\dots\,\lambda_{j+l-1}$, where $l$ is clearly the length of the chain, has that $\mu_{k+1}=\frac{\lambda_{k+1}}{\lambda_j}\mu_j$, which means that 
\end{document}
