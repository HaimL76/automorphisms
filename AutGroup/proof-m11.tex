\documentclass[12pt,fleqn]{article}
\usepackage{graphicx} % Required for inserting images
\makeatletter
\newcommand*{\rom}[1]{\expandafter\@slowromancap\romannumeral #1@}
\makeatother
\usepackage{amsfonts, amssymb}
\usepackage{mathrsfs, mathdots} 
\usepackage{amsmath}
\usepackage{float}
\usepackage{amsthm}
\usepackage{tikz-cd}
\usepackage{xcolor}
\usepackage{xparse}
\usepackage{setspace}
\usepackage{xfrac}
\usepackage{yfonts}
\usepackage{cancel}
\newcommand*\circled[1]{\tikz[baseline=(char.base)]{
            \small \node[shape=circle,draw,inner sep=1pt] (char) {#1};}}
\newtheorem{theorem}{Theorem}[subsection]
\newtheorem{proposition}[theorem]{Proposition}
\newtheorem{corollary}[theorem]{Corollary}
\newtheorem{lemma}[theorem]{Lemma}
\newtheorem{notations}[theorem]{Notations}
\newtheorem{definition}[theorem]{Definition}
\newtheorem{example}[theorem]{Example}
\setcounter{MaxMatrixCols}{20}
\ExplSyntaxOn
\NewDocumentCommand{\cycle}{ O{\;} m }
 {
  (
  \alec_cycle:nn { #1 } { #2 }
  )
 }

\seq_new:N \l_alec_cycle_seq
\cs_new_protected:Npn \alec_cycle:nn #1 #2
 {
  \seq_set_split:Nnn \l_alec_cycle_seq { , } { #2 }
  \seq_use:Nn \l_alec_cycle_seq { #1 }
 }
\ExplSyntaxOff
\usepackage{titling}
\newcommand{\subtitle}[1]{%
  \posttitle{%
    \par\end{center}
    \begin{center}\large#1\end{center}
    \vskip0.5em}%
}
\usepackage{arydshln}
\setcounter{tocdepth}{2}
\setlength{\dashlinedash}{1.2pt}
\setlength{\dashlinegap}{1.5pt}
\setlength{\arrayrulewidth}{0.2pt}
\title{proof-m11}
\author{haiml76 }
\date{May 2024}

\begin{document}
\section{The computation of $G_n(\mathbb{Q}_p)$}
\subsection{The computation of the first block $M_{11}$}
For all the following, we extend our notations of elements in the quotient $\sfrac{\gamma_1}{\gamma_2}$ by denoting $x=\sum_{i=0}^n\lambda_i e_{i,i+1}$, for all $x\in\sfrac{\gamma_1}{\gamma_3}$, and since $e_{0,1}$ and $e_{n-1,n}$ do not exist in our algebra, it is well understood that $\lambda_0=\lambda_n=0$. We also denote by $z_j$, $w_j$, $v_j$ or $u_z$ some element in $\gamma_j\mathcal{L}_{n,p}$ whose accurate value is of no interest for the matter of the discussion.
\begin{lemma}
\label{prop.centralizer.dimension}
Let $x=\sum_{i=0}^n\lambda_i e_{i,i+1}$, where $\lambda_i\in\mathbb{Q}_p$ are not all zero. Then $\dim\sfrac{\mathcal{C}_{\sfrac{\gamma_1}{\gamma_3}}(x)}{\gamma_2}=\mathfrak{l}(x)+\mathfrak{m}(x)$, where $\mathfrak{l}(x)$ is the number of sequences of consecutive non-zero coefficients of the form $\lambda_j,\lambda_{j+1},\dots,\lambda_{j+k-1},\lambda_{j+k}$ and $\lambda_{j-1}=\lambda_{j+k+1}=0$, and $\mathfrak{m}(x)$ is the number of zero coefficients $\lambda_j=0$, such that also $\lambda_{j-1}=\lambda_{j+1}=0$.
\end{lemma}
\begin{proof}
Let $y=\sum_{i=1}^{n-1}\mu_i e_{i,i+1}$, where $\lambda_i\in\mathbb{Q}_p$, be an element in the quotient $\sfrac{\gamma_1}{\gamma_2}$.
For every $1\leq i\leq n-1$, denote by $(\mathfrak{C}_i)$ the constraint equation $[\lambda_i e_{i,i+1},\mu_{i+1}e_{i+1,i+2}]-[\lambda_{i+1}e_{i+1,i+2},\mu_i e_{i,i+1}]=(\lambda_i\mu_{i+1}-\lambda_{i+1}\mu_i)e_{i,i+2}=0$, and it is clear that $y\in\sfrac{\mathcal{C}_{\sfrac{\gamma_1}{\gamma_3}}(x)}{\gamma_2}$ if and only if all the $(\mathfrak{C}_i)$ constraints are satisfied. We observe that each $\mu_i$ participates in two constraints, $(\mathfrak{C}_{i-1})$ and $(\mathfrak{C}_i)$, that is, $\lambda_{i-1}\mu_i-\lambda_i\mu_{i-1}=\lambda_i\mu_{i+1}-\lambda_{i+1}\mu_i=0$. If $\lambda_i=0$, then $\lambda_i\mu_{i-1}=\lambda_i\mu_{i+1}=0$, hence by constraint $(\mathfrak{C}_{i-1})$ we have that $\lambda_{i-1}\mu_i=0$, and by constraint $(\mathfrak{C}_i)$ we have that $\lambda_{i+1}\mu_i=0$. Hence, if either $\lambda_{i-1}$ or $\lambda_{i+1}$ are non-zero, then $\mu_i=0$. But if $\lambda_{i-1}=\lambda_{i+1}=0$, then both constraints are satisfied for any choice of $\mu_i$, which increases $\dim\sfrac{\mathcal{C}_{\sfrac{\gamma_1}{\gamma_3}}(x)}{\gamma_2}$ by $1$. We need to prove that for any sequence of $k$ consecutive zero coefficients of $x$, where $k\geq 3$, that is $\lambda_{j+1}=\lambda_{j+2}=\cdots=\lambda_{j+k}=0$, for $1\leq j\leq n-2$, we have that the sequence $\mu_{j+2},\mu_{j+3},\dots,\mu_{j+k-1}$ of $k-2$ consecutive coefficients of $y$ is made of scalars of any choice, thus $\dim\sfrac{\mathcal{C}_{\sfrac{\gamma_1}{\gamma_3}}(x)}{\gamma_2}$ is increased by $k-2$. We prove that by simple induction on $k$. For $k=3$, we just proved that if $\lambda_{i-1}=\lambda_i=\lambda_{i+1}=0$, then $\mu_i$ can be any scalar. For $k+1$, we look at the sequence of $k+1$ zero coefficients, $\lambda_{j+1}=\lambda_{j+2}=\cdots=\lambda_{j+k}=\lambda_{j+k+1}=0$. By constraints $(\mathfrak{C}_{j+k-1})$ and $(\mathfrak{C}_{j+k})$, we have that $\lambda_{j+k-1}\mu_{j+k}-\lambda_{j+k}\mu_{j+k-1}=\lambda_{j+k}\mu_{j+k+1}-\lambda_{j+k+1}\mu_{j+k}=0$, and since $\lambda_{j+k-1}=\lambda_{j+k}=\lambda_{j+k+1}=0$, we have that $\mu_{j+k}$ can be any scalar, as we proved earlier. By the assumption, we have that all $k-2$ previous coefficients, that is $\mu_{j+2},\dots,\mu_{j+k-1}$ can also be any scalars, so in total the whole sequence of $k-1=(k+1)-2$ coefficients of $y$ can be any scalars, which proves the induction step. Suppose that we have $m$ sequences of three or more consecutive zero coefficients in $x$, whose lengths are $k_1,k_2,\dots,k_m$, then $\mathfrak{m}(x)=\sum_{l=1}^m k_l-2m$ is the total number of zero coefficients $\lambda_j=0$ such that also $\lambda_{j-1}=\lambda_{j+1}=0$, as proposed. Using again the two consecutive constraints, $(\mathfrak{C}_{i-1})$ and $(\mathfrak{C}_i)$,
suppose now that $\lambda_i\neq 0$. If $\lambda_{i-1}=0$, then by constraint $(\mathfrak{C}_{i-1})$ we must have that $\mu_{i-1}=0$, but if $\lambda_{i-1}\neq 0$, then by this constraint we have $\mu_i=\frac{\lambda_i\mu_{i-1}}{\lambda_{i-1}}$, which means that $\mu_i$ depends on $\mu_{i-1}$. Precisely the same way for constraint $(\mathfrak{C}_i)$, we have that if $\lambda_{i+1}=0$ then $\mu_{i+1}=0$, otherwise $\mu_{i+1}=\frac{\lambda_{i+1}\mu_i}{\lambda_i}$, which means that $\mu_{i+1}$ depends on $\mu_i$, and if also $\lambda_{i-1}\neq 0$, then $\mu_{i+1}=\frac{\lambda_{i+1}\frac{\lambda_i\mu_{i-1}}{\lambda_{i-1}}}{\lambda_i}=\frac{\lambda_{i+1}\mu_{i-1}}{\lambda_{i-1}}$, which means that both $\mu_{i+1}$ and $\mu_i$ depend on $\mu_{i-1}$. We need to prove this is true for any sequence of $k$ consecutive non-zero coefficients of $x$, that is, for a given sequence of coefficients, $\lambda_{j+1},\lambda_{j+2},\dots,\lambda_{j+k}$, where $1\leq j\leq n-1$, we need to prove that $\mu_{j+1}$ can be any scalar, while $\mu_{j+2},\mu_{j+3},\dots,\mu_{j+k}$ all depend on $\mu_{j+1}$. Here again, we use a simple induction on $k$. For $k=1,2,3$, we already proved this. For $k+1$, we look into a sequence of $k+1$ consecutive non-zero coefficients of $x$, that is $\lambda_{j+1},\dots,\lambda_{j+k},\lambda_{j+k+1}$. By constraint $(\mathfrak{C}_{j+k})$ we have $\lambda_{j+k}\mu_{j+k+1}-\lambda_{j+k+1}\mu_{j+k}=0$, and since $\lambda_{j+k}\neq 0$, we have $\mu_{j+k+1}=\frac{\lambda_{j+k+1}\mu_{j+k}}{\lambda_{j+k}}$, but by the assumption, $\mu_{j+k}=\frac{\lambda_{j+k}\mu_{j+k-1}}{\lambda_{j+k-1}}=\frac{\lambda_{j+k}}{\lambda_{j+k-1}}\frac{\lambda_{j+k-1}\mu_{j+k-2}}{\lambda_{j+k-2}}=\frac{\lambda_{j+k}}{\lambda_{j+k-1}}\frac{\lambda_{j+k-1}}{\lambda_{j+k-2}}\frac{\lambda_{j+k-2}\mu_{j+k-3}}{\lambda_{j+k-3}}=\cdots=\frac{\lambda_{j+k}\mu_{j+1}}{\lambda_{j+1}}$, hence $\mu_{j+k+1}=\frac{\lambda_{j+k+1}\mu_{j+k}}{\lambda_{j+k}}=\frac{\lambda_{j+k+1}}{\lambda_{j+k}}\frac{\lambda_{j+k}\mu_{j+1}}{\lambda_{j+1}}=\frac{\lambda_{j+k+1}\mu_{j+1}}{\lambda_{j+1}}$, which proves the induction step. Looking again at $(\mathfrak{C}_{i-1})$ and $(\mathfrak{C}_i)$, we observe that if $\lambda_{i-1}$ and $\lambda_{i+1}$ are non-zero and $\lambda_i=0$, then $\lambda_i\mu_{i-1}=\lambda_i\mu_{i+1}=0$, hence by constraint $(\mathfrak{C}_{i-1})$ we have that $\lambda_{i-1}\mu_i=0$, and by constraint $(\mathfrak{C}_i)$ we have that $\lambda_{i+1}\mu_i=0$, so $\mu_i=0$ by both constraints, which means that there is no dependency between $\mu_{i+1}$ and $\mu_{i-1}$. This shows that any zero coefficient between two non-zero coefficients of $x$ creates two separate sequences, each sequences increases $\dim\sfrac{\mathcal{C}_{\sfrac{\gamma_1}{\gamma_3}}(x)}{\gamma_2}$ by $1$, hence, if we have $l$ sequences of consecutive non-zero coefficients of $x$, they increase $\dim\sfrac{\mathcal{C}_{\sfrac{\gamma_1}{\gamma_3}}(x)}{\gamma_2}$ by $l$, and we denote $\mathfrak{l}(x)=l$. All this shows that $\dim\sfrac{\mathcal{C}_{\sfrac{\gamma_1}{\gamma_3}}(x)}{\gamma_2}=\mathfrak{l}(x)+\mathfrak{m}(x)$, as proposed.
\end{proof}
\begin{proposition}
\label{prop.n.geq.5.centralizer.codimension}
Let $\mathcal{L}_{n,p}$ be the $\mathbb{Q}_p$-Lie algebra associated with $\mathcal{U}_n(\mathbb{Q}_p)$, where $n\geq 5$, then $\dim\mathcal{C}_{\sfrac{\gamma_1}{\gamma_3}}(x)=\dim\sfrac{\gamma_1}{\gamma_3}-1$ if and only if $x\in\{\lambda e_{12}+\gamma_2\mathcal{L}_{n,p}\}$ or $x\in\{\lambda e_{n-1,n}+\gamma_2\mathcal{L}_{n,p}\}$, for a non-zero scalar $\lambda\in\mathbb{Q}_p$.
\end{proposition}
\begin{proof}
We recall first that for any algebra $\mathcal{L}_{n,p}$, the first two elements of the lower central series, namely $\sfrac{\gamma_1}{\gamma_2}$ and $\sfrac{\gamma_2}{\gamma_3}$, are of sizes $n-1$ and $n-2$, respectively. Consider the Lie brackets operation $[x_1,x_2]$, where $x_1\in\sfrac{\gamma_1}{\gamma_2}$ and $x_2\in\sfrac{\gamma_2}{\gamma_3}$, or the opposite, then $[x_1,x_2]$ can be either zero, or $[x_1,x_2]\in\gamma_3$, which means that if we consider the centralizer of some $x\in\mathcal{L}_{n,p}$ in $\sfrac{\gamma_1}{\gamma_3}$, then any element $y\in\sfrac{\gamma_2}{\gamma_3}$ would either commute with $x$ or yield an element in $\gamma_3$, which also means that it commutes with $x$ in $\sfrac{\gamma_1}{\gamma_3}$. This shows that $\dim\mathcal{C}_{\sfrac{\gamma_1}{\gamma_3}}(x)\geq\dim\sfrac{\gamma_2}{\gamma_3}=n-2$. Suppose that $x=\lambda_1 e_{12}+z$ or $x=\lambda_{n-1}e_{n-1,n}+z$, where $z\in\gamma_2\mathcal{L}_{n,p}$, then looking only at the elements of $\sfrac{\gamma_1}{\gamma_2}$ in the linear combination that forms $x$, we have one sequence of one non-zero coefficient, hence $\mathfrak{l}(x)=1$, and one sequence of $n-1-1=n-2$ zero coefficients of $x$, hence we have that $\mathfrak{m}(x)=n-2-1=n-3$. By \ref{prop.centralizer.dimension} we have that $\dim\sfrac{\mathcal{C}_{\sfrac{\gamma_1}{\gamma_3}}(x)}{\gamma_2}=\mathfrak{l}(x)+\mathfrak{m}(x)=1+(n-3)=n-2$. Adding all the elements of $\sfrac{\gamma_2}{\gamma_3}$, we have that $\dim\mathcal{C}_{\sfrac{\gamma_1}{\gamma_3}}(x)=(n-2)+(n-2)=2n-4=(n-1)+(n-2)-1=\dim\sfrac{\gamma_1}{\gamma_2}+\dim\sfrac{\gamma_2}{\gamma_3}-1=\dim\sfrac{\gamma_1}{\gamma_3}-1$, as proposed. To prove the opposite direction, we assume that $\dim\mathcal{C}_{\sfrac{\gamma_1}{\gamma_3}}(x)=\dim\sfrac{\gamma_1}{\gamma_3}-1=2n-4$, and since we already know that all the elements of $\dim\sfrac{\gamma_2}{\gamma_3}$ are in the centralizer of $x$, we have that $\dim\sfrac{\mathcal{C}_{\sfrac{\gamma_1}{\gamma_3}}(x)}{\gamma_2}=\dim\mathcal{C}_{\sfrac{\gamma_1}{\gamma_3}}(x)-\dim\sfrac{\gamma_2}{\gamma_3}=(2n-4)-(n-2)=n-2=(n-1)-1=\dim\sfrac{\gamma_1}{\gamma_2}-1$. Suppose that either $x=\lambda_i e_{i,i+1}+z$ where $1<i<n-1$, or $x=\lambda_{i_1}e_{i_1,i_1+1}+\cdots+\lambda_{i_m}e_{i_m,i_m+1}$ where we denote the number of $\sfrac{\gamma_1}{\gamma_2}$ coefficients by $2\leq m\leq n-1$, in both cases $z\in\gamma_2\mathcal{L}_{n,p}$. In the first case, we have that $\mathfrak{l}(x)=1$, because there is only one non-zero coefficient of $x$ in $\sfrac{\gamma_1}{\gamma_2}$, and we have two sequences of zero coefficients of $x$, that is $\lambda_0=\lambda_1=\lambda_2=\cdots=\lambda_{i-2}=\lambda_{i-1}=0$ and $\lambda_{i+1}=\lambda_{i+2}=\lambda_{i+3}=\cdots=\lambda_{n-2}=\lambda_{n-1}=\lambda_n=0$, which means we have a sequence of zeros of length $(i-1)+1=i$ and a sequence of length $(n-1)-i+1=n-i$, hence by \ref{prop.centralizer.dimension} we have that $\mathfrak{m}(x)\leq(i-2)+(n-i-2)=n-4$, which means that $\dim\sfrac{\mathcal{C}_{\sfrac{\gamma_1}{\gamma_3}}(x)}{\gamma_2}=\mathfrak{l}(x)+\mathfrak{m}(x)\leq 1+(n-4)=n-3=(n-1)-2=\dim\sfrac{\gamma_1}{\gamma_2}-2$. Adding all the elements of $\sfrac{\gamma_2}{\gamma_3}$, we have that $\dim\mathcal{C}_{\sfrac{\gamma_1}{\gamma_3}}(x)=\dim\sfrac{\mathcal{C}_{\sfrac{\gamma_1}{\gamma_3}}(x)}{\gamma_2}+\dim\sfrac{\gamma_2}{\gamma_3}\leq(n-3)+(n-2)=2n-5<2n-4$, which contradicts the assumption. Suppose that $x$ is of the second form, then $x$ may have one sequence of $k$ non-zero coefficients, where $k>1$, or $x$ may have two or more sequences of one or more non-zero coefficients. In the case that $x$ has one sequence of $k$ non-zero coefficients, $\mathfrak{l}(x)=1$, and the number of consecutive zero coefficients is either less or equal to $(n-1)-k+1=n-k$, which results in $\mathfrak{m}(x)\leq n-k-2$, if the sequence of non-zero coefficients contains either $\lambda_1$ or $\lambda_{n-1}$, or the number of consecutive zero coefficients is divided into two separate sequences with a total number of consecutive zeros which is less or equal to $(n-1)-k+2=n-k+1$, which results in $\mathfrak{m}(x)\leq n-k+1-4=n-k-3$, in the case that the sequence of non-zero coefficients of $x$ does not contains either $\lambda_1$ or $\lambda_{n-1}$, which means that $\mathfrak{l}(x)+\mathfrak{m}(x)\leq 1+(n-k-2)=n-k-1$, and since $k>1$, we have that $\dim\sfrac{\mathcal{C}_{\sfrac{\gamma_1}{\gamma_3}}(x)}{\gamma_2}=\mathfrak{l}(x)+\mathfrak{m}(x)\leq n-2-1=n-3=(n-1)-2$, which again contradicts the assumption. Suppose that $x$ has $l$ sequences of non-zero coefficients in $\sfrac{\gamma_1}{\gamma_2}$, of lengths $k_1,k_2,\dots,k_l$, then $\mathfrak{l}(x)=l$, and the number of zeros is $n-1-\sum_{j=1}^l k_j+2=n+1-\sum_{j=1}^l k_j$ in $l+1$ sequences, considering the two options from earlier, whether any of the sequences of non-zero coefficients contains $\lambda_1$ or $\lambda_{n-1}$, or does not contain either of them. We observe that $n+1-\sum_{j=1}^l k_j$ is bounded by $n+1-l$, which is the case where $k_1=k_2=\cdots=k_l=1$. Suppose that $l=2$, and that each of the $3$ sequences of zero coefficients is of length greater or equal to $3$, then $\mathfrak{m}(x)\leq n+1-2-3\cdot 2=n-7$, hence $\dim\sfrac{\mathcal{C}_{\sfrac{\gamma_1}{\gamma_3}}(x)}{\gamma_2}=\mathfrak{l}(x)+\mathfrak{m}(x)\leq 2+n-7=n-5$, hence $\dim\mathcal{C}_{\sfrac{\gamma_1}{\gamma_3}}(x)=\dim\sfrac{\mathcal{C}_{\sfrac{\gamma_1}{\gamma_3}}(x)}{\gamma_2}+\dim\sfrac{\gamma_2}{\gamma_3}\leq (n-5)+(n-2)=2n-7<2n-2=\dim\sfrac{\gamma_1}{\gamma_3}-1$. We prove this is true for any number of sequences of non-zero coefficients of $x$ simple induction on $l$. For $l=2$ we already proved that, for $l+1$ we take the upper bound case, which means that we add a new non-zero coefficient which splits a sequence of zero coefficients such that each of the two parts of the sequence that we split has at least $3$ zero coefficients. Denote by $\mathfrak{l}_l(x)=l$ and $\mathfrak{m}_l(x)$ for $l$, and denote by $\mathfrak{l}_{l+1}(x)=l+1$ and $\mathfrak{m}_{l+1}(x)$ for $l+1$, then obviously $\mathfrak{l}_{l+1}+\mathfrak{m}_{l+1}\leq\mathfrak{l}_l+\mathfrak{m}_l$, because $\mathfrak{l}_{l+1}-\mathfrak{l}_l=1$ and $\mathfrak{m}_{l}-\mathfrak{m}_{l+1}\geq 1$, because the newly added non-zero coefficient does not only remove one zero coefficient from the total count, but may also remove one or two of its neighboring zero coefficients, since they are no longer surrounded by another zero coefficient, which proves the induction step. All this proves that if $\dim\mathcal{C}_{\sfrac{\gamma_1}{\gamma_3}}(x)=\dim\sfrac{\gamma_1}{\gamma_3}-1$, then the only option for $x$ is either $x=\lambda_1 e_{12}+z$ or $x=\lambda_{n-1}e_{n-1,n}+z$, where $z\in\gamma_2\mathcal{L}_{n,p}$, as proposed.
\end{proof}
\begin{proposition}
\label{prop.n.geq.4.centralizer.codimension}
Let $\mathcal{L}_{4,p}$ be the $\mathbb{Q}_p$-Lie algebra associated with $\mathcal{U}_4(\mathbb{Q}_p)$, then $\dim\mathcal{C}_{\sfrac{\gamma_1}{\gamma_3}}(x)=\dim\sfrac{\gamma_1}{\gamma_3}-1$ if and only if $x\in\{\lambda_1 e_{12}+\lambda_3 e_{34}+\gamma_2\mathcal{L}_{4,p}\}$, for $\lambda_1,\lambda_3\in\mathbb{Q}_p$ not both zero.
\end{proposition}
\begin{proof}
Suppose that $x=\lambda e_{12}+\mu e_{34}+z$, where $x\in\gamma_2\mathcal{L}_{4,p}$, then by $\ref{prop.centralizer.dimension}$ if $\lambda\neq 0$ and $\mu\neq 0$, we have that $\mathfrak{l}(x)=2$, and $\mathfrak{m}(x)=0$ because there are three isolated zero coefficients, $\lambda_0=\lambda_2=\lambda_4=0$, hence $\dim\sfrac{\mathcal{C}_{\sfrac{\gamma_1}{\gamma_3}}(x)}{\gamma_2}=\mathfrak{l}(x)+\mathfrak{m}(x)=2$, which means that $\dim\mathfrak{C}_{\sfrac{\gamma_1}{\gamma_3}}(x)=\dim\sfrac{\mathcal{C}_{\sfrac{\gamma_1}{\gamma_3}}(x)}{\gamma_2}+\dim\sfrac{\gamma_2}{\gamma_3}=2+(4-2)=4=(4-1)+(4-2)-1=\dim\sfrac{\gamma_1}{\gamma_2}+\dim\sfrac{\gamma_2}{\gamma_3}-1=\dim\sfrac{\gamma_1}{\gamma_3}-1$. If $\lambda_1=0$ or $\lambda_3=0$, that is $x=\lambda_1 e_{12}+z$ or $x=\lambda_{}$, where $z\in\gamma_2\mathcal{L}_{4,p}$, then $\mathfrak{l}(x)=1$ and we have a sequence of $3$ consecutive zero coefficients, either $\lambda_2=\lambda_3=\lambda_4=0$ or $\lambda_0=\lambda_1\lambda_2=0$, which means that $\mathfrak{l}(x)=3-2=1$, hence $\mathfrak{l}(x)+\mathfrak{m}(x)=1+1=2$, hence $\dim\mathcal{C}_{\sfrac{\gamma_1}{\gamma_3}}(x)=\dim\sfrac{\gamma_1}{\gamma_3}-1$, as proposed. To prove the opposite, we review the different options. Suppose that $x\in\gamma_2\mathcal{L}_{4,p}$, which means that $\lambda_0=\lambda_1=\lambda_2=\lambda_3=\lambda_4=0$, hence $\mathfrak{l}(x)=0$ and $\mathfrak{m}(x)=5-2=3$, so $\dim\sfrac{\mathcal{C}_{\sfrac{\gamma_1}{\gamma_3}}(x)}{\gamma_2}=\mathfrak{l}(x)+\mathfrak{m}(x)=0+3$, and therefore $\dim\mathcal{C}_{\sfrac{\gamma_1}{\gamma_3}}(x)=\dim\sfrac{\mathcal{C}_{\sfrac{\gamma_1}{\gamma_3}}(x)}{\gamma_2}+\dim\sfrac{\gamma_2}{\gamma_3}=3+(4-2)=5>\dim\sfrac{\gamma_1}{\gamma_3}-1$. The other option is that $x=\lambda_1 e_{12}+\lambda_2 e_{23}+\lambda_3 e_{34}+z$, where $\lambda_2\neq 0$, where $\lambda_1$ and $\lambda_3$ may be either zero or non-zero, and where $z\in\gamma_2\mathcal{L}_{4,p}$. Since $\lambda_2\neq 0$ there is no sequence of zero coefficients of $x$ which is greater than $2$, and also, we observe that whether $\lambda_1\neq 0$ or $\lambda_3\neq 0$ or both are non-zero, they are part of the sequence of non-zero coefficients which has $\lambda_2$, hence $\mathfrak{l}(x)=1$ and $\mathfrak{m}(x)=0$, so $\dim\mathcal{C}_{\sfrac{\gamma_1}{\gamma_3}}(x)=\dim\sfrac{\mathcal{C}_{\sfrac{\gamma_1}{\gamma_3}}(x)}{\gamma_2}+\dim\sfrac{\gamma_2}{\gamma_3}=\mathfrak{l}(x)+\mathfrak{m}+2=1+2=3<4=\dim\sfrac{\gamma_1}{\gamma_3}-1$, which means that if $x$ is not as proposed, then $\dim\mathcal{C}_{\sfrac{\gamma_1}{\gamma_3}}(x)\neq\dim\sfrac{\gamma_1}{\gamma_3}-1$.
\end{proof}
\begin{proposition}
Let $\mathcal{L}_{4,p}$ be the $\mathbb{Q}_p$-lie algebra associated with $\mathcal{U}_4(\mathbb{Q}_p)$, then the first block $M_{11}$ of the coefficient matrix $M$ representing $\varphi$ is either diagonal or anti-diagonal.
\end{proposition}
\begin{proof}
We compute the centralizers on the quotient $\sfrac{\gamma_1}{\gamma_3}$,
\[\mathcal{C}_{\sfrac{\gamma_1}{\gamma_3}}(\varphi(e_{12}))=\mathcal{C}_{\sfrac{\gamma_1}{\gamma_3}}(e_{12})=\{e_{12},\cancel{e_{23}},e_{34}\}\]
\[\mathcal{C}_{\sfrac{\gamma_1}{\gamma_3}}(\varphi(e_{23}))=\mathcal{C}_{\sfrac{\gamma_1}{\gamma_3}}(e_{23})=\{\cancel{e_{12}},e_{23},\cancel{e_{34}}\}\]
\[\mathcal{C}_{\sfrac{\gamma_1}{\gamma_3}}(\varphi(e_{34}))=\mathcal{C}_{\sfrac{\gamma_1}{\gamma_3}}(e_{34})=\{e_{12},\cancel{e_{23}},e_{34}\}\]
Which means that,
\[\mathrm{codim}\mathcal{C}_{\sfrac{\gamma_1}{\gamma_3}}(\varphi(e_{12}))=\mathrm{codim}\mathcal{C}_{\sfrac{\gamma_1}{\gamma_3}}(e_{12})=1\]
\[\mathrm{codim}\mathcal{C}_{\sfrac{\gamma_1}{\gamma_3}}(\varphi(e_{23}))=\mathrm{codim}\mathcal{C}_{\sfrac{\gamma_1}{\gamma_3}}(e_{23})=2\]
\[\mathrm{codim}\mathcal{C}_{\sfrac{\gamma_1}{\gamma_3}}(\varphi(e_{34}))=\mathrm{codim}\mathcal{C}_{\sfrac{\gamma_1}{\gamma_3}}(e_{34})=1\]
But by \ref{prop.n.geq.4.centralizer.codimension}, for $e_{12}$ and $e_{34}$ we have that $\varphi(e_{12})=a_{12}^{12}e_{12}+a_{34}^{12}e_{34}+z_2$ where $a_{12}^{12}$ and $a_{34}^{12}$ are not both zero, and $\varphi(e_{34})=a_{12}^{34}e_{12}+a_{34}^{34}e_{34}+w_2$ where $a_{12}^{34}$ and $a_{34}^{34}$ are as above. Also, by \ref{prop.centralizer.dimension} we have that if $x=\varphi(e_{23})=a_{12}^{23}e_{12}+a_{23}^{23}e_{23}+a_{34}^{23}e_{34}+a_{13}^{23}e_{23}+a_{24}^{23}e_{24}+a_{14}^{23}e_{14}$, then $\dim\mathcal{C}_{\sfrac{\gamma_1}{\gamma_3}}(x)=\dim\mathcal{C}_{\sfrac{\gamma_1}{\gamma_3}}(e_{23})$ Therefore assume that,
\[\varphi(e_{12})= a_{12}^{12}e_{12}+a_{34}^{12}e_{34}+a_{13}^{12}e_{13}+a_{24}^{12}e_{24}+a_{14}^{12}e_{14}\]
\[\varphi(e_{23})= a_{12}^{23}e_{12}+a_{23}^{23}e_{23}+a_{34}^{23}e_{34}+a_{13}^{23}e_{13}+a_{24}^{23}e_{24}+a_{14}^{23}e_{14}\]
\[\varphi(e_{34})= a_{12}^{34}e_{12}+a_{34}^{34}e_{34}+a_{13}^{34}e_{13}+a_{24}^{34}e_{24}+a_{14}^{34}e_{14}\]
And we have the equations,
$a_{12}^{12}b_{23}^{12}e_{13}=0$, $a_{34}^{12}b_{23}^{12}e_{24}=0\Rightarrow b_{23}^{12}=0$, $(a_{12}^{23}b_{23}^{23}-a_{23}^{23}b_{12}^{23})e_{13}=0\Rightarrow b_{23}^{23}=\frac{a_{23}^{23}}{a_{12}^{23}}b_{12}^{23}$, $(a_{23}^{23}b_{34}^{23}-a_{34}^{23}b_{23}^{23})e_{24}=0\Rightarrow b_{34}^{23}=\frac{a_{34}^{23}}{a_{23}^{23}}b_{23}^{23}=\frac{a_{34}^{23}}{a_{12}^{23}}b_{12}^{23}$, $a_{34}^{34}b_{23}^{34}e_{24}=0$, $a_{12}^{34}b_{23}^{34}e_{13}=0\Rightarrow b_{23}^{34}=0$, hence the co dimensions of the centralizers for the quotient $\sfrac{\gamma_1}{\gamma_3}$ are precisely the number of independent constraints. Looking at a larger quotient $\sfrac{\gamma_1}{\gamma_4}$, we compute the centralizers, \[\mathcal{C}_{\sfrac{\gamma_1}{\gamma_4}}(\varphi(e_{12}))=\mathcal{C}_{\sfrac{\gamma_1}{\gamma_4}}(e_{12})=\{e_{12},\cancel{e_{23}},e_{34},e_{13},\cancel{e_{24}}\}\]
\[\mathcal{C}_{\sfrac{\gamma_1}{\gamma_4}}(\varphi(e_{23}))=\mathcal{C}_{\sfrac{\gamma_1}{\gamma_4}}(e_{23})=\{\cancel{e_{12}},e_{23},\cancel{e_{34}},e_{13},e_{24}\}\]
\[\mathcal{C}_{\sfrac{\gamma_1}{\gamma_4}}(\varphi(e_{34}))=\mathcal{C}_{\sfrac{\gamma_1}{\gamma_4}}(e_{34})=\{e_{12},\cancel{e_{23}},e_{34},\cancel{e_{13}},e_{24}\}\]
Which means that,
\[\mathrm{codim}\mathcal{C}_{\sfrac{\gamma_1}{\gamma_4}}(\varphi(e_{12}))=\mathrm{codim}\mathcal{C}_{\sfrac{\gamma_1}{\gamma_4}}(e_{12})=2\]
\[\mathrm{codim}\mathcal{C}_{\sfrac{\gamma_1}{\gamma_4}}(\varphi(e_{23}))=\mathrm{codim}\mathcal{C}_{\sfrac{\gamma_1}{\gamma_4}}(e_{23})=2\]
\[\mathrm{codim}\mathcal{C}_{\sfrac{\gamma_1}{\gamma_4}}(\varphi(e_{34}))=\mathrm{codim}\mathcal{C}_{\sfrac{\gamma_1}{\gamma_4}}(e_{34})=2\]
And we have the equations,
\[( a_{12}^{12}b_{24}^{12}- b_{12}^{12}a_{24}^{12}+a_{13}^{12}b_{34}-b_{13}^{12}a_{34}^{12})e_{14}=0\]
\[(-a_{34}^{34}b_{13}^{34}+a_{13}^{34}b_{34}^{34}-a_{24}^{34}b_{12}^{34}+a_{12}^{34}b_{24}^{34})e_{14}=0\]
Hence we have that,
$b_{24}^{12}=\frac{ b_{12}^{12}a_{24}^{12}-a_{13}^{12}b_{34}^{12}+b_{13}^{12}a_{34}^{12}}{ a_{12}^{12}}$ or $b_{13}^{12}=\frac{ b_{12}^{12}a_{24}^{12}-a_{13}^{12}b_{34}^{12}-b_{24}^{12}a_{12}^{12}}{-a_{34}^{12}}$, and we have that $b_{13}^{34}=\frac{a_{13}^{34}b_{34}^{34}-a_{24}^{34}b_{12}^{34}+a_{12}^{34}b_{24}^{34}}{a_{34}^{34}}$ or
$b_{24}^{34}=\frac{-a_{13}^{34}b_{34}^{34}+a_{24}^{34}b_{12}^{34}+a_{34}^{34}b_{13}^{34}}{a_{12}^{34}}$.
Which means that the number of constraints matches precisely the co dimension of each element and its image also in the quotient $\sfrac{\gamma_1}{\gamma_4}$. 
Now we compute,
\[\varphi(e_{13})=[\varphi(e_{12}),\varphi(e_{23})]= a_{12}^{12}a_{23}^{23}e_{13}+a_{12}^{12}a_{24}^{23}e_{14}+a_{13}^{12}a_{34}^{23}e_{14}-a_{23}^{23}a_{34}^{12}e_{24}-a_{13}^{23}a_{34}^{12}e_{14}-a_{12}^{23}a_{24}^{12}e_{14}=\]\[=a_{12}^{12}a_{23}^{23}e_{13}-a_{23}^{23}a_{34}^{12}e_{24}+(a_{13}^{12}a_{34}^{23}+a_{12}^{12}a_{24}^{23}-a_{13}^{23}a_{34}^{12}-a_{12}^{23}a_{24}^{12})e_{14}\]
\[\varphi(e_{24})=[\varphi(e_{23}),\varphi(e_{34})]= a_{12}^{23}a_{24}^{34}e_{14}+a_{23}^{23}a_{34}^{34}e_{24}-a_{12}^{34}a_{23}^{23}e_{13}-a_{12}^{34}a_{24}^{23}e_{14}+a_{13}^{23}a_{34}^{34}e_{14}-a_{13}^{34}a_{34}^{23}e_{14}=\]\[=
a_{23}^{23}a_{34}^{34}e_{24}-a_{12}^{34}a_{23}^{23}e_{13}+(a_{12}^{23}a_{24}^{34}-a_{12}^{34}a_{24}^{23}+a_{13}^{23}a_{34}^{34}-a_{13}^{34}a_{34}^{23})e_{14}\]
And we compute,
\[
[\varphi(e_{12}),\varphi(e_{13})]=\]\[=[a_{12}^{12}e_{12}+a_{34}^{12}e_{34}+a_{13}^{12}e_{13}+a_{24}^{12}e_{24}+a_{14}^{12}e_{14},a_{12}^{12}a_{23}^{12}e_{13}-a_{23}^{23}a_{34}^{12}e_{24}+(a_{12}^{12}a_{24}^{23}e_{14}-a_{13}^{23}a_{34}^{12})e_{14}]=\]\[=[-a_{12}^{12}e_{12},a_{23}^{23}a_{34}^{12}e_{24}]-[a_{12}^{12}a_{23}^{23}e_{13}a_{34}^{12}e_{34}]=-a_{12}^{12}a_{23}^{23}a_{34}^{12}e_{14}-a_{12}^{12}a_{23}^{23}a_{34}^{12}e_{14}=-2a_{12}^{12}a_{23}^{23}a_{34}^{12}e_{14}=0
\]
Hence we compute also,
\[
[\varphi(e_{34}),\varphi(e_{24})]=\]\[=[a_{12}^{34}e_{12}+a_{34}^{34}e_{34}+a_{13}^{34}e_{13}+a_{24}^{34}e_{24}+a_{14}^{34}e_{14},a_{23}^{23}a_{34}^{34}e_{24}-a_{12}^{34}a_{23}^{23}e_{13}+(a_{12}^{23}a_{24}^{34}-a_{12}^{34}a_{24}^{23}+a_{13}^{23}a_{34}^{34}-a_{13}^{34}a_{34}^{23})e_{14}=\]\[=a_{12}^{34}a_{23}^{23}a_{34}^{34}e_{14}-(-a_{12}^{34}a_{23}^{23}a_{34}^{34}e_{14})=2 a_{12}^{34}a_{23}^{23}a_{34}^{34}e_{14}=0
\]
The two equations above are possible only if at least one of the coefficients $a_{12}^{12}$, $a_{23}^{23}$ and $a_{34}^{12}$ is zero, and if at least one of the coefficients $a_{12}^{34}$, $a_{23}^{23}$ and $a_{34}^{34}$ is zero. But we cannot have that $a_{23}^{23}$ is zero, because then by \ref{prop.n.geq.4.centralizer.codimension} $\dim\mathcal{C}_{\sfrac{\gamma_1}{\gamma_3}}(\varphi(e_{23}))=\dim\mathcal{C}_{\sfrac{\gamma_1}{\gamma_3}}(e_{23})=\dim\sfrac{\gamma_1}{\gamma_3}-1$, which is a contradiction. Suppose that $a_{12}^{12}=0$, then by \ref{prop.n.geq.4.centralizer.codimension}, $a_{34}^{12}\neq 0$, but then $a_{34}^{34}=0$ and by the same proposition, $a_{12}^{34}\neq 0$. We can also assume that $a_{34}^{12}=0$, and we obtain that $a_{12}^{12}\neq 0$, $a_{12}^{34}=0$ and $a_{34}^{34}\neq 0$. Compute now, \[
[\varphi(23),\varphi(13)]=-a_{12}^{23}a_{23}^{23}a_{34}^{12}e_{14}-a_{12}^{23}a_{23}^{23}a_{34}^{12}e_{14}=-2a_{12}^{23}a_{23}^{23}a_{34}^{12}e_{14}=0
\]
And,\[
[\varphi(23),\varphi(24)]=a_{12}^{23}a_{23}^{23}a_{34}^{34}e_{14}-(-a_{12}^{34}a_{23}^{23}a_{34}^{23})e_{14}=2a_{12}^{34}a_{23}^{23}a_{34}^{23}e_{14}=0
\]
But these two equations can be true only if either of the coefficients $a_{12}^{23}$, $a_{23}^{23}$ and $a_{34}^{12}$ is zero, and only if either of the coefficients $a_{12}^{34}$, $a_{23}^{23}$ and $a_{34}^{23}$ is zero. We already know that $a_{23}^{23}\neq 0$, then if we assume that $a_{12}^{12}\neq 0$ and $a_{12}^{34}=0$, then $a_{34}^{23}=0$, and since $a_{34}^{12}=0$, then $a_{12}^{23}=0$.
\end{proof}
\begin{proposition}
\label{m11.n.5}
Let $\varphi\in G_n(\mathbb{Q}_p)$, where $n\geq 5$, then $M_{11}$ the first block of $M$, the coefficient matrix of $\varphi$, is either diagonal or anti-diagonal.
\end{proposition}
\begin{proof}
We first look at $\varphi(e_{12})$. We observe that \[\mathcal{C}_{\sfrac{\gamma_1}{\gamma_3}}(e_{12})=\langle e_{12},\cancel{e_{23}},e_{34},e_{45},\dots,e_{n-1,n},e_{13},e_{24},e_{35},\dots,e_{n-2,n}\rangle\] hence $\dim\mathcal{C}_{\sfrac{\gamma_1}{\gamma_3}}(\varphi(e_{12}))=\dim\mathcal{C}_{\sfrac{\gamma_1}{\gamma_3}}(e_{12})=\dim\sfrac{\gamma_1}{\gamma_3}-1$, hence by \ref{prop.n.geq.5.centralizer.codimension} we have that either $\varphi(e_{12})=\lambda_1 e_{12}+z_2$ or $\varphi(e_{12})=\lambda_{n-1}e_{n-1,n}+w_2$. From the same considerations, we have that either $\varphi(e_{n-1})=\rho_1 e_{12}+v_2$ or $\rho_{n-1}e_{n-1,n}+u_2$. Clearly, if $\varphi(e_{12})=\lambda_1 e_{12}+z_2$ then $\varphi(e_{n-1,n})=\lambda_{n-1,n} e_{n-1,n}+w_2$ and vice versa, because $\varphi$ is a monomorphism. We take, throughout this proof, $\varphi(e_{12})=\lambda_1 e_{12}+z_2$, and we shall see later that this convention holds for our computational purposes. We now look at $\varphi(e_{23})$, and observe that \[\mathcal{C}_{\sfrac{\gamma_1}{\gamma_3}}(e_{23})=\langle\cancel{e_{12}},e_{23},\cancel{e_{34}},e_{45},\dots,e_{n-1,n},e_{13},e_{24},e_{35},\dots,e_{n-2,n}\rangle\] 
but $e_{13}\in\mathcal{C}_{\sfrac{\gamma_1}{\gamma_3}}(e_{23})$ if and only if $\varphi(e_{13})\in\mathcal{C}_{\sfrac{\gamma_1}{\gamma_3}}(\varphi(e_{23}))$, because $\varphi$ is an automorphism, which means that  $\varphi(e_{13})=\varphi([e_{12},e_{23}])=[\varphi(e_{12}),\varphi(e_{23})]\neq 0$ in the quotient, hence $[\varphi(e_{12}),\varphi(e_{23})]=[\lambda_1 e_{12}+z_2,\sum_{i=1}^{n-1}\rho_i e_{i,i+1}+w_2]=\sum_{i=1}^{n-1}\lambda_1\rho_i [e_{12},e_{i,i+1}]+\lambda_1 w_2+z_2\sum_{i=1}^{n-1}\rho_i e_{i,i+1}+z_2 w_2\neq 0$, and we observe that since we are in the quotient $\sfrac{\gamma_1}{\gamma_3}$, and any element of $\gamma_2\mathcal{L}_{n,p}$ commutes with any element of the algebra in this quotient, then we must have $\sum_{i=1}^{n-1}\lambda_1\rho_i [e_{12},e_{i,i+1}]\neq 0$, but this is true only if $\rho_2\neq 0$, thus $\sum_{i=1}^{n-1}\lambda_1\rho_i [e_{12},e_{i,i+1}]=\lambda_1\rho_2[e_{12},e_{23}]=\lambda_1\rho_2 e_{13}\neq 0$, hence the linear combination that forms $\varphi(e_{23})$ must have $\lambda_2 e_{23}$. Assume that $\varphi(e_{23})=\lambda_2 e_{23}+\sum_{i=1}^{n-1}\lambda_i e_{i,i+1}+z_2$, where $i\neq 2$ and not all $\lambda_i$ are zero. We observe that if $\varphi(e_{23})=x=\lambda_1 e_{12}+\lambda_2 e_{23}+z_2$, then by \ref{prop.centralizer.dimension} $\dim\mathcal{C}_{\sfrac{\gamma_1}{\gamma_3}}(x)=\dim\mathcal{C}_{\sfrac{\gamma_1}{\gamma_3}}(\lambda_2 e_{23}+w_2)$. Therefore, we need to consider a larger centralizer, \[\mathcal{C}_{\sfrac{\gamma_1}{\gamma_4}}(e_{23})=\langle \cancel{e_{12}},e_{23},\cancel{e_{34}},e_{45},\dots,e_{n-1,n},e_{13},e_{24},\cancel{e_{35}},e_{45},\dots,e_{n-2,n},e_{14},\dots,e_{n-3,n}\rangle\] hence $\mathrm{codim}\mathcal{C}_{\sfrac{\gamma_1}{\gamma_4}}(e_{23})=3$, and following the assumption on $\varphi(e_{23})$ we have that $\varphi(e_{23})=x=\lambda_1 e_{12}+\lambda_2 e_{23}+\sum_{i=1}^{n-2}a_{i,i+2}e_{i,i+2}+z_3$, and we look into an arbitrary element $y\in\mathcal{C}_{\sfrac{\gamma_1}{\gamma_4}}(x)$, that is $y=\sum_{i=1}^{n-1}\mu_i e_{i,i+1}+\sum_{i=1}^{n-2}b_{i,i+2}e_{i,i+2}+w_3$, so we have the following equations 
\circled{1} $(\lambda_1\mu_2-\lambda_2\mu_1)e_{13}=0$, \circled{2}
$\lambda_2\mu_3 e_{24}=0\Rightarrow\mu_3=0$, \circled{3}
$(\lambda_1 b_{24}-a_{24}\mu_1)e_{14}=0$, \circled{4}
$(\lambda_2 b_{35}-a_{35}\mu_2+a_{24}\mu_4)e_{25}=0$, and we observe that none of these equations is a linear combination of one or more of the other equations. This means that we have four independent constraints, which means that $\mathrm{codim}\mathcal{C}_{\sfrac{\gamma_1}{\gamma_4}}(x)<\mathrm{codim}\mathcal{C}_{\sfrac{\gamma_1}{\gamma_4}}(e_{23})$.
These four equations can be formed in a $4\times 7$ matrix \[
A=\begin{pmatrix}
-\lambda_2 & \lambda_1 & 0 & 0 & 0 & 0 & 0\\
0 & 0 & \lambda_2 & 0 & 0 & 0 & 0\\
-a_{24} & 0 & 0 & 0 & 0 & \lambda_1 & 0\\
0 & -a_{35} & 0 & a_{24} & 0 & 0 & \lambda_2\\
\end{pmatrix}
\]
which is ordered in relation to the vector \[
v=\begin{pmatrix}
\mu_1\\
\mu_2\\
\mu_3\\
\mu_4\\
b_{13}\\
b_{24}\\
b_{35}\\
\end{pmatrix}
\]
such that $Av=0$. We observe that since $\mathrm{codim}\mathcal{C}_{\sfrac{\gamma_1}{\gamma_4}}(e_{23})=3$, then we must have that any $4\times 4$ minor $B$ of $A$ has that $\det B=0$. Let \[B=\begin{pmatrix}
\lambda_1 & 0 & 0 & 0\\
0 & \lambda_2 & 0 & 0\\
0 & 0 & \lambda_1 & 0\\
-a_{35} & 0 & 0 & \lambda_2\\
\end{pmatrix}\]
be such a minor, so $\det B=\lambda_1^2\lambda_2^2=0$, but we know that $\lambda_2\neq 0$, hence we must have that $\lambda_1=0$, which rules out that $\varphi(e_{23})=\lambda_1 e_{12}+\lambda_2 e_{23}+z_2$. Assume that $\varphi(e_{23})=x=\lambda_2 e_{23}+\lambda_j e_{j,j+1}+z_2$, for some $3\leq j\leq n-1$, we have several cases. If $j=3$, then $\mathfrak{l}(x)=\mathfrak{l}(\lambda_2 e_{23})=1$, because in both cases we have only one sequence of non-zero coefficients, but $\mathfrak{m}(x)=\mathfrak{m}(\lambda_2 e_{23})-1$, because if $\lambda_3=0$ then the zero coefficient in index $4$ is surrounded by two zeros, which is not the case if $\lambda_3\neq 0$, hence $\dim\mathcal{C}_{\sfrac{\gamma_1}{\gamma_3}}(x)=\dim\mathcal{C}_{\sfrac{\gamma_1}{\gamma_3}}(\lambda_2 e_{23})-1$, so also we rule out that $\varphi(e_{23})=\lambda_2 e_{23}+\lambda_3 e_{34}+z_2$. For any $4\leq j\leq n-1$, that is $\varphi(e_{23})=x=\lambda_2 e_{23}+\lambda_j e_{j,j+1}+z_2$, we observe that if $n\geq 6$ then $\mathfrak{l}(x)=\mathfrak{l}(\lambda_2 e_{23})+1$ because we added a new sequence of non-zero coefficients, but $\mathfrak{m}(x)\leq\mathfrak{m}(\lambda_2 e_{23})-2$, because if $4\leq j\leq n-2$ then indices $j-1$, $j$ and $j+1$ are omitted from the count of zeros that are surrounded by two zeros, and if $j=n-1$ then we only omit $n-1$ and $n-2$ from this count, hence $\dim\mathcal{C}_{\sfrac{\gamma_1}{\gamma_3}}(x)\leq\dim\mathcal{C}_{\sfrac{\gamma_1}{\gamma_3}}(\lambda_2 e_{23})-1$. If $n=5$ and $j=4$, that is $\varphi(e_{23})=\lambda_2 e_{23}+\lambda_4 e_{45}+z_2$, we have that $\mathfrak{l}(x)=\mathfrak{l}(\lambda_2 e_{23}+w_2)$ and $\mathfrak{m}(x)=\mathfrak{m}(\lambda_2 e_{23}+w_2)$ hence $\dim\mathcal{C}_{\sfrac{\gamma_1}{\gamma_3}}(x)=\dim\mathcal{C}_{\sfrac{\gamma_1}{\gamma_3}}(\lambda_2 e_{23}+w_2)$, so we need to consider the centralizer of the larger quotient $\mathcal{C}_{\sfrac{\gamma_1}{\gamma_4}}(e_{23})$, and we have the following equations: \circled{1} $-\lambda_2\mu_1 e_{13}=0$, \circled{2} $\lambda_2\mu_3 e_{24}=0$, \circled{3} $-\lambda_4\mu_3 e_{35}=0$, which yield two constraints, $\mu_1=\mu_3=0$, and the equation \circled{4} $(\lambda_2 b_{35}-a_{35}\mu_2+a_{24}\mu_4-\lambda_4 b_{24})e_{25}=0$, hence we have three independent constraints, which is matching to $\dim\mathcal{C}_{\sfrac{\gamma_1}{\gamma_4}}(x)=3=\dim\mathcal{C}_{\sfrac{\gamma_1}{\gamma_4}}(\lambda_2 e_{23}+z_2)$. This means that in this case we need to take even the centralizer of a larger quotient, $\mathcal{C}_{\sfrac{\gamma_1}{\gamma_5}}(e_{e_{23}})=\langle \cancel{e_{12}},e_{23},\cancel{e_{34}},e_{45},e_{13},\cancel{e_{24}},e_{35},e_{14},e_{25},e_{15}\rangle$, which means that $\mathrm{codim}\mathcal{C}_{\sfrac{\gamma_1}{\gamma_5}}(x)=\mathrm{codim}\mathcal{C}_{\sfrac{\gamma_1}{\gamma_5}}(e_{23})=3=\mathrm{codim}\mathcal{C}_{\sfrac{\gamma_1}{\gamma_4}}(x)$, but on the other hand we have another equation, \circled{5} $(a_{13}b_{35}-a_{35}b_{13}+a_{14}\mu_4-\lambda_4 b_{14})e_{15}=0$.
Therefore we have a $4\times 8$ matrix \[
A=\begin{pmatrix}
-\lambda_2 & 0 & 0 & 0 & 0 & 0 & 0 & 0\\
0 & 0 & \lambda_2 & 0 & 0 & 0 & 0 & 0\\
0 & -a_{35} & 0 & a_{24} & 0 & -\lambda_4 & \lambda_2 & 0\\
0 & 0 & 0 & a_{14} & -a_{35} & 0 & a_{13} & -\lambda_4\\
\end{pmatrix}
\]
which is ordered in relation to the vector \[
v=\begin{pmatrix}
\mu_1\\
\mu_2\\
\mu_3\\
\mu_4\\
b_{13}\\
b_{24}\\
b_{35}\\
b_{14}\\
\end{pmatrix}
\]
such that $Av=0$. Here again, we take a $4\times 4$ minor \[B=\begin{pmatrix}
-\lambda_2 & 0 & 0 & 0\\
0 & \lambda_2 & 0 & 0\\
0 & 0 & -\lambda_4 & 0\\
0 & 0 & 0 & -\lambda_4\\
\end{pmatrix}\]
and we have that $\det B=-\lambda_2^2\lambda_4^2=0$, but $\lambda_2\neq 0$ hence $\lambda_4=0$, which rules out the option that $\varphi(e_{23})=\lambda_2 e_{23}+\lambda_4 e_{45}+z_2$. For $n\geq 6$ and any $4\leq j\leq n-1$, it is obvious that if $\varphi(e_{23})=\lambda_2 e_{23}+\lambda_j e_{j,j+1}+z_2$, then $\mathfrak{m}(x)\leq\mathfrak{m}(e_{23})-2$, while $\mathfrak{l}(x)=\mathfrak{l}(e_{23})+1$ hence $\dim\mathcal{C}_{\sfrac{\gamma_1}{\gamma_3}}(x)\leq\dim\mathcal{C}_{\sfrac{\gamma_1}{\gamma_3}}(e_{23})-1$, so we ruled out all the possibilities of the form $\varphi(e_{23})=\lambda_2 e_{23}+\lambda_j e_{j,j+1}$, for all $1\leq j\leq n-1$, and $j\neq 2$, and one checks that all this is true also for $2$ or more non-zero coefficients besides $\lambda_2$. Therefore, we conclude that $\varphi(e_{23})=\lambda_2 e_{23}+z_2$. Continue to $\varphi(e_{34})$, we observe that since $[\varphi(e_{23}),\varphi(e_{34})]=\varphi(e_{24})\neq 0$, we must have that the linear combination which forms $\varphi(e_{34})$ contains $\lambda_3 e_{34}$. Suppose that $\varphi(e_{34})=\lambda_3 e_{34}+\lambda_i e_{i,i+1}$. Assume $i=1$, that is, $\varphi(e_{34})=x=\lambda_1 e_{12}+\lambda_3 e_{34}+z_2$, where $z_2\in\gamma_2\mathcal{L}_{n,p}$, we have that $\mathfrak{l}(\lambda_1 e_{12}+\lambda_3 e_{34})=\mathfrak{l}(\lambda_3 e_{34})+1$, but $\mathfrak{m}(\lambda_1 e_{12}+\lambda_3 e_{34})=\mathfrak{m}(\lambda_3 e_{34})-1$, hence $\dim\mathcal{C}_{\sfrac{\gamma_1}{\gamma_3}}(\lambda_1 e_{12}+\lambda_3 e_{34})=\mathfrak{l}(\lambda_1 e_{12}+\lambda_3 e_{34})+\mathfrak{m}(\lambda_1 e_{12}+\lambda_3 e_{34})=(\mathfrak{l}(\lambda_3 e_{34})+1)+(\mathfrak{m}(\lambda_3 e_{34})-1)=\dim\mathcal{C}_{\sfrac{\gamma_1}{\gamma_3}}(\lambda_3 e_{34})$, so we need to take a larger quotient, as we did for $\varphi(e_{23})$. Computing $\mathcal{C}_{\sfrac{\gamma_1}{\gamma_4}}(x)$, we have the following equations: \circled{1} $\lambda_1\mu_2 e_{13}=0$, \circled{2} $-\lambda_3\mu_2 e_{24}=0$, \circled{3} $\lambda_3\mu_4 e_{35}=0$, \circled{4} $(\lambda_1 b_{24}-a_{24}\mu_1+a_{13}\mu_3-\lambda_3 b_{13})e_{14}=0$, \circled{5} $(\lambda_3 b_{46}-a_{46}\mu_3+a_{35}\mu_5)e_{36}=0$, but obviously, if $n=5$ then $e_{36}$ does not exist in $\mathcal{L}_{5,p}$ hence equation \circled{5} is omitted. We observe that for any $n\geq 6$ we have that $\mathcal{C}_{\sfrac{\gamma_1}{\gamma_4}}(e_{34})=\{e_{12},e_{34},e_{56},e_{67},\dots,e_{n,n-1},e_{24},e_{35},e_{57},e_{68},\dots,e_{n-2,n},e_{14},\dots,e_{n-3,n}\}$, which means that $\mathrm{codim}\mathcal{C}_{\sfrac{\gamma_1}{\gamma_4}\mathcal{L}_{n,p}}(e_{34})=4$, but if $n=5$ then $e_{46}$ does not exist in $\mathcal{L}_{5,p}$, hence $\mathrm{codim}\mathcal{C}_{\sfrac{\gamma_1}{\gamma_4}\mathcal{L}_{5,p}}(e_{34})=3$. We observe that equations \circled{1} and \circled{2} yield the same constraint $\mu_2=0$, hence if $n\geq 6$ we have equations \circled{1}, \circled{3}, \circled{4} and \circled{5} which means that we have $4$ constraints on the dimension of the centralizer, but if $n=5$ we omit equation \circled{5} which means that we have $3$ constraints on the dimension of the centralizer, but this shows that for every $n\geq 5$, we have that $\dim\mathcal{C}_{\sfrac{\gamma_1}{\gamma_4}}(\lambda_2 e_{12}+\lambda_3 e_{34}+\sum_{i=1}^{n-2}a_{i,i+2}e_{i,i+2}+z_3)=\dim\mathcal{C}_{\sfrac{\gamma_1}{\gamma_4}}(e_{34})$. Therefore, we take a larger quotient $\sfrac{\gamma_1}{\gamma_5}$ and add more equations: \circled{6} $(\lambda_1 b_{25}+a_{25}\mu_1+a_{13}b_{35}-a_{35}b_{13})e_{15}=0$, \circled{7} $(-a_{36}\mu_2+a_{24}b_{46}-a_{46}b_{24}+a_{25}\mu_5)e_{26}=0$. Here again, we observe that if $n=5$ then $e_{26}$ does not exist in $\mathcal{L}_{5,p}$ hence equation \circled{7} is omitted, hence and if $n\geq 6$ then the co dimension of the centralizer is $4+2=6$, and if $n=5$ then the co dimension of the centralizer is $3+1=4$, but the complement set to the centralizer of $e_{34}$ in $\sfrac{\gamma_1}{\gamma_5}$ for any $n\geq 6$ is $\{e_{23},e_{45},e_{13},e_{46},e_{47}\}$ hence for $n\geq 6$ we have that the co dimension of the centralizer is $5<6$, and for $n=5$ we have that $e_{46}$ and $e_{47}$ are omitted from the complement set, hence the co dimension of the centralizer is $3<4$, which shows that for any $n\geq 5$ we have that $\dim\mathcal{C}_{\sfrac{\gamma_1}{\gamma_5}}(\lambda_2 e_{12}+\lambda_3 e_{34}+\sum_{i=1}^{n-2}a_{i,i+2}e_{i,i+2}+\sum_{i=1}^{n-3}a_{i,i+3}e_{i,i+3}+z_4)=\dim\mathcal{C}_{\sfrac{\gamma_1}{\gamma_5}}(e_{34})-1$. We realize that in order to match the number of constraints to the dimension of the centralizer, we need to remove one equation or present it as a scalar multiplication of another constraint. We observe that since there is a match, for the quotient $\sfrac{\gamma_1}{\gamma_4}$, then the equation that must be removed or compromised is equation $\circled{6}$, because it is an equation on the coefficients of $e_{15}\notin\sfrac{\gamma_1}{\gamma_4}$. We prove that by setting the four equations in the following $4\times 9$ coefficient matrix \[A=\begin{pmatrix}
0 & -\lambda_3 & 0 & 0 & 0 & 0 & 0 & 0 & 0\\
0 & 0 & 0 & \lambda_3 & 0 & 0 & 0 & 0 & 0\\
-a_{24} & 0 & a_{13} & 0 & -\lambda_3 & \lambda_1 & 0 & 0 & 0\\
-a_{25} & 0 & 0 & 0 & -a_{35} & 0 & a_{13} & 0 & \lambda_1\\
\end{pmatrix}\]
which is ordered in relation to the vector \[
v=\begin{pmatrix}
\mu_1\\
\mu_2\\
\mu_3\\
\mu_4\\
b_{13}\\
b_{24}\\
b_{35}\\
b_{14}\\
b_{25}\\
\end{pmatrix}
\]
such that $Av=0$. But $\mathrm{codim}\mathcal{C}_{\sfrac{\gamma_1}{\gamma_5}}(e_{34})=3$ hence $\mathrm{rank}A=3$, which means that any $4\times 4$ minor $B$ of matrix $A$ has that $\mathrm{det}B=0$. Take the minor \[
B_1=\begin{pmatrix}
-\lambda_3 & 0 & 0 & 0\\
0 & \lambda_3 & 0 & 0\\
0 & 0 & -\lambda_3 & 0\\
0 & 0 & -a_{35} & \lambda_1\\ 
\end{pmatrix}
\]
which means that we take from $A$ the columns $2$, $4$, $5$ and $9$, and since $\det B_1=\lambda_3^3\lambda_1=0$, but we know that $\lambda_3\neq 0$, hence we must have that $\lambda_1=0$. Replacing column $9$ by column $7$, we get another minor  \[
B_2=\begin{pmatrix}
-\lambda_3 & 0 & 0 & 0\\
0 & \lambda_3 & 0 & 0\\
0 & 0 & -\lambda_3 & 0\\
0 & 0 & -a_{35} & a_{13}\\ 
\end{pmatrix}
\]
which leads to the conclusion that $a_{13}=0$. We can already observe here that taking the minor \[
B_3=\begin{pmatrix}
-\lambda_3 & 0 & 0 & 0\\
0 & \lambda_3 & 0 & 0\\
0 & 0 & -\lambda_3 & -a_{24}\\
0 & 0 & -a_{35} & -a_{25}\\ 
\end{pmatrix}
\]
by taking columns $2$, $4$, $5$ and $1$, and computing the equation $\det B_3=-\lambda_3^2(\lambda_3 a_{25}-a_{24}a_{35})=0$, based on the fact that $\lambda_3\neq 0$, yields the relation $\lambda_3 a_{25}-a_{24}a_{35}=0$ between coefficients of $x$, which gives us information about coefficients in the next blocks, but we shall analyze these relations later. So we ruled out that $\varphi(e_{34})=\lambda_1 e_{12}+\lambda_3 e_{34}+z_2$, we assume now that $\varphi(e_{34})=x=\lambda_2 e_{23}+\lambda_3 e_{34}+z_2$, but then we have that $\mathfrak{m}(x)=\mathfrak{m}(\lambda_3 e_{34})-1$ while $\mathfrak{l}(x)=\mathfrak{l}(\lambda_3 e_{34})$, because in both cases there is only one sequence of non-zero coefficients, hence $\dim\mathcal{C}_{\sfrac{\gamma_1}{\gamma_3}}(x)<\dim\mathcal{C}_{\sfrac{\gamma_1}{\gamma_3}}(\lambda_3 e_{34})$, so we rule out also that $\varphi(e_{34})=\lambda_2 e_{23}+\lambda_3 e_{34}+z_2$. Assume that $\varphi(e_{34})=x=\lambda_3 e_{34}+\lambda_4 e_{45}+z_2$. For $n\geq 6$, we observe that it is a similar case to $\lambda_2 e_{23}+\lambda_3 e_{34}+z_2$, but for $n=5$ we observe that $\mathfrak{l}(x)=\mathfrak{l}(\lambda_3 e_{34})=1$ and $\mathfrak{m}(x)=\mathfrak{m}(\lambda_3 e_{34})=1$, so we need to look at the centralizer of a larger quotient, $\mathcal{C}_{\sfrac{\gamma_1}{\gamma_4}}(e_{34})$, so we have the following equations: \circled{1} $-\lambda_3\mu_2 e_{24}=0$, \circled{2} $(\lambda_3\mu_4-\lambda_4\mu_3)e_{35}=0$, $(-a_{24}\mu_1+a_{13}\mu_3-\lambda_3 b_{13})e_{14}=0$, \circled{4} $(a_{24}\mu_4-\lambda_4 b_{24})e_{25}=0$, but $\mathrm{codim}\mathcal{C}_{\sfrac{\gamma_1}{\gamma_4}}(e_{34})=3$, because $e_{46}\notin\mathcal{L}_{5,p}$, which means that the four equations must yield only three independent constraints, hence we have the following $4\times 6$ matrix: \[A=\begin{pmatrix}
0 & -\lambda_3 & 0 & 0 & 0 & 0\\
0 & 0 & -\lambda_4 & \lambda_3 & 0 & 0\\
-a_{24} & 0 & a_{13} & 0 & -\lambda_3 & 0\\
0 & 0 & 0 & a_{24} & 0 & -\lambda_4\\
\end{pmatrix}\]
which is ordered in relation to the vector \[
v=\begin{pmatrix}
\mu_1\\
\mu_2\\
\mu_3\\
\mu_4\\
b_{13}\\
b_{24}\\
\end{pmatrix}
\]
such that $Av=0$. We take the minor \[B=\begin{pmatrix}
-\lambda_3 & 0 & 0 & 0\\
0 & \lambda_3 & 0 & 0\\
0 & 0 & -\lambda_3 & 0\\
0 & a_{24} & 0 & -\lambda_4\\
\end{pmatrix}\]
We observe that $\det B=-\lambda_3^3\lambda_4=0$, but we know that $\lambda_3\neq 0$ hence we must have that $\lambda_4=0$, which rules out that $\varphi(e_{34})=\lambda_3 e_{34}+\lambda_4 e_{45}+z_2$. For $n=6$, assume that $\varphi(e_{34})=\lambda_3 e_{34}+\lambda_5 e_{56}+z_2$, and we observe that $\mathfrak{l}(x)=\mathfrak{l}(e_{34})+1$, but $\mathfrak{m}(x)=\mathfrak{m}(e_{34})-1$, hence $\dim\mathcal{C}_{\sfrac{\gamma_1}{\gamma_3}}(x)=\dim\mathcal{C}_{\sfrac{\gamma_1}{\gamma_3}}(e_{34})$, so here again we look at a larger quotient, and we have the following equations: \circled{1} $-\lambda_3\mu_2 e_{24}=0$, \circled{2} $\lambda_3\mu_4 e_{35}=0$, \circled{3} $-\lambda_5\mu_4 e_{46}=0$, \circled{4} $(-\mu_1 a_{24}+a_{13}\mu_3-\lambda_3 b_{13})e_{14}=0$, \circled{5} $(\lambda_3 b_{46}-a_{46}\mu_3+a_{35}\mu_5-b_{35}\lambda_5)e_{36}$, and we observe that equations \circled{2} and \circled{3} yield the same constraint $\mu_4=0$, hence we have four independent constraints, which is matching to $\mathrm{codim}\mathcal{C}_{\sfrac{\gamma_1}{\gamma_4}}(e_{34})=4$. Therefore, we look at a larger quotient, and add the equations \circled{6} $-a_{25}\mu_1+a_{13}b_{35}-a_{35}b_{13} e_{15}=0$ and \circled{7} $(a_{24}b_{46}-a_{46}b_{24}+a_{25}\mu_5-\lambda_5 b_{25})e_{26}$, but $\mathrm{codim}\mathcal{C}_{\sfrac{\gamma_1}{\gamma_5}}(e_{34})=4$, because $e_{47}\notin\mathcal{L}_{6,p}$, hence these equations cannot yield more independent constraints, hence we have the following $5\times 11$ matrix: \[A=\begin{pmatrix}
0 & -\lambda_3 & 0 & 0 & 0 & 0 & 0 & 0 & 0 & 0 & 0\\
0 & 0 & 0 & \lambda_3 & 0 & 0 & 0 & 0 & 0 & 0 & 0\\
-a_{24} & 0 & a_{13} & 0 & 0 & -\lambda_3 & 0 & 0 & 0 & 0 & 0\\
0 & 0 & -a_{46} & 0 & 0 & a_{35} & 0 & -\lambda_5 & \lambda_3 & 0 & 0\\
0 & 0 & 0 & 0 & a_{25} & 0 & -a_{46} & 0 & a_{24} & 0 & \lambda_5\\
\end{pmatrix}\]
which is ordered in relation to the vector \[
v=\begin{pmatrix}
\mu_1\\
\mu_2\\
\mu_3\\
\mu_4\\
\mu_5\\
b_{13}\\
b_{24}\\
b_{35}\\
b_{46}\\
b_{14}\\
b_{25}\\
\end{pmatrix}
\]
such that $Av=0$. We take the $5\times 5$ minor \[
B=\begin{pmatrix}
-\lambda_3 & 0 & 0 & 0 & 0\\
0 & \lambda_3 & 0 & 0 & 0\\
0 & 0 & -\lambda_3 & 0 & 0\\
0 & 0 & a_{35} & -\lambda_5 & 0\\
0 & 0 & 0 & 0 & \lambda_5\\
\end{pmatrix}
\]
and we observe that $\det B=-\lambda_3^3\lambda_5^2$, but we know that $\lambda_3\neq 0$, hence we must have that $\lambda_5=0$, which rules out that $\varphi(e_{34})=\lambda_3 e_{34}+\lambda_5 e_{56}+z_2$. For $n\geq 7$, if $\varphi(e_{34})=x=\lambda_3 e_{34}+\lambda_i e_{i,i+1}+z_2$, for any $5\leq i\leq n-1$, then $\dim\mathcal{C}_{\sfrac{\gamma_1}{\gamma_3}}(x)<\dim\mathcal{C}_{\sfrac{\gamma_1}{\gamma_3}}(\lambda_3 e_{34})$, and one checks that if $\varphi(e_{34})=\lambda_3 e_{34}+\sum_{i=1}\lambda_i e_{i,i+1}$, where at least two indices besides $3$ are not zero, then $\dim\mathcal{C}_{\sfrac{\gamma_1}{\gamma_3}}(x)<\dim\mathcal{C}_{\sfrac{\gamma_1}{\gamma_3}}(\lambda_3 e_{34})$, so any option besides $\varphi(e_{34})=\lambda_3 e_{34}+z_2$ is ruled out. For $\varphi(e_{45})$, we already know from the same considerations we made earlier for $e_{23}$ and $e_{34}$ that we need to have $\lambda_4 e_{45}$ in the linear combination that forms $\varphi(e_{45})$. To rule out any other option for $\varphi(e_{45})$, we first observe that since we have $4$ coefficient indices before index $4$, including $\lambda_0=0$, then if $\lambda_1\neq$ we remove both $\lambda_1$ and its neighboring $\lambda_2$ from the count of zero coefficients that are surrounded by two zeros, if $\lambda_2\neq 0$ then we remove both $\lambda_2$ and $\lambda_1$ from this count, and if $\lambda_3\neq 0$ we remove $\lambda_2$ alone from this count, which means that if either $x=\varphi(e_{45})=\lambda_1 e_{12}+\lambda_4 e_{45}+z_2$ or $x=\varphi(e_{45})=\lambda_2 e_{23}+\lambda_4 e_{45}+z_2$, where $z_2\in\gamma_2\mathcal{L}_{n,p}$, then $\mathfrak{m}(x)=\mathfrak{m}(e_{45})-1$, and if $x=\varphi(e_{45})=\lambda_3 e_{34}+\lambda_4 e_{45}+z_2$ then $\mathfrak{m}(x)=\mathfrak{m}(e_{45})-2$. We also observe that if either $x=\varphi(e_{45})=\lambda_1 e_{12}+\lambda_4 e_{45}+z_2$ or $x=\varphi(e_{45})=\lambda_2 e_{23}+\lambda_4 e_{45}+z_2$ then $\mathfrak{l}(x)=\mathfrak{l}(e_{45})+1$, and if $x=\varphi(e_{45})=\lambda_3 e_{34}+\lambda_4 e_{45}+z_2$ then $\mathfrak{l}(x)=\mathfrak{l}(e_{45})$, since $\lambda_3$ and $\lambda_4$ are part of the same sequence of non-zero coefficients of $x$, which means in either of these three cases, either $\mathfrak{l}(x)+\mathfrak{m}(x)=(\mathfrak{l}(e_{34})+1)+(\mathfrak{m}(e_{45})-2)=\mathfrak{l}(e_{34})+\mathfrak{m}(e_{45})-1$ or $\mathfrak{l}(x)+\mathfrak{m}(x)=\mathfrak{l}(e_{34})+(\mathfrak{m}(e_{45})-1)$, which means that none of these cases can hold for $x=\varphi(e_{45)}$. One easily checks that there is no linear combination of the form $x_4=\lambda_1 e_{12}+\lambda_2 e_{23}+\lambda_3 e_{34}$, where at least two coefficients are non-zero, such that $x=\varphi(e_{45})=x_4+\lambda_4 e_{45}+z_2$ satisfies $\dim\mathcal{C}_{\sfrac{\gamma_1}{\gamma_3}}(x)=\dim\mathcal{C}_{\sfrac{\gamma_1}{\gamma_3}}(e_{45})$



We observe that this is true for the entire first block, that is, $\varphi(e_{j,j+1})=\lambda_j e_{j,j+1}$, where $1\leq j\leq n-1$, which means that the block $M_{11}$ is diagonal, if we assume that $\varphi(e_{12})=\lambda_1 e_{12}$, and it is immediate to follow the course of this proof with the assumption that $\varphi(e_{12})=\lambda_{n-1}e_{n-1,n}+w$, where $w\in\gamma_2\mathcal{L}_{n,p}$, and prove that in this case the block $M_{11}$ is anti-diagonal.
\end{proof}
\begin{proposition}
\label{m11.n.4}
Let $\varphi\in G_4(\mathbb{Q}_p)$, then $M_{11}$ is either diagonal or anti-diagonal.
\end{proposition}
\begin{proof}
We observe that $\mathcal{C}_{\sfrac{\gamma_1}{\gamma_3}}(e_{12})=\langle e_{12},e_{34},e_{13},e_{24}\rangle$, hence $\dim\mathcal{C}_{\sfrac{\gamma_1}{\gamma_3}}(\varphi(e_{12}))=\dim\mathcal{C}_{\sfrac{\gamma_1}{\gamma_3}}(e_{12})=4=(4-1)+(4-2)-1=\dim\sfrac{\gamma_1}{\gamma_3}-1$, but by \ref{prop.n.geq.4.centralizer.codimension}, we have that $\varphi(e_{12})=x=\lambda_1 e_{12}+\lambda_3 e_{34}+z_2$, where $\lambda_1$ and $\lambda_3$ are not both zero. Looking at the centralizer of a larger quotient, we have that $\mathcal{C}_{\sfrac{\gamma_1}{\gamma_4}}(e_{12})=\langle e_{12},e_{34},e_{13},e_{14}\rangle$. For any $y=\mu_1 e_{12}+\mu_2 e_{23}+\mu_3 e_{34}+a_{13}e_{13}+a_{24}e_{24}+z_3$, such that $[x,y]=0$, we have the following equations: \circled{1} $\lambda_1\mu_2 e_{13}=0$, \circled{2} $-\lambda_3\mu_2 e_{24}=0$, \circled{3} $(\lambda_1 b_{24}-a_{24}\mu_1+a_{13}\mu_3-\lambda_3 b_{13})e_{14}=0$, but both equations \circled{1} and \circled{2} yield the same constraint $\mu_2=0$, which means that we have two constraints on the centralizer of $x$, hence $\mathrm{codim}\mathcal{C}_{\sfrac{\gamma_1}{\gamma_4}}(x)=\mathrm{codim}\mathcal{C}_{\sfrac{\gamma_1}{\gamma_4}}(e_{12})=\mathrm{codim}\mathcal{C}_{\sfrac{\gamma_1}{\gamma_4}}(\lambda_1 e_{12})=2$, which means that we cannot determine $\varphi(e_{12})$ by dimension considerations alone. We observe that $\mathcal{C}_{\sfrac{\gamma_1}{\gamma_3}}(e_{23})=\langle e_{23},e_{13},e_{24}\rangle$, which means that $\dim\mathcal{C}_{\sfrac{\gamma_1}{\gamma_3}}(e_{23})=3<4=\dim\sfrac{\gamma_1}{\gamma_3}-1$, hence by \ref{prop.n.geq.4.centralizer.codimension} we have that $\varphi(e_{23})\neq\lambda_1 e_{12}+\lambda_3 e_{34}+z_2$ where $\lambda_1$ and $\lambda_3$ are not both zero, which means that we must have $\lambda_2 e_{23}$ in the linear combination that forms $\varphi(e_{23})$. Suppose that $\varphi(e_{23})=x=\lambda_1 e_{12}+\lambda_2 e_{23}+z_2$, we have the constraints \circled{1} $\lambda_2\mu_3 e_{23}=0$ and \circled{2} $(\lambda_1\mu_2-\lambda_2\mu_1)e_{13}=0$, which means that $\mu_3=0$ and $\mu_1=\frac{\lambda_1}{\lambda_2}\mu_2$, hence $\mathrm{codim}\mathcal{C}_{\sfrac{\gamma_1}{\gamma_3}}(x)=2=\mathrm{codim}\mathcal{C}_{\sfrac{\gamma_1}{\gamma_3}}(e_{23})=\mathrm{codim}\mathcal{C}_{\sfrac{\gamma_1}{\gamma_3}}(\lambda_2 e_{23}+z_2)$. We take the centralizer of a larger quotient, $\mathcal{C}_{\sfrac{\gamma_1}{\gamma_4}}(e_{23})=\langle e_{23},e_{13},e_{24},e_{14}\rangle$, which means that $\mathrm{codim}\mathcal{C}_{\sfrac{\gamma_1}{\gamma_4}}(e_{23})=2$, but then we have an additional equation \circled{4} $(\lambda_1 b_{24}-a_{24}\mu_1+a_{13}\mu_3)e_{14}=0$, but since we already know that $\mu_3=0$, then equation \circled{4} turns into $(\lambda_1 b_{24}-a_{24}\mu_1)e_{14}=0$, which means that $b_{24}=\frac{a_{24}}{\lambda_1}\mu_1=\frac{a_{24}}{\lambda_1}\frac{\lambda_1}{\lambda_2}\mu_2=\frac{a_{24}}{\lambda_2}\mu_2$, hence the number of constraints remains $2$, hence $\mathrm{codim}\mathcal{C}_{\sfrac{\gamma_1}{\gamma_4}}(x)=2=\mathrm{codim}\mathcal{C}_{\sfrac{\gamma_1}{\gamma_4}}(e_{23})=\mathrm{codim}\mathcal{C}_{\sfrac{\gamma_1}{\gamma_4}}(\lambda_2 e_{23})$. This means that we cannot determine also $\varphi(e_{23})$ only by computing dimensions of centralizers.
\end{proof}
\begin{proposition}
\label{prop.involution}
Let $\mathcal{L}_{n,p}$ be the $\mathbb{Q}$-Lie algebra associated with $\mathcal{U}_n(\mathbb{Q}_p)$, and let $\mathcal{B}_n=\{e_{12},e_{23},\dots,e_{1n}\}$ be its basis. Then, the map $\eta:\mathcal{B}_n\rightarrow \mathcal{B}_n$, defined by \[\eta(e_{ij}):=(-1)^{j-i-1}e_{n+1-j,n+1-i}\] is an involution, hence, also an $\mathcal{L}_{n,p}$-automorphism.
\end{proposition}
\begin{proof}
Clearly, we calculate $\eta^2(e_{ij})$ by taking $k=n+1-j$ and $l=n+1-i$. Hence, \[\eta(\eta(e_{ij}))=\eta((-1)^{j-i-1}e_{kl})=(-1)^{j-i-1}\eta(e_{ij})=(-1)^{j-i-1}(-1)^{l-k-1}e_{n+1-l,n+1-k}=\]\[=(-1)^{j-i-1}(-1)^{n+1-i-(n+1-j)-1}e_{n+1-(n+1-i),n+1-(n+1-j)}=(-1)^{j-i-1}(-1)^{j-i-1}e_{ij}=\]\[=(-1)^{2(j-i-1)}e_{ij}=e_{ij}\]
To complete the proof, we need to show that $\eta$ is indeed a homomorphism,
\[[\eta(e_{ij}),\eta(e_{kl})]=[(-1)^{j-i-1}e_{n+1-j,n+1-i},(-1)^{l-k-1}e_{n+1-l,n+1-k}]=\]\[=(-1)^{j-i-1}(-1)^{l-k-1}[e_{n+1-j,n+1-i},e_{n+1-l,n+1-k}]=\]\[(-1)^{j-i+l-k-2}[e_{n+1-j,n+1-i},e_{n+1-l,n+1-k}]\]
We observe that if $n+1-i=n+1-l$ or $n+1-j=n+1-k$ then the Lie brackets evaluate to a non-zero element, but this happens if and only if $i=l$ or $j=k$, respectively. Assume $j=k$, then \[(-1)^{j-i+l-k-2}[e_{n+1-j,n+1-i},e_{n+1-l,n+1-k}]=(-1)^{l-i-2}[e_{n+1-j,n+1-i},e_{n+1-l,n+1-j}]=\]\[(-1)^{l-i-2}\cdot-e_{n+1-l,n+1-i}=(-1)^{l-i-1}e_{n+1-l,n+1-i}=\eta(e_{il})=\eta[e_{ij},e_{jl}]\] 
\end{proof}
\begin{corollary}
\label{m11.diagonal}
Let $\mathcal{L}_{n,p}$, where $n\geq 4$, then the block $M_{11}$ is diagonal.
\end{corollary}
\begin{proof}
By \ref{m11.n.5} and \ref{m11.n.4}, we have that $M_{11}$ is either diagonal or anti-diagonal. Assume that $M_{11}$ is anti-diagonal, that is, $\varphi(e_{12})=\lambda_{n-1}e_{n-1,n}+z_{n-1},\varphi(e_{23})=\lambda_{n-2} e_{n-2,n-1}+z_{n-2},\dots,\varphi(e_{n-1,n})=\lambda_1 e_{12}+z_1$, where $z_1,z_2,\dots,z_{n-1}\in\gamma_2\mathcal{L}_{n,p}$, then we define a new automorphism by the composition $\psi=\varphi\eta$. Let $e_{i,i+1}\in\sfrac{\gamma_1}{\gamma_2}$, where $1\leq i\leq n-1$, then $\psi(e_{i,i+1})=\varphi\eta(e_{i,i+1})=\varphi((-1)^{i+1-i-1}e_{n+1-(i+1),n+1-i})=\varphi(e_{n-i,n-i+1})
$, hence $\psi(e_{12})=\varphi(e_{n-1,n})=\lambda_1 e_{12}+z_1,\psi(e_{23})=\varphi(e_{n-2,n-1})=\lambda_2 e_{23}+z_2,\dots,\psi(e_{n-1,n})=\varphi(e_{12})=\lambda_{n-1}e_{n-1,n}+z_{n-1}$, which means that $M_{11}$, as the first block of the matrix representing $\psi$, is diagonal.  
\end{proof}
\begin{corollary}
\label{main.diagonal.blocks}
Let $\mathcal{L}_{n,p}$, where $n\geq 4$, then all the blocks $M_{11},M_{22},\dots,M_{n-1,n-1}$ are diagonal.
\end{corollary}
\begin{proof}
By simple induction on the index $1\leq k\leq n-1$ of the diagonal block. For $k=1$ we proved this in \ref{m11.diagonal}, for $k+1$ we observe that for every $1\leq i\leq n-k$, we have the identity $\varphi(e_{i,i+k+1})=\varphi([e_{i,i+j},e_{i+j,i+k+1}])=[\varphi(e_{i,i+j}),\varphi(e_{i+j,i+k+1})]$, for $1\leq j\leq k$, but by the assumption we have that $\varphi(e_{i,i+j})=a_{i,i+j}e_{i,i+j}+z_{j+1}$ and $\varphi(e_{i+j,i+k+1})=a_{i+j,i+k+1}e_{i+j,i+k+1}+z_{k+2-j}$, where $z_{j+1}\in\gamma_{j+1}\mathcal{L}_{n,p}$ and $z_{k+2-j}\in\gamma_{k+2-j}\mathcal{L}_{n,p}$, hence $\varphi(e_{i,i+k+1})=[a_{i,i+j}e_{i,i+j}+z_{j+1},a_{i+j,i+k+1}e_{i+j,i+k+1}+z_{k+2-j}]=a_{i,i+j}a_{i+j,i+k+1}e_{i,i+k+1}+a_{i,i+j}e_{i,i+j}z_{k+2-j}+a_{i+j,i+k+1}e_{i+j,i+k+1}z_{j+1}+z_{k+2-j}z_{j+1}=a_{i,i+j}a_{i+j,i+k+1}e_{i,i+k+1}+z_{k+2-j}$, where $z_{k+2-j}\in\gamma_{k+2-j}\mathcal{L}_{n,p}$, which proves the induction step.
\end{proof}
\begin{corollary}
Let $\mathcal{L}_{n,p}$, where $n\geq 4$, then the main diagonal is
\[\begin{pmatrix}
\lambda_1 \\
& \lambda_2  \\
& & \ddots\\
& & & \lambda_{n-1}\\
& & & &\lambda_1\lambda_2\\
& & & & &\ddots\\
& & & & & &\lambda_{n-1}\lambda_{n-2}\\
& & & & & & &\ddots\\
& & & & & & & &\lambda_1\lambda_2\cdots\lambda_{n-2}\lambda_{n-1}\\
\end{pmatrix}\]
where $\lambda_1,\lambda_2,\dots,\lambda_{n-1}\in\mathbb{Q}_p$, such that for every $1\leq k\leq n-1$ and for every $1\leq i\leq n-k$, we have that $\varphi(e_{i,i+k})=\lambda_i\lambda_{i+1}\lambda_{i+2}\cdots\lambda_{i+k-2}\lambda_{i+k-1}e_{i,i+k}+z_{k+1}$, where $z_{k+1}\in\gamma_{k+2}\mathcal{L}_{n,p}$
\end{corollary}
\begin{proof}
Following the proof of \ref{main.diagonal.blocks}, we have that for every $1\leq k\leq n-1$ and for every $1\leq i\leq n-k$, $\varphi(e_{i,i+k})=\lambda_{i,i+k}e_{i,i+k}+z_{k+1}$, where $z_{k+1}\in\gamma_{k+1}\mathcal{L}_{n,p}$, and we use the same method of induction on $k$. For $k=1$, we already proved in \ref{m11.n.5} that $\varphi(e_{12})=\lambda_1 e_{12}+z_1,\varphi(e_{23})=\lambda_2 e_{23}+z_2,\dots,\varphi(e_{n-1,n})=\lambda_{n-1}e_{n-1,n}+z_{n-1}$, where $z_i\in\gamma_i\mathcal{L}_{n,p}$, for every $1\leq i\leq n-1$. For $k+1$, we have that $\varphi(e_{i,i+k+1})=\varphi([e_{i,i+j},e_{i+j,i+k+1}])=[\varphi(e_{i,i+j}),\varphi(e_{i+j,i+k+1})]$, for every $1\leq j\leq k$, but by the assumption we have that $\varphi(e_{i,i+j})=\lambda_i\lambda_{i+1}\cdots\lambda_{i+j}+z_{i+j+1}$ and $\varphi(e_{i+j,i+k+1})=\lambda_{i+j}\lambda_{i+j+1}\cdots\lambda_{i+k+1}+z_{i+k+2}$, where $z_{i+j+1}\in\gamma_{i+j+1}\mathcal{L}_{n,p}$ and $z_{i+k+2}\in\gamma_{i+k+2}\mathcal{L}_{n,p}$, hence $\varphi(e_{i,i+k+1})=\varphi([e_{i,i+j},e_{i+j,i+k+1}])=[\varphi(e_{i,i+j}),\varphi(e_{i+j,i+k+1})]=[\lambda_i\lambda_{i+1}\cdots\lambda_{i+j}+z_{i+j+1},\lambda_{i+j}\lambda_{i+j+1}\cdots\lambda_{i+k+1}+z_{i+k+2}]=\lambda_i\lambda_{i+1}\lambda_{i+2}\lambda_{i+k}\cdots\lambda_{i+k+1}+z_{i+k+2}$, which proves the induction step.
\end{proof}
\begin{proposition}
Let $\varphi\in G_n(\mathbb{Q}_p)$, and let $M$ be the coefficient matrix of $\varphi$, then the $(n-1)\times(n-2)$ block $M_{12}$, then the two diagonal sequences, $m_{11}=a_{13}^{12},m_{22}=a_{24}^{23},\dots,m_{n-2,n-2}=a_{n-2,n}^{n-2,n-1}$ and $m_{21}=a_{13}^{23},m_{32}=a_{24}^{34},\dots,m_{n-1,n-2}=a_{n-2,n}^{n-1,n}$ are non-zero coefficients and all the other elements of $M_{12}$ are zero,\[M_{12}=\begin{pmatrix}
a_{13}^{12} & 0 & 0 & 0 & 0 & 0\\
a_{13}^{23} & a_{24}^{23} & 0 & 0 & 0 & 0\\
0 & a_{24}^{34} & a_{35}^{34} & 0 & 0 & 0\\
0 & 0 & a_{35}^{45} & a_{46}^{45} & 0 & 0\\
0 & 0 & 0 & a_{46}^{56} & \ddots & 0\\
0 & 0 & 0 & 0 & \ddots & a_{n-2,n}^{n-2,n-1}\\
0 & 0 & 0 & 0 & 0 & a_{n-2,n}^{n-1,n}\\
\end{pmatrix}
\]
\end{proposition}
\begin{proof}
We saw in the proof of \ref{m11.n.5} that for $2\leq i\leq n-2$, we have the two equations $\lambda_i\mu_{i+1}^i e_{i,i+2}=0$ and $-\mu_{i-1}^i\lambda_{i}e_{i-1,i+1}=0$, while for $i=1$ we have the equation $\lambda_1\mu_2^1 e_{13}=0$, and for $i=n-1$ we have the equation $-\mu_{n-2}^{n-1}\lambda_{n-1}e_{n-2,n}=0$, therefore $\mu_2^1=0$, and $\mu_{n-2}^{n-1}=0$, and $\mu_{i-1}^i=\mu_{i+1}^i=0$ for all $2\leq i\leq n-2$, and since $\mathrm{codim}\mathcal{C}_{\sfrac{\gamma_1}{\gamma_1}}(\varphi(e_{i,i+2}))=\mathrm{codim}\mathcal{C}_{\sfrac{\gamma_1}{\gamma_1}}(e_{i,i+2})=2$, then we know that $\mu_j^i\neq 0$, for every $j\neq i-1,i+1$, for all $2\leq i\leq n-2$. We saw earlier that $\varphi(e_{i,i+1})=\lambda_i e_{i,i+1}+a_{13}^i e_{13}+a_{24}^i e_{24}+a_{35}^i e_{35}+\cdots+a_{n-2,n}^i e_{n-2,n}+z_3$, so we have the equation $(\lambda_{i+1}b_{i+2,i+4}^i-\mu_{i+1}^i a_{i+2,i+4}^i +a_{i+1,i+3}^i\mu_{i+3}^i-b_{i+1,i+3}^i\lambda_{i+3})e_{i+1,i+4}=0$, but $\lambda_{i+1}=\lambda_{i+3}=0$, and since $\mu_{i-1}^i=\mu_{i+1}^i=0$, and we know that there cannot be more $\mu$ coefficients that are zero by constraint, then $\mu_{i+1}^i=0$ and $\mu_{i+3}^i\neq 0$, hence the equation becomes $a_{i+1,i+3}^i\mu_{i+3}^i e_{i+1,i+4}=0$, and since $\mu_{i+3}^i\neq 0$, we must have that $a_{i+1,i+3}^i=0$. Precisely the same way we have the equation $(\lambda_{i-3}b_{i-2,i}^i-\mu_{i-3}^i a_{i-2,i}^i+a_{i-3,i-1}^i\mu_{i-1}^i-b_{i-3,i-1}^i\lambda_{i-1})e_{i-3,i}$, but $\lambda_{i-3}=\lambda_{i-1}=0$, and as we saw earlier, $\mu_{i-1}^i=0$ and $\mu_{i-3}^i\neq 0$, hence $a_{i-2,i}^i=0$. In the same way as for \ref{m11.n.4}, we observe that $\varphi(e_{i,i+2})=[\varphi(e_{i,i+1}),\varphi(e_{i+1,i+2})]=\lambda_i\lambda_{i+1}e_{i,i+2}+\lambda_i a_{i+1,i+3}^{i+1}e_{i,i+3}-a_{i-2,i}^{i+1}\lambda_i e_{i-2,i+1}e_{i-2,i+1}+\lambda_{i+1}a_{i+2,i+4}^i e_{i+1,i+4}-a_{i-1,i+1}^i\lambda_{i+1}e_{i-1,i+2}+z_4$, and we compute $[\varphi(e_{i,i+1}),\varphi(e_{i,i+2})]=\lambda_i\lambda_{i+1}a_{i+2,i+4}^i e_{i,i+4}=0$, but that is possible only if either of $\lambda_i$, $\lambda_{i+1}$ and $a_{i+2,i+4}^i$ is zero, and since $\lambda_i\neq 0$ and $\lambda_{i+1}\neq 0$, then we must have that $a_{i+2,i+4}^i=0$. Precisely the same way, we compute $[\varphi(e_{i+1,i+2}),\varphi(e_{i,i+2})]=-a_{i-2,i}^{i+1}\lambda_i\lambda_{i+1}e_{i-2,i+2}=0$, but that is possible only if either of $\lambda_i$, $\lambda_{i+1}$ and $a_{i-2,i}^{i+1}$ is zero, and since $\lambda_i\neq 0$ and $\lambda_{i+1}\neq 0$, then we must have that $a_{i-2,i}^{i+1}=0$. Therefore, for all $1\leq i\leq n-2$ we have that $a_{i+1,i+3}^i=a_{i+2,i+4}^i=a_{i-2,i}^i=0$, and we have the equation $(\lambda_j b_{j+1,j+3}^i-\mu_j^i a_{j+1,j+3}^i+a_{j,j+2}^i\mu_{j+2}^i-b_{j+1,j+3}^i\lambda_{j+2})e_{j,j+3}=0$, for all $j\geq i+2$. For $j=i+2$, $(\lambda_{i+2}b_{i+3,i+5}^i-\mu_{i+2}^i a_{i+3,i+5}^i+a_{i+2,i+4}^i\mu_{i+4}^i-b_{i+3,i+5}^i\lambda_{i+4})e_{i+2,i+5}=0$, and we know that $\lambda_{i+2}=\lambda_{i+4}=0$, and we saw that $a_{i+1,i+3}^i=0$, hence the equation becomes $-\mu_{i+2}^i a_{i+3,i+5}^i=0$, but since we know that the only zero $\mu$ coefficients are $\mu_{i-1}=\mu_{i+1}=0$, then $\mu_{i+2}\neq 0$, hence $a_{i+3,i+5}^i=0$.








$\lambda_2=\lambda_4=0$ and $\mu_2^{12}=0$, hence this equation becomes $a_{24}^{12}\mu_4^{12}=0$, and since $\mu_4^{12}\neq 0$, then $a_{24}^{12}=0$. Also, $\varphi(e_{23})=\lambda_2 e_{23}+a_{13}^{23}e_{13}+a_{24}^{23}e_{24}+a_{35}^{23}e_{35}+w_3$, and we have the equation $(\lambda_3 b_{46}^{23}-\mu_3^{23}a_{46}^{23}+a_{35}^{23}\mu_5^{23}-b_{35}^{23}\lambda_5)e_{36}=0$, but $\lambda_3=\lambda_5=0$ and $\mu_3^{23}=0$, hence this equation becomes $a_{35}^{23}\mu_5^{35}=0$, and since $\mu_5\neq 0$, we have that $a_{35}^{23}=0$. Therefore $\varphi(e_{13})=[\varphi(e_{12}),\varphi(e_{23})]=[\lambda_1 e_{12}+a_{13}^{12}e_{13}+a_{35}^{12}e_{35}+z_3,\lambda_2 e_{23}+a_{13}^{23}e_{13}+a_{24}^{23}e_{24}+w_3]=\lambda_1\lambda_2 e_{13}+\lambda_1 a_{24}^{23}e_{14}-\lambda_2 a_{35}^{12}e_{25}-a_{35}^{12}a_{13}^{23}e_{15}+u_4$. We observe that $[\varphi(e_{12}),\varphi(e_{13})]=\varphi([e_{12},e_{13}])=0=-\lambda_1\lambda_2 a_{35}^{12}e_{15}+\lambda_1$ 

















For every $1\leq i\leq n-1$, let $y_i=\sum_{j=1}^{n-1}\mu_j^i e_{j,j+1}+z_2\in\mathcal{C}_{\sfrac{\gamma_1}{\gamma_3}}(\varphi(e_{i,i+1}))$ be an element in the centralizer of $\varphi(e_{i,i+1})$ in the quotient $\sfrac{\gamma_1}{\gamma_3}$, then for every $1\leq i\leq n-1$, we have the following set of equations: \[\{(\lambda_{j}b_{j+1,j+3}^i-\mu_{j}^i a_{j+1,j+3}^i+a_{j,j+2}^i\mu_{j+2}^i-\lambda_{j+2}b_{j,j+2}^i)e_{j,j+3}=0\}_{j=1}^{n-3}\]
We already know that for every $1\leq i\leq n-1$, only $\lambda_i\neq 0$, while $\lambda_j=0$ for every $1\leq j\neq i\leq n-1$. In addition, we have the two following sets of equations: \[\{\lambda_i\mu_{i+1}^i e_{i,i+2}=0\}_{i=1}^{n-2},\{-\lambda_i\mu_{i-1}^i e_{i-1,i+1}=0\}_{i=2}^{n-1}\]
which means that $\mu_{i-1}^i=0$, for every $2\leq i\leq n-1$ and $\mu_{i+1}^i=0$, for every $1\leq i\leq n-2$. We have already seen that besides these constraints, there are no other constraints on $\mathcal{C}_{\sfrac{\gamma_1}{\gamma_3}}(\varphi(e_{i,i+1}))$, hence we have that $\mu_j^i\neq 0$, for every $j$ except for $j=i-1$ and $j=i+1$. Therefore, every equation of the form $(\lambda_{j}b_{j+1,j+3}^i-\mu_{j}^i a_{j+1,j+3}^i+a_{j,j+2}^i\mu_{j+2}^i-\lambda_{j+2}b_{j,j+2}^i)e_{j,j+3}=0$, for every $j=i+1$, has that $\lambda_j b_{j+1,j+3}^i=\lambda_{i+1}b_{i+2,i+4}^i=0$, because $\lambda_{i+1}=0$, and $\mu_j^i a_{j+1,j+3}^i=\mu_{i+1}^i a_{i+2,i+4}^i=0$, because $\mu_{i+1}^i=0$ and $-\lambda_{j+2}b_{j,j+2}^i=\lambda_{i+3}b_{i+1,i+3}^i=0$, because $\lambda_{i+3}=0$, but since $\mu_{j+2}^i=\mu_{i+3}^i\neq 0$, then the equation becomes $a_{j,j+2}^i\mu_{j+2}^i e_{j,j+3}=a_{i+1,i+3}^i\mu_{i+3}^i e_{i+1,i+4}=0$, which means that $a_{i+1,i+3}^i=0$. In the same way, for every $j=i-3$, has that $\lambda_j b_{j+1,j+3}^i=\lambda_{i-3}b_{i-2,i}^i=0$, because $\lambda_{i-3}=0$, and $\mu_{j+2}^i a_{j,j+2}^i=\mu_{i-1}^i a_{i-3,i-1}^i=0$, because $\mu_{i-1}^i=0$ and $-\lambda_{j+2}b_{j,j+2}^i=\lambda_{i-1}b_{i-3,i-1}^i=0$, because $\lambda_{i-1}=0$, but since $\mu_j^i=\mu_{i-3}^i\neq 0$, then the equation becomes $a_{j+1,j+3}^i\mu_j^i e_{j,j+3}=a_{i-2,i}^i\mu_{i-3}^i e_{i-3,i}=0$, which means that $a_{i-2,i}^i=0$. We prove that for every $j>i$, $a_{j,j+2}^i=0$, by simple induction. For $j=i+1$, we just saw that $a_{i+1,i+3}^i=0$, for $j+1$, we have the equation $(\lambda_j b_{j+1,j+3}^i-\mu_j^i a_{j+1,j+3}^i+a_{j,j+2}\mu_{j+2}^i-\lambda_{j+2}b_{j,j+2}^i)e_{j,j+3}=0$, but $\lambda_j=\lambda_{j+2}=0$, because $j>i$, and by the assumption, $a_{j,j+2}^i=0$, hence the equation becomes $a_{j+1,j+3}^i\mu_j^i e_{j,j+3}=0$, and we know that $\mu_j^i\neq 0$, because $j>i+1$, therefore we must have that $a_{j+1,j+3}^i=0$, which proves the induction step. In the same way we prove that for every $j\leq i-2$, $a_{j,j+2}^i=0$. Let $k=2$, then for $j=i-k=i-2$ we already saw that $a_{i-2,i}^i=0$, for $j\leq i-(k+1)=i-3$, we look at the same equation and observe that $\lambda_j=\lambda_{j+2}=0$, because $j,j+2<i$, and by the assumption we have that $a_{j+1,j+3}^i=0$, hence the equation becomes $a_{j,j+2}^i\mu_{j+2}^i=0$, but $\mu_{j+2}^i\neq 0$, because $j\leq i-3$, therefore we must have that $a_{j,j+2}^i=0$.
\end{proof}
\end{document}