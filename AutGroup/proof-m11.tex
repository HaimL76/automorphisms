\documentclass[12pt]{article}
\usepackage{graphicx} % Required for inserting images
\makeatletter
\newcommand*{\rom}[1]{\expandafter\@slowromancap\romannumeral #1@}
\makeatother
\usepackage{amsfonts, amssymb}
\usepackage{mathrsfs, mathdots} 
\usepackage{amsmath}
\usepackage{float}
\usepackage{amsthm}
\usepackage{tikz-cd}
\usepackage{xcolor}
\usepackage{xparse}
\usepackage{setspace}
\usepackage{xfrac}
\usepackage{yfonts}

\newtheorem{theorem}{Theorem}[subsection]
\newtheorem{proposition}[theorem]{Proposition}
\newtheorem{corollary}[theorem]{Corollary}
\newtheorem{lemma}[theorem]{Lemma}
\newtheorem{notations}[theorem]{Notations}
\newtheorem{definition}[theorem]{Definition}
\newtheorem{example}[theorem]{Example}

\ExplSyntaxOn
\NewDocumentCommand{\cycle}{ O{\;} m }
 {
  (
  \alec_cycle:nn { #1 } { #2 }
  )
 }

\seq_new:N \l_alec_cycle_seq
\cs_new_protected:Npn \alec_cycle:nn #1 #2
 {
  \seq_set_split:Nnn \l_alec_cycle_seq { , } { #2 }
  \seq_use:Nn \l_alec_cycle_seq { #1 }
 }
\ExplSyntaxOff
\usepackage{titling}
\newcommand{\subtitle}[1]{%
  \posttitle{%
    \par\end{center}
    \begin{center}\large#1\end{center}
    \vskip0.5em}%
}
\usepackage{arydshln}
\setcounter{tocdepth}{2}
\setlength{\dashlinedash}{1.2pt}
\setlength{\dashlinegap}{1.5pt}
\setlength{\arrayrulewidth}{0.2pt}
\title{proof-m11}
\author{haiml76 }
\date{May 2024}

\begin{document}
\section{The computation of $G_n(\mathbb{Q}_p)$}
\subsection{The computation of the first block $M_{11}$}
\begin{proposition}
Let $x=\sum_{i=1}^{n-1}\lambda_i e_{i,i+1}$, where $\lambda_i\in\mathbb{Q}_p$ not all zero. Then $\dim\mathcal{C}_{\gamma_3}(x)=l+m$, where $\textbf{l}$ is the number of sequences of non-zero coefficients of the form $\lambda_j,\lambda_{j+1},\dots,\lambda_{j+k-1},\lambda_{j+k}$ and $\lambda_{j-1}=\lambda_{j+k+1}=0$\footnote{We extend our notation of indices, to include also the case where $j=1$ or $j+k=n-1$, and define that $\lambda_{j-1}=\lambda_0=0$ or $\lambda_{j+k+1}=\lambda_n=0$, respectively}, and $\textbf{m}$ is the number of zero coefficients $\lambda_j=0$, such that also $\lambda_{j-1}=\lambda_{j+1}=0$.
\end{proposition}
\begin{proof}
Let $y=\sum_{i=1}^{n-1}\mu_i e_{i,i+1}\in\mathcal{C}_{\gamma_3}(x)$, where $\lambda_i\in\mathbb{Q}_p$.
For every $1\leq i\leq n-1$, denote by $c_i$ the constraint equation $[\lambda_i e_{i,i+1},\mu_{i+1}e_{i+1,i+2}]-[\lambda_{i+1}e_{i+1,i+2},\mu_i e_{i,i+1}]=(\lambda_i\mu_{i+1}-\lambda_{i+1}\mu_i)e_{i,i+2}=0$. Let $1\leq j\leq n-1$ and $1\leq k\leq n-1-j$ be two indices, such that $\lambda_{j-1}=\lambda_{j+k+1}=0$, and $\lambda_j,\lambda_{j+1},\dots,\lambda_{j+k-1},\lambda_{j+k}$ are all non-zero, then by constraints $c_j,c_{j+1},\dots,c_{m-1}$, we have that $\mu_m=\frac{\lambda_m}{\lambda_{m-1}}\mu_{m-1}=\frac{\lambda_m}{\lambda_{m-1}}\frac{\lambda_{m-1}}{\lambda_{m-2}}\mu_{m-2}=\frac{\lambda_m}{\lambda_{m-2}}\mu_{m-2}=\cdots=\frac{\lambda_m}{\lambda_j}\mu_j$, for every $j+1\leq m\leq j+k-1$, which means that all $\mu$ coefficients of $y$, with indices from $j+1$ to $j+k$, depend on the first coefficient, namely $\mu_j$. We denote the free choice of $\mu_j$ by $\mu_j=\ast$. One easily checks that we can choose freely any coefficient $\mu_m$ from $j+1$ to $j+k$, instead of $\mu_j$, and all other coefficients in that range will depend on our choice of $\mu_m$. By constraint $c_{j-1}$, we have that $\lambda_{j-1}\mu_j-\lambda_j\mu_{j-1}=0$, but $\lambda_{j-1}=0$, hence $\lambda_j\mu_{j-1}$ must vanish, but $\lambda_j\neq 0$, which obviously means that $\mu_{j-1}=0$. Similarly, we have that $\mu_{j+k+1}=0$, due to constraint $c_{j+k}$. By constraint $c_{j+k+1}$, we have that $\lambda_{k+k+1}\mu_{j+k+2}-\lambda_{j+k+2}\mu_{j+k+1}=0$, but $\lambda_{j+k+1}=\mu_{j+k+1}=0$, hence, $\lambda_{j+k+1}\mu_{j+k+2}$ must vanish, but $\lambda_{j+k+1}=0$, which means that we need to look at constraint $c_{j+k+2}$, that is, $\lambda_{j+k+2}\mu_{j+k+3}-\lambda_{j+k+3}\mu_{j+k+2}=0$. We check the different options.
If $\lambda_{j+k+2}=0$, then $\lambda_{j+k+3}\mu_{j+k+2}$ must vanish. Therefore, if $\lambda_{j+k+3}\neq 0$, then $\mu_{j+k+2}=0$, but if $\lambda_{j+k+3}=0$, then $\mu_{j+k+2}=\ast$. If $\lambda_{j+k+2}\neq 0$, then again $\mu_{j+k+2}=\ast$. If $\lambda_{j+k+2}\neq 0$, then $\mu_{j+k+2}=\ast$, and we continue the same way as for $\lambda_j$ and its following coefficients.
\end{proof}
\end{document}
