\documentclass[12pt]{article}
\usepackage{graphicx} % Required for inserting images
\makeatletter
\newcommand*{\rom}[1]{\expandafter\@slowromancap\romannumeral #1@}
\makeatother
\usepackage{amsfonts, amssymb}
\usepackage{mathrsfs, mathdots} 
\usepackage{amsmath}
\usepackage{float}
\usepackage{amsthm}
\usepackage{tikz-cd}
\usepackage{xcolor}
\usepackage{xparse}
\usepackage{setspace}
\usepackage{xfrac}
\usepackage{yfonts}

\newtheorem{theorem}{Theorem}[subsection]
\newtheorem{proposition}[theorem]{Proposition}
\newtheorem{corollary}[theorem]{Corollary}
\newtheorem{lemma}[theorem]{Lemma}
\newtheorem{notations}[theorem]{Notations}
\newtheorem{definition}[theorem]{Definition}
\newtheorem{example}[theorem]{Example}

\ExplSyntaxOn
\NewDocumentCommand{\cycle}{ O{\;} m }
 {
  (
  \alec_cycle:nn { #1 } { #2 }
  )
 }

\seq_new:N \l_alec_cycle_seq
\cs_new_protected:Npn \alec_cycle:nn #1 #2
 {
  \seq_set_split:Nnn \l_alec_cycle_seq { , } { #2 }
  \seq_use:Nn \l_alec_cycle_seq { #1 }
 }
\ExplSyntaxOff
\usepackage{titling}
\newcommand{\subtitle}[1]{%
  \posttitle{%
    \par\end{center}
    \begin{center}\large#1\end{center}
    \vskip0.5em}%
}
\usepackage{arydshln}
\setcounter{tocdepth}{2}
\setlength{\dashlinedash}{1.2pt}
\setlength{\dashlinegap}{1.5pt}
\setlength{\arrayrulewidth}{0.2pt}
\title{proof-m11}
\author{haiml76 }
\date{May 2024}

\begin{document}
\section{}
\subsection{}
\begin{proposition}
\label{prop.centralizer.dimension}
Let $\mathfrak{m}\in G_n(\mathbb{Q}_p)$ be a coefficient matrix representing an $\mathcal{L}_{n,p}$-automorphism, and let $\mathfrak{n}\in\mathcal{N}_n(\mathbf{Q}_p)$ and $\mathfrak{h}\in\mathcal{H}_n(\mathbf{Q}_p)$ be the two matrices in the unique decomposition $\mathfrak{m}=\mathfrak{n}\mathfrak{h}$, then $m\in G(\mathbb{Z}_p)$ if and only if $n\in\mathcal{N}(\mathbb{Z}_p)$ and $h\in\mathcal{H}(\mathbb{Z}_p)$.
\end{proposition}
\begin{proof}
We prove only the non-trivial direction. Assume $\mathfrak{m}\in G(\mathbf{Z}_p)$. By the construction of the automorphism coefficient matrices, the diagonal of $\mathfrak{m}$ is $\lambda_1,\lambda_2,\dots,\lambda_{n-1},\lambda_1\lambda_2,\lambda_2\lambda_3,\dots,\lambda_{n-2}\lambda_{n-1},\dots,\lambda_1\lambda_2\cdots\lambda_{n-1}$, where $\lambda_1,\lambda_2,\dots,\lambda_{n-1}\in\mathbb{Q}_p$, but we assumed $\mathfrak{m}\in G_n(\mathbb{Z}_p)$, hence we must have all the $\lambda$ coefficients and their products in $\mathbb{Z}_p$ itself. Moreover, if $\mathfrak{m}\in G_n(\mathbb{Z}_p)$, it means that $\det\mathfrak{m}\in\mathbb{Z}_p^\ast$, in words, the determinant of $\mathfrak{m}$ is invertible in $\mathbb{Z}_p$. The matrix $\mathfrak{h}$ has precisely the same diagonal, and determinant, as $\mathfrak{m}$, and zero in all the other cells. As seen earlier, the determinant of $\mathfrak{h}$, and of $\mathfrak{m}$, for every $n\geq 2$, is $\prod_{k=1}^{n-1}\lambda_k^{k(n-k)}$, so if $\det\mathfrak{h}$ is invertible, hence the valuation $v(\det\mathfrak{h})=v(\prod_{k=1}^{n-1}\lambda_k^{k(n-k)})=\sum_{k=1}^{n-1}v(\lambda_k^{k(n-k)})=\sum_{k=1}^{n-1}\sum_{l=1}^{k(n-k)}v(\lambda_k)=0$, but $\lambda_k\in\mathbb{Z}_p$, for every $1\leq k\leq n-1$, hence $v(\lambda_k)\geq 0$, but if the total sum is zero, then we must have $v(\lambda_k)=0$, for every $k$, which means that $\lambda_k\in\mathbb{Z}_p^\ast$, hence $\mathfrak{h}\in G_n(\mathbb{Z}_p)$. But if $\mathfrak{m},\mathfrak{h}\in G_n(\mathbb{Z}_p)$, then $\mathfrak{n}=\mathfrak{m}\mathfrak{h^{-1}}\in G_n(\mathbb{Z}_p)$.
\end{proof}
\end{document}
