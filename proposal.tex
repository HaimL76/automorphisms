\documentclass[12pt]{article}
\makeatletter
\newcommand*{\rom}[1]{\expandafter\@slowromancap\romannumeral #1@}
\makeatother
\usepackage{amsfonts, amssymb}
\usepackage{mathrsfs, mathdots} 
\usepackage{amsmath}
\usepackage{float}
\usepackage{amsthm}
\usepackage{tikz-cd}
\usepackage{xcolor}
\usepackage{xparse}
\usepackage{setspace}
\usepackage{xfrac}
\usepackage{yfonts}

\newtheorem{theorem}{Theorem}[subsection]
\newtheorem{proposition}[theorem]{Proposition}
\newtheorem{corollary}[theorem]{Corollary}
\newtheorem{lemma}[theorem]{Lemma}
\newtheorem{notations}[theorem]{Notations}
\newtheorem{definition}[theorem]{Definition}
\newtheorem{example}[theorem]{Example}

\ExplSyntaxOn
\NewDocumentCommand{\cycle}{ O{\;} m }
 {
  (
  \alec_cycle:nn { #1 } { #2 }
  )
 }

\seq_new:N \l_alec_cycle_seq
\cs_new_protected:Npn \alec_cycle:nn #1 #2
 {
  \seq_set_split:Nnn \l_alec_cycle_seq { , } { #2 }
  \seq_use:Nn \l_alec_cycle_seq { #1 }
 }
\ExplSyntaxOff

\begin{document}
\begin{abstract}
Let $G$ be any group. For any natural number $n\in\mathbb{N}$, let $a_n$ be the number of subgroups $H\leq G$, such that $[G:H]=n$. Assume $G$ is finitely-generated, then $a_n<\infty$, and we can define a $\zeta$-function of the form $\zeta_G(s):=\sum_{i=1}^\infty a_n n^{-s}$, where $s\in\mathbb{C}$. Assume, in addition, that $G$ is also nilpotent and torsion-free, then this function has properties of the Riemann $\zeta$-function, mainly the decomposition of $\zeta$ to an Euler product of local factors indexed by primes. A version of this $\zeta$-function counts pro-isomorphic subgroups, and an analogous function may be defined for appropriate Lie rings. we study the pro-isomorphic $\zeta$-functions for a family of nilpotent Lie rings of unbounded nilpotency class. We shall compute te automorphism groups of these Lie rings explicitly, prove uniformity of the pro-isomorphic $\zeta$-functions local factors, and aim to determine them explicitly. This study of nilpotent Lie rings can then be reflected in the study of pro-isomorphic subgroups and their associated $\zeta$-functions.
\end{abstract}
\section{Scientific Background}
\subsection{Introduction}
We start our discussion with the following proposition, which stands at the very base of our subject.
\begin{proposition} \label{prop:finite.number.subgroups}
Let $G$ be any finitely generated group, and let $n\in\mathbb{N}$ any natural number. Then there is a finite number of subgroups $H\leq G$, such that $[G:H]=n$
\end{proposition}
\begin{proof}
Let $H\leq G$ be a subgroup, such that $[G:H]=n$, then $\sfrac{G}{H}:=\{g_1H,g_2H,\dots,g_nH\}$ is the set containing all left-cosets of $H$ in $G$. We may consider the action of $G$ by left multiplication on $\sfrac{G}{H}$. i.e., for all $g\in G$, and for all left-cosets $g_iH\in\sfrac{G}{H}$, we have that $g(g_iH):=(gg_i)H=g_jH$, where $g_jH\in\sfrac{G}{H}$ is some left-coset. This means that $g$ maps every index $i\in[n]$ to some index $j\in[n]$, which means that $g$ operates as a permutation on $[n]$. Therefore, there exists a homomorphism $f:G\rightarrow\mathcal{S}_n$, from $G$ to the symmetric group of order $n$. 
Clearly $g_k\in H$ iff $g_kH=H$, which means that $f(g_k)(k)=k$.  This means that $H=\{g\in G : f(g)(k)=k\}$, which clearly shows that the number of subgroups $H\leq G$ of index $n$ is less or equal to the number of maps $f : G
\rightarrow\mathcal{S}_n$, which are maps from a finitely generated group to a finite group, and since group homomorphisms are uniquely determined by the maps of their generators, it is clear that the number of these maps is finite.
\end{proof}
This proposition gives rise to an entire subject in group theory, called \textbf{subgroup growth}. We denote by $a_n(G)$ the number of subgroups of $G$ of index $n$, and look at the sequence $\{a_n(G)\}$. The subject of subgroup growth aims to relate the properties of this sequence to the algebraic structure of $G$. For instance, Lubotzky, Mann and Segal showed that $a_n(G)$ grows polynomially iff $G$ is virtually nilpotent of finite rank. That is, $G$ has a finite-index nilpotent subgroup, and all finitely-generated subgroups may be generated by a finite number of generators. This research concentrates on the growth of \textbf{pro-isomorphic} subgroups, which we now define.
\begin{definition}
\label{def:profinite.closure}
Let $G$ be any group, and let $\mathcal{N}:=\{N_i\trianglelefteq G\}_{i\in I}$ the set of all normal subgroups of $G$. We define a partial order on $\mathcal{N}$, by inclusion, and assign $G$ an infinite set of indices. The inverse limit $\widehat{G}=\underset{\leftarrow}{lim}\{\sfrac{G}{N_k}\}_{k\in I}:=\{(h_k)_{k\in I}\in\prod_{k\in I}\sfrac{G}{N_k} : \pi_{ji}(h_j)=h_i,\forall i\leq j\}$, where $\pi_{ji}:=\sfrac{G}{N_j}\rightarrow\sfrac{G}{N_i}$ is the natural projection, is called the \textbf{profinite closure} of $G$.
\end{definition}
\begin{definition}
\label{def:pro.isomorphic}
Let $G$ be any group. a subgroup $H\leq G$ is called \textbf{pro-isomorphic}, if $\widehat{H}\cong\widehat{G}$.
\end{definition}
\begin{definition}
\label{def:zeta.pro.isomorphic}
Let $G$ be any group, and let $\hat{a_n}(G):=\#\{H\leq G : \widehat{H}\cong\widehat{G}, [G:H]=n\}$, in words, the number of pro-isomorphic subgroups of $G$, of index $n$. The \textbf{Pro-Isomorphic $\zeta$-Function} of $G$ is defined by $\hat{\zeta_G}(s):=\sum_{n=1}^{\infty}\hat{a_n}(G)n^{-s}$, for some $s\in\mathbb{C}$. 
\end{definition}
In this research, we discuss only groups for which $\hat{a_n}(G)<\infty$, for every $n\in\mathbb{N}$. A sufficient condition for this would be that $G$ is finitely-generated, by proposition \ref{prop:finite.number.subgroups}.
\begin{example}
$G=(\mathbb{Z},+)$. $\mathbb{Z}$ is an abelian group, and every $H\leq \mathbb{Z}$ is of the form $H=n\mathbb{Z}=\langle n\rangle$, for some $n\in\mathbb{N}$, which means that $H\cong \mathbb{Z}$, as both are infinite cyclic groups, and so, $\widehat{H}\cong\widehat{\mathbb{Z}}$. Since we have only one $\mathbb{Z}$-subgroup of index $n$, for every $n\in\mathbb{N}$, then $a_n(\mathbb{Z})=\hat{a_n}(\mathbb{Z})=1$, thus, its pro-isomorphic $\zeta$-function is $\hat{\zeta_{\mathbb{Z}}}=\sum_{i=1}^{\infty}n^{-s}=\zeta(s)$, the Riemann $\zeta$-function. 
\end{example}
After establishing the basic definitions, we observe a fact that is a major motivation for this research, which says that the Riemann $\zeta$-function decomposes to an infinite product of local $\zeta_p$-functions, that is, $\zeta(s)=\prod_p\zeta_p(s)=\prod_p\sum_{k=0}^\infty p^{-ks}=\prod_p\frac{1}{1-p^{-s}}$, where the product runs over all the prime numbers. Following this fact, regarding the Riemann $zeta$-function, we observe that for any finitely-generated, nilpotent and torsion-free group, $G$, we have the same decomposition as above, for the pro-isomorphic $\zeta$-function, $\hat{\zeta_G}(s)=\prod_p\hat{\zeta_{G,p}}(s)$, where $\hat{\zeta_{G,p}}(s):=\sum_{k=0}^\infty a_{p^{ks}}(G)p^{-ks}$
We hereby bring several basic definitions of group nilpotency, which are very important for this research.
\begin{definition}
\label{def.lower.central.series}
Let $G$ be any group, then the \textbf{Lower Central Series} of $G$ is a sequence of subgroups of $G$, defined by the recursive rule, $G_n:=[G,G_{n-1}]$, for every $n\in\mathbb{N}$, where $G_0:=G$.  We recall that $[G,G_n]\leq G$ is the subgroup of commutators, $\{gg_ng^{-1}g_n^{-1} : g\in G,g_n\in G_n\}$
\end{definition}
\begin{definition}
\label{def.nilpotency.class}
Let $G$ be any group. the \textbf{Nilpotency Class} of $G$ is $min\{n\in\mathbb{N} : G_n=[G,G_{n-1}]=\{e\}\}$, in words, the smallest natural number, such that the subgroup of commutators of the form $[G,G_n]$ is the trivial group. We can extend this definition, and say that the trivial group nilpotency class is $0$.
\end{definition}
\begin{definition}
\label{def.nilpotency.class}
Let $G$ be a group. If $G$ if of a finite nilpotency class, $n\in\mathbb{N}$, then $G$ is said to be a \textbf{Nilpotent} group.
\end{definition}
\subsection{Linearization}
For finitely-generated torsion-free nilpotent groups, we associate nilpotent Lie algebras over $\mathbb{Z}_p$, the ring of $p$-adic integers. We show here the basic properties of $\mathbb{Z}_p$-algebras, as subalgebras of $\mathbb{Q}_p$-algebras, where $\mathbb{Q}_p$ is the fraction field of $\mathbb{Z}_p$. In the part that describes the goals of this research, we present a specific structure of nilpotent groups, and their associated Lie algebras over the $p$-adic integers.
We begin this part of our discussion by a very basic fact, which says that the group of automorphisms of $\mathcal{L}_{p,n}$, namely $G_n(\mathbb{Q}_p)$, where $p$ is a prime number, and $n=dim\mathcal{L}_{p,n}$, is a subgroup of $GL_n(\mathbb{Q}_p)$, which means that $G_n(\mathbb{Q}_p)$ is a group of invertible $n\times n$ matrices over $\mathbb{Q}_p$, which is actually true for any field.
This basic fact comes immediately from choosing a basis for $\mathcal{L}_{p,n}$, $\mathcal{B}=\{b_1,\dots,b_n\}$, and showing that for every $\mathcal{L}_{p,n}$-automorphism, $\varphi\in G_n(\mathbb{Q}_p)$, and for every $v\in\mathcal{L}_{p,n}$, the image $\varphi(v)$ can be uniquely determined by one invertible linear transformation, This obviously comes from the fact that every $v\in\mathcal{L}_{pn}$ is uniquely represented as a linear combination of elements of the basis, i.e. $v=\sum_{i=1}^n\lambda_ib_i$, then the image $\varphi(v)=\varphi(\sum_{i=1}^n\lambda_i b_i)=\sum_{i=1}^n\varphi(\lambda_i b_i)=\sum_{i=1}^n\lambda_i\varphi(b_i)$. Clearly, $\{\varphi(b_1),\dots,\varphi(b_n)\}$ itself forms a basis of $\mathcal{L}_{p,n}$, and so, $\varphi$ can be represented as a basis transition matrix, which proves the above.
After establishing the basic fact, regarding the structure of $\mathcal{L}_p,n$-automorphisms as invertible matrices over $\mathbb{Q}_p$, we observe that $\mathcal{L}_{p,n}$ can be restricted to an algebra over the ring of $p$-adic integers, which we denote as $L_{p,n}<\mathcal{L}_{p,n}$. This comes from the fact that $\mathbb{Z}_p<\mathbb{Q}_p$, so if $\mathcal{B}_n=\{b_1,\dots,
b_n\}$ is a basis for for $\mathcal{L}_{p,n}$, and if $v=\sum_{i=1}^n\alpha_i b_i$, and $u=\sum_{i=1}^n\beta_i b_i$, where $\alpha_i,\beta_i\in\mathbb{Z}_p$, then clearly, $v+u,vu\in L_{p,n}$.
For the construction of the algebras we shall study, in this research, we need to show the opposite direction of the above claim. Suppose we have $\mathcal{A}_n$, a $\mathbb{Z}$-algebra, with a basis $\mathcal{B}_n=\{b_1,\dots,b_n\}$. We observe that $\mathbb{Z}_p$, as an abelian group, has a natural structure of $\mathbb{Z}$-module, and therefore, we can take the tensor product $\mathcal{A}_n\otimes_{\mathbb{Z}}\mathbb{Z}_p$, and so we have, for all $a\in\mathcal{A}_n$, and for all $r,s\in\mathbb{Z}_p$, that $s(a\otimes r)=a\otimes rs= rs(a\otimes 1)$, which is well-defined, because of the multiplication in $\mathbb{Z}_p$, and so, given $B=\{b_1,b2_,\dots,b_n\}$, a basis for $L_n(\mathbb{Z}_p)$, we have a natural bijection between $b_i$ and $b_i\otimes 1$, for $1\leq i\leq n$, which means that $\{b_1\otimes 1,b_2\otimes 1,\dots,b_n\otimes 1\}$ is a basis for $\mathcal{A}_n\otimes_{\mathbb{Z}}\mathbb{Z}_p$, we denote $L_{p,n}:=\mathcal{A}_n\otimes_{\mathbb{Z}}\mathbb{Z}_p$, and we got that $L_{p,n}$ is a $\mathbb{Z}_p$-algebra, with the same basis, $\mathcal{B}_n$, but with scalars from $\mathbb{Z}_p$, which proves that $L_n<\mathcal{L}_n$.
The same construction exactly extends $L_{p,n}$ to $\mathcal{L}_{p,n}$.\par
The above discussion brings us closer to the essence of our research background, for now we are able to observe the following important fact. If $\mathcal{B}_n$ is a basis for $\mathcal{L}_{p,n}$, then, for every $\varphi\in G_n(\mathbb{Q}_p)$, we have that $\varphi(L_{p,n})\subseteq L_{p,n}$ iff $\varphi(b_1),\dots,\varphi(b_n)\in L_{p,n}$, in words, the image of every vector in $L_{p,n}$ is in $L_{p,n}$ itself if and only if the coefficients of the $\mathbb{Q}_p$-linear transformation $\varphi$, i.e., the rows of the $n\times n$ matrix that represents $\varphi$, are in $\mathbb{Z}_p$ itself. The less obvious direction comes from the fact that if $v=\sum_{i=1}^n\lambda_i b_i$, where $\lambda_1,\dots,\lambda_n\in\mathbb{Z}_p$, then $\varphi(v)=\varphi(\sum_{i=1}^n\lambda_ib_i)=\sum_{i=1}^n\varphi(\lambda_ib_i)=\sum_{i=1}^n\lambda_i\varphi(b_i)$, but $\varphi(b_1),\dots,\varphi(b_n)\in L_p$, so $\sum_{i=1}^n\lambda_i\varphi(b_i)$ is a $\mathbb{Z}_p$-linear combination, hence $\varphi(v)\in L_p$.\par
Another important step in the direction of our research would be to introduce the following object. $G_n^{+}(\mathbb{Q}_p):=G_n(\mathbb{Q}_p)\cap\mathcal{M}_n(\mathbb{Z}_p)$, in words, all the $\mathcal{L}_{p,n}$-automorphisms, which are matrices over $\mathbb{Z}_p$. We immediately observe that $G_n^{+}(\mathbb{Q}_p)$ is a monoid, since 
$G_n(\mathbb{Q}_p)$ is a group, thus a monoid, and $\mathcal{M}_n(\mathbb{Z}_p)$ is a monoid, so, their intersection is a monoid.\par
A very important attribute of this object is that it absorbs $L_{p,n}$-automorphisms by multiplication from right, i.e., for every $g\in G_n^{+}(\mathbb{Q}_p)$, the right coset $G_n(\mathbb{Z}_p)g\subseteq G_n^{+}(\mathbb{Q}_p)$. This comes from the fact that if $\varphi\in G_n(\mathbb{Z}_p)$, then clearly $\varphi\in G(\mathbb{Q}_p)$, but on the other hand, $\varphi(L_{p,n)}\subseteq L_{p,n}$, as we saw earlier, which means that $\varphi$ is a $n\times n$ matrix with coefficients from $\mathbb{Z}_p$, so $\varphi\in\mathcal{M}_n(\mathbb{Z}_p)$, as well, and the two inclusions give us that $\varphi\in G({\mathbb
{Q}_p})\cap\mathcal{M}_n(\mathbb{Z}_p)$, hence, $\varphi g\in G({\mathbb
{Q}_p})\cap\mathcal{M}_n(\mathbb{Z}_p)$. This gives way to the following result,
$G^+(\mathbb{Q}_p)=\bigsqcup_{i=1}^m G(\mathbb{Z}_p)g_i$, where $[G(\mathbb{Q}_p):G(\mathbb{Z}_p)]=m$, in words, the monoid $G_n^{+}(\mathbb{Q}_p)$ is a disjoint union of right-cosets of $G_n(\mathbb{Z}_p)$ in $G_n(\mathbb{Q}_p)$.\par
The discussion above reveals the construction we base our research upon.
We observe that there is a bijection between $G(\mathbb{Z}_p)\backslash G^+(\mathbb{Q}_p)$ and $\{M\leq L_{p,n} : M\cong L_{p,n}\}$, in words, we have a bijective map between each right-coset of $G_n(\mathbb{Z}_p)$ in $G_n^{+}(\mathbb{Q}_p)$ and each $L_{p,n}$-subalgebra which is isomorphic to $L_{p,n}$ itself.
The general idea behind this observation is that if we take an automorphism in a right-coset of $G_n(\mathbb{Z}_p)$, namely $\varphi\in G_n(\mathbb{Z}_p)g\in G_n(\mathbb{Z}_p)\backslash G_n^+(\mathbb{Q}_p)$, and denote $M=\varphi(L_{p,n})$, since $\varphi\in G_n^{+}(\mathbb{Q}_p)=G_n(\mathbb{Q}_p)\cap\mathcal{M}_n(\mathbb{Z}_p)$, we have that $M=\varphi(L_{p,n})\subseteq L_{p,n}$, as we saw earlier. If we choose a different representative of the same right-coset, namely $\psi\in G(\mathbb{Z}_p)g$, we have that $\tau=\psi\varphi^{-1}\in G(\mathbb{Z}_p)$, which means that $\tau(L_{p,n})=L_{p,n}$. But $\tau\varphi=\psi\varphi^{-1}\varphi=\psi$, which means that $\psi(L_{p,n})=\tau\varphi(L_{p,n})=\varphi(\tau(L_{p,n}))=\varphi(L_{p,n})=M$, so we have that $M$ is the image of any representative of $G(\mathbb{Z}_p)g$. We further observe that the restriction $\varphi|_{L_{p,n}}$ is a one-to-one map from $L_{p,n}$ onto $L_{p,n}$ itself, and therefore, it is an isomorphism. If we take $M=\varphi_{L_{p,n}}(L_{p,n})$, we have that $M\cong L_{p,n}$, as the image of the restriction of any choice of representative of $G(\mathbb{Z}_p)g$, which generally shows where this bijection comes from.\par
We end this part, as a preparation for the final part of this technical background review, with the following result, which says that for each right-coset of $G_n(\mathbb{Z}_p)$ in $G_n^{+}(\mathbb{Q}_p)$, namely, $G_n(\mathbb{Z}_p)g$, taking any representative of this coset $\varphi\in G_n(\mathbb{Z}_p)g$, and taking the image $M=\varphi(L_{p,n})\leq L_{p,n}$, then $[L_{p,n}:M]=|det(g)|_p^{-1}$.
\subsection{$p$-adic Integration}
In this final part of the technical background review, we finally get to the motivation for all the construction we have presented in the first parts.
We start with a basic observation about topological groups, which says that if $\Gamma$ is a topological group, and $U\subseteq\Gamma$, is an open subset of $\Gamma$, then $\gamma U:=\{\gamma u : \gamma\in\Gamma, u\in U\}$ is also an open subset of $\Gamma$. To show this is true, We define a map $f=f_{\gamma^{-1}}:\Gamma\rightarrow\Gamma$, by $f(g):=\gamma^{-1}g$, for any $g\in\Gamma$. Clearly, $f$ is continuous, as a composition of the group inverse and multiplication maps, both continuous in $\Gamma$, therefore, taking, for every open set $U\subseteq\Gamma$, the inverse image  $f^{-1}(U)=\{g\in G : f(g)=\gamma^{-1}g\in U\}=\{\gamma h : f(\gamma h)=\gamma^{-1}\gamma h=h\in U\}=\{\gamma h : h\in U\}=\gamma U$, proves that $\gamma U$ is also an open subset in $\Gamma$.\par
We shall now define a very central object for our research. Prior to defining this object, we actually claim, without proving, that it does exist, under the prerequisites of the definition.
\begin{definition}
\label{prop.compact.subset.right.haar.measure}
Let $\Gamma$ be a locally compact topological group, i.e., $\forall\gamma\in\Gamma$, there is an open environment $U_{\gamma}$ of $\gamma$, and a compact subset $K_{\gamma}$, such that $\gamma\in U_{\gamma}\subset K_{\gamma}$. Then there is a measure $\mu$, with the following property: for any measurable subset, $U\subseteq\Gamma$, and any $\gamma\in\Gamma$, $\mu(U\gamma)=\mu(U)$, where $U\gamma:=\{u\gamma : u\in U\}$, and $\mu$ is unique up to multiplication in constant. $\mu$ is called a \textbf{Right Haar Measure}
\end{definition}
Equipped with the newly-defined right Haar measure, we can finally make use of the construction from above. We start by claiming, without proof, that if $p$ be a prime number, then $G_n(\mathbb{Q}_p)$, that is, the group of automoprhisms, for any $n$-dimensional $\mathbb{Z}_p$-algebra, $L_{p,n}$, is a locally compact topological group.\par
We continue to claim that $G_n(\mathbb{Q}_p)$, the group of $\mathcal{L}_{p,n}$-automorphisms, has a unique right Haar measure $\mu$, with the property that $\mu(G(\mathbb{Z}_p))=1$.\par
Moreover, we claim that for every $g\in G_n(\mathbb{Q}_p)$, we also have that $\mu(G(\mathbb{Z}_p)g)=\mu(G(\mathbb{Z}_p))=1$.
With this observation, we go directly to the calculation of the $p$-adic valuation of the determinant as an integral, and say that if $p$ is a prime number, $s\in\mathbb{C}$, and $g\in G_n^+(\mathbb{Q}_p)$, then $|det(g)|_p^s=\displaystyle\int_{h\in G_n^+(\mathbb{Q}_p)}|det(h)|_h^sd\mu$.\par
This comes from the above claim, that for every right-coset of $G_n(\mathbb{Z}_p)$ in $G_n^{+}(\mathbb{Q}_p)$, we have that for every $L_{p,n}$-automorphism, $g\in G(\mathbb{Z}_p)h$, the inverse of the $p$-adic valuation of the determinant, $|det(g)|_p^{-1}$, does not depend on the choice of representative, which means that $|det(g)|_p$ is constant on the entire right-coset, so we have that $|det(g)|_p=|det(h)|_p$, hence, $|det(g)|_p^s=\displaystyle\int_{h\in G^+(\mathbb{Q}_p)}|det(h)|_p^sd\mu=\displaystyle\int_{h\in G^+(\mathbb{Q}_p)}|det(g)|_p^sd\mu=|det(g)|_p^s\displaystyle\int_{h\in G^+(\mathbb{Q}_p)}d\mu$. But, $\mu=\mu(G(\mathbb{Z}_p)h)$, and we saw earlier that $\mu(G(\mathbb{Z}_p)h)=\mu(G(\mathbb{Z}_p))=1$, so $\displaystyle\int_{h\in G^+(\mathbb{Q}_p)}|det(h)|_p^sd=|det(g)|_p^s\displaystyle\int_{h\in G^+(\mathbb{Q}_p)}d\mu=|det(g)|_p^s\cdot 1=|det(g)|_p^s$\par
Now we can conclude all this construction by the following observation, that if $p$ is a prime number, and $s\in\mathbb{C}$, then $\hat{\zeta_{L,p}}(s)=\displaystyle\sum_{G(\mathbb{Z})g\in G(\mathbb{Z}_p)\backslash G^+(\mathbb{Q}_p)}|det(g)|_p^s=\displaystyle\sum_{G(\mathbb{Z})g\in G(\mathbb{Z}_p)\backslash G^+(\mathbb{Q}_p)}\displaystyle\int_{h\in G(\mathbb{Z}_p)g}|det(h)|_p^s d\mu=\displaystyle\int_{h\in G^+(\mathbb{Q}_p)}|det(h)|_p^s d\mu$, which brings us back to our initial quest, which is, finding a way to calculate the local pro-isomorphic $\hat\zeta_{G,p}$-function, for every prime number $p$.
We end this technical background review by a theorem, brought with no proof,
\begin{theorem}
\label{thm.rational.function}
Let $p$ be a prime number, $s\in\mathbb{C}$, then there exists a rational function, $w_p(s):=\frac{f(x)}{g(x)}$, where $f,g\in\mathbb{Z}_p[x]$, which satisfies $\hat{\zeta_{L,p}}(s)=w_p(p^{-s})$.
\end{theorem}
\section{Research Goals and Methodology}
\subsection{The group $U_n(\mathbb{Z}_p)$}
We start by this following definition.
\begin{definition}
\label{def.unipotent.matrix}
Let $\mathcal{R}$ be a commutative ring. Let $U_n(\mathcal{R})\leq GL_n(\mathcal{R})$ be the subgroup of upper triangular matrices, i.e. $U_n(\mathcal{R})=\Bigg\{
\begin{pmatrix}
1 & a_{12} &\\
  & \ddots & \ddots\\
  & & 1\\
\end{pmatrix}\Bigg\}$
\end{definition}
Looking deeper into the structure of $U_n(\mathbb{Z}_p)$, we observe that $U_n(\mathbb{Z}_p)$ can be generated by matrices of the form $E_{ij}=\begin{pmatrix}
1 & 0 & \dots & 0\\
  & \ddots & 1 & 0\\
  &  & 1 & 0\\
  & & & 1\\
  \end{pmatrix}$, where besides the elements on the main diagonal, only the element in row $i$ and column $j$, satisfying the condition that $i<j$, is $1$, and all the other elements are $0$. It is easy to observe that $E_{ij}^m=\begin{pmatrix}
1 & 0 & \dots & 0\\
  & \ddots & m & 0\\
  &  & 1 & 0\\
  & & & 1\\
  \end{pmatrix}$, for every $m\in\mathbb{Z}$, by simple induction.\par
Now we introduce a fact that is very basic to our research and will come in handy when we move from the group $U_n(\mathbb{Z}_p)$ to the appropriate Lie algebra. Taking any two such elementary matrices, $E_{ij},E_{kl}$, and calculating their commutator, $[E_{ij},E_{kl}]=E_{ij}E_{kl}E_{ij}^{-1}E_{kl}^{-1}$, we can easily check that $E_{ij}E_{kl}=E_{il}$ iff $j=k$, and $E_{ij}E_{kl}=-E_{kj}$ iff ${i=l}$, and $E_{ij}E_{kl}=I_n$ in any other case.
For example, $\Bigg[\begin{pmatrix}
    1 & 1 & 0\\
     & 1 & 0\\
     &  & 1\\
\end{pmatrix},\begin{pmatrix}
    1 & 0 & 0\\
     & 1 & 1\\
     &  & 1\\
\end{pmatrix}\Bigg]=\begin{pmatrix}
    1 & 0 & 1\\
     & 1 & 0\\
     &  & 1\\
\end{pmatrix}$
This fact puts $U_n(\mathbb{Z}_p)$ in the category of nilpotent groups, since, taking $\mathcal{E}(\mathbb{Z}_p):\{E_{ij} : i<j\}$, the set of $U_n(\mathbb{Z}_p)$ generators from the set of $U_n(\mathbb{Z}_p)$ generators, and observing its lower central series, we see that $\mathcal{E}_{n-1}=\{I_n\}$, which says that $U_n(\mathbb{Z}_p)$ is of nilpotency class $n-1$.\par
Considering the behavior of this said set of generators, we observe that we need a significantly smaller set, because, if we start only with elementary matrices of the form $E_{i,i+1}$, where $1\leq i\leq n-1$, we can obtain any matrix $E_{ij}$ of the wider set, by taking any chain of commutators of the form $[E_{i,i+1},[E_{i+1,i+2},[\dots,[E_{i+k,j}]]]]$, where $i+k+1=j$.
We end this $U_n(\mathbb{Z}_p)$ review by stating another important fact, that $U_n(\mathbb{Z}_p)$ is torsion-free.
Again, we show this by looking at the set of generators, $\mathcal{E}$, and we observe that for any $E_{ij}\in\mathcal{E}$, since $E_{ij}^m$, for any $m\in\mathbb{Z}$, has one element equals to $m$, as we saw above, clearly, $m\neq 0$, which means that there is no $m\in\mathbb{N}$, for which $E_{ij}^m=I_n$. These facts $U_n(\mathbb{Z}_p)$ place it as a group of our interest, for this research, and bring us next to its associated Lie algebra.
\subsection{The algebra $L_{p,n}$}
We start with $E_{ij}$ from above and define matrices of the form $e_{ij}=E_{ij}-I_n$, in words, $e_{ij}$ is obtained by replacing all the $1$ on the main diagonal with $0$. Then clearly, $e_{ij}e_{kl}=e_{il}$ where $j=k$, $e_{ij}e_{kl}=-e_{kj}$ where ${i=j}$, and $e_{ij}e_{kl}=0$ in any other case.
\begin{corollary}
\label{cor.lie.algebra}
Let $\mathcal{B}_n$ be the set $\{e_{ij} : i<j\}$, of all the matrices of the form described in \ref{prop.lie.algebra.commutator}. Then $\mathcal{B}_n$, with the standard matrix addition, and a multiplication operation $\ast$, defined by $e_{ij}*e_{jk}=e_{ij}e_{jk}-e_{jk}e_{ij}$, is a basis for a Lie algerba over $\mathbb{Z}_p$, which shall be denoted by $L_n(\mathbb{Z}_p)$, or $L_n$, for abbreviation, which is the $\mathbb{Z}_p$-algebra of all the matrices $A\in\mathcal{M}_n(\mathbb{Z}_p)$, with $0$ on the main diagonal. The multiplication operation $\ast$ shall be denoted by Lie Brackets, that is, $e_{ij}*e_{jk}=[e_{ij},e_{jk}]$.
\end{corollary}
\begin{proof}
Since we have defined the multiplication operation as the standard Lie brackets, for matrix Lie algebras, i.e., $[A,B]=AB-BA$, for all the matrices $A,B$ in the algebra, one easily checks that all the axioms of a Lie algebra hold for this definition. Obviously, $L_n$ is a $\mathbb{Z}_p$-span of $\mathcal{B}_n$, since every matrix of the form $A=\begin{pmatrix}
0 & a_{12} & a_{13} & \dots & a_{1n}\\
0 & 0 & a_{23} & \dots & a_{2n}\\
\vdots & \vdots & \vdots & \vdots & \vdots\\
0 & 0 & 0 & \dots & a_{n-1n}\\
0 & 0 & 0 & 0 & 0\\
\end{pmatrix}$ is a linear combination of matrices of $\mathcal{B}_n$, i.e., $A=\sum_{i=1}^{n-1}\sum_{j=i+1}^n a_{ij}e_{ij}$. We can observe that if $B=\sum_{i=1}^{n-1}\sum_{j=i+1}^n b_{ij}e_{ij}=0_n$, then clearly all the $b_{ij}$ are $0$. We conclude that $\mathcal{B}_n$ is a basis for $L_n$.
\end{proof}
\begin{proposition}
\label{prop.lie.algebra.dimension}
Let $p$ be a prime number, and let $n\in\mathbb{N}$ be any natural number, then dim $L_n(\mathbb{Z}_p)=\binom{n}{2}$
\end{proposition}
\begin{proof}
From \ref{cor.lie.algebra}, we have that a basis for $L_n(\mathbb{Z}_p)$ is the set of all $e_{ij}$, where $i<j$. For each row $1\leq i\leq n-1$, we have $n-i$ elements of the form $e_{ij}$, which gives, in total, $\frac{n(n-1)}{2}=\frac{n!}{2!(n-2)!}=\binom{n}{2}$ elements of the the basis.
\end{proof}
\begin{proposition}
\label{prop.nilpotent.lie.algebra}
Let $p$ be a prime number, and let $n\in\mathbb{N}$ be any natural number, then $L_n(\mathbb{Z}_p)$ is a nilpotent Lie algebra.
\end{proposition}
\begin{proof}
It is followed directly from \ref{prop.lie.algebra.commutator}, and from \ref{prop.unipotent.group.nilpotent}, since $[e_{ij},e_{jk}]=[E_{ij},E_{jk}]-I_n=E_{ik}-I
_n=e_{ik}$, where the first brackets are Lie Brackets of $L_n(\mathbb{Z}_p)$, and the second brackets are a group commutator of $U_n(\mathbb{Z}_p)$.
\end{proof}
By considering the behavior of $\mathcal{L}_n(\mathbb{Q}_p)$ under the Lie brackets, we can learn about the structure of $Aut_{\mathbb{Q}_p}(\mathcal{L}_n)$. As a basic fact, every $\mathcal{L}_n(\mathbb{Q}_p)$-automorphism $\varphi$ must obey the $\mathcal{L}_n$ Lie brackets, meaning that for all $x,y\in\mathcal{L}_n$, we must have that $\varphi([x,y])=[\varphi(x),\varphi(y)]$. Let $B=\{b_1,b_2,\dots,b_m\}$ be a basis for $\mathcal{L}_n$, we have that $x=\sum_{i=1}^m\lambda_i b_i$, and $y=\sum_{i=1}^m\rho_i b_i$, so $\varphi([x,y])=[\varphi(\sum_{i=1}^m\lambda_i b_i),\varphi(\sum_{i=1}^m\rho_i b_i)]=[\sum_{i=1}^m\varphi(\lambda_i b_i),\sum_{i=1}^m\varphi(\rho_i b_i)]=[\sum_{i=1}^m\lambda_i\varphi(b_i),\sum_{i=1}^m\rho_i\varphi(b_i)]=\sum_{i=1}^m\sum_{j=1}^m[\lambda_i\varphi(b_i),\rho_j\varphi(b_j)]=\sum_{i=1}^m\sum_{j=1}^m\lambda_i\rho_j[\varphi(b_i),\varphi(b_j)]$. This technique can be demonstrated in the most simple case, which is the Heisenberg group.
\subsection{The Heisenberg group}
\begin{definition}
\label{def.heisenberg.group}
The \textbf{Heisenberg group} is the unipotent group of $3\times 3$ matrices, over $\mathbb{Q}_p$, namely $U_3(\mathbb{Q}_p)$. Every matrix $A\in U_3$ is of the form $$
\begin{pmatrix}
1 & a_{12} & a_{13}\\
0 & 1 & a_{23}\\
0 & 0 & 1\\
\end{pmatrix}
$$ where $a_1,a_2,a_3\in\mathbb{Q}_p$.
\end{definition}
The $\mathbb{Q}_p$-algebra associated with $U_3$ consists of matrices of the form $$A-I_3=
\begin{pmatrix}
0 & a_{12} & a_{13}\\
0 & 0 & a_{23}\\
0 & 0 & 0\\
\end{pmatrix}=a_{12}e_{12}+a_{13}e_{13}+a_{23}e_{23}
$$. Let $\varphi\in Aut_{\mathbb{Q}_p}(\mathcal{L}_p)$ be an $\mathcal{L}_{p,n}$-automorphism. The image of every $A\in\mathcal{L}_{p,n}$, as a linear combination of elements of the basis, is a linear combination of the images of these elements. So, let $v=(x,y,z)=xe_{12}+y_{23}+z_{13}$, where $x,y,z\in\mathbb{Q}_p$, we have that $\varphi(v)=\begin{pmatrix}
x & y & z\\
\end{pmatrix}\begin{pmatrix}
a_{11} & a_{12} & a_{13}\\
a_{21} & a_{22} & a_{23}\\
a_{31} & a_{32} & a_{33}\\
\end{pmatrix}=\begin{pmatrix}
a_{11}x+a_{12}x+a_{13}x & a_{21}y+a_{22}y+a_{23}y & a_{31}z+a_{32}z+a_{33}z\\
\end{pmatrix}=\\\begin{pmatrix}
(a_{11}+a_{12}+a_{13})x & (a_{21}+a_{22}+a_{23})y & (a_{31}+a_{32}+a_{33})z\\
\end{pmatrix}=\begin{pmatrix}
\varphi(x) & \varphi(y) & \varphi(z)\\
\end{pmatrix}
$, which means that $$
\varphi(e_{12})=a_{11}e_{12}+a_{12}e_{23}+a_{13}e_{13}$$
$$\varphi(e_{23})=a_{21}e_{12}+a_{22}e_{23}+a_{23}e_{13}$$
$$\varphi(e_{13})=a_{31}e_{12}+a_{32}e_{23}+a_{33}e_{13}
$$.
We want to find relations between the elements of $\varphi$. Considering the fact that $[\varphi(x),\varphi(y)]=\varphi([x,y])=$, we observe that the Lie brackets on images of any two commuting elements of the basis give $0$, as they are images of $0$, i.e., for every $x,y\in\mathcal{L}_n$, such that $[x,y]=0$, we have that $[\varphi(x),\varphi(y)]=\varphi([x,y])=\varphi(0)=0$. Hence, the only images that do not vanish under Lie brackets are $[\varphi(e_{12}),\varphi(e_{23})]=[a_{11}e_{12}+a_{12}e_{23}+a_{13}e_{13},a_{21}e_{12}+a_{22}e_{23}+a_{23}e_{13}]=a_{11}a_{21}[e_{12},e_{12}]+a_{11}a_{22}[e_{12},e_{23}]+\dots+a_{13}a_{23}[e_{13},e_{13}]=\varphi([e_{12},e_{23}])=\varphi(e_{13})=a_{31}e_{12}+a_{32}e_{23}+a_{33}e_{13}$. Considering again only the non-vanishing Lie brackets, we have that $[\varphi(e_{12}),\varphi(e_{23})]=a_{11}a_{22}[e_{12},e_{23}]+a_{12}a_{21}[e_{23},e_{12}]=a_{11}a_{22}e_{13}-a_{12}a_{21}e_{13}=(a_{11}a_{22}-a_{12}a_{21})e_{13}=a_{31}e_{12}+a_{32}e_{23}+a_{33}e_{13}=\varphi(e_{13})$. Comparing the scalars, for the three elements of the basis, gives the following relations, $$
a_{31}=0$$
$$a_{32}=0$$
$$a_{33}=(a_{11}a_{22}-a_{12}a_{21})\neq 0
$$
which gives the following matrix, $$\varphi(v)=\begin{pmatrix}
a_{11} & a_{12} & a_{13}\\
a_{21} & a_{22} & a_{23}\\
0 & 0 & det(A)\\
\end{pmatrix}
$$
where $A$ is the minor $$
A=\begin{pmatrix}
a_{11} & a_{12}\\
a_{21} & a_{22}\\
\end{pmatrix}
$$
We can observe that for every $v\in\mathcal{L}_{p,n}$, writing $M=\varphi(v)$ lines in the following way,
$$
M=\begin{pmatrix}
\varphi(e_{12})\\
\varphi(e_{23})\\
\varphi(e_{n-1n})\\
\varphi(e_{13})\\
\vdots\\
\varphi(e_{n-1n})\\
\vdots\\
\varphi(e_{1n})\\
\end{pmatrix}\\
$$
where $m=\binom{n}{2}$, divides $M$ to a block matrix, $$M=\begin{pmatrix}
M_{11} & \vline & M_{12}&\vline & \dots& \vline & M_{1n-1} & \vline&M_{1n}\\
\hline
M_{21} & \vline & M_{22}&\vline & \dots &\vline & M_{2n-1} &\vline& M_{2n}\\
\hline
\vdots & \vline & \vdots&\vline & \dots &\vline & \vdots &\vline& \vdots\\
\hline
M_{n1} & \vline & M_{n2}&\vline & \dots &\vline & M_{nn-1} &\vline& M_{nn}\\
\end{pmatrix}
$$
where $M_{ij}\in\mathcal{M}_{k\times l}(\mathbb{Q}_p)$, $k=$dim$(\gamma_i\mathcal{L}),l=$dim$(\gamma_j\mathcal{L})$. From this, we can understand that the blocks on the main diagonal of $M$ are squared matrices, $A_{ii}\in\mathcal{M}_{n-i}$. From the calculation on $\mathcal{L}_{p,3}$, we understand also that any element $e_{ii+k}\in\gamma_k\mathcal{L}_{p,n}$ must vanish in the images of elements from higher nilpotency classes, i.e. $\varphi(e_{i,i+l})$, where $l>k$, which means that all the elements under every squared block on the main diagonal must be zero, so $M$ has the form, $$M=\begin{pmatrix}
M_{11} & \vline & M_{12}&\vline & M_{13} & \dots& \vline & M_{1m-1} & \vline&M_{1m}\\
\hline
0 & \vline & M_{22}&\vline & M_{23} & \dots &\vline & M_{2m-1} &\vline& M_{2m}\\
\hline
\vdots & \vline & \vdots&\vline & \vdots & \dots &\vline & \vdots &\vline& \vdots\\
\hline
0 & \vline & 0&\vline & 0 & \dots &\vline & M_{2m-1} &\vline& M_{2m}\\
\hline
0 & \vline & 0 &\vline & 0 & \dots &\vline & 0 &\vline& M_{mm}\\
\end{pmatrix}
$$
We observe that the matrix $M_{ij}$ blocks represent quotients of the form $\sfrac{\gamma_i\mathcal{L}_{p,n}}{\gamma_{i+1}\mathcal{L}_{p,n}},\sfrac{\gamma_j\mathcal{L}_{p,n}}{\gamma_{j+1}\mathcal{L}_{p,n}}$. We shall state, as a fact, that the block $M_{11}$ is either diagonal or anti-diagonal, i.e., $$M_{11}=\begin{pmatrix}
\lambda_1 & & &\\
& \lambda_2 & &\\
& & \ddots &\\
& & & \lambda_n\\
\end{pmatrix}$$
or
$$M_{11}=\begin{pmatrix}
& & & \lambda_1\\
& & \lambda_2 &\\
& \iddots & &\\
\lambda_n & & &\\
\end{pmatrix}$$
In the case of an anti-diagonal block, we have the following proposition,
\begin{proposition}
\label{prop.involution}
Let $p$ be a prime number, and let $n\in\mathbb{N}$. $B_n=\{e_1,\dots,e_{m-1}\}$, where $m=\binom{n}{2}$. Then, the map $\eta_n:B_n\rightarrow B_n$, defined by $\eta_n(e_i):=e_{m-i}$ is a $\mathcal{L}_{p,n}$-automorphism, which is also an involution.
\end{proposition}
\begin{proof}
Clearly, $\eta_n$ is the anti-diagonal $m\times m$ matrix,$$
\eta_n=\begin{pmatrix}
& & & 1 &\\
& & 1 &\\
& \iddots & &\\
1 & & &\\
\end{pmatrix}
$$
$\eta_n$ is an invertible matrix, which operates on any vector $$v=(a_1,a_2,\dots,a_{m-1})=\sum_{i=1}^{m-1}a_ie_i$$ in the following way,$$
\eta_n(v)=\eta_n\bigg(\sum_{i=1}^{m-1}a_ie_i\bigg)=\begin{pmatrix}
a_1 & a_2 & \dots & a_{m-1}\\
\end{pmatrix}\begin{pmatrix}
& & & 1 &\\
& & 1 &\\
& \iddots & &\\
1 & & &\\
\end{pmatrix}=\begin{pmatrix}
a_{m-1} & a_{m-2} & \dots & a_1\\
\end{pmatrix}=$$
$$\begin{pmatrix}
\eta_n(a_1) & \eta_n(a_2) & \dots & \eta_n(a_{m-1})\\
\end{pmatrix}=\sum_{i=1}^{m-1}\eta_n(a_ie_i)=\sum_{i=1}^{m-1}a_i\eta_n(e_i)
$$
And, $\eta_n^2(v)=\eta_n(\eta_n(v))=\eta_n\bigg(\eta_n\bigg(\sum_{i=1}^{m-1}a_ie_i\bigg)\bigg)=\eta_n\bigg(\sum_{i=1}^{m-1}a_i\eta_n(e_i)\bigg)=\sum_{i=1}^{m-1}a_i\eta_n^2(e_i)=\sum_{i=1}^{m-1}a_i\eta_n(e_{n-i})=\sum_{i=1}^{m-1}a_ie_i$
\end{proof}
From this proposition, we realize that if $M_{11}$ is anti-diagonal, then $\eta_n\varphi$ is the automorphism which has that $M_{11}$ is diagonal.
\begin{proposition}
\label{prop.main.diagonal.blocks}
Let $p$ be a prime number, and let $n\in\mathbb{N}$, and let $M=\varphi\in\mathcal{L}_{p,n}$. Then, all the blocks on the main diagonal, $M_{ii},\dots,M_{n-1n-1}$, are diagonal, of the form,$$
M=\varphi=\begin{pmatrix}
\lambda_1 & & & & & & & &\\
& \lambda_2 & & & & & & &\\
& & \ddots & & & & & &\\
& & & \lambda_n & & & & &\\
& & & & \lambda_1\lambda_2 & & & &\\
& & & & & \lambda_2\lambda_3 & & &\\
& & & & & & \ddots & &\\
& & & & & & & \lambda_{n-1}\lambda_n & &\\
& & & & & & & & \ddots &\\
& & & & & & & & & \lambda_1\lambda_2\cdots\lambda_n\\
\end{pmatrix}
$$
\end{proposition}
\begin{proof}
By simple induction. We have already assumed that $M_{11}$ is diagonal. Every sequential block $M_{ii}$ contains the coefficients of elements of $\sfrac{\gamma_i\mathcal{L}_{p,n}}{\gamma_{i+1}\mathcal{L}_{p,n}}$ as summands in images of elements of the same quotient algebra. So, $\varphi(e_{ii+2})=\sum_{i=1}^{n-2}a_{ii+2}e_{ii+2}$, but $e_{ii+2}=[e_{ii+1},e_{i+1i+2}]$, so $\varphi(e_{ii+2})=[\varphi(e_{ii+1}),\varphi(e_{i+1i+2})]$, hence, $\lambda_{i+2}=a_{ii+2}=a_{ii+1}a_{i+1i+2}=\lambda_{i}\lambda_{i+1}$, which proves the proposition.
\end{proof}
\begin{proposition}
\label{prop.h.matrix.determinant}
Let $n\in\mathbb{N}$, and let $$
A_n=\begin{pmatrix}
\lambda_1 & & & & & & & &\\
& \lambda_2 & & & & & & &\\
& & \ddots & & & & & &\\
& & & \lambda_n & & & & &\\
& & & & \lambda_1\lambda_2 & & & &\\
& & & & & \lambda_2\lambda_3 & & &\\
& & & & & & \ddots & &\\
& & & & & & & \lambda_{n-1}\lambda_n & &\\
& & & & & & & & \ddots &\\
& & & & & & & & & \lambda_1\lambda_2\cdots\lambda_n\\
\end{pmatrix}
$$
where $\lambda_1,\lambda_2,\dots,\lambda_n\in\mathbb{Q}_p$, then, $det(A_n)=\prod_{i=1}^{n}\lambda_i^{i(n+1-i)}$.
\end{proposition}
\begin{proof}
We observe that the determinants, for $n=1,2,3,\dots$, form a recursive sequence, $$
det(A_1)=\lambda_1$$
$$det(A_2)=det(A_1)\lambda_1\lambda_2^2$$
$$det(A_3)=det(A_2)\lambda_1\lambda_2^2\lambda_3^3$$
$$\vdots$$
$$det(A_n)=det(A_{n-1})\lambda_1\lambda_2^2\lambda_3^3\cdots\lambda_n^n$$
Calculating the general element, $a_n=det(A_n)$, we see that we have $n$ times $\lambda_1$, $n-1$ times $\lambda_2^2$, $n-2$ times $\lambda_3^3$, and so forth. In general, we have $n-i+1$ times $\lambda_i^i$, which means that we have $i(n-i+1)$ times $\lambda_i$, and in total, $a_n=det(A_n)=\prod_{i=1}^n\lambda_i^{i(n+1-i)}$.
\end{proof}
This means that every $M=\varphi\in\mathcal{L}_{p,n}$ is of the form,$$M=\varphi=\begin{pmatrix}
\begin{matrix}\lambda_1 & & &\\
& \lambda_2 & &\\
& & \ddots &\\
& & & \lambda_n\\
\end{matrix} & \vline & M_{12}&\vline & M_{13} & \dots& \vline & M_{1m-1} & \vline&M_{1m}\\
\hline
0 & \vline & \begin{matrix}\lambda_1\lambda_2 & & &\\
& \lambda_2\lambda_3 & &\\
& & \ddots &\\
& & & \lambda_{n-1}\lambda_n\\
\end{matrix}&\vline & M_{23} & \dots &\vline & M_{2m-1} &\vline& M_{2m}\\
\hline
\vdots & \vline & \vdots&\vline & \vdots & \dots &\vline & \vdots &\vline& \vdots\\
\hline
0 & \vline & 0&\vline & 0 & \dots &\vline & M_{2m-1} &\vline& M_{2m}\\
\hline
0 & \vline & 0 &\vline & 0 & \dots &\vline & 0 &\vline& \lambda1\lambda2\cdots\lambda_n\\
\end{pmatrix}$$
The above discussion gives rise to the decomposition of each $\varphi\in\mathcal{L}_{p,n}$ to two matrices, one is the diagonal matrix $$h=\begin{pmatrix}
\lambda_1 & & & & & & & &\\
& \lambda_2 & & & & & & &\\
& & \ddots & & & & & &\\
& & & \lambda_n & & & & &\\
& & & & \lambda_1\lambda_2 & & & &\\
& & & & & \lambda_2\lambda_3 & & &\\
& & & & & & \ddots & &\\
& & & & & & & \lambda_{n-1}\lambda_n & &\\
& & & & & & & & \ddots &\\
& & & & & & & & & \lambda_1\lambda_2\cdots\lambda_n\\
\end{pmatrix}
$$
and the other matrix is $$n=\begin{pmatrix}
1 &* &* &* &* &* &* & *\\
& 1 &* &* &* &* &* &*\\
& & \ddots & *& *& *& *& *&\\
& & & 1 &* &* & *& *\\
& & & & 1 &* &* & *\\
& & & & & 1 & *& *\\
& & & & & & \ddots & *\\
& & & & & & & 1
\end{pmatrix}
$$
So, we have the following proposition,
\begin{proposition}
\label{prop.automorphism.matrix,decomposition}
Let $p$ be a prime number, and let $n\in\mathbb{N}$, and let $M=\varphi\in\mathcal{L}_{p,n}$. Then, $M=\varphi=nh$, where $n$ and $h$ are of the above form.
\end{proposition}
\begin{proof}
Trivially, $h$ is an invertible matrix, and its inverse is the matrix $$h^{-1}=\begin{pmatrix}
\lambda_1^{-1} & & & & & & & &\\
& \lambda_2^{-1} & & & & & & &\\
& & \ddots & & & & & &\\
& & & \lambda_n^{-1} & & & & &\\
& & & & (\lambda_1\lambda_2)^{-1} & & & &\\
& & & & & (\lambda_2\lambda_3)^{-1} & & &\\
& & & & & & \ddots & &\\
& & & & & & & (\lambda_{n-1}\lambda_n)^{-1} & &\\
& & & & & & & & \ddots &\\
& & & & & & & & & (\lambda_1\lambda_2\cdots\lambda_n)^{-1}\\
\end{pmatrix}
$$
Easy to check that $n=Mh^{-1}$ is also an invertible matrix, with $1$ on the main diagonal, and $0$ below it.
\end{proof}
We observe that all the matrices with non-zero elements on the main diagonal, and $0$ everywhere else form an abelian subgroup of $G_n(\mathbb{Q}_p)$, since multiplying such matrices yields a matrix of the same specification. Let $$h_{\alpha}=\begin{pmatrix}
\alpha_1 & & &\\
& \alpha_2 & &\\
& & \ddots &\\
& & & \alpha_m\\
\end{pmatrix},h_{\beta}=\begin{pmatrix}
\beta_1 & & &\\
& \beta_2 & &\\
& & \ddots &\\
& & & \beta_m\\
\end{pmatrix}$$
Then, $$h_{\alpha}h_{\beta}=\begin{pmatrix}
\alpha_1\beta_1 & & &\\
& \alpha_2\beta_2 & &\\
& & \ddots &\\
& & & \alpha_m\beta_m\\
\end{pmatrix}=\begin{pmatrix}
\beta_1\alpha_1 & & &\\
& \beta_2\alpha_2 & &\\
& & \ddots &\\
& & & \beta_m\alpha_m\\
\end{pmatrix}=h_{\beta}h_{\alpha}
$$
Obviously, this subgroup, which we shall denote as $H<G_n(\mathbb{Q}_p)$ is not normal, as we observe by taking the $n$ matrix described above, and multiplying $A=nhn^{-1}$, clearly $A\notin H$. On the other hand, the set of all $n$ matrices if a normal subgroup of $G_n(\mathbb{Q}_p)$, because if $$n_{\alpha}=\begin{pmatrix}
1 & \alpha_{12} & \alpha_{13} & \dots & \alpha_{1m}\\
& 1 & \alpha_{23} & \dots & \alpha_{2m}\\
& & \ddots &  \vdots& \vdots &\\
& &  &  1& \alpha_{m-1m} &\\
& & & & 1\\
\end{pmatrix},n_{\beta}=\begin{pmatrix}
1 & \beta_{12} & \beta_{13} & \dots & \beta_{1m}\\
& 1 & \beta_{23} & \dots & \beta_{2m}\\
& & \ddots &  \vdots& \vdots &\\
& &  &  1& \beta_{m-1m} &\\
& & & & 1\\
\end{pmatrix}
$$
Then $$n_\alpha n_\beta=\begin{pmatrix}
1 & \alpha_{12}+\beta_{12} & * & \dots & *\\
& 1 & * & \dots & *\\
& & \ddots &  \vdots& \vdots &\\
& &  &  1& \alpha_{m-1m}+\beta_{m-1m} &\\
& & & & 1\\
\end{pmatrix}
$$
which proves that all the $n$ matrices form a subgroup, which we shall denote by $N\in G_n(\mathbb{Q}_p)$.
taking any matrix, $g\in G_n(\mathbb{Q}_p)$, and taking the product $A=gng^{-1}$, if we look at the main diagonals, we see that the product is of the general form $$
gng^{-1}=\begin{pmatrix}
\lambda_1 & a_{12} & \dots & a_{1m}\\
0 & \lambda_2 & \dots & a_{2m}\\
& & \ddots &*\\
& & & \lambda_m\\
\end{pmatrix}\begin{pmatrix}
1 & b_{12} & \dots & b_{1m}\\
0 & 1 & \dots & b_{2m}\\
& & \ddots &*\\
& & & 1\\
\end{pmatrix}\begin{pmatrix}
\lambda_1^{-1} & c_{12} & \dots & c_{1m}\\
0 & \lambda_2^{-1} & \dots & c_{2m}\\
& & \ddots &*\\
& & & \lambda_m^{-1}\\
\end{pmatrix}=$$
$$
\begin{pmatrix}
1 & d_{12} & \dots & d_{1m}\\
0 & 1 & \dots & d_{2m}\\
& & \ddots &*\\
& & & 1\\
\end{pmatrix}\in N
$$
So, $N\lhd G_n(\mathbb{Q}_p)$ is a normal subgroup.
This discussion gives rise to the decomposition of $G_n(\mathbb{Q}_p)$. 
Since only $N$ is a normal subgroup of $G_n(\mathbb{Q}_p)$, we decompose $G_n(\mathbb{Q}_p)$ to a semi-direct product, $G\cong N\rtimes H$, where the map $\phi:H\rightarrow Aut(N)$, given by $\phi(h)(n):=hnh^{-1}$, for every $h\in H$, and $n\in N$, is a homomorphism, as we can see by the fact that for every $h_1,h_2\in H$, and for every $n\in N$, $\phi(h_1)\phi(h_2)(n)=h_1n(h_2nh_2^{-1})h_1^{-1}=h_1h_2nh_2^{-1}h_1^{-1}=(h_1h_2)n(h_1h_2)^{-1}=\phi(h1h2)
(n)$
This means that calculating the integral, for $G_n(\mathbb{Q}_p)$, reduces to calculating a double integral, $\displaystyle\int_{N\rtimes H}$. We mean to show in the research that the normal subgroup $N$ can itself be decomposed to a semi-direct product of several subgroups, thus simplyifing the integration.



By \ref{prop.integer.automorphism}, we have that any $L_p(\mathbb{Z}_p)$-automorphism must be in $G_n(\mathbb{Z}_p)$, in words, any $\varphi\in G(\mathbb{Z}_p)$ is an invertible matrix with elements in $\mathbb{Z}_p$. Our goal is to find a way to compute $G(\mathbb{Z}_p)$, the automorphism group of $L_n(\mathbb{Z}_p)$, for any $n\in\mathbb{N}$. After finding a general formula for this calculation, we shall be able to show a way to compute the $n$-multiple $p$-adic integral of the form $\displaystyle\int\int\dots\int\int_{D_1\times D_2\dots\times D_{n-1}\times D_n}f(h_1,h_2,\dots,h_{n-1},h_n)d(\mu_1,\mu_2,\dots,\mu_{n-1},\mu_n)$, where $D_i$ is the set of $G(\mathbb{Z}_p)$-cosets, for $G(\mathbb{Z}_p)$, the group of $\mathbb{Z}_p$-automorphisms on the algebra $L_i(\mathbb{Z}_p)$, and $h_i$ is any element of this group, and $\mu_i$ is the Haar measure on this group. By Fubini, this multiple integral can be calculated as the iterated integral $$\displaystyle\int_{D_n}\left(\int_{D_{n-1}}\dots\left(\int_{D_2}\left(\int_{D_1}f(h_1,h_2,\dots,h_{n-1},h_n)d\mu_1\right)d\mu_2\right)\dots d\mu_{n-1}\right)d\mu_n$$. Alternatively, if we do not find an explicit formula for this calculation, we will show the general approach for this calculation, and prove the necessary conditions for its validity. 
\section{Notations}
\begin{itemize}
\item $\mathbb{Z}_p$, the ring of $p$-adic integers.
\item $\mathbb{Q}_p$, the fraction field of $\mathbb{Z}_p$.
\item $L_p$, a $\mathbb{Z}_p$-algebra over the ring of p-adic integers.
\item $\mathcal{L}_p$, a $\mathbb{Q}_p$-algebra, over the fraction field of $\mathbb{Z}_p$.
\item $G(L_p):=Aut_{\mathbb{Z}_p}(L_p)$, the group of $\mathbb{Z}_p$-automorphisms of $L_p$.
\item $G(\mathcal{L}_p):=Aut_{\mathbb{Q}_p}(\mathcal{L}_p)$, the group of $\mathbb{Q}_p$-automorphisms of $\mathcal{L}_p$.
\end{itemize}
\end{document}