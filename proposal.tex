\documentclass[12pt]{article}
\makeatletter
\newcommand*{\rom}[1]{\expandafter\@slowromancap\romannumeral #1@}
\makeatother
\usepackage{amsfonts, amssymb}
\usepackage{mathrsfs, mathdots} 
\usepackage{amsmath}
\usepackage{float}
\usepackage{amsthm}
\usepackage{tikz-cd}
\usepackage{xcolor}
\usepackage{xparse}
\usepackage{setspace}
\usepackage{xfrac}
\usepackage{yfonts}

\newtheorem{theorem}{Theorem}[subsection]
\newtheorem{proposition}[theorem]{Proposition}
\newtheorem{corollary}[theorem]{Corollary}
\newtheorem{lemma}[theorem]{Lemma}
\newtheorem{notations}[theorem]{Notations}
\newtheorem{definition}[theorem]{Definition}
\newtheorem{example}[theorem]{Example}

\ExplSyntaxOn
\NewDocumentCommand{\cycle}{ O{\;} m }
 {
  (
  \alec_cycle:nn { #1 } { #2 }
  )
 }

\seq_new:N \l_alec_cycle_seq
\cs_new_protected:Npn \alec_cycle:nn #1 #2
 {
  \seq_set_split:Nnn \l_alec_cycle_seq { , } { #2 }
  \seq_use:Nn \l_alec_cycle_seq { #1 }
 }
\ExplSyntaxOff
\usepackage{arydshln}

\setlength{\dashlinedash}{1.2pt}
\setlength{\dashlinegap}{1.5pt}
\setlength{\arrayrulewidth}{0.2pt}
\begin{document}
\begin{abstract}
Let $G$ be any group. For any natural number $n\in\mathbb{N}$, let $a_n$ be the number of subgroups $H\leq G$, such that $[G:H]=n$. Assume $G$ is finitely-generated, then $a_n<\infty$, and we can define a Dirichlet series of the form $\zeta_G(s):=\sum_{i=1}^\infty a_n n^{-s}$, where $s\in\mathbb{C}$. Assume, in addition, that $G$ is also nilpotent and torsion-free, then this function has some properties of the Riemann $\zeta$-function, such as the Euler decomposition of $\zeta$ into a product of local factors indexed by primes. A version of this $\zeta$-function counts pro-isomorphic subgroups, and an analogous function may be defined for appropriate Lie rings. We study here the pro-isomorphic $\zeta$-functions for a family of nilpotent Lie rings of unbounded nilpotency class. We shall compute the automorphism groups of these Lie rings explicitly, prove uniformity of the local factors of the pro-isomorphic $\zeta$-functions local factors, and aim to determine them explicitly.
\end{abstract}
\section{Scientific Background}
\subsection{Introduction}
We start our discussion with the following proposition, which stands at the very foundation of our subject.
\begin{proposition} \label{prop:finite.number.subgroups}
Let $G$ be any finitely generated group, and let $n\in\mathbb{N}$ any natural number. Then there is a finite number of subgroups $H\leq G$, such that $[G:H]=n$
\end{proposition}
\begin{proof}
Let $H\leq G$ be a subgroup, such that $[G:H]=n$, and let $\sfrac{G}{H}:=\{g_1H,g_2H,\dots,g_nH\}$ be the set containing all left-cosets of $H$ in $G$. We may consider the action of $G$ by left multiplication on $\sfrac{G}{H}$: for all $g\in G$, and for all left-cosets $g_iH\in\sfrac{G}{H}$, we have that $g(g_iH):=(gg_i)H=g_jH$, where $g_jH\in\sfrac{G}{H}$ is some left-coset. This means that $g$ maps every index $i\in[n]$ to some index $j\in[n]$, which means that $g$ operates as a permutation on $[n]$. Therefore, there exists a homomorphism $f:G\rightarrow\mathcal{S}_n$, from $G$ to the symmetric group of order $n$. Let $1\leq k\leq n$ be the index for which $g_kH=H$.
For all $g\in G$, $g\in H$ if and only if $gH=H$, which means that $gg_kH=g_kH$, which means that $f(g)(k)=k$, in words, $k$ is fixed by the image $f(g)$. This means that $H=\{g\in G : f(g)(k)=k\}$, so $H$ may be recovered from the homomorphism $f$, which clearly shows that the number of subgroups $H\leq G$ of index $n$ is less or equal to the number of maps $f : G
\rightarrow\mathcal{S}_n$. $G$ is finitely-generated, and $\mathcal{S}_n$ is finite. We know that group homomorphisms are uniquely determined by the maps of their generators. Therefore, there are finitely many maps from a finite set of generators to a finite group, which clearly shows that the number of subgroups $H\leq G$ of index $n$ is necessarily finite.
\end{proof}
This proposition gives rise to an entire subject in group theory, called \textbf{subgroup growth}. We denote by $a_n(G)$ the number of subgroups of $G$ of index $n$, and look at the sequence $\{a_n(G)\}$. The subject of subgroup growth aims to relate the properties of this sequence to the algebraic structure of $G$. For instance, Lubotzky, Mann and Segal showed\cite{LubotzkyMannSegal} that $a_n(G)$ grows polynomially if and only if $G$ is virtually nilpotent of finite rank. That is, $G$ has a finite-index nilpotent subgroup, and all finitely-generated subgroups may be generated by a bounded number of generators. This research concentrates on the growth of \textbf{pro-isomorphic} subgroups, which we now define.
\begin{definition}
\label{def:profinite.closure}
Let $G$ be any group, and let $\mathcal{N}:=\{N_i\trianglelefteq G\}_{i\in I}$ be the set of all normal subgroups of $G$. We define a partial order on $\mathcal{N}$ by reverse inclusion, that is for every two indices $i,j$ we say that $i\leq j$ if and only if $N_j\subseteq N_i$, hence for every $i\leq j$ there exists a natural projection map $\pi_{ji}:\sfrac{G}{N_
j}\rightarrow\sfrac{G}{N_i}$. The inverse limit \[\widehat{G}=\underset{\leftarrow}{lim}\{\sfrac{G}{N_k}\}_{k\in I}:=\{(h_k)_{k\in I}\in\prod_{k\in I}\sfrac{G}{N_k} : \pi_{ji}(h_j)=h_i,\forall i\leq j\}\] is called the \textbf{profinite closure} of $G$.
\end{definition}
\begin{definition}
\label{def:pro.isomorphic}
Let $G$ be any group. A subgroup $H\leq G$ is called \textbf{pro-isomorphic} if $\widehat{H}\cong\widehat{G}$.
\end{definition}
\begin{definition}
\label{def:zeta.pro.isomorphic}
Let $G$ be any group, and let \[\hat{a_n}(G):=\#\{H\leq G : \widehat{H}\cong\widehat{G}, [G:H]=n\}\] be the number of pro-isomorphic subgroups of $G$ of index $n$. Assume that $\hat{a_n}(G)<\infty$ for all $n$. The \textbf{pro-isomorphic $\zeta$-function} of $G$ is defined by $\hat{\zeta_G}(s):=\sum_{n=1}^{\infty}\hat{a_n}(G)n^{-s}$ for $s\in\mathbb{C}$.
\end{definition}
If our $\zeta$-function, which is a special case of the Dirichlet series, has some properties of convergence, on some subset of $\mathbb{C}$, one may reconstruct its coefficients, $\hat{a_n(G)}$, which, in our case, are the number of subgroups of our interest, using the \textbf{Perron's forumla}, which is an implementation of a \textbf{reverse Mellin transform}, as discussed, for example, in \cite{MontgomeryVaughan}. This method, including the specific properties of convergence required for the reconstruction, is out of the scope of our research, and therefore will not be further discussed, at this stage.\par
It is known that if $\hat{a_n}(G)$ grows polynomially, then $\hat{\zeta_G}(s)$ converges on some right half-plane of $\mathbb{C}$. For instance, we take the group $G=(\mathbb{Z},+)$. The group $\mathbb{Z}$ is an abelian group, and every subgroup is of the form $H=n\mathbb{Z}=\langle n\rangle$, for some $n\in\mathbb{N}$, which means that $H\cong \mathbb{Z}$, as both are infinite cyclic groups, and so, $\widehat{H}\cong\widehat{\mathbb{Z}}$. Since we have only one subgroup of index $n$, for every $n\in\mathbb{N}$, then $a_n(\mathbb{Z})=\hat{a_n}(\mathbb{Z})=1$. Thus, its pro-isomorphic $\zeta$-function is $\hat{\zeta_{\mathbb{Z}}}=\sum_{i=1}^{\infty}n^{-s}=\zeta(s)$, the Riemann $\zeta$-function, which is known to converge for $Re(s)>1$.\par
After establishing the basic definitions, we observe a fact that is a major motivation for this research, which says that the Riemann $\zeta$-function decomposes to an infinite product of local zeta-functions, that is, $\zeta(s)=\prod_p\zeta_p(s)=\prod_p\sum_{k=0}^\infty p^{-ks}=\prod_p\frac{1}{1-p^{-s}}$, where the product runs over all the prime numbers. Following this fact, regarding the Riemann $zeta$-function, we observe that for any finitely-generated, nilpotent and torsion-free group, $G$, we have the same decomposition as above for the pro-isomorphic $\zeta$-function: $\hat{\zeta_G}(s)=\prod_p\hat{\zeta_{G,p}}(s)$, where $\hat{\zeta_{G,p}}(s):=\sum_{k=0}^\infty \hat{a_{p^k}}(G)p^{-ks}$
We hereby bring several basic definitions of group nilpotency, which are very important for this research.\par
The general construction of $\zeta$-functions, as well as their Euler decomposition to local $\zeta_p$-functions, for the study of subgroup growth, were well established by Grunewald, Segal and Smith, in \cite{GrunewaldSegalSmith}.
\begin{definition}
\label{def.lower.central.series}
Let $G$ be any group, then the \textbf{lower central series} of $G$ is a sequence of subgroups of $G$ defined by the recursive rule $G_n:=[G,G_{n-1}]$, for every $n\in\mathbb{N}$, where $G_0:=G$.  We recall that $[G,G_n]\leq G$ is the subgroup generated by the collection of commutators $\{gg_ng^{-1}g_n^{-1} : g\in G,g_n\in G_n\}$
\end{definition}
\begin{definition}
\label{def.nilpotency.class}
Let $G$ be any group. The \textbf{nilpotency class} of $G$ is $min\{n\in\mathbb{N} : G_n=[G,G_{n-1}]=\{e\}\}$, in words, the smallest natural number, such that the subgroup of commutators of the form $[G,G_n]$ is the trivial group. We can extend this definition, and say that the trivial group nilpotency class is $0$.
\end{definition}
\begin{definition}
\label{def.nilpotency.class}
Let $G$ be a group. If $G$ is of finite nilpotency class then $G$ is said to be a \textbf{nilpotent} group.
\end{definition}
\subsection{Linearization}
We want to transfer the ideas from the above discussion about groups to a linear context, where we can use tools from linear algebra.
Hence, for finitely-generated torsion-free nilpotent groups $G$, we associate nilpotent Lie algebras over $\mathbb{Z}$. This, in general, is called the \textbf{Maltsev correspondance}. 
If $L$ is a $\mathbb{Z}$-Lie algebra, namely a free $\mathbb{Z}$-module of a finite rank with a Lie bracket, then consider the (finite) number $\hat{a_n}(L)$ of subalgebras $M\leq L$, such that $M\otimes\mathbb{Z}_p\cong L\otimes\mathbb{Z}_p$ for all primes $p$. The Dirichlet series $\hat{\zeta_L}(S):=\sum_{n=1}^{\infty}\hat{a_n}(L)n^{-1}$, is called the \textbf{pro-isomorphic $\zeta$-function} of $L$. By the Maltsev correspondence, to every finitely-generated, nilpotent, torsion-free group $G$, one may associate a Lie algebra $L(G)$, such that $\hat{\zeta_{G,p}}(s)=\hat{\zeta_{L,p}}(s)$, for but finitely many primes $p$. If $G$ has nilpotency class $2$, one may obtain the equality for all primes. Let $L$ be a $\mathbb{Z}$-Lie algebra, and fix a $\mathbb{Z}$-basis $\mathcal{B}=\{b_1,\dots,b_d\}$. Let $\mathcal{L}_{n,p}=L\otimes_{\mathbb{Z}}\mathbb{Q}_p$, for any $p$. This is a $\mathbb{Q}_p$-Lie algebra, and our choice of basis allows us to identify the automorphism group $G(\mathbb{Q}_p)=Aut_{\mathbb{Q}_p}(\mathcal{L}_{n,p})$ with a subgroup of $GL_d(\mathbb{Q}_p)$. Note that $\mathcal{L}_{n,p}$ contains a $\mathbb{Z}_p$-lattice, $L_{n,p}=L\otimes_{\mathbb{Z}}\mathbb{Z}_p$. If $\varphi\in G(\mathbb{Q}_p)$, then $\varphi(L_{n,p})=L_{n,p}$ if and only if $\varphi\in G(\mathbb{Z}_p)=G(\mathbb{Q}_p)\cap GL_d(\mathbb{Z}_p)$. Here $GL_d(\mathbb{Z}_p)$ is the group of $d\times d$ matrices which are invertible over $\mathbb{Z}_p$. Similarly, $\varphi(L_{n,p})\subseteq L_{n,p}$ if and only if $G^{+}(\mathbb{Q}_p):=G(\mathbb{Q}_p)\cap M_d(\mathbb{Z}_p)$, where $M_d(\mathbb{Z}_p)$ is the collection of $d\times d$ matrices with entries in $\mathbb{Z}_p$. Note that $G^{+}(\mathbb{Q}_p)$ is a monoid, not a group.\par
Denote by $G(\mathbb{Z}_p)g$, where $g\in G^{+}(\mathbb{Q}_p)$, a right-coset of $G(\mathbb{Z}_p)$, one checks that the monoid $G^{+}(\mathbb{Q}_p)$ is a disjoint union of right-cosets of $G(\mathbb{Z}_p)$.\par
The discussion above reveals the construction we base our research upon.
We observe that there is a bijection between the set $G(\mathbb{Z}_p)\backslash G^+(\mathbb{Q}_p)$ of right-cosets and $\{M\leq L_{n,p} : M\cong L_{n,p}\}$ the set of $L_{n,p}$-subalgebras which are isomorphic to $L_{n,p}$ itself. This bijection takes $G(\mathbb{Z}_p)g$ to $M=\varphi(L_{n,p})$. For any $\varphi\in G(\mathbb{Z}_p)g$, this is well-defined. One checks that for every $\psi\in G(\mathbb{Z}_p)g$, we have that $\psi(L_{n,p})=\varphi(L_{n,p})=M$.
We end this part, as a preparation for the final part of this technical background review, with the following result, which states that for each right-coset, $G(\mathbb{Z}_p)g$, if $M=\varphi(L_{n,p})$, where $\varphi\in G(\mathbb{Z}_p)g$, then $[L_{n,p}:M]=|\det\varphi|_p^{-1}$, and therefore,\par $\hat{\zeta_{L,p}}(s)=\underset{\overset{\scriptscriptstyle M\leq L_{n,p}}{\scriptscriptstyle M\cong L_{n,p}}}{\sum}[L_{n,p}:M]^{-s}=\underset{\scriptscriptstyle G(\mathbb{Z}_p)\varphi\in G(\mathbb{Z}_p)\backslash G^+(\mathbb{Q}_p)}{\sum}|\det\varphi|_p^s$.
\subsection{$p$-adic Integration}
In this final part of the technical background review, we finally get to the motivation for all the construction we have presented in the first parts. We now define a very central object for our research. We shall assume, without proof, the existence of such an object, under the prerequisites of the definition.
\begin{definition}
\label{def.right.haar.measure}
Let $\Gamma$ be a locally compact topological group, i.e., for all $\gamma\in\Gamma$, there is an open neighborhood of $\gamma\in U_{\gamma}$, and a compact subset $K_{\gamma}$, such that $U_{\gamma}\subset K_{\gamma}$. Then there is a measure $\mu$, with the following property: for any measurable subset, $U\subseteq\Gamma$, and any $\gamma\in\Gamma$, $\mu(U\gamma)=\mu(U)$, where $U\gamma:=\{u\gamma : u\in U\}$. Such a measure $\mu$ is called a \textbf{right Haar measure}, and is unique up to multiplication by a non-zero constant.
\end{definition}
Equipped with the right Haar measure, we can finally make use of the construction from above. We start by claiming, without proof, that for every prime number $p$, the group $G(\mathbb{Q}_p)$ is a locally compact topological group. We also claim that the right Haar measure has the property that $\mu(G(\mathbb{Z}_p))=1$. The measure of all the right-cosets of $G(\mathbb{Z}_p)$ equals to the measure of $G(\mathbb{Z}_p)$ itself, i.e., for every $g\in G^{+}(\mathbb{Q}_p)$, we have that $\mu(G(\mathbb{Z}_p)g)=\mu(G(\mathbb{Z}_p))=1$.
With this observation, we go directly to the calculation of the $p$-adic norm of the determinant of every $L_{n,p}$-automorphism, as a $p$-adic integral over our measure space.
First, we observe that given any $L_{n,p}$-automorphism in some right-coset, $\varphi\in G(\mathbb{Z}_p)\varphi$, we have that $|\det\varphi|_p^s=\displaystyle\int_{G(\mathbb{Z}_p)\varphi}|\det\varphi|_p^sd\mu$, because $\mu(G(\mathbb{Z}_p)\varphi)=1$, and $|\det\varphi|_p^{-1}$ is fixed on $G(\mathbb{Z}_p)\varphi$.\par
Going back to our desired function, we observe that\par $\hat{\zeta_{L,p}}(s)=\underset{\scriptscriptstyle G(\mathbb{Z}_p)\varphi\in G(\mathbb{Z}_p)\backslash G^+(\mathbb{Q}_p)}{\sum}|\det\varphi|_p^s=\underset{\scriptscriptstyle G(\mathbb{Z}_p)\varphi\in G(\mathbb{Z}_p)\backslash G^+(\mathbb{Q}_p)}{\sum}\displaystyle\int_{G(\mathbb{Z}_p)\varphi}|\det\varphi|_p^sd\mu=\displaystyle\int_{G^{+}(\mathbb{Z}_p)}|\det\varphi|_p^sd\mu$.\par 
This calculation of the local $\zeta_p$-function as a $p$-adic integral was established by the work of du Sautoy and Lubotzky, in \cite{DuSautoyLubotzky}.
This integral is the main object we shall study, in this research.
We end this part, of the technical background review, by a definition and a theorem, which stand in the center of our research goals.
\begin{definition}
Let $L$ be a $\mathbb{Z}$-algebra. Then $\hat{\zeta_L}(s)$ is called \textbf{uniform}, if there exists a rational function, $W\in\mathbb{Q}(X,Y)$, such that for every prime number $p$, the local function $\zeta_{L,p}(s)=W(p,p^{-1})$.\par 
Sometimes, we prefer to say that $\zeta_{L,p}(s)$ is \textbf{uniform}, if there exists a function $W_p\in\mathbb{Q}(X)$, such that $\zeta_{L,p}(s)=W_p(p^{-s})$.\par
Both definitions are just two dif and only iferent ways of writing the rational function $W_p(p^{-s})=W(p,p^{-s})$.\par
Here, $\mathbb{Q}(X)$ and $\mathbb{Q}(X,Y)$ are the fields of rational functions in one variable and two variables, respectively.\par
The term uniform expresses that the rational function depends, for every $p$, only on $p$ and its inverse.\par
\end{definition}
\begin{theorem}
\label{thm.rational.function}
Let $p$ be a prime number, $s\in\mathbb{C}$, then $\hat{\zeta_{L,p}}(s)$ is uniform. 
\end{theorem}
The uniformity is also established in the work of Grunewald, Segal and Smith, see \cite{GrunewaldSegalSmith}.
\section{Research Goals and Methodology}
\subsection{The unipotent group $U_n$}
We start by defining the families of groups and corresponding Lie algebras that will be considered in this project.
\begin{definition}
\label{def.unipotent.matrix}
Let $\mathcal{R}$ be a commutative ring. Then $U_n(\mathcal{R})\leq GL_n(\mathcal{R})$ is the subgroup of upper unitriangular matrices over $\mathcal{R}$, i.e. $U_n(\mathcal{R})=\Bigg\{
\begin{pmatrix}
1 & a_{12} &\\
  & \ddots & \ddots\\
  & & 1\\
\end{pmatrix}\Bigg\}$.
\end{definition}
 One easily checks that $U_n(\mathcal{R})$ is a special case of a \textbf{unipotent} group.
Looking into the structure of $U_n(\mathcal{R})$, we observe that we can find a set of generators, $\mathcal{U}_n(\mathcal{R})$. Denote by $E_{ij}$, where $i<j$, an elementary matrix of the form $\begin{pmatrix}
1 & 0 & \dots & 0\\
  & \ddots & 1 & 0\\
  &  & 1 & 0\\
  & & & 1\\
  \end{pmatrix}\in U_n(\mathcal{R})$, where, in addition to the main diagonal, only the element in row $i$ and column $j$ is $1$, and all the other elements are $0$. One checks that if $i<j$ and $k<l$, then the commutator is \[
  [E_{ij},E_{kl}]=\begin{cases}
    E_{il}, & j=k\\
    -E_{kj}, & {i=l}\\
    I_n, & \text{otherwise}\\
    \end{cases}
    \]
  In addition, it is easy to observe that $E_{ij}^m=\begin{pmatrix}
1 & 0 & \dots & 0\\
  & \ddots & m & 0\\
  &  & 1 & 0\\
  & & & 1\\
  \end{pmatrix}$, for every $m\in\mathbb{Z}$, by simple induction.
All this means that we can generate $U_n(\mathcal{R})$ entirely by the set $\mathcal{U}_n:=\{E_{1,2},\dots,E_{n-1,n}\}$. From this, it is also clear that $U_n(\mathcal{R})$ is nilpotent, of nilpotency class $n$, because the longest chain of commutators in $U_n(\mathcal{R})$, i.e., chaining all the elements in the set of generators, $\mathcal{U}_n$, by the order of their indices, yields $[E_{1,2},[E_{2,3},[\dots,[E_{n-1,n}]]]]=E_{1,n}$, which means that $\gamma_{n-1}U_n(\mathcal{R})=\{E_{1,n}\}$, which means that $\gamma_n U_n(\mathcal{R})=I_n$.\par
This also shows that $U_n(\mathcal{R})$ is torsion-free if and only if $\mathcal{R}$ itself is torsion-free, as can be readily seen from the fact that $E_{ij}^m$ has an $m$ in row $i$ and column $j$, and $m\neq 0$ if and only if $\mathcal{R}$ is torsion-free, for instance, $\mathcal{R}=\mathbb{Z}$.\par
These facts, regarding $U_n(\mathbb{Z})$ place it as a group of our interest, for this research, and bring us next to its associated Lie algebra.
\subsection{The Lie algebras $L_{n,p}$}
We start with the elementary matrices $E_{i,j}$ where $i<j$, from above, and define strictly upper triangular matrices of the form $e_{i,j}=E_{i,j}-I_n$, in words, $e_{i,j}$ is obtained by replacing all the $1$ on the main diagonal with $0$. It is readily seen that the standard brackets operation on these matrices is compatible with the commutator on matrices in $\mathcal{U}_n$, i.e., $[e_{i,i+1},e_{j,j+1}]=[E_{i,i+1}-I_n,E_{j,j+1}-I_n]=[E_{i,i+1},E_{j,j+1}]-I_n=e_{i,i+1}e_{j,j+1}-e_{j,j+1}e_{i,i+1}$. Considering now $\mathcal{R}=\mathbb{Z}$, we construct a nilpotent $\mathbb{Z}$-Lie algebra of strictly upper triangular matrices over $\mathbb{Z}$, which we denote by $L_n$, with the standard brackets operation as its Lie brackets. As discussed above, this $\mathbb{Z}$-Lie algebra can be extended to a $\mathbb{Z}_p$-algebra, which we denote by $L_{n,p}$, and then to a $\mathbb{Q}_p$-algebra, which we denote by $\mathcal{L}_{n,p}$.
it is readily seen that the set of matrices of the form $e_{ij}$ where $i<j$, spans the whole $\mathbb{Z}$-Lie algebra $L_n$ and is $\mathbb{Z}$-linearly independent. Therefore, it forms the standard basis for $L_n$ as a free module over $\mathbb{Z}$, $\mathcal{B}_n:=\{e_{1,2},e_{1,3},\dots,e_{1,n},e_{2,3},\dots,e_{2,n},\dots,e_{n-1,n}\}$. One easily checks that rank$L_n$=$|\mathcal{B}_n|=\binom{n}{2}$, which is the number of elements above the main diagonal for every $n\in\mathbb{N}$. Obviously, the same goes also for the extensions of $L_n$, namely $L_{n,p}$ and $\mathcal{L}_{n,p}$. And so, we have reached the target of our research, which is studying the $\hat{\zeta_{L,p}}$-function on the $\mathbb{Z}_p$-Lie algebra associated with $U_n(\mathbb{Z}_p)$.
\subsection{Research goals}
As stated above, this research will focus on studying $U_n(\mathbb{Z}_p)$ and its associated $\mathbb{Z}_p$-Lie algebra, namely $L_{n,p}$. The project consists of three major steps:\par
1. \textbf{Calculating the automorphism group of the $\mathbb{Q}_p$-Lie algebras $\mathcal{L}_{n,p}$, for all $n\in\mathbb{N}$ and all primes $p$.}\par
2. \textbf{Showing that the pro-isomorphic zeta-functions $\hat{\zeta_{L_{n,p}}}(s)$ are uniform for all $n\in\mathbb{N}$.}\par
3. \textbf{Giving an explicit uniform formula for the zeta-functions $\hat{\zeta_{L_{n,p}}}(s)$ for specific values of $n$, if not for all $n\in\mathbb{N}$.}\par
As we elaborate further, steps 1 and 2 are already known entirely for $n\leq 5$, and step 3 is known for $n\leq 4$.
We start with the first step of calculating $Aut_{\mathbb{Q}_p}(\mathcal{L}_{n,p})$. These automorphism groups have been studied for decades from a different point of view.  There are classical results showing that any automorphism may be expressed as a product of automorphisms of a specific type; see, for instance, the main result of Gibbs \cite{Gibbs}.  These results are not explicit enough for our purposes; indeed, the submonoid $G^+(\mathbb{Q}_p)$ arises, for us, as the domain of integration of a $p$-adic integral.  In order to compute this integral, we need to decompose the automorphism group $G(\mathbb{Q}_p)$ into a repeated semi-direct product of groups with a simple structure.
\par
After we have analyzed the structure of $G(\mathbb{Z}_p)$, we will need to construct the monoid $G^{+}(\mathbb{Q}_p)$ and its $G(\mathbb{Z}_p)$ right-cosets, as we have seen above. This will give us both the function to integrate, which is $\det\varphi$ for every $G(\mathbb{Z}_p)$ right-coset $G(\mathbb{Z}_p)\varphi$, and the domain of integration, which is the monoid $G^+(\mathbb{Q}_p)$. We will use this information to analyze the behavior of the $p$-adic integral we have described above and prove that its calculation depends only on $p$, thus showing that the $\hat{\zeta_{L,p}}$-function is uniform.
\subsection{The Heisenberg group}
We show the very basic approach to this problem, by the simplest example, the \textbf{Heisenberg group}, which is simply $U_3(\mathbb{Z})$, and its associated $\mathbb{Z}$-Lie algebra. However, this example is far from complying with the general case of $n>3$, as we shall see later, and we only use it to demonstrate the basic technique we shall be using. 
Since $U_3(\mathbb{Z})$ is the group of unipotent $3\times 3$ matrices over $\mathbb{Z}$, we observe that the basis for the associated $\mathbb{Z}$-Lie algebra is $\mathcal{B}_3=\{e_{12},e_{13},e_{23}\}$. We also observe that for every $L_n$ we can apply a linear order to the basis $\mathcal{B}_n$, where $e_{ij}<e_{kl}$ if the difference $j-i$ is less than $l-k$, or if both differences are equal, and $i<k$. In other words, we apply an order that divides $\mathcal{B}_n$ to basis elements of $\sfrac{L_n}{\gamma_2 L_n},\sfrac{\gamma_2 L_n}{\gamma_3 L_n},\dots,\sfrac{\gamma_{n-2}L_n}{\gamma_{n-1}L_n},\gamma_{n-1}L_n$. Therefore, we set the basis of $L_3$ according to this order, $\mathcal{B}_3=\{e_{12},e_{23},e_{13}\}$.
Every $\varphi\in G(\mathbb{Z})$ must obey the Lie brackets. This means that if we analyze $\varphi$ by its operation on the basis elements, then $\varphi$ must be some $3\times 3$ matrix over $\mathbb{Z}$, such that multiplying, from the right, with any vector $v=xe_{12}+ye_{23}+ze_{13}\in L_{n,p}$, yields a vector $u=x\varphi(e_{12})+y\varphi(e_{23})+z\varphi(e_{13})\in L_{n,p}$, such that $\varphi(u)=\varphi^2(v)=v$, i.e., \[\begin{pmatrix}
x & y & z\\
\end{pmatrix}\begin{pmatrix}
a_{11} & a_{12} & a_{13}\\
a_{21} & a_{22} & a_{23}\\
a_{31} & a_{32} & a_{33}\\
\end{pmatrix}=\begin{pmatrix}
\varphi(x) & \varphi(y) & \varphi(z)\\
\end{pmatrix}\]
then we observe that $a_{31}e_{12}+a_{32}e_{23}+a_{33}e_{13}=\varphi(e_{13})=\varphi[(e_{12}),(e_{23})]=[\varphi(e_{12}),\varphi(e_{23})]=[a_{11}e_{12}+a_{12}e_{23}+a_{13}e_{13},a_{21}e_{12}+a_{22}e_{23}+a_{23}e_{13}]=(a_{11}a_{22}-a_{12}a_{21})e_{13}$, which gives the following relations, $$
a_{31}=0$$
$$a_{32}=0$$
$$a_{33}=(a_{11}a_{22}-a_{12}a_{21})\neq 0
$$
which means that $\varphi$ is the following matrix, $\varphi=\begin{pmatrix}
a_{11} & a_{12} & a_{13}\\
a_{21} & a_{22} & a_{23}\\
0 & 0 & \det A\\
\end{pmatrix}
$
where $A=\begin{pmatrix}
a_{11} & a_{12}\\
a_{21} & a_{22}\\
\end{pmatrix}
$, and $A$ must be invertible, otherwise $\varphi$ is not bijective. Based on the construction from earlier, of $L_{n,p}$ and $\mathcal{L}_{n,p}$, all the above applies also for $G(\mathbb{Z}_p)$ and for $G(\mathbb{Q}_p)$, respectively.
\subsection{$U_n(\mathbb{Z}_p)$ groups for $n>3$}
Mark N. Berman, in his doctoral thesis\cite{Berman}, has displayed an explicit formula for $\zeta_{L_{4,p}}$, and proved that $\zeta_{L_{5,p}}$ is, indeed, uniform.
Following his work, we first observe that for every $v\in L_{n,p}$ the $\mathbb{Z}_p$-Lie algebra associated with $U_n(\mathbb{Z}_p)$, where $n\geq3$, we present $\varphi(v)$ as the multiplication of $v$ by a matrix from the right $\varphi(v)=vM$. As stated earlier, $M$ is an $r\times r$ matrix, where $r$=rank$L_{n,p}=\binom{n}{2}$, whose lines are set by the order we have defined above, i.e., considering the standard basis \[\mathcal{B}_n=\{e_{
12},e_{23},\dots,e_{n-1,n},e_{13},\dots,e_{n-1.n},\dots,e_{1n}\}\] then $M$ is the following matrix,
$$
M=\begin{pmatrix}
\varphi(e_{12})\\
\varphi(e_{23})\\
\varphi(e_{n-1,n})\\
\hdashline
\varphi(e_{13})\\
\vdots\\
\varphi(e_{n-2,n})\\
\hdashline
\vdots\\
\hdashline
\varphi(e_{1n})\\
\end{pmatrix}\\
$$
Following this division of the basis by quotients of the nilpotnecy classes of $L_{n,p}$, we observe that $M$, therefore, is divided into a block matrix, $$M=\begin{pmatrix}
M_{11} & \vline & M_{12}&\vline & \dots& \vline & M_{1n-1} & \vline&M_{1n}\\
\hline
M_{21} & \vline & M_{22}&\vline & \dots &\vline & M_{2n-1} &\vline& M_{2n}\\
\hline
\vdots & \vline & \vdots&\vline & \ddots &\vline & \vdots &\vline& \vdots\\
\hline
M_{n1} & \vline & M_{n2}&\vline & \dots &\vline & M_{nn-1} &\vline& M_{nn}\\
\end{pmatrix}
$$
each block is $M_{ij}\in\mathcal{M}_{k\times l}(\mathbb{Z}_p)$, where $k$=rank$\gamma_iL_{n,p}$ and $l$=rank$\gamma_jL_{n,p}$. From this, we can understand that the blocks on the main diagonal of $M$ are squared matrices, $A_{ii}\in\mathcal{M}_{n-i}$. From the calculation on the Heisenberg group, we understand also that any element $e_{ii+k}\in\gamma_kL_p$ must vanish in the images of elements of higher nilpotency classes, i.e. $\varphi(e_{i,i+l})$, where $l>k$, which means that all the elements under every squared block on the main diagonal must be zero, so $M$ has the form, $$M=\begin{pmatrix}
M_{11} & \vline & M_{12}&\vline & M_{13} & \dots& \vline & M_{1m-1} & \vline&M_{1m}\\
\hline
0 & \vline & M_{22}&\vline & M_{23} & \dots &\vline & M_{2m-1} &\vline& M_{2m}\\
\hline
\vdots & \vline & \vdots&\vline & \vdots & \ddots &\vline & \vdots &\vline& \vdots\\
\hline
0 & \vline & 0&\vline & 0 & \dots &\vline & M_{2m-1} &\vline& M_{2m}\\
\hline
0 & \vline & 0 &\vline & 0 & \dots &\vline & 0 &\vline& M_{mm}\\
\end{pmatrix}
$$
We observe that the matrix $M_{ij}$ blocks represent quotients of the form $\sfrac{\gamma_i\mathcal{L}_{p,n}}{\gamma_{i+1}\mathcal{L}_{p,n}},\sfrac{\gamma_j\mathcal{L}_{p,n}}{\gamma_{j+1}\mathcal{L}_{p,n}}$. We shall state, as a fact, that the block $M_{11}$ is either diagonal or anti-diagonal, i.e., $$M_{11}=\begin{pmatrix}
\lambda_1 & & &\\
& \lambda_2 & &\\
& & \ddots &\\
& & & \lambda_n\\
\end{pmatrix}$$
or
$$M_{11}=\begin{pmatrix}
& & & \lambda_1\\
& & \lambda_2 &\\
& \iddots & &\\
\lambda_n & & &\\
\end{pmatrix}$$
In the case of an anti-diagonal block, we have the following proposition,
\begin{proposition}
\label{prop.involution}
Let $p$ be a prime number, and let $n\in\mathbb{N}$. $B_n=\{e_1,\dots,e_{m-1}\}$, where $m=\binom{n}{2}$. Then, the map $\eta_n:B_n\rightarrow B_n$, defined by $\eta_n(e_i):=e_{m-i}$ is a $\mathcal{L}_{p,n}$-automorphism, which is also an involution.
\end{proposition}
\begin{proof}
Clearly, $\eta_n$ is the anti-diagonal $m\times m$ matrix,$$
\eta_n=\begin{pmatrix}
& & & 1 &\\
& & 1 &\\
& \iddots & &\\
1 & & &\\
\end{pmatrix}
$$
$\eta_n$ is an invertible matrix, which operates on any vector $$v=(a_1,a_2,\dots,a_{m-1})=\sum_{i=1}^{m-1}a_ie_i$$ in the following way,$$
\eta_n(v)=\eta_n\bigg(\sum_{i=1}^{m-1}a_ie_i\bigg)=\begin{pmatrix}
a_1 & a_2 & \dots & a_{m-1}\\
\end{pmatrix}\begin{pmatrix}
& & & 1 &\\
& & 1 &\\
& \iddots & &\\
1 & & &\\
\end{pmatrix}=\begin{pmatrix}
a_{m-1} & a_{m-2} & \dots & a_1\\
\end{pmatrix}=$$
$$\begin{pmatrix}
\eta_n(a_1) & \eta_n(a_2) & \dots & \eta_n(a_{m-1})\\
\end{pmatrix}=\sum_{i=1}^{m-1}\eta_n(a_ie_i)=\sum_{i=1}^{m-1}a_i\eta_n(e_i)
$$
And, $\eta_n^2(v)=\eta_n(\eta_n(v))=\eta_n\bigg(\eta_n\bigg(\sum_{i=1}^{m-1}a_ie_i\bigg)\bigg)=\eta_n\bigg(\sum_{i=1}^{m-1}a_i\eta_n(e_i)\bigg)=\sum_{i=1}^{m-1}a_i\eta_n^2(e_i)=\sum_{i=1}^{m-1}a_i\eta_n(e_{n-i})=\sum_{i=1}^{m-1}a_ie_i$
\end{proof}
From this proposition, we realize that if $M_{11}$ is anti-diagonal, then $\eta_n\varphi$ is the automorphism which has that $M_{11}$ is diagonal.
\begin{proposition}
\label{prop.main.diagonal.blocks}
Let $p$ be a prime number, and let $n\in\mathbb{N}$, and let $M=\varphi\in\mathcal{L}_{p,n}$. Then, all the blocks on the main diagonal, $M_{ii},\dots,M_{n-1n-1}$, are diagonal, of the form,$$
M=\varphi=\begin{pmatrix}
\lambda_1 & & & & & & & &\\
& \lambda_2 & & & & & & &\\
& & \ddots & & & & & &\\
& & & \lambda_{n-1} & & & & &\\
& & & & \lambda_1\lambda_2 & & & &\\
& & & & & \lambda_2\lambda_3 & & &\\
& & & & & & \ddots & &\\
& & & & & & & \lambda_{n-2}\lambda_{n-1} & &\\
& & & & & & & & \ddots &\\
& & & & & & & & & \lambda_1\lambda_2\cdots\lambda_{n-1}\\
\end{pmatrix}
$$
\end{proposition}
\begin{proof}
By simple induction. We have already assumed that $M_{11}$ is diagonal. Every sequential block $M_{ii}$ contains the coefficients of elements of $\sfrac{\gamma_i\mathcal{L}_{p,n}}{\gamma_{i+1}\mathcal{L}_{p,n}}$ as summands in images of elements of the same quotient algebra. So, $\varphi(e_{ii+2})=\sum_{i=1}^{n-2}a_{ii+2}e_{ii+2}$, but $e_{ii+2}=[e_{ii+1},e_{i+1i+2}]$, so $\varphi(e_{ii+2})=[\varphi(e_{ii+1}),\varphi(e_{i+1i+2})]$, hence, $\lambda_{i+2}=a_{ii+2}=a_{ii+1}a_{i+1i+2}=\lambda_{i}\lambda_{i+1}$, which proves the proposition.
\end{proof}
\begin{proposition}
\label{prop.h.matrix.determinant}
Let $n\in\mathbb{N}$, and let $$
A_n=\begin{pmatrix}
\lambda_1 & & & & & & & &\\
& \lambda_2 & & & & & & &\\
& & \ddots & & & & & &\\
& & & \lambda_{n-1} & & & & &\\
& & & & \lambda_1\lambda_2 & & & &\\
& & & & & \lambda_2\lambda_3 & & &\\
& & & & & & \ddots & &\\
& & & & & & & \lambda_{n-2}\lambda_{n-1} & &\\
& & & & & & & & \ddots &\\
& & & & & & & & & \lambda_1\lambda_2\cdots\lambda_{n-1}\\
\end{pmatrix}
$$
where $\lambda_1,\lambda_2,\dots,\lambda_{n-1}\in\mathbb{Q}_p$, then, $\det(A_n)=\prod_{i=1}^{n}\lambda_i^{i(n+1-i)}$.
\end{proposition}
\begin{proof}
We observe that the determinants, for $n=1,2,3,\dots$, form a recursive sequence, $$
\det(A_1)=\lambda_1$$
$$\det(A_2)=\det(A_1)\lambda_1\lambda_2^2$$
$$\det(A_3)=\det(A_2)\lambda_1\lambda_2^2\lambda_3^3$$
$$\vdots$$
$$\det(A_n)=\det(A_{n-1})\lambda_1\lambda_2^2\lambda_3^3\cdots\lambda_n^n$$
Calculating the general element, $a_n=\det(A_n)$, we see that we have $n$ times $\lambda_1$, $n-1$ times $\lambda_2^2$, $n-2$ times $\lambda_3^3$, and so forth. In general, we have $n-i+1$ times $\lambda_i^i$, which means that we have $i(n-i+1)$ times $\lambda_i$, and in total, $a_n=\det(A_n)=\prod_{i=1}^n\lambda_i^{i(n+1-i)}$.
\end{proof}
This means that every $M=\varphi\in\mathcal{L}_{p,n}$ is of the form,$$M=\begin{pmatrix}
\begin{matrix}\lambda_1 & & &\\
& \lambda_2 & &\\
& & \ddots &\\
& & & \lambda_{n-1}\\
\end{matrix} & \vline & M_{12}&\vline & M_{13} & \dots& \vline & M_{1,m-2} & \vline&M_{1,m-1}\\
\hline
0 & \vline & \begin{matrix}\lambda_1\lambda_2 & & &\\
& \lambda_2\lambda_3 & &\\
& & \ddots &\\
& & & \lambda_{n-1}\lambda_n\\
\end{matrix}&\vline & M_{23} & \dots &\vline & M_{2,m-2} &\vline& M_{2,m-1}\\
\hline
\vdots & \vline & \vdots&\vline & \vdots & \dots &\vline & \vdots &\vline& \vdots\\
\hline
0 & \vline & 0&\vline & 0 & \dots &\vline & \ddots &\vline& M_{m-2,m-1}\\
\hline
0 & \vline & 0 &\vline & 0 & \dots &\vline & 0 &\vline& \lambda1\lambda2\cdots\lambda_{n-1}\\
\end{pmatrix}$$
The above discussion gives rise to the decomposition of each $\varphi\in\mathcal{L}_{p,n}$ to two matrices, one is the diagonal matrix $$h=\begin{pmatrix}
\lambda_1 & & & & & & & &\\
& \lambda_2 & & & & & & &\\
& & \ddots & & & & & &\\
& & & \lambda_n & & & & &\\
& & & & \lambda_1\lambda_2 & & & &\\
& & & & & \lambda_2\lambda_3 & & &\\
& & & & & & \ddots & &\\
& & & & & & & \lambda_{n-1}\lambda_n & &\\
& & & & & & & & \ddots &\\
& & & & & & & & & \lambda_1\lambda_2\cdots\lambda_n\\
\end{pmatrix}
$$
and the other matrix is $$n=\begin{pmatrix}
1 &* &* &* &* &* &* & *\\
& 1 &* &* &* &* &* &*\\
& & \ddots & *& *& *& *& *&\\
& & & 1 &* &* & *& *\\
& & & & 1 &* &* & *\\
& & & & & 1 & *& *\\
& & & & & & \ddots & *\\
& & & & & & & 1
\end{pmatrix}
$$
So, we have the following proposition,
\begin{proposition}
\label{prop.automorphism.matrix,decomposition}
Let $p$ be a prime number, and let $n\in\mathbb{N}$, and let $M=\varphi\in\mathcal{L}_{p,n}$. Then, $M=\varphi=nh$, where $n$ and $h$ are of the above form.
\end{proposition}
\begin{proof}
Trivially, $h$ is an invertible matrix, and its inverse is the matrix $$h^{-1}=\begin{pmatrix}
\lambda_1^{-1} & & & & & & & &\\
& \lambda_2^{-1} & & & & & & &\\
& & \ddots & & & & & &\\
& & & \lambda_n^{-1} & & & & &\\
& & & & (\lambda_1\lambda_2)^{-1} & & & &\\
& & & & & (\lambda_2\lambda_3)^{-1} & & &\\
& & & & & & \ddots & &\\
& & & & & & & (\lambda_{n-1}\lambda_n)^{-1} & &\\
& & & & & & & & \ddots &\\
& & & & & & & & & (\lambda_1\lambda_2\cdots\lambda_n)^{-1}\\
\end{pmatrix}
$$
Easy to check that $n=Mh^{-1}$ is also an invertible matrix, with $1$ on the main diagonal, and $0$ below it.
\end{proof}
We observe that all the matrices with non-zero elements on the main diagonal, and $0$ everywhere else form an abelian subgroup of $G_n(\mathbb{Q}_p)$, since multiplying such matrices yields a matrix of the same specification. Let $$h_{\alpha}=\begin{pmatrix}
\alpha_1 & & &\\
& \alpha_2 & &\\
& & \ddots &\\
& & & \alpha_m\\
\end{pmatrix},h_{\beta}=\begin{pmatrix}
\beta_1 & & &\\
& \beta_2 & &\\
& & \ddots &\\
& & & \beta_m\\
\end{pmatrix}$$
Then, $$h_{\alpha}h_{\beta}=\begin{pmatrix}
\alpha_1\beta_1 & & &\\
& \alpha_2\beta_2 & &\\
& & \ddots &\\
& & & \alpha_m\beta_m\\
\end{pmatrix}=\begin{pmatrix}
\beta_1\alpha_1 & & &\\
& \beta_2\alpha_2 & &\\
& & \ddots &\\
& & & \beta_m\alpha_m\\
\end{pmatrix}=h_{\beta}h_{\alpha}
$$
Obviously, this subgroup, which we shall denote as $H<G_n(\mathbb{Q}_p)$ is not normal, as we observe by taking the $n$ matrix described above, and multiplying $A=nhn^{-1}$, clearly $A\notin H$. On the other hand, the set of all $n$ matrices if a normal subgroup of $G_n(\mathbb{Q}_p)$, because if $$n_{\alpha}=\begin{pmatrix}
1 & \alpha_{12} & \alpha_{13} & \dots & \alpha_{1m}\\
& 1 & \alpha_{23} & \dots & \alpha_{2m}\\
& & \ddots &  \vdots& \vdots &\\
& &  &  1& \alpha_{m-1m} &\\
& & & & 1\\
\end{pmatrix},n_{\beta}=\begin{pmatrix}
1 & \beta_{12} & \beta_{13} & \dots & \beta_{1m}\\
& 1 & \beta_{23} & \dots & \beta_{2m}\\
& & \ddots &  \vdots& \vdots &\\
& &  &  1& \beta_{m-1m} &\\
& & & & 1\\
\end{pmatrix}
$$
Then $$n_\alpha n_\beta=\begin{pmatrix}
1 & \alpha_{12}+\beta_{12} & * & \dots & *\\
& 1 & * & \dots & *\\
& & \ddots &  \vdots& \vdots &\\
& &  &  1& \alpha_{m-1m}+\beta_{m-1m} &\\
& & & & 1\\
\end{pmatrix}
$$
which proves that all the $n$ matrices form a subgroup, which we shall denote by $N\in G_n(\mathbb{Q}_p)$.
taking any matrix, $g\in G_n(\mathbb{Q}_p)$, and taking the product $A=gng^{-1}$, if we look at the main diagonals, we see that the product is of the general form $$
gng^{-1}=\begin{pmatrix}
\lambda_1 & a_{12} & \dots & a_{1m}\\
0 & \lambda_2 & \dots & a_{2m}\\
& & \ddots &*\\
& & & \lambda_m\\
\end{pmatrix}\begin{pmatrix}
1 & b_{12} & \dots & b_{1m}\\
0 & 1 & \dots & b_{2m}\\
& & \ddots &*\\
& & & 1\\
\end{pmatrix}\begin{pmatrix}
\lambda_1^{-1} & c_{12} & \dots & c_{1m}\\
0 & \lambda_2^{-1} & \dots & c_{2m}\\
& & \ddots &*\\
& & & \lambda_m^{-1}\\
\end{pmatrix}=$$
$$
\begin{pmatrix}
1 & d_{12} & \dots & d_{1m}\\
0 & 1 & \dots & d_{2m}\\
& & \ddots &*\\
& & & 1\\
\end{pmatrix}\in N
$$
So, $N\lhd G_n(\mathbb{Q}_p)$ is a normal subgroup.
This discussion gives rise to the decomposition of $G_n(\mathbb{Q}_p)$. 
Since only $N$ is a normal subgroup of $G_n(\mathbb{Q}_p)$, we decompose $G_n(\mathbb{Q}_p)$ to a semi-direct product, $G\cong N\rtimes H$, where the map $\phi:H\rightarrow Aut(N)$, given by $\phi(h)(n):=hnh^{-1}$, for every $h\in H$, and $n\in N$, is a homomorphism, as we can see by the fact that for every $h_1,h_2\in H$, and for every $n\in N$, $\phi(h_1)\phi(h_2)(n)=h_1n(h_2nh_2^{-1})h_1^{-1}=h_1h_2nh_2^{-1}h_1^{-1}=(h_1h_2)n(h_1h_2)^{-1}=\phi(h1h2)
(n)$
This means that calculating the integral, for $G_n(\mathbb{Q}_p)$, reduces to calculating a double integral, $\displaystyle\int_{N\rtimes H}$. We mean to show in the research that the normal subgroup $N$ can itself be decomposed to a semi-direct product of several subgroups, thus simplyifing the integration.



By \ref{prop.integer.automorphism}, we have that any $L_{n,p}(\mathbb{Z}_p)$-automorphism must be in $G_n(\mathbb{Z}_p)$, in words, any $\varphi\in G(\mathbb{Z}_p)$ is an invertible matrix with elements in $\mathbb{Z}_p$. Our goal is to find a way to compute $G(\mathbb{Z}_p)$, the automorphism group of $L_n(\mathbb{Z}_p)$, for any $n\in\mathbb{N}$. After finding a general formula for this calculation, we shall be able to show a way to compute the $n$-multiple $p$-adic integral of the form $\displaystyle\int\int\dots\int\int_{D_1\times D_2\dots\times D_{n-1}\times D_n}f(h_1,h_2,\dots,h_{n-1},h_n)d(\mu_1,\mu_2,\dots,\mu_{n-1},\mu_n)$, where $D_i$ is the set of $G(\mathbb{Z}_p)$-cosets, for $G(\mathbb{Z}_p)$, the group of $\mathbb{Z}_p$-automorphisms on the algebra $L_i(\mathbb{Z}_p)$, and $h_i$ is any element of this group, and $\mu_i$ is the Haar measure on this group. By Fubini, this multiple integral can be calculated as the iterated integral $$\displaystyle\int_{D_n}\left(\int_{D_{n-1}}\dots\left(\int_{D_2}\left(\int_{D_1}f(h_1,h_2,\dots,h_{n-1},h_n)d\mu_1\right)d\mu_2\right)\dots d\mu_{n-1}\right)d\mu_n$$. Alternatively, if we do not find an explicit formula for this calculation, we will show the general approach for this calculation, and prove the necessary conditions for its validity. 
\subsection{Possible Extension}
To conclude this part, of the research goals review, we shall display a possible extension, considering a general ring of integers, as our associated Lie algebra. Taking some number field, $K=\sfrac{K}{\mathbb{Q}}$, of a finite degree, $d$, over $\mathbb{Q}$, and its ring of integers, $\mathcal{O}_K$, which is a $d$-dimensional $\mathbb{Z}$-algebra. We can try applying everything we have established, regarding algebras over $\mathbb{Z}_p$ and $\mathbb{Q}_p$, to, for example, algebras over $\mathbb{Z}[\sqrt{p}]$ and $\mathbb{Q}[\sqrt{p}]$, respectively. This approach is only one possible way to extend our research, and therefore, we shall not go into the details of it, at this stage.
\begin{thebibliography}{2}
\bibitem{Berman} M. N. Berman,
Proisomorphic zeta functions of groups
, Ph.D. thesis, University of Oxford,
2005.
\bibitem{DuSautoyLubotzky} M.P.F. du Sautoy and A. Lubotzky, Functional equations and uniformity for
local zeta functions of nilpotent groups, Amer. J. Math. 118 (1996), no. 1, 39–
90.
\bibitem{Gibbs} John A. Gibbs, Automorphisms of certain unipotent groups, J. Algebra 14 (1970), 203-228.
\bibitem{GrunewaldSegalSmith} F. J. Grunewald, D. Segal, and G. C. Smith, Subgroups of finite index in nilpotent groups,
Invent. Math. 93 (1988), no. 1, 185–223, DOI 10.1007/BF01393692. MR943928.
\bibitem{LubotzkyMannSegal} Alexander Lubotzky, Avinoam Mann, and Dan Segal,
Finitely generated groups of polynomial
subgroup growth
, Israel J. Math.
82
(1993), no. 1-3, 363–371, DOI 10.1007/BF02808118.
MR1239055
\bibitem{MontgomeryVaughan} Hugh L. Montgomery and Robert C. Vaughan, Multiplicative Number Theory I. Classical Theory, Cambridge Studies in Advanced Mathematics 97, chapter 5.
\end{thebibliography}
\end{document}