\documentclass{article}
\makeatletter
\newcommand*{\rom}[1]{\expandafter\@slowromancap\romannumeral #1@}
\makeatother
\usepackage{amsfonts, amssymb}
\usepackage{mathrsfs, mathdots} 
\usepackage{amsmath}
\usepackage{float}
\usepackage{amsthm}
\usepackage{tikz-cd}
\usepackage{xcolor}
\usepackage{xparse}
\usepackage{setspace}
\usepackage{xfrac}
\usepackage{yfonts}
\setcounter{MaxMatrixCols}{11}

\newtheorem{theorem}{Theorem}[subsection]
\newtheorem{proposition}[theorem]{Proposition}
\newtheorem{corollary}[theorem]{Corollary}
\newtheorem{lemma}[theorem]{Lemma}
\newtheorem{notations}[theorem]{Notations}
\newtheorem{definition}[theorem]{Definition}
\newtheorem{example}[theorem]{Example}
\newtheorem{remark}[theorem]{Remark}

\newtheorem{theorem2}{Theorem}[section]
\newtheorem{proposition2}[theorem2]{Proposition}
\newtheorem{corollary2}[theorem2]{Corollary}
\newtheorem{lemma2}[theorem2]{Lemma}
\newtheorem{notations2}[theorem2]{Notations}
\newtheorem{definition2}[theorem2]{Definition}
\newtheorem{example2}[theorem2]{Example}
\newtheorem{remark2}[theorem2]{Remark}

\ExplSyntaxOn
\NewDocumentCommand{\cycle}{ O{\;} m }
 {
  (
  \alec_cycle:nn { #1 } { #2 }
  )
 }

\seq_new:N \l_alec_cycle_seq
\cs_new_protected:Npn \alec_cycle:nn #1 #2
 {
  \seq_set_split:Nnn \l_alec_cycle_seq { , } { #2 }
  \seq_use:Nn \l_alec_cycle_seq { #1 }
 }
\ExplSyntaxOff
\usepackage{titling}
\newcommand{\subtitle}[1]{%
  \posttitle{%
    \par\end{center}
    \begin{center}\large#1\end{center}
    \vskip0.5em}%
}
\usepackage{arydshln}
\setcounter{tocdepth}{2}
\setlength{\dashlinedash}{1.2pt}
\setlength{\dashlinegap}{1.5pt}
\setlength{\arrayrulewidth}{0.2pt}
\usepackage{graphicx} % Required for inserting images

\begin{document}
\section{The computation of $G_{n}(\mathbb{Z}_{p})$}
\begin{proposition2}
Define $\eta=\eta_{n}:\mathcal{L}_{n,p}\rightarrow\mathcal{L}_{n,p}$ by \[\eta(e_{ij}):=(-1)^{j-i-1}e_{n+1-j,n+1-i},\]\\ for all $1\leq{i}<{j}\leq{n}$. Then
$\eta\in{G_{n}(\mathbb{Q}_{p})}$.
\end{proposition2}
\begin{proof}
Let $e_{ij},e_{jk}\in\mathcal{L}_{n,p}$, then $\eta([e_{ij},e_{jk}])=\eta(e_{ik})=(-1)^{k-i-1}e_{n+1-k,n+1-i}=(-1)^{k-j+j-i-1}[e_{n+1-k,n+1-j},e_{n+1-j,n+1-i}]=$\[=[(-1)^{k-j}e_{n+1-k,n+1-j},(-1)^{j-i-1}e_{n+1-j,n+1-i}]=\]\[=-[(-1)^{k-j-1}e_{n+1-k,n+1-j},(-1)^{j-i-1}e_{n+1-j,n+1-i}]=\]\[=-[\eta(e_{jk}),\eta(e_{ij})]=[\eta(e_{ij}),\eta(e_{jk})].\]
Therefore $\eta$ is compatible with the Lie bracket of $\mathcal{L}_{n,p}$.
\end{proof}
From the definition of $\eta$, one observes that for all $1\leq{r}\leq{n-1}$, $\eta$ operates on the elements of $\sfrac{\gamma_{r}\mathcal{L}_{n,p}}{\gamma_{r+1}\mathcal{L}_{n,p}}$ as the self-inverse permutation: \[  \begin{pmatrix}
    1 & 2 & \cdots & n-r-1 & n-r \\
    n-r & n-r-1 & \cdots & 2 & 1
  \end{pmatrix}\]
Which means that, denoting by $M^{\eta}$ the matrix representing $\eta$, each diagonal block $M^{\eta}_{rr}$ is an anti-diagonal matrix with $1$ on the anti-diagonal.
\begin{corollary2}
Let $\eta\in{G_{n}(\mathbb{Q}_{p})}$ be the automorphism defined above, and let $G_{n}^{0}(\mathbb{Q}_{p}):=\{\varphi\in{G_{n}(\mathbb{Q}_{p})} : \mathrm{block}\,M_{11}\,\mathrm{is}\,\mathrm{diagonal}\}$ be the subgroup of all diagonal automorphisms of $\mathcal{L}_{n,p}$. Then $G_{n}(\mathbb{Q}_{p})=G_{n}^{0}(\mathbb{Q}_{p})\coprod{G_{n}^{0}(\mathbb{Q}_{p})\eta}$.
\end{corollary2}
One checks that ${G_{n}^{0}}\backslash{G_{n}}=\{G_{n}^{0}(\mathbb{Q}_{p}),G_{n}^{0}(\mathbb{Q}_{p})\eta\}$, hence $G_{n}(\mathbb{Q}_{p})$ is a disjoint union of the two right-cosets of $G_{n}^{0}(\mathbb{Q}_{p})$. Moreover, since $[G_{n}(\mathbb{Q}_{p}):G_{n}^{0}(\mathbb{Q}_{p})]=2$, we have that $G_{n}^{0}(\mathbb{Q}_{p})\triangleleft{G_{n}(\mathbb{Q}_{p})}$ and ${G_{n}^{0}}\backslash{G_{n}}$ is a quotient group.
This result can also be obtained by looking at $G_{n}(\mathbb{Q}_{p})$ as an algebraic subgroup of $GL_{r}(\mathbb{Q}_{p})$, where $r=\dim\mathcal{L}_{n,p}$. Let $\mathbb{Q}_{p}^{r^2}$ be the affine $r^2$-space over $\mathbb{Q}_{p}$, with points of the form $x=(x_{11},x_{12},\dots,x_{1r},x_{21},x_{22},\dots,x_{2r},\dots,x_{r1},\dots,x_{rr})$. Let $t\in\mathbb{Q}_{p}$. Let $f^{0}(x,t)=t\prod_{i=1}^{r}x_{ii}-1$ be a polynomial in $r^2+1$ variables over $\mathbb{Q}_{p}$, where all the variables which are not $t$ and $x_{ii}$ have a coefficient of zero. Then $V^{0}=V(f^{0}):=\{(A^{0},\frac{1}{\det{A^{0}}})\}$, where every $A^{0}=(a_{ij})$ is a block upper-triangular matrix, each of the blocks on its main diagonal is itself a diagonal matrix. The matrix $A$ is taken as the sequence of elements $(a_{11},\dots,a_{1r},\dots,a_{r1},\dots,a_{rr})\in\mathbb{Q}_{p}^{r^2}$. Let $f^{1}(x,t)=t(x_{1,n-1}x_{2,n-2}\cdots{x_{n-1,1}}x_{n,2n-3}\cdots{x_{2n-3,n}}\cdots{x_{rr}})-1$ be another polynomial in $r^2+1$ variables, then $V^{1}=V(f^{1}):=\{(A^{1},\frac{1}{\det{A^{1}}})\}$, where every $A^{1}=(a_{ij})$ is a block upper-triangular matrix, each of the blocks on its main diagonal is an anti-diagonal matrix.
Let $V=V^{0}\coprod{V^{1}}$. One checks that $V$ is closed under matrix multiplication, thus it forms a subvariety of $GL_{r}(\mathbb{Q}_{p})$, and hence it is an algebraic subgroup of $GL_{r}(\mathbb{Q}_{p})$. Now we use the following proposition: Let $G$ be an algebraic group, with $G^{0}$ as its irreducible component that contains $e$, then $G^{0}\triangleleft{G}$ with the other irreducible components as cosets. We shall briefly state that the proof to this proposition is that morphisms of algebraic varieties are continuous maps, thus mapping connected sets to connected sets. But $e\in{G^{0}}$, hence the image of $(e,e)$ under multiplication is $ee=e$, which shows that $G^{0}\times{G^{0}}$ is mapped into $G^{0}$, i.e., $G^{0}$ is a subgroup of $G$.

Replacing $\varphi$ by $\varphi\circ\eta$ if necessary, we may assume without loss of generality that $M_{11}$ is diagonal. Indeed, let $G_{n}^{0}(\mathbb{Q}_{p})\leq{G_{n}(\mathbb{Q}_{p})}$ be the subgroup of automorphisms with diagonal block $M_{11}$, then\\ $G_{n}(\mathbb{Q}_{p})=G_{n}^{0}(\mathbb{Q}_{p})\coprod{G_{n}(\mathbb{Q}_{p})}\eta$, and as in \cite[Proposition 2.1]{DuSautoyLubotzky} we may replace the domain of integration in \eqref{eq:p-adic.integral} by $G_{n}^{0}(\mathbb{Q}_{p})\cap{G_{n}^{+}(\mathbb{Q}_{p})}$ after a suitable renormalization of the Haar measure. We proceed to determine $G_{n}^{0}(\mathbb{Q}_{p})$.
We check, for any $\lambda_{1},\lambda_{2},\dots,\lambda_{n-1}\in\mathbb{Q}_{p}^{\ast}$, that the diagonal matrix \[h=\mathrm{diag}(\lambda_{1},\lambda_{2},\dots,\lambda_{n-1},\lambda_{1}\lambda_{2},\lambda_{2}\lambda_{3}\dots,\lambda_{n-2}\lambda_{n-1},\dots,\lambda_{1}\lambda_{2}\cdots\lambda_{n-2}\lambda_{n-1})\]
represents an automorphism of $\mathcal{L}_{n,p}$. Thus every $g\in{G_{n}^{0}(\mathbb{Q}_{p})}$ has a unique decomposition $g=uh$, where $h$ is of the above form and $u$ has $1$'s on the diagonal. It is easy to see that $\det{h}=\prod_{i=1}^{n-1}\lambda_i^{i(n-i)}$ by induction on $n$. Since $\det{u}=1$, it follows that $|\det{g}|_{p}^{s}=|\det{h}|_{p}^{s}$.

The collection $H(\mathbb{Q}_{p})$ of diagonal matrices $h$ as above is the reductive part of $G_{n}^{0}(\mathbb{Q}_{p})$. The collection $N(\mathbb{Q}_{p})$ of matrices $u$ as above is the unipotent radical of $G_{n}^{0}(\mathbb{Q}_{p})$. We aim to determine the structure of the unipotent radical $N(\mathbb{Q}_p)$ by decomposing it into an iterative semidirect product of abelian subgroups.
For all $2\leq{r}\leq{n-1}$ we denote by $N_{r}\leq{N(\mathbb{Q}_{p})}$ the subgroup of all automorphisms of $\mathcal{L}_{n,p}$, such that\\ $M_{11}=I_{n}$ and $M_{12}=M_{13}=\cdots=M_{1,r-1}=0$. In other words, $N_{r}$ is the kernel of the natural map $G_{n}^{0}(\mathbb{Q}_{p})\rightarrow{Aut_{\mathbb{Q}_{p}}(\sfrac{\mathcal{L}_{n,p}}{\gamma_{r}\mathcal{L}_{n,p}})}$. We can describe $N_{n-1}$ explicitly as the following set of block matrices \[
N_{n-1}=\left\{\begin{pmatrix}
I_{n-1} & 0 & \cdots & 0 & M_{1,n-1}\\
0 & I_{n-2} & \cdots & 0 & 0\\
\vdots & \vdots & \ddots & \vdots & \vdots\\
0 & 0 & \cdots & I_{2} & 0\\
0 & 0 & \cdots & 0 & 1\\
\end{pmatrix}\right\}
\]
where $M_{1,n-1}$ is an arbitrary $(n-1)\times{1}$ matrix with entries in $\mathbb{Q}_{p}$ and the $0$ blocks are zero matrices of suitable size. The map $\varphi\mapsto{M_{1,n-1}}$ for all $\varphi\in{N_{n-1}}$ gives an isomorphism $N_{n-1}\cong\mathbb{Q}_{p}^{n-1}$.







\begin{proposition2}
\label{block.1r.structure}
Let $\varphi_{r}\in{N(\mathbb{Q}_{p})}$ be a unipotent automorphism of $\mathcal{L}_{n,p}$ such that the matrix upper blocks $M_{1k}$, for all $2\leq{k}\leq{r-1}$, are zero matrices. Consider the $(n-1)\times{(n-r)}$ matrix $M_{1r}=(a_{ij})$. Then,
\begin{enumerate}
    \item Let $2\leq{r}<{n-2}$. If $a_{ij}\neq{0}$, then either $i=j$ or $i=j+r-1$, and we have the relation $a_{i+r,i+1}=-a_{ii}$.
    \item Let $r=n-2$. If $a_{ij}\neq{0}$, then either $i=j$ or $i=j+r-1$ or $(i,j)\in\{(1,2),(n-1,1)\}$, with the same relation as above.
\end{enumerate}
\end{proposition2}
\begin{proof}
From the relation $[\varphi_{r}(e_{k,k+1}),\varphi_{r}(e_{l,l+1})]=0$ where $l>k+1$, we deduce that $a_{ij}\neq{0}$ only if either $i=j$ or $i=j+r-1$ or\\ $(i,j)\in\{(r+1,1),(r+2,1),(n-r-2,n-r),(n-r-1,n-r)\}$. If $r<n-2$ then it follows from the conditions
\begin{align*}
[\varphi_{r}(e_{n-r-2,n-r-1}),\varphi_{r}(e_{n-r-2,n-r})]=0\\
[\varphi_{r}(e_{n-r-1,n-r}),\varphi_{r}(e_{n-r-2,n-r})]=0\\
[\varphi_{r}(e_{r+1,r+2}),\varphi_{r}(e_{r+1,r+3})]=0\\
[\varphi_{r}(e_{r+2,r+3}),\varphi_{r}(e_{r+1,r+3})]=0\\
\end{align*}
that the four exceptional cases cannot occur. When $r=n-2$, we have that $(n-r-2,n-r)=(0,2)$ and $(r+2,1)=(n,1)$ so these cases do not exist, but so are the four conditions above, which means that the two remaining cases, $(r+1,1)=(n-1,1)$ and $(n-r-1,n-r)=(1,2)$, do not necessarily vanish.
\end{proof}
\begin{proposition2}
Denote by $N_{r}:=\{\varphi_{r} : 2\leq{r}\leq{n-2}\}\subset{N(\mathbb{Q}_{p})}$ the set of all automorphisms of the form described in \ref{block.1r.structure}, then $N_{r}\leq{N(\mathbb{Q}_{p})}$. Note that $N_2=N(\mathbb{Q}_{p})$.
\end{proposition2}
\begin{proposition2}
\label{phi.r.k}
Let $2\leq{r}\leq{n-2}$, and let $0\leq{k}\leq{n-r}$, and let $a\in\mathbb{Q}_{p}$. We extend our notation of basis elements to include $e_{01}=e_{n,n+1}=0$.
\begin{enumerate}
\item There is an automorphism $\varphi_{r,k}(a)\in{N(\mathbb{Q}_{p})}$ determined by \[
    \varphi_{r,k}(a)(e_{i,i+1}):=\begin{cases}
        e_{k,k+1}+a{e_{k,k+r}} & : i=k\\
        e_{k+r,k+r+1}-a{e_{k+1,k+r+1}} & : i=k+r\\
        e_{i,i+1} & : i\notin\{k,k+r\}
    \end{cases}
\]
\item Suppose that $r=n-2$, let $(k,l)\in\{(1,2),(n-1,1)\}$, and let $a\in\mathbb{Q}_{p}$. There is an automorphism $\varphi_{n-2,k,l}(a)\in{G_{n}^{0}(\mathbb{Q}_{p})}$ determined by \[
    \varphi_{n-2,k,l}(a)(e_{i,i+1}):=\begin{cases}
        e_{k,k+1}+a{e_{l,l+r}} & : i=k\\
        e_{i,i+1} & : i\neq{k}
    \end{cases}
\]
    We denote $\varphi'_{n-2}(a):=\varphi_{n-2,1,2}(a)$ and $\varphi''_{n-2}(a):=\varphi_{n-2,n-1,1}(a)$.
\end{enumerate}
\end{proposition2}
\begin{proof}
We need to verify that for all $1\leq{i<j}\leq{n}$ and $1\leq{l<m}\leq{n}$ we have the following relations \[
    [\varphi_{r,k}(a)(e_{ij}),\varphi_{r,k}(a)(e_{lm})]=\begin{cases}
    \varphi_{r,k}(a)(e_{im}) & : j=l\\
        -\varphi_{r,k}(a)(e_{lj}) & : i=m\\
        0 & : \mathrm{otherwise}
    \end{cases}
\]
We can verify explicitly for $n=4$ that these relations are true. Alternatively, Berman did this in \cite[\S{3.3.7}]{BermanThesis}. For $n>4$, let $m=n$, then \[[\varphi_{r,k}(a)(e_{ij}),\varphi_{r,k}(a)(e_{lm})]=[\varphi_{r,k}(a)(e_{ij}),\varphi_{r,k}(a)(e_{ln})].\]
If $i>1$, then we consider the inclusion $\iota:\mathcal{L}_{n-1,p}\hookrightarrow\mathcal{L}_{n,p}$, mapping each $e_{i,i+1}\in\mathcal{L}_{n-1,p}$ to $e_{i+1,i+2}\in\mathcal{L}_{n,p}$ for all $1\leq{i}\leq{n-2}$. By the assumption on $\mathcal{L}_{n-1,p}$, we have that \[
(\iota\circ\iota^{-1})([\varphi_{r,k}(a)(e_{ij}),\varphi_{r,k}(a)(e_{ln})])=\iota([\iota^{-1}(\varphi_{r,k}(a)(e_{ij})),\iota^{-1}(\varphi_{r,k}(a)(e_{ln}))])=\]\[=\iota([\varphi_{r,k}(a)(e_{i-1,j-1}),\varphi_{r,k}(a)(e_{l-1,n-1})])=\]\[=\begin{cases}
\iota(\varphi_{r,k}(a)(e_{i-1,n-1}))=\varphi_{r,k}(a)(e_{in}) & : j=l\\
        0 & : j\neq{l}
    \end{cases}
\]
If $i=1$, then \[[\varphi_{r,k}(a)(e_{ij}),\varphi_{r,k}(a)(e_{ln})]=[\varphi_{r,k}(a)(e_{1j}),\varphi_{r,k}(a)(e_{ln})]=\]\[=[\varphi_{r,k}(a)(e_{1j}),[\varphi_{r,k}(a)(e_{l,n-1}),\varphi_{r,k}(a)(e_{n-1,n})]].\]
By the Jacobi identity, we have that \[[\varphi_{r,k}(a)(e_{1j}),[\varphi_{r,k}(a)(e_{l,n-1}),\varphi_{r,k}(a)(e_{n-1,n})]]=\]\[=-[\varphi_{r,k}(a)(e_{n-1,n}),[\varphi_{r,k}(a)(e_{1j}),\varphi_{r,k}(a)(e_{l,n-1})]].\]
Now we use the inclusion $\iota':\mathcal{L}_{n-1,p}\hookrightarrow\mathcal{L}_{n,p}$, where $\iota'(e_{i,i+1})=e_{i,i+1}$ for all $1\leq{i}\leq{n-1}$, to obtain, same as above, that \[-[\varphi_{r,k}(a)(e_{n-1,n}),[\varphi_{r,k}(a)(e_{1j}),\varphi_{r,k}(a)(e_{l,n-1})]]=-[\varphi_{r,k}(a)(e_{n-1,n}),\varphi_{r,k}(a)(e_{1,n-1})].\]
To continue, we need to prove the following auxiliary proposition:
\begin{proposition2}
\label{phi.r.k.general.element}
\[\varphi_{r,k}(a)(e_{k+1-h,k+1})=e_{k+1-h,k+1}+ae_{k+1-h,k+r} : h>0\]
\[\varphi_{r,k}(a)(e_{k+r,k+r+h})=e_{k+r,k+r+h}-ae_{k+1,k+r+h} : h>0\]
\[\varphi_{r,k}(a)(e_{ij})=e_{ij} : i\neq{k+r}\land{j\neq{k+1}}\] 
\end{proposition2}
\begin{proof}
For $h=1$, $\varphi_{r,k}(a)(e_{k,k+1})=e_{k,k+1}+ae_{k,k+r}$. For $h'=h>1$, $\varphi_{r,k}(a)(e_{k+1-h',k+1})=[\varphi_{r,k}(a)(e_{k-h,k-h+1}),\varphi_{r,k}(a)(e_{k-h+1,k+1})]$. By the assumption, we have that $[\varphi_{r,k}(a)(e_{k-h,k-h+1}),\varphi_{r,k}(a)(e_{k-h+1,k+1})]=$\[=[\varphi_{r,k}(a)(e_{k-h,k-h+1}),e_{k+1-h,k+1}+ae_{k+1-h,k+r}].\]
But for all $h>0$, we have that $k-h\neq{k}$ and $k-h+1\neq{k+1}$, and hence \[[\varphi_{r,k}(a)(e_{k-h,k-h+1}),e_{k+1-h,k+1}+ae_{k+1-h,k+r}]=\]\[=[e_{k-h,k-h+1},e_{k+1-h,k+1}+ae_{k+1-h,k+r}=e_{k-h,k+1}+ae_{k-h,k+r}=\]\[=e_{k+1-h',k+1}+ae_{k+1-h',k+r}.\]
We prove the two other cases in the same way.
\end{proof}
By \ref{phi.r.k.general.element} we have that 
\[\varphi_{r,k}(a)(e_{1,n-1})=\begin{cases}
        e_{1,n-1}+ae_{1,n} & : r=2\land{k=n-2}\\
        e_{1,n-1} & : \mathrm{otherwise}
    \end{cases}
\]
while $k+r\neq{1}$ for all $r,k$. Thus, $-[\varphi_{2,n-2}(a)(e_{n-1,n}),\varphi_{2,n-2}(a)(e_{1,n-1})]=-[\varphi_{2,n-2}(a)(e_{n-1,n}),e_{1,n-1}+ae_{1,n}]=[e_{n-1,n},e_{1,n-1}+ae_{1,n}]=e_{1n}$. 

For $(r,k)\notin(2,n-2)$, if $k+r=n-1$ then $-[\varphi_{r,k}(a)(e_{n-1,n}),\varphi_{r,k}(a)(e_{1,n-1})]=-[e_{n-1,n}-ae_{n-r,n},e_{1,n-1}]=e_{1n}$, otherwise $-[\varphi_{r,k}(a)(e_{n-1,n}),\varphi_{r,k}(a)(e_{1,n-1})]=-[e_{n-1,n},e_{1,n-1}]=e_{1n}$
\end{proof}
Fix the two parameters $2\leq{r}\leq{n-2}$ and $0\leq{k}\leq{n-r}$, and denote by $N_{r,k}:=\{\varphi_{r,k}(a) : a\in\mathbb{Q}_{p}\}\subset{N(\mathbb{Q}_{p})}$ the set of all automorphisms of this form. Also denote $N'_{n-2}:=\{\varphi'_{n-2}(a) : a\in\mathbb{Q}_{p}\}$ and $N''_{n-2}:=\{\varphi''_{n-2}(a) : a\in\mathbb{Q}_{p}\}.$
\begin{proposition2}
Let $N_{r,k}$, $N'_{n-2}$ and $N''_{n-2}$ be the subsets defined above, then
\begin{enumerate}
    \item $N_{r,k},N'_{n-2},N''_{n-2}\leq{N(\mathbb{Q}_{p})}.$
    \item $N_{r,k},N'_{n-2},N''_{n-2}\cong{\mathbb{Q}_{p}}.$
\end{enumerate}
\end{proposition2}
\begin{proof}
A simple check shows that these subsets are subgroups of $N(\mathbb{Q}_p)$. Define $\tau_{r,k}:\mathbb{Q}_{p}\rightarrow{N_{r,k}}$. For every $a,b\in\mathbb{Q}_{p}$, it is easy to see that the image of the sum, $\tau_{r,k}(a+b)=\tau_{r,k}(a)\cdot\tau_{r,k}(b)$, is the product of the images of $a$ and $b$, and that $\tau_{r,k}^{-1}(I)=\{0\}$.
\end{proof}
The following proposition follows from a simple computation.
\begin{proposition2}
\label{prop:psi.automorphism}
Consider $\varphi_{r}\in{N_{r}}$.
\begin{enumerate}
    \item 
If $r<n-2$, denote by $\psi_{r}$ the automorphism \[\psi_{r}:=\varphi_{r}\circ\varphi_{r,n-r}(-a_{n-r,n-r})\circ\cdots\circ\varphi_{r,1}(-a_{11})\circ\varphi_{r,0}(-a_{r+1,1}).\]
Then $\psi_{r}\in{N_{r+1}}$.
    \item 
If $r=n-2$, denote by $\psi_{n-2}$ the automorphism \[\psi_{n-2}:=\varphi_{r}\circ\varphi'_{n-2}(-a_{12})\circ\varphi''_{n-2}(-a_{n-1,1})\circ\]\[\circ\varphi_{n-2,2}(-a_{n-2,2})\circ\varphi_{n-2,1}(-a_{n-2,1})\circ\varphi_{n-2,0}(-a_{n-2,0}).\]
Then $\psi_{n-2}\in{N_{n-1}}$.
\end{enumerate}
\end{proposition2}
\begin{proof}
By the definition, $\varphi_{r}$ has $1$ on the main diagonal and all the upper blocks $M_{1k}$, for $2\leq{k}\leq{r-1}$ are zero matrices. One checks that the composition of $\varphi_{r}$ and the chain of compositions $\prod_{k=1}^{n-r}\varphi_{r,k}(-a_{kk})\circ\varphi_{r,0}(-a_{r+1,1})$ yields also a matrix with $1$ on the main diagonal whose upper blocks $M_{1k}$, for all $2\leq{k}\leq{r}$, are zero matrices, thus $\psi_{r}\in{N_{r+1}}$. Same applies for the case $r=n-2$, considering the specific structure of $M_{1,n-2}$, as described in the second part of \ref{block.1r.structure}.
\end{proof}
\begin{corollary2}
\label{cor:Nr.decomposition}
We have the following decompositions:
\begin{enumerate}
    \item 
For all $2\leq{r}<{n-2}$, we have \[N_{r}=N_{r+1}\rtimes(N_{r,0}\rtimes(\cdots(N_{r,n-r-1}\rtimes{N_{r,n-r}})\cdots)).\]
\item For $r=n-2$, we have 
\[N_{n-2}=N_{n-1}\rtimes({N_{n-2,0}}\rtimes(N_{n-2,1}\rtimes(N_{n-2,2}\rtimes(N''_{n-2}\rtimes{N'_{n-2}})))).\]
\end{enumerate}
\end{corollary2}
\begin{proof}
This is immediate from Proposition \ref{prop:psi.automorphism}, since we have that \[\varphi_{r}=\psi_{r}\circ\varphi_{r,0}(-a_{r+1,1})\circ\varphi_{r,1}(-a_{11})\circ\cdots\circ\varphi_{r,n-r}(-a_{n-r,n-r}).\]
\end{proof}
Corollary \ref{cor:Nr.decomposition} provides a recursive decomposition of the unipotent radical $N(\mathbb{Q}_{p})$ as an iterated semidirect product of $N_{n-1}$ and subgroups isomorphic to $\mathbb{Q}_{p}$.

As we saw earlier, the calculation of $\zeta_{L_{n,p}}^{\wedge}(s)$ requires understanding $G_{n}(\mathbb{Z}_p)$ and $G_{n}^{+}(\mathbb{Q}_p)$ first. As $G_{n}(\mathbb{Z}_{p})$ is a group, its structure is easily deduced from the above.
\begin{proposition2}
\label{prop:G.n.Zp.decomposition}
For all $n\geq{4}$ the group $G_{n}^{0}(\mathbb{Z}_{p})$ has the decomposition $G_{n}^{0}(\mathbb{Z}_{p})=N(\mathbb{Z}_{p})\rtimes{H(\mathbb{Z}_{p})}$, where \[N(\mathbb{Z}_{p}):=M_{\binom{n}{2}}(\mathbb{Z}_{p})\cap{N(\mathbb{Q}_{p})},\]
\[H(\mathbb{Z}_{p}):=\{\mathrm{diag}(\lambda_{1},\lambda_{2},\dots,) : \lambda_1,\dots,\lambda_{n-1}\in\mathbb{Z}_{p}^{\ast}\}.\]
Moreover, $N(\mathbb{Z}_{p})$ itself has the decomposition: \[N(\mathbb{Z}_{p})=\Tilde{N}_{2}(\mathbb{Z}_{p})=
\Tilde{N}_{n-1}\rtimes({\Tilde{N}_{n-2,0}}\rtimes(\Tilde{N}_{n-2,1}\rtimes(\Tilde{N}_{n-2,2}\rtimes(\Tilde{N}''_{n-2}\rtimes{\Tilde{N}'_{n-2}}))))\rtimes\cdots\]\[\cdots\rtimes{(\Tilde{N}_{2,0}\rtimes(\cdots(\Tilde{N}_{2,n-3}\rtimes{\Tilde{N}_{2,n-2}})\cdots))}
,\]
where $\Tilde{N}_{r}=N_{r}\cap{N(\mathbb{Z}_{p})}$ and $\Tilde{N}_{r,k}=\{\varphi_{r,k}(a) : a\in\mathbb{Z}_{p}\}$.
\end{proposition2}
By contrast, describing the structure of the monoid $G_{n}^{+}(\mathbb{Q}_p)$ is expected to be a substantial challenge.

By applying Fubini's theorem for semidirect products of topological groups \cite[Proposition 28]{Nachbin}, we have that \[\zeta_{L_{n,p}}^{\wedge}(s)=\displaystyle\int_{G_{n}^{+}(\mathbb{Q}_p)}|\det{\varphi}|_p^sd\mu_{G_{n}(\mathbb{Z}_p)\varphi}=\displaystyle\int_{H^+(\mathbb{Q}_p)}\left(\displaystyle\int_{N_{h}^+}|\det{uh}|_p^sd\mu_{N(\mathbb{Q}_p)}\right)d\mu_{H(\mathbb{Q}_p)},\]
where \[H^+(\mathbb{Q}_p):=\{\mathrm{diag}(\lambda_{1},\dots,\lambda_{n-1},\lambda_{1}\lambda_{2},\dots,\lambda_{1}\lambda_{2}\cdots\lambda_{n-2}\lambda_{n-1}) : \lambda_{i}\in\mathbb{Z}_{p}\setminus\{0\}\},\]
that is, $H^+(\mathbb{Q}_p)$ consists of all $h\in{H(\mathbb{Q}_{p})}$ that appear in the decomposition $\varphi=uh$ for some $\varphi\in{G_{n}^{+}(\mathbb{Q}_{p})}$, and, for a given $h\in{H^{+}(\mathbb{Q}_{p})}$, we set $N_{h}^{+}:=\{u\in{N(\mathbb{Q}_{p}) : uh\in{G_{n}^{+}(\mathbb{Q}_{p})}}\}$. The integrand is constant on $N_{h}^{+}$, so computing the inner integral amounts to finding the measure of $N_{h}^+$, which is complicated, but using the decomposition from Corollary \ref{cor:Nr.decomposition}, we can simplify $N_{h}^{+}$ at the price of replacing a single integral by multiple integrals.

Let $\mathcal{N}_{1},\mathcal{N}_{2},\dots,\mathcal{N}_{m}$, where $m=\binom{n}{2}$, be an enumeration of the subgroups
\[N_{2,n-2},N_{2,n-3},\dots,N_{2,0},N_{3,n-3},\dots,N_{3,0},\dots\]\[\dots,N_{n-2,2},N_{n-2,1},N_{n-2,0},N'_{n-2},N''_{n-2},N_{n-1}.\]
Every $\varphi\in{G_{n}^{0}(\mathbb{Q}_{p})}$ can be written uniquely as $\varphi=u_{m}\cdots{u_{1}h}$, where $u_{i}\in{\mathcal{N}_{i}}$. Thus, by Fubini \[
\zeta_{L_{n,p}}^{\wedge}(s)=\displaystyle\int_{H^{+}}\displaystyle\int_{\mathcal{N}_{1}^{+}(h)}\displaystyle\int_{\mathcal{N}_{2}^{+}(h,u_{1})}\cdots\displaystyle\int_{\mathcal{N}_{m}^{+}(h,u_{1},\dots,u_{m-1})}|\det{h}|_{p}^{s}d\mu_{H}d\mu_{\mathcal{N}_{1}}\cdots{d\mu_{\mathcal{N}_{m}}},\]
where each $\mu_{\mathcal{N}_{i}}$ is Haar measure on $\mathcal{N}_{i}(\mathbb{Q}_{p})$ normalized so that\\ $\mu_{\mathcal{N}_{i}}(\mathcal{N}_{i}(\mathbb{Z}_{p}))=1$, and \[\mathcal{N}_{i}^{+}(h,u_{1},\dots,u_{i-1}):=\{u_{i}\in\mathcal{N}_{i} : \exists{u_{i+1},\dots,u_{m}}\,\mathrm{such}\,\mathrm{that}\,u_{m}\cdots{u_{1}}h\in{G_{n}^{+}(\mathbb{Q}_{p})}\}.\]
\end{document}
