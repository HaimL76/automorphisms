\documentclass{article}

% Language setting
% Replace `english' with e.g. `spanish' to change the document language
\usepackage[english]{babel}


% Useful packages
\usepackage{amsmath}
\usepackage{amssymb}
\usepackage{graphicx}

\title{Your Paper}
\author{You}

\begin{document}
\maketitle
Denote $G_{5}:=G_{5}(\mathbb{Z}_{p})$, and $G_{5}^{+}:=G_{5}^{+}(\mathbb{Q}_{p})$.

$\zeta_{L_{5,p}}^{\wedge}(s)=\displaystyle\int_{{G_{5}^{+}}}{|\det{g}|_{p}^{s}}d\mu(G_{5})=\displaystyle\int_{{G_{5}^{+}}}{|\det{uh}|_{p}^{s}}d\mu(G_{5})$, where $h\in{H}$ and $u\in{N_{h}}$. 

Each $u$ is unipotent, hence $\zeta_{L_{5,p}}^{\wedge}(s)=\displaystyle\int_{{G_{5}^{+}}}{|\det{h}|_{p}^{s}}d\mu(G_{5})=\displaystyle\int_{{G_{5}^{+}}}{|\lambda_{1}^{4}\lambda_{2}^{6}\lambda_{3}^{6}\lambda_{4}^{4}|_{p}^{s}}d\mu(G_{5})=$

$=\displaystyle\int_{{G_{5}^{+}}}{\Bigl[|\lambda_{1}^{4}|_{p}|\lambda_{2}^{6}|_{p}|\lambda_{3}^{6}|_{p}|\lambda_{4}^{4}|_{p}\Bigr]^{s}}d\mu(G_{5})$, by the inductive formula we have found for every $|h|$. 

We denote $v_{i}:=v_{p}(\lambda_{i})$, 

and so $\zeta_{L_{5,p}}^{\wedge}(s)=\displaystyle\int_{{G_{5}^{+}}}{\Bigl[p^{-4v_{1}}p^{-6v_{2}}p^{-6v_{3}}p^{-4v_{4}}\Bigr]^{s}}d\mu(G_{5})=\displaystyle\int_{{G_{5}^{+}}}{p^{-(4v_{1}+6v_{2}+6v_{3}+4v_{4})s}}d\mu(G_{5})$. 

We denote $I(\underline{\lambda}):=p^{-(4v_{1}+6v_{2}+6v_{3}+4v_{4})s}$.
Now we use the natural matrix decomposition of the $N_{h}$ matrix of Berman's, which means that

$\zeta_{L_{5,p}}^{\wedge}(s)=\displaystyle\int_{{G_{5}^{+}}}{I(\underline{\lambda})}d\mu(G_{5})=\displaystyle\int_{\underline\lambda}\int_{\underline{a}}\int_{\underline{b}}\int_{\underline{c}}{I(\underline{\lambda})}d\mu(\underline{c})d\mu(\underline{b})d\mu(\underline{a})d\mu(\underline{\lambda})$. Since $I(\underline{\lambda})$ depends only on $\lambda_{1},\lambda_{2},\lambda_{3},\lambda_{4}$, which appear only in the computation of the outermost integral, we consider them as constants for all the inner integrals, which means that  we have $\zeta_{L_{5,p}}^{\wedge}(s)=\displaystyle\int_{\underline\lambda}I(\underline{\lambda})\int_{\underline{a}}\int_{\underline{b}}\int_{\underline{c}}1d\mu(\underline{c})d\mu(\underline{b})d\mu(\underline{a})d\mu(\underline{\lambda})$, hence all the inner integrals evaluate to the measure of their domains of integration.
now we compute the innermost integral by considering $\underline{a}$, $\underline{b}$ and $\underline{\lambda}$ as constants, and integrating only over $\underline{c}$. Considering the multiplication $uh$, we observe that for each element $c_{j}$, we must have that $\rho_{j}=c_{j}\lambda_{1}\lambda_{2}\lambda_{3}\lambda_{4}\in\mathbb{Z}_{p}$, which means that $v(\rho_{i})=v(c_{i}\lambda_{1}\lambda_{2}\lambda_{3}\lambda_{4})\geq{0}\Rightarrow{v(c_{i})+v_{1}+v_{2}+v_{3}+v_{4}}\geq{0}\Rightarrow{v(c_{i})\geq{-(v_{1}+v_{2}+v_{3}+v_{4})}}$. But this means that $c_{i}\in{p^{-(v_{1}+v_{2}+v_{3}+v_{4})}\mathbb{Z}_{p}}$, and since the domain of integration for this integral is $\underline{c}=\{c_{1},c_{2},c_{3},c_{4}\}$, then $\mu(\underline{c})=|c_{j}|_{p}^{4}=p^{4(v_{1}+v_{2}+v_{3}+v_{4})}$. Denote $I(\underline{\lambda},\underline{c}):=I(\underline{\lambda})p^{4(v_{1}+v_{2}+v_{3}+v_{4})}$, we now have that $\zeta_{L_{5,p}}^{\wedge}(s)=\displaystyle\int_{\underline\lambda}I(\underline{\lambda},\underline{c})\int_{\underline{a}}\int_{\underline{b}}1d\mu(\underline{b})d\mu(\underline{a})d\mu(\underline{\lambda})$.

Denote $\lambda_{13}:=\lambda_{1}\lambda_{2}\lambda_{3}$, $\lambda_{24}:=\lambda_{2}\lambda_{3}\lambda_{4}$, and $\lambda_{14}:=\lambda_{1}\lambda_{2}\lambda_{3}\lambda_{4}$. We now consider the constraints on $\underline{b}$. 

$b_{11}\lambda_{13},b_{31}\lambda_{13},b_{41}\lambda_{13}\in\mathbb{Z}_{p}$, and $b_{12}\lambda_{24},b_{22}\lambda_{24}\in\mathbb{Z}_{p}$. These constaints are obtained by multiplying elements in block $M_{13}$ with elements in $h$, but one observes that we have $b_{22}$ also in location $(5,10)$ of the matrix, and $b_{31}$ in location $(7,10)$, which means that $b_{22}\lambda_{14},b_{31}\lambda_{14}\in\mathbb{Z}_{p}$. But since we already have $b_{22}\lambda_{24},b_{31}\lambda_{13}\in\mathbb{Z}_{p}$, the constraints $b_{22}\lambda_{14}$ and $b_{31}\lambda_{14}$ do not contribute any new information. In addition, we have one of the elements of $\underline{b}$ that forms a constraint together with elements from $\underline{a}$, namely $(a_{11}a_{22}-b_{11})\lambda_{24}\in\mathbb{Z}_{p}$. The constraints $b_{31}\lambda_{13},b_{41}\lambda_{13},b_{12}\lambda_{24},b_{22}\lambda_{24}\in\mathbb{Z}_{p}$ from above translate to $p^{-2(v_{1}+v_{2}+v_{3})}p^{-2(v_{2}+v_{3}+v_{4})}=p^{-2(v_{1}+2v_{2}+2v_{3}+v_{4})}$. On the other hand, $b_{11}$ is a part of two constraints, hence we must have both $b_{11}\in{p^{-(v_{1}+v_{2}+v_{3})}\mathbb{Z}_{p}}$ and $a_{11}a_{22}-b_{11}\in{p^{-(v_{2}+v_{3}+v_{4})}\mathbb{Z}_{p}}\Rightarrow{b_{11}\in{a_{11}a_{22}+p^{-(v_{2}+v_{3}+v_{4})}\mathbb{Z}_{p}}}$, which means that we need to compute the measure of the intersection of the two modules $\mu\Bigl(p^{-(v_{1}+v_{2}+v_{3})}\mathbb{Z}_{p}\cap{a_{11}a_{22}+p^{-(v_{2}+v_{3}+v_{4})}\mathbb{Z}_{p}\Bigr)}$. Denote $\alpha:=v_{1}+v_{2}+v_{3}$, $\beta:=v_{2}+v_{3}+v_{4}$ and $x:=a_{11}a_{22}$, so $b_{11}\in{p^{-\alpha}\mathbb{Z}_{p}\cap{x+p^{-\beta}}\mathbb{Z}_{p}}$. Assume $\beta\geq\alpha\Rightarrow{-\beta\leq-\alpha\Rightarrow{p^{-\alpha}\mathbb{Z}_{p}\subset{p^{-\beta}\mathbb{Z}_{p}}}}\Rightarrow{b_{11}\in{p^{-\beta}\mathbb{Z}_{p}\cap{x+p^{-\beta}}\mathbb{Z}_{p}}}$. 
\end{document}