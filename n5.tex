\documentclass{article}

% Language setting
% Replace `english' with e.g. `spanish' to change the document language
\usepackage[english]{babel}


% Useful packages
\usepackage{amsmath}
\usepackage{amssymb}
\usepackage{graphicx}

\title{Your Paper}
\author{You}

\begin{document}
\maketitle
Denote $G_{5}:=G_{5}(\mathbb{Z}_{p})$, and $G_{5}^{+}:=G_{5}^{+}(\mathbb{Q}_{p})$.

$\zeta_{L_{5,p}}^{\wedge}(s)=\displaystyle\int_{{G_{5}^{+}}}{|\det{g}|_{p}^{s}}d\mu(G_{5})=\displaystyle\int_{{G_{5}^{+}}}{|\det{uh}|_{p}^{s}}d\mu(G_{5})$, where $h\in{H}$ and $u\in{N_{h}}$. 

Each $u$ is unipotent, hence $\zeta_{L_{5,p}}^{\wedge}(s)=\displaystyle\int_{{G_{5}^{+}}}{|\det{h}|_{p}^{s}}d\mu(G_{5})=\displaystyle\int_{{G_{5}^{+}}}{|\lambda_{1}^{4}\lambda_{2}^{6}\lambda_{3}^{6}\lambda_{4}^{4}|_{p}^{s}}d\mu(G_{5})=$

$=\displaystyle\int_{{G_{5}^{+}}}{\Bigl[|\lambda_{1}^{4}|_{p}|\lambda_{2}^{6}|_{p}|\lambda_{3}^{6}|_{p}|\lambda_{4}^{4}|_{p}\Bigr]^{s}}d\mu(G_{5})$, by the inductive formula we have found for every $|h|$. 

We denote $v_{i}:=v_{p}(\lambda_{i})$, 

and so $\zeta_{L_{5,p}}^{\wedge}(s)=\displaystyle\int_{{G_{5}^{+}}}{\Bigl[p^{-4v_{1}}p^{-6v_{2}}p^{-6v_{3}}p^{-4v_{4}}\Bigr]^{s}}d\mu(G_{5})=\displaystyle\int_{{G_{5}^{+}}}{p^{-(4v_{1}+6v_{2}+6v_{3}+4v_{4})s}}d\mu(G_{5})$. 

We denote $I_{1}:=p^{-(4v_{1}+6v_{2}+6v_{3}+4v_{4})s}$.
Now we use the natural matrix decomposition of the $N_{h}$ matrix of Berman's, which means that

$\zeta_{L_{5,p}}^{\wedge}(s)=\displaystyle\int_{{G_{5}^{+}}}{I_{1}}d\mu(G_{5})=\displaystyle\int_{\underline\lambda}\int_{\underline{a}}\int_{\underline{b}}\int_{\underline{c}}{I_{1}}d\mu(\underline{c})d\mu(\underline{b})d\mu(\underline{a})d\mu(\underline{\lambda})$. Since $I_{1}$ depends only on $\lambda_{1},\lambda_{2},\lambda_{3},\lambda_{4}$, which appear only in the computation of the outermost integral, we consider them as constants for all the inner integrals, which means that  we have $\zeta_{L_{5,p}}^{\wedge}(s)=\displaystyle\int_{\underline\lambda}I_{1}\int_{\underline{a}}\int_{\underline{b}}\int_{\underline{c}}1d\mu(\underline{c})d\mu(\underline{b})d\mu(\underline{a})d\mu(\underline{\lambda})$, hence all the inner integrals evaluate to the measure of their domains of integration.
now we compute the innermost integral by considering $\underline{a}$, $\underline{b}$ and $\underline{\lambda}$ as constants, and integrating only over $\underline{c}$. Considering the multiplication $uh$, we observe that for each element $c_{j}$, we must have that $\rho_{j}=c_{j}\lambda_{1}\lambda_{2}\lambda_{3}\lambda_{4}\in\mathbb{Z}_{p}$, which means that $v(\rho_{i})=v(c_{i}\lambda_{1}\lambda_{2}\lambda_{3}\lambda_{4})\geq{0}\Rightarrow{v(c_{i})+v_{1}+v_{2}+v_{3}+v_{4}}\geq{0}\Rightarrow{v(c_{i})\geq{-(v_{1}+v_{2}+v_{3}+v_{4})}}$. But this means that $c_{i}\in{p^{-(v_{1}+v_{2}+v_{3}+v_{4})}\mathbb{Z}_{p}}$, and since the domain of integration for this integral is $\underline{c}=\{c_{1},c_{2},c_{3},c_{4}\}$, then $\mu(\underline{c})=|c_{j}|_{p}^{4}=p^{4(v_{1}+v_{2}+v_{3}+v_{4})}$. Denote $I_{2}:=I_{1}p^{4(v_{1}+v_{2}+v_{3}+v_{4})}$, we now have that $\zeta_{L_{5,p}}^{\wedge}(s)=\displaystyle\int_{\underline\lambda}I_{2}\int_{\underline{a}}\int_{\underline{b}}1d\mu(\underline{b})d\mu(\underline{a})d\mu(\underline{\lambda})$.

Denote $\lambda_{13}:=\lambda_{1}\lambda_{2}\lambda_{3}$, $\lambda_{24}:=\lambda_{2}\lambda_{3}\lambda_{4}$, and $\lambda_{14}:=\lambda_{1}\lambda_{2}\lambda_{3}\lambda_{4}$. We now consider the constraints on $\underline{b}$. 

$b_{11}\lambda_{13},b_{31}\lambda_{13},b_{41}\lambda_{13}\in\mathbb{Z}_{p}$, and $b_{12}\lambda_{24},b_{22}\lambda_{24}\in\mathbb{Z}_{p}$. These constaints are obtained by multiplying elements in block $M_{13}$ with elements in $h$, but one observes that we have $b_{22}$ also in location $(5,10)$ of the matrix, and $b_{31}$ in location $(7,10)$, which means that $b_{22}\lambda_{14},b_{31}\lambda_{14}\in\mathbb{Z}_{p}$. But since we already have $b_{22}\lambda_{24},b_{31}\lambda_{13}\in\mathbb{Z}_{p}$, the constraints $b_{22}\lambda_{14}$ and $b_{31}\lambda_{14}$ do not contribute any new information. In addition, we have one of the elements of $\underline{b}$ that forms a constraint together with elements from $\underline{a}$, namely $(a_{11}a_{22}-b_{11})\lambda_{24}\in\mathbb{Z}_{p}$. The constraints $b_{31}\lambda_{13},b_{41}\lambda_{13},b_{12}\lambda_{24},b_{22}\lambda_{24}\in\mathbb{Z}_{p}$ from above translate to $p^{-2(v_{1}+v_{2}+v_{3})}p^{-2(v_{2}+v_{3}+v_{4})}=p^{-2(v_{1}+2v_{2}+2v_{3}+v_{4})}$. On the other hand, $b_{11}$ is a part of two constraints, hence we must have both $b_{11}\in{p^{-(v_{1}+v_{2}+v_{3})}\mathbb{Z}_{p}}$ and $a_{11}a_{22}-b_{11}\in{p^{-(v_{2}+v_{3}+v_{4})}\mathbb{Z}_{p}}\Rightarrow{b_{11}\in{a_{11}a_{22}+p^{-(v_{2}+v_{3}+v_{4})}\mathbb{Z}_{p}}}$, which means that we need to compute the measure $\mu(A)$, where $A=p^{-(v_{1}+v_{2}+v_{3})}\mathbb{Z}_{p}\cap{a_{11}a_{22}+p^{-(v_{2}+v_{3}+v_{4})}\mathbb{Z}_{p}}$. Denote $\alpha:=v_{1}+v_{2}+v_{3}$, $\beta:=v_{2}+v_{3}+v_{4}$ and $x:=a_{11}a_{22}$, and we need to find a formula for a generic intersection of the form $A={p^{-\alpha}\mathbb{Z}_{p}\cap{x+p^{-\beta}}\mathbb{Z}_{p}}$. We need to find a formula for this generic form. Since $b_{11}$ is in the intersection, we have that $b_{11}=z=x+y$ where $y\in{p^{-\beta}}$ and $z\in{p^{-\alpha}\mathbb{Z}_{p}}\Rightarrow{z-x\in{p^{-\beta}\mathbb{Z}_{p}}}$. Assume $\beta\geq\alpha\Rightarrow{-\beta\leq-\alpha}$, and since $v_{p}(b_{11})=v_{p}(z-x)\geq{\min\{v_{p}(z),v_{p}(x)\}}$, and $v_{p}(z)\geq{-\alpha}\geq{-\beta}$, then we have two cases. If $v_{p}(x)\geq{-\beta}$, then $v_{p}(z-x)\geq{\beta}\Rightarrow{z-x\in{p^{-\beta}\mathbb{Z}_{p}}}$. But $-\alpha\geq{-\beta}\Rightarrow{p^{-\alpha}\mathbb{Z}_{p}\subseteq{p^{-\beta}\mathbb{Z}_{p}}}\Rightarrow{A=p^{-\alpha}\mathbb{Z}_{p}}$. If $v_{p}(x)<{-\beta}$, then $v_{p}(z-x)=v_{p}(x)<-\beta\Rightarrow{z-x}\notin{p^{-\beta}\mathbb{Z}_{p}}$, which means that $A=\varnothing$. One checks that if we assume $\alpha\geq\beta$, then we obtain that $A=p^{-\beta}\mathbb{Z}_{p}$ if $v_{p}(x)\geq{-\alpha}$, and $A=\varnothing$ if $v_{p}(x)<-\alpha$. Therefore, $\mu(A)=p^{\min\{\alpha,\beta\}}$ for every $x$ such that $v_{p}(x)\geq\min\{-\alpha,-\beta\}=-\max\{\alpha,\beta\}$, which means, in our case, that $v_{p}(x)=v_{p}(a_{11}a_{22})\geq-\max\{v_{1}+v_{2}+v_{3},v_{2}+v_{3}+v_{4}\}=-v_{2}-v_{3}-\max\{v_{1},v_{4}\}$. Thus, denoting $I_{3}:=I_{2}p^{-(v_{2}+v_{3})-\max\{v_{1},v_{4}\}}$, we have that $\zeta_{L_{5,p}}^{\wedge}(s)=\displaystyle\int_{\underline\lambda}I_{3}\int_{\underline{a}}1d\mu(\underline{b})d\mu(\underline{a})d\mu(\underline{\lambda})$. Denote $v_{ij}:=v_{p}(a_{ij})$. For the constraints on $\underline{a}$, we have

$a_{11}\lambda_{1}\lambda_{2},-a_{11}\lambda_{2}\lambda_{3},-a_{11}\lambda_{2}\lambda_{3}\lambda_{4}\in\mathbb{Z}_{p}\Rightarrow{v_{11}\geq{-(v_{1}+v_{2})},v_{11}\geq{-(v_{2}+v_{3})}}\Rightarrow{v_{11}\geq{-v_{2}-\min\{v_{1},v_{3}\}}}$. 

$a_{21}\lambda_{1}\lambda_{2},a_{21}\lambda_{1}\lambda_{2}\lambda_{3},a_{21}\lambda_{1}\lambda_{2}\lambda_{3}\lambda_{4}\in\mathbb{Z}_{p}\Rightarrow{v_{21}\geq{-(v_{1}+v_{2})}}$.

$a_{22}\lambda_{2}\lambda_{3},-a_{22}\lambda_{3}\lambda_{4},a_{22}\lambda_{1}\lambda_{2}\lambda_{3}\in\mathbb{Z}_{p}\Rightarrow$

$\Rightarrow{v_{22}\geq{-(v_{2}+v_{3})},v_{22}\geq{-(v_{3}+v_{4})},v_{22}\geq{-(v_{1}+v_{2}+v_{3}}})\Rightarrow{v_{22}\geq{-v_{3}-\min\{v_{2},v_{4}\}}}$.

$a_{33}\lambda_{3}\lambda_{4},a_{33}\lambda_{2}\lambda_{3}\lambda_{4},a_{33}\lambda_{1}\lambda_{2}\lambda_{3}\lambda_{4}\in\mathbb{Z}_{p}\Rightarrow{v_{33}\geq{-(v_{3}+v_{4})}}$.

$a_{21}a_{22}\lambda_{1}\lambda_{2}\lambda_{3}\in\mathbb{Z}_{p}\Rightarrow{v_{21}+v_{22}\geq{-(v_{1}+v_{2}+v_{3})}}$.

$-a_{11}a_{33}\lambda_{2}\lambda_{3}\lambda_{4}\in\mathbb{Z}_{p}\Rightarrow{v_{11}+v_{33}\geq{-(v_{2}+v_{3}+v_{4})}}$.

$a_{21}a_{33}\lambda_{1}\lambda_{2}\lambda_{3}\lambda_{4}\in\mathbb{Z}_{p}\Rightarrow{v_{21}+v_{33}\geq{-(v_{1}+v_{2}+v_{3}+v_{4})}}$.

And we also have the constraint found earlier, 

$v_{11}+v_{22}\geq{-(v_{2}+v_{3}+\max\{v_{1},v_{4}\})}$.

We have three constraints on $a_{21}$
\begin{enumerate}
    \item $v_{21}\geq{-(v_{1}+v_{2})}$
    \item $v_{21}\geq{-(v_{1}+v_{2}+v_{3}+v_{22})}$
    \item $v_{21}\geq{-(v_{1}+v_{2}+v_{3}+v_{4}+v_{33})}$
\end{enumerate}
But the third constraint does not add new information, because we already have the two separate constraints $v_{21},v_{33}\geq{-(v_{1}+v_{2}+v_{3}+v_{4})}$.

The two valid constraints translate to 

$v_{21}\geq{\min\{-(v_{1}+v_{2}),-(v_{1}+v_{2}+v_{3}+v_{22})\}=-(v_{1}+v_{2})-\{0,v_{3}+v_{22}\}}$.

In the same way, we obtain the constraint $v_{33}\geq{-(v_{3}+v_{4})-\min\{0,v_{2}+v_{11}\}}$.

Thus, we decompose the inner integral for $\underline{a}$ into separate integrals, to obtain $\zeta_{L_{5,p}}^{\wedge}(s)=\displaystyle\int_{\underline\lambda}I_{3}\int_{\underline{a}}1d\mu(\underline{b})d\mu(\underline{a})d\mu(\underline{\lambda})=$

$=\displaystyle\int_{\underline\lambda}I_{3}\int_{a_{11}}\int_{a_{22}}\int_{a_{33}}\int_{a_{21}}1d\mu(\underline{b})d\mu(\underline{a})d\mu(\underline{\lambda})$.

Hence, we have the measures $\mu(a_{21})=p^{v_{1}+v_{2}+\min\{0,v_{3}+v_{22}\}}$ and $\mu(a_{33})=p^{v_{3}+v_{4}+\min\{0,v_{2}+v_{11}\}}$.
Denote $I_{4}:=I_{3}p^{v_{1}+v_{2}}p^{v_{3}+v_{4}}$. We have 

$\zeta_{L_{5,p}}^{\wedge}(s)=\displaystyle\int_{\underline\lambda}I_{3}\int_{\underline{a}}1d\mu(\underline{b})d\mu(\underline{a})d\mu(\underline{\lambda})=$

$=\displaystyle\int_{\underline\lambda}I_{3}\int_{a_{11}}\int_{a_{22}}\int_{a_{33}}\int_{a_{21}}1d\mu(\underline{b})d\mu(\underline{a})d\mu(\underline{\lambda})=$

$=\displaystyle\int_{\underline\lambda}I_{4}\int_{a_{11}}p^{\min\{0,v_{2}+v_{11}\}}\int_{a_{22}}p^{\min\{0,v_{3}+v_{22}\}}d\mu(\underline{b})d\mu(\underline{a})d\mu(\underline{\lambda}).$

By the constraints we found earlier on $a_{22}$, we have the following.
\begin{enumerate}
    \item $v_{22}\geq{-v_{3}-\min\{v_{2},v_{4}\}}$
    \item $v_{22}\geq{-(v_{2}+v_{3})-\max\{v_{1},v_{4}\}-v_{11}}$
\end{enumerate}
which translates into $v_{22}\geq{-v_{3}-\min\{\min\{v_{2},v_{4}\},v_{2}+\max\{v_{1},v_{4}\}+v_{11}\}}=$

$=-v_{3}-\min\{v_{2},v_{4},v_{2}+\max\{v_{1},v_{4}\}+v_{11}\}$.

Denote $\alpha:=v_{2}+\max\{v_{1},v_{4}\}+v_{11}$ and $\beta:=\min\{v_{2},v_{4},\alpha\}$. We already have the constraint $v_{11}\geq{-(v_{2}+\min\{v_{1},v_{3}\})}$, which means that, in either case, $v_{11}\geq{-(v_{1}+v_{2})}\geq{-\min\{v_{1},v_{4}\}-v_{2}}$

$\Rightarrow{\alpha=v_{2}+\max\{v_{1},v_{4}\}+v_{11}\geq{\max\{v_{1},v_{4}\}-\min\{v_{1},v_{4}\}}}\geq{0}$

$\Rightarrow{\beta=\min\{v_{2},v_{4},\alpha\}}\geq{0}\Rightarrow{v_{3}+\alpha>0}\Rightarrow{v_{22}\geq{-(v_{3}+\beta)}}$. 

For the inner integral $\displaystyle\int_{a_{22}}p^{\min\{0,v_{3}+v_{22}\}}d\mu(a_{22})$, we have two cases. If $v_{3}+v_{22}\geq{0}$, then $\min\{0,v_{3}+v_{22}\}=0\Rightarrow{\displaystyle\int_{a_{22}}p^{\min\{0,v_{3}+v_{22}\}}d\mu(a_{22})}=\displaystyle\int_{v_{22}\geq{-v_{3}}}1d\mu(a_{22})=p^{v_{3}}$. 

If $v_{3}+v_{22}<0$, then $\displaystyle\int_{a_{22}}p^{\min\{0,v_{3}+v_{22}\}}d\mu(a_{22})=\displaystyle\int_{v_{22}<-v_{3}}p^{v_{3}+v_{22}}d\mu(a_{22})$. But we saw earlier that $v_{22}\geq{-(v_{3}+\beta)}$, hence $-v_{3}-\beta\leq{v_{22}}\leq{-v_{3}-1}\Rightarrow{-\beta\leq{v_{3}+v_{22}}\leq{-1}}$, which means that we can compute the integral over $a_{22}$ as a sum of $\beta$ integrals, $\displaystyle\int_{-v_{3}-\beta\leq{v_{22}}\leq{-v_{3}-1}}p^{v_{3}+v_{22}}d\mu(a_{22})=\sum_{\tau=1}^{\beta}\displaystyle\int_{v_{22}=-v_{3}-\tau}p^{-\tau}d\mu(a_{22})$. To evaluate each integral in the sum, we need to calculate the measure of its domain, namely $\mu(\{v_{22}=-(v_{3}+\tau)\})=\mu(\{v_{22}\geq{-(v_{3}+\tau+1)}\}\setminus{\{v_{22}\geq{-(v_{3}+\tau)}\}})=\mu(p^{-(v_{3}+\tau+1)}\mathbb{Z}_{p}\setminus{p^{-(v_{3}+\tau)}\mathbb{Z}_{p}})=p^{v_{3}+\tau+1}-p^{v_{3}+\tau}=p^{v_{3}+\tau}(p-1)$, which means that each integral evaluates as $p^{v_{3}+\tau}(p-1)p^{\tau}=p^{v_{3}}(p-1)$, and the sum is over $\beta$ such integrals, so we have that $\displaystyle\int_{a_{22}}p^{\min\{0,v_{3}+v_{22}\}}d\mu(a_{22})=p^{v_{3}}+\beta{p^{v_{3}}(p-1)}$, where $\beta$ depends also on $v_{11}$.

Hence, we need to compute the integral 

$\displaystyle\int_{a_{11}}p^{\min\{0,v_{2}+v_{11}\}}p^{v_{3}}[1+\beta(p-1)]d\mu(a_{11})$. We denote $I_{5}:=I_{4}p^{v_{3}}$, so $\zeta_{L_{5,p}}^{\wedge}(s)=$

$=\displaystyle\int_{\underline\lambda}I_{5}\displaystyle\int_{a_{11}}p^{\min\{0,v_{2}+v_{11}\}}[1+\beta(p-1)]d\mu(a_{11})$. But similar to what we saw earlier, $p^{\min\{0,v_{2}+v_{11}\}}$ has two cases. If $v_{11}\geq{-v_{2}}$, then $p^{\min\{0,v_{2}+v_{11}\}}=p^{0}$. If $v_{11}<-v_{2}$, then $p^{\min\{0,v_{2}+v_{11}\}}=p^{v_{2}+v_{11}}$.
We saw earlier that $v_{11}\geq{-(v_{2}+v_{3})}\Rightarrow{v_{11}+v_{2}\geq{-v_{3}}}$, so for this case, we have that $-v_{3}\leq{v_{11}+v_{2}}\leq{0}$, which means that $\displaystyle\int_{a_{11}}p^{\min\{0,v_{2}+v_{11}\}}[1+\beta(p-1)]d\mu(a_{11})=\displaystyle\int_{v_{11}\leq{-v_{2}}}1+\beta(p-1)d\mu(a_{11})+\displaystyle\int_{v_{11}+v_{2}\geq{-v_{3}}}p^{v_{2}+v_{11}}[1+\beta(p-1)]d\mu(a_{11})=\displaystyle\int_{v_{11}\leq{-v_{2}}}1+\beta(p-1)d\mu(a_{11})+\sum_{\tau=1}^{v_{3}}\displaystyle\int_{v_{11}\geq{-(v_{3}+v_{2})}}p^{-\tau}[1+\beta(p-1)]d\mu(a_{11})$. Now we need to resolve $\beta=\min\{v_{2},v_{4},v_{2}+\max\{v_{1},v_{4}\}+v_{11}\}$, hence we need to divide the inner integral to different orderings of $v_{1},v_{2},v_{3},v_{4}$.

Case 1: $v_{1}\geq{v_{2}}\geq{v_{3}}\geq{v_{4}}$.
For this case, we have that $\beta=\min\{v_{2},v_{4},v_{2}+\max\{v_{1},v_{4}\}+v_{11}\}=\min\{v_{4},v_{1}+v_{2}+v_{11}\}$.

The two possible minimum values are equal when $v_{4}=v_{1}+v_{2}+v_{11}$, that is, when $v_{2}+v_{11}=-(v_{1}-v_{4})$. But for this case we have two subcases. If $v_{1}-v_{4}\leq{v_{3}}$, then, since $v_{11}\geq{-(v_{2}+v_{3})}$, we have that $v_{11}+v_{2}\geq{-v_{3}}\geq{-(v_{1}-v_{4})}\Rightarrow{v_{1}+v_{2}+v_{11}\geq{v_{4}}}\Rightarrow{\beta=v_{4}}$, hence 

$\displaystyle\int_{\underline{\lambda}}I_{5}\int_{a_{11}}p^{\min\{0,v_{2}+v_{11}\}}(1+v_{4}(1-p^{-1}))d\mu(a_{11})=$

$=\displaystyle\int_{\underline{\lambda}}I_{5}(1+v_{4}(1-p^{-1}))\int_{v_{11}\geq{-(v_{2}+v_{3})}}p^{\min\{0,v_{2}+v_{11}\}}d\mu(a_{11})$, but same as earlier $\displaystyle\int_{v_{11}\geq{-(v_{2}+v_{3})}}p^{\min\{0,v_{2}+v_{11}\}}d\mu(a_{11})=p^{0}\mu(\{v_{11}+v_{2}\geq{0}\})+\displaystyle\int_{{v_{2}+v_{11}}<{0}}p^{v_{2}+v_{11}}d\mu(a_{11})=1\mu(\{v_{11}\geq{-v_{2}}\})+\displaystyle\int_{{v_{2}+v_{11}}<{0}}p^{v_{2}+v_{11}}d\mu(a_{11})=p^{v_{2}}+\displaystyle\int_{-v_{3}\leq{v_{11}}<-v_{2}}p^{v_{2}+v_{11}}d\mu(a_{11})=p^{v_{2}}+\sum_{\tau=-v_{3}}^{-(v_{2}+1)}\displaystyle\int_{v_{11}=\tau}p^{v_{2}+\tau}d\mu(a_{11})=$

$=p^{v_{2}}+\sum_{\tau=1}^{v_{3}}\displaystyle\int_{v_{11}=-(v_{2}+\tau)}p^{\tau}d\mu(a_{11})=p^{v_{2}}+\sum_{\tau=1}^{v_{3}}p^{-\tau}(p^{v_{2}+\tau}-p^{v_{2}+\tau-1})=p^{v_{2}}+\sum_{\tau=1}^{v_{3}}p^{v_{2}}(1-p^{-1})=p^{2}+v_{3}=p^{v_{2}}(1+v_{3}(1-p^{-1}))$. Denote $I_{6}:=I_{5}p^{v_{2}}(1+v_{3}(1-p^{-1}))$, thus we have $\zeta_{L_{5,p}}^{\wedge}(s)=\displaystyle\int_{\underline{\lambda}}I_{6}d\mu(\lambda)$. We compute the complete expression $I_{6}=p^{4(v_{1}+v_{2}+v_{3}+v_{4})}p^{-(v_{2}+v_{3})-\max\{v_{1},v_{4}\}}p^{v_{1}+v_{2}}p^{v_{3}+v_{4}}p^{v_{3}}p^{v_{2}}(1+v_{3}(1-p^{-1}))p^{-(4v_{1}+6v_{2}+6v_{3}+4v_{4})s}=p^{7v_{1}+11v_{2}+11v_{3}+8v_{4}}(1+v_{3}(1-p^{-1}))(1+v_{4}(1+p^{-1}))p^{-(4v_{1}+6v_{2}+6v_{3}+4v_{4})s}=p^{(7-4s)v_{1}}p^{(11-6s)v_{2}}p^{(11-6s)v_{3}}p^{(8-4s)v_{4}}(1+v_{3}(1-p^{-1}))(1+v_{4}(1-p^{-1}))$ and integrate it over $\underline{\lambda}$, which translates to the infinite sum 

$S:=\sum_{v_{1}\geq{v_{2}\geq{v_{3}}\geq{v_{4}}}}p^{(7-4s)v_{1}}p^{(11-6s)v_{2}}p^{(11-6s)v_{3}}p^{(8-4s)v_{4}}(1+v_{3}(1-p^{-1}))(1+v_{4}(1+p^{-1}))$.

We notice that we have no constraint which dictates an order relation between $v_{1}$ and $v_{2}$, thus we have the following constraints

$v_{2}\geq{v_{3}}\geq{v_{4}}$ and $v_{1}\geq{v_{3}+v_{4}}\geq{v_{3}}\geq{v_{4}}$. But this allows us to break the computed sum into separate sums, where the index of summation must preserve the constraints between $v_{1},v_{2},v_{3},v_{4}$.

Denote $u_{4}:=v_{4}$, $u_{3}:=v_{3}-v_{4}$, ${u_{2}:=v_{2}-v_{3}}$ and $u_{1}:=v_{1}-(v_{3}+v_{4})$. With these notations, we have 

$S=\sum_{u_{1}=0}^{\infty}\sum_{u_{2}=0}^{\infty}\sum_{u_{3}=0}^{\infty}\sum_{u_{4}=0}^{\infty}p^{(7-4s)(u_{1}+u_{3}+2u_{4})}p^{(11-6s)(u_{2}+u_{3}+u_{4})}p^{(11-6s)(u_{3}+u_{4})}p^{(8-4s)u_{4}}(1+(u_{3}+u_{4})(1-p^{-1}))(1+u_{4}(1-p^{-1}))=$

$\sum_{u_{1}=0}^{\infty}p^{(7-4s)u_{1}}\sum_{u_{2}=0}^{\infty}p^{(11-6s)u_{2}}\sum_{u_{3}=0}^{\infty}p^{(29-16s)u_{3}}\sum_{u_{4}=0}^{\infty}p^{(44-24s)u_{4}}(1+(u_{3}+2u_{4})(1-p^{-1})+(u_{3}u_{4}+u_{4}^{2}(1-p^{-1})))$.
Denote $w_{1}:=p^{(7-4s)}$, $w_{2}:=p^{(11-6s)}$, $w_{3}:=p^{(29-16s)}$ and $w_{4}:=p^{(44-24s)}$.
We shall compute each summand separately.

$S_{1}:=\sum_{u_{1}}^{\infty}w_{1}^{u_{1}}\sum_{u_{2}}^{\infty}w_{2}^{u_{2}}\sum_{u_{3}}^{\infty}w_{3}^{u_{3}}\sum_{u_{4}}^{\infty}w_{4}^{u_{4}}=\frac{1}{1-w_{1}^{u_{1}}}\frac{1}{1-w_{2}^{u_{2}}}\frac{1}{1-w_{3}^{u_{3}}}\frac{1}{1-w_{4}^{u_{4}}}$.

$S_{2}:=\sum_{u_{1}}^{\infty}w_{1}^{u_{1}}\sum_{u_{2}}^{\infty}w_{2}^{u_{2}}\sum_{u_{3}}^{\infty}w_{3}^{u_{3}}\sum_{u_{4}}^{\infty}w_{4}^{u_{4}}(u_{3}+2u_{4})(1-p^{-1})=$

$=(1-p^{-1})\sum_{u_{1}}^{\infty}w_{1}^{u_{1}}\sum_{u_{2}}^{\infty}w_{2}^{u_{2}}\sum_{u_{3}}^{\infty}w_{3}^{u_{3}}\sum_{u_{4}}^{\infty}w_{4}^{u_{4}}(u_{3}+2u_{4})=$

$=(1-p^{-1})\sum_{u_{1}}^{\infty}w_{1}^{u_{1}}\sum_{u_{2}}^{\infty}w_{2}^{u_{2}}\sum_{u_{3}}^{\infty}w_{3}^{u_{3}}\sum_{u_{4}}^{\infty}w_{4}^{u_{4}}(u_{3}+2u_{4})=$

$=(1-p^{-1})\Bigl[\sum_{u_{1}}^{\infty}w_{1}^{u_{1}}\sum_{u_{2}}^{\infty}w_{2}^{u_{2}}\sum_{u_{3}}^{\infty}u_{3}w_{3}^{u_{3}}\sum_{u_{4}}^{\infty}w_{4}^{u_{4}}+$

$+\sum_{u_{1}}^{\infty}w_{1}^{u_{1}}\sum_{u_{2}}^{\infty}w_{2}^{u_{2}}\sum_{u_{3}}^{\infty}w_{3}^{u_{3}}\sum_{u_{4}}^{\infty}2u_{4}w_{4}^{u_{4}}\Bigr]=(1-p^{-1})\Bigl[\frac{1}{1-w_{1}}\frac{1}{1-w_{2}}\frac{w_{3}}{(1-w_{3})^{2}}\frac{1}{1-w_{4}}+2\frac{1}{1-w_{1}}\frac{1}{1-w_{2}}\frac{1}{1-w_{3}}\frac{w_{4}}{(1-w_{4})^{2}}\Bigr]$.

$S_{3}:=\sum_{u_{1}}^{\infty}w_{1}^{u_{1}}\sum_{u_{2}}^{\infty}w_{2}^{u_{2}}\sum_{u_{3}}^{\infty}w_{3}^{u_{3}}\sum_{u_{4}}^{\infty}w_{4}^{u_{4}}u_{4}(1-p^{-1})=$

$=(1-p^{-1})\sum_{u_{1}}^{\infty}w_{1}^{u_{1}}\sum_{u_{2}}^{\infty}w_{2}^{u_{2}}\sum_{u_{3}}^{\infty}u_{3}w_{3}^{u_{3}}\sum_{u_{4}}^{\infty}w_{4}^{u_{4}}u_{4}=$

$=\sum_{u_{1}}^{\infty}w_{1}^{u_{1}}\sum_{u_{2}}^{\infty}w_{2}^{u_{2}}\sum_{u_{3}}^{\infty}w_{3}^{u_{3}}\sum_{u_{4}}^{\infty}\frac{1}{1-w_{1}}\frac{1}{1-w_{2}}\frac{1}{1-w_{3}}\frac{w_{4}}{(1-w_{4})^{2}}$.

The second sub case is where $\beta=\min\{v_4,\alpha\}=\alpha<v_{4}$, where $\alpha=v_1+v_2+v_{11}$. 
But $v_{11}\geq{-(v_2+v_3)}\Rightarrow{v_1+v_2+v_{11}\geq{v_1-v_3}}$, which means that ${v_{4}>v_1-v_3}$ is a necessary condition for this sub case. We use a strong inequality here, because we have already counted the case where $v_4=v_1+v_2+v_{11}$.

Thus, $v_1-v_3\leq\alpha<v_4$, which means that for this case, the value of $\beta$ is not constant, but rather $\beta\in\{v_1-v_3,v_1-v_3+1,v_1-v_3+2,\dots,v_4-2,v_4-1\}$.

Hence, the lower bound for $v_{11}$ remains $-(v_2+v_3)$, but our upper bound comes from either $\min\{0,v_2+v_{11}\}$ or $\alpha$. We have \[\min\{0,v_2+v_{11}\}\leq{0}\Rightarrow{v_2+v_{11}\leq{0}}\Rightarrow{v_{11}\leq{-v_2}}\] 
and \[\alpha<v_4\Rightarrow{v_1+v_2+v_{11}\leq{v_4-1}}\Rightarrow{v_{11}\leq{v_4-(v_1+v_2+1)}}\]
But $v_4\leq{v_1}\Rightarrow{v_4-v_1\leq{0}}$, and hence $v_4-v_1-v_2-1\leq{-v_2-1}$, which means that $-(v_2+v_3)\leq{v_{11}}\leq{v_4-(v_1+v_2+1)}$

Therefore, we have

\[S_{11}=\displaystyle\int_{a_{11}}p^{\min\{0,v_{2}+v_{11}\}}[1+\beta(1-p^{-1})]d\mu(a_{11})=\displaystyle\int_{a_{11}}p^{\min\{0,v_{2}+v_{11}\}}[1+\alpha(1-p^{-1})]d\mu(a_{11})=\]
\[=\sum_{v_{11}=-(v_2+v_3)}^{v_{4}-(v_1+v_2+1)}p^{\min\{0,v_2+v_{11}\}}[1+(v_1+v_2+v_{11})(1-p^{-1})]d\mu(a_{11})\]

We notice that for this case we do not have $\min\{0,v_2+v_11\}=0$, because $v_{11}+v_2=v_4-v_1-1=0\Rightarrow{v_4-v_1=1}$, which contradicts $v_1\geq{v_4}$. Therefore 
\[S_{11}=\sum_{v_{11}=-(v_2+v_3)}^{v_{4}-(v_1+v_2+1)}p^{v_2+v_{11}}[1+(v_1+v_2+v_{11})(1-p^{-1})]d\mu(a_{11})\]

$=\displaystyle\int_{\underline\lambda}I_{5}\sum_{v_{11}=-(v_2+v_3)}^{v_{4}-v_1-(v_2-1)}p^{v_{2}+v_{11}}[1+(v_1+v_2+v_{11})(1-p^{-1})]d\mu(a_{11})=$

$=\displaystyle\int_{\underline\lambda}I_{5}\Bigl[p^{-v_{3}}[1+(v_1-v_3)(1-p^{-1})]\mu(p^{-(v_{2}+v_3)}\mathbb{Z}_{p}\setminus{p^{-(v_{2}+v_3)+1}\mathbb{Z}_{p}})+$

$+p^{-v_{3}+1}[1+(v_1-v_3+1)(1-p^{-1})]\mu(p^{-(v_{2}+v_3)+1}\mathbb{Z}_{p}\setminus{p^{-(v_{2}+v_3)+2}\mathbb{Z}_{p}})+$

$\vdots$

$+p^{-(v_{4}+v_1)+1}[1+(v_4+1)(1-p^{-1})]\mu(p^{-(v_{4}+v_1)+1}\mathbb{Z}_{p}\setminus{p^{-(v_{4}+v_1)+2}\mathbb{Z}_{p}})\Bigr]=$

$=\displaystyle\int_{\underline\lambda}I_{5}\Bigl[p^{-v_{3}}[1+(v_1-v_3)(1-p^{-1})]p^{v_{2}+v_3}-p^{v_{2}+v_3-1}+$

$+p^{-v_{3}+1}[1+(v_1-v_3+1)(1-p^{-1})]p^{v_{2}+v_3-1}-p^{v_{2}+v_3-2})+$

$\vdots$

$+p^{-(v_{4}+v_1)+1}[1+(v_4+1)(1-p^{-1})]p^{v_{4}+v_1-1}-p^{v_{4}+v_1-2}\mathbb{Z}_{p})\Bigr]=$
\end{document}