\documentclass{article}

% Language setting
% Replace `english' with e.g. `spanish' to change the document language
\usepackage[english]{babel}


% Useful packages
\usepackage{amsmath}
\usepackage{amssymb}
\usepackage{graphicx}

\title{Your Paper}
\author{You}

\begin{document}
\maketitle
Denote $G_{5}:=G_{5}(\mathbb{Z}_{p})$, and $G_{5}^{+}:=G_{5}^{+}(\mathbb{Q}_{p})$.

$\zeta_{L_{5,p}}^{\wedge}(s)=\displaystyle\int_{{G_{5}^{+}}}{|\det{g}|_{p}^{s}}d\mu(G_{5})=\displaystyle\int_{{G_{5}^{+}}}{|\det{uh}|_{p}^{s}}d\mu(G_{5})$, where $h\in{H}$ and $u\in{N_{h}}$. 

Each $u$ is unipotent, hence $\zeta_{L_{5,p}}^{\wedge}(s)=\displaystyle\int_{{G_{5}^{+}}}{|\det{h}|_{p}^{s}}d\mu(G_{5})=\displaystyle\int_{{G_{5}^{+}}}{|\lambda_{1}^{4}\lambda_{2}^{6}\lambda_{3}^{6}\lambda_{4}^{4}|_{p}^{s}}d\mu(G_{5})=$

$=\displaystyle\int_{{G_{5}^{+}}}{\Bigl[|\lambda_{1}^{4}|_{p}|\lambda_{2}^{6}|_{p}|\lambda_{3}^{6}|_{p}|\lambda_{4}^{4}|_{p}\Bigr]^{s}}d\mu(G_{5})$, by the inductive formula we have found for every $|h|$. 

We denote $v_{i}=v(\lambda_{i})$, 

and so $\zeta_{L_{5,p}}^{\wedge}(s)=\displaystyle\int_{{G_{5}^{+}}}{\Bigl[p^{-4v_{1}}p^{-6v_{2}}p^{-6v_{3}}p^{-4v_{4}}\Bigr]^{s}}d\mu(G_{5})=\displaystyle\int_{{G_{5}^{+}}}{p^{-(4v_{1}+6v_{2}+6v_{3}+4v_{4})s}}d\mu(G_{5})$. 

We denote $\Lambda:=p^{-(4v_{1}+6v_{2}+6v_{3}+4v_{4})s}$.
Now we use the natural matrix decomposition of the $N_{h}$ matrix of Berman's, which means that

$\zeta_{L_{5,p}}^{\wedge}(s)=\displaystyle\int_{{G_{5}^{+}}}{\Lambda}d\mu(G_{5})=\displaystyle\int_{\underline\lambda}\int_{\underline{a}}\int_{\underline{b}}\int_{\underline{c}}{\Lambda}d\mu(\underline{c})d\mu(\underline{b})d\mu(\underline{a})d\mu(\underline{\lambda})$. Since $\Lambda$ depends only on $\lambda_{1},\lambda_{2},\lambda_{3},\lambda_{4}$, which appear only in the computation of the outermost integral, we consider them as constants for all the inner integrals, which means that  we have $\zeta_{L_{5,p}}^{\wedge}(s)=\displaystyle\int_{\underline\lambda}\Lambda\int_{\underline{a}}\int_{\underline{b}}\int_{\underline{c}}1d\mu(\underline{c})d\mu(\underline{b})d\mu(\underline{a})d\mu(\underline{\lambda})$, hence all the inner integrals evaluate to the measure of their domains of integration.
now we compute the innermost integral by considering $\underline{a}$, $\underline{b}$ and $\underline{\lambda}$ as constants, and integrating only over $\underline{c}$. Considering the multiplication $uh$, we observe that for each element $c_{j}$, we must have that $\rho_{j}=c_{j}\lambda_{1}\lambda_{2}\lambda_{3}\lambda_{4}\in\mathbb{Z}_{p}$, which means that $v(\rho_{i})=v(c_{i}\lambda_{1}\lambda_{2}\lambda_{3}\lambda_{4})\geq{0}\Rightarrow{v(c_{i})+v_{1}+v_{2}+v_{3}+v_{4}}\geq{0}\Rightarrow{v(c_{i})\geq{-(v_{1}+v_{2}+v_{3}+v_{4})}}$. But this means that $c_{i}\in{p^{-(v_{1}+v_{2}+v_{3}+v_{4})}\mathbb{Z}_{p}}$, and since the domain of integration for this integral is $\underline{c}=\{c_{1},c_{2},c_{3},c_{4}\}$, then $\mu(\underline{c})=|c_{j}|_{p}^{4}=p^{4(v_{1}+v_{2}+v_{3}+v_{4})}$. Denote $C:=p^{4(v_{1}+v_{2}+v_{3}+v_{4})}$, we now have that $\zeta_{L_{5,p}}^{\wedge}(s)=\displaystyle\int_{\underline\lambda}C\Lambda\int_{\underline{a}}\int_{\underline{b}}1d\mu(\underline{b})d\mu(\underline{a})d\mu(\underline{\lambda})$.
\end{document}